%%%%%%%%%%%%%%%%%%%%%%%%%%%%%%%%%%%%%%%%%
% The Legrand Orange Book
% LaTeX Template
% Version 3.1 (February 18, 2022)
%
% This template originates from:
% https://www.LaTeXTemplates.com
%
% Authors:
% Vel (vel@latextemplates.com)
% Mathias Legrand (legrand.mathias@gmail.com)
%
% License:
% CC BY-NC-SA 4.0 (https://creativecommons.org/licenses/by-nc-sa/4.0/)
%
% Compiling this template:
% This template uses biber for its bibliography and makeindex for its index.
% When you first open the template, compile it from the command line with the 
% commands below to make sure your LaTeX distribution is configured correctly:
%
% 1) pdflatex main
% 2) makeindex main.idx -s indexstyle.ist
% 3) biber main
% 4) pdflatex main x 2
%
% After this, when you wish to update the bibliography/index use the appropriate
% command above and make sure to compile with pdflatex several times 
% afterwards to propagate your changes to the document.
%
%%%%%%%%%%%%%%%%%%%%%%%%%%%%%%%%%%%%%%%%%

%----------------------------------------------------------------------------------------
%	PACKAGES AND OTHER DOCUMENT CONFIGURATIONS
%----------------------------------------------------------------------------------------

\documentclass[
	11pt, % Default font size, select one of 10pt, 11pt or 12pt
	fleqn, % Left align equations
	a4paper, % Paper size, use either 'a4paper' for A4 size or 'letterpaper' for US letter size
	oneside, % Uncomment for oneside mode, this doesn't start new chapters and parts on odd pages (adding an empty page if required), this mode is more suitable if the book is to be read on a screen instead of printed
]{LegrandOrangeBook}

% Book information for PDF metadata, remove/comment this block if not required 
\hypersetup{
	pdftitle={Chess, Life and Anything Else}, % Title field
	pdfauthor={Ming Lu}, % Author field
	pdfsubject={Peace}, % Subject field
	pdfkeywords={chess, meditation, myths...}, % Keywords
	pdfcreator={LaTeX}, % Content creator field
}

\addbibresource{sample.bib} % Bibliography file

\definecolor{ocre}{RGB}{243, 102, 25} % Define the color used for highlighting throughout the book

\chapterimage{zen-stone.jpg} % Chapter heading image
\chapterspaceabove{6.5cm} % Default whitespace from the top of the page to the chapter title on chapter pages
\chapterspacebelow{6.75cm} % Default amount of vertical whitespace from the top margin to the start of the text on chapter pages

\usepackage{chessboard}
\usepackage{xskak}
\usepackage{sectsty}
\usepackage{tocloft}
\usepackage{epigraph}
\usepackage{multicol}

%----------------------------------------------------------------------------------------

\begin{document}

%----------------------------------------------------------------------------------------
%	TITLE PAGE
%----------------------------------------------------------------------------------------

\titlepage % Output the title page
	{\includegraphics[width=\paperwidth]{background.pdf}} % Code to output the background image, which should be the same dimensions as the paper to fill the page entirely; leave empty for no background image
	{ % Title(s) and author(s)
		\centering\sffamily % Font styling
		{\Huge\bfseries Chess, Life and Anything Else\par} % Book title
		\vspace{16pt} % Vertical whitespace
		%{\LARGE A Practical Guide\par} % Subtitle
		\vspace{24pt} % Vertical whitespace
		{\huge\bfseries Ming Lu\par} % Author name
	}

%----------------------------------------------------------------------------------------
%	COPYRIGHT PAGE
%----------------------------------------------------------------------------------------

\thispagestyle{empty} % Suppress headers and footers on this page

~\vfill % Push the text down to the bottom of the page

\noindent Copyright \copyright\ 2025 Ming Lu\\ % Copyright notice

\noindent \textsc{Published by Publisher}\\ % Publisher

\noindent \textsc{\href{https://www.latextemplates.com/template/legrand-orange-book}{book-website.com}}\\ % URL


%----------------------------------------------------------------------------------------
%	TABLE OF CONTENTS
%----------------------------------------------------------------------------------------

\pagestyle{empty} % Disable headers and footers for the following pages
\newcommand{\Color}[1]{\hypersetup{linkcolor=#1}\color{#1}}

\titlespacing*{\part}{10em}{0em}{0em}
% Optionally change font style
\renewcommand{\cftpartfont}{\Large\bfseries\Color{ocre}} 
\renewcommand{\cftchapfont}{\Large\bfseries\Color{ocre}} 
\renewcommand{\cftsecfont}{\large}  % Bold section font
\renewcommand{\cftsubsecfont}{\large\itshape} % Italic subsection font
\renewcommand{\cftsecpagefont}{\large}   % Page numbers for sections
\renewcommand{\cftsubsecpagefont}{\large} % Page numbers for subsections

\begingroup % start a TeX group
\color{ocre}% or whatever color you wish to use
\tableofcontents % Output the table of contents
\listoffigures % Output the list of figures, comment or remove this command if not required
\listoftables % Output the list of tables, comment or remove this command if not required
\endgroup


\pagestyle{fancy} % Enable default headers and footers again

\cleardoublepage % Start the following content on a new page

% Define section and subsection colors
\partfont{\color{ocre}\Huge}
\chapterfont{\color{ocre}}
\sectionfont{\color{ocre}}
\subsectionfont{\color{ocre}}

%----------------------------------------------------------------------------------------
%	Main Content
%----------------------------------------------------------------------------------------
\part*{Chess and Everything}
\chapter{Chess (and) History}
\section{Diplomacy with Chess}
\begin{figure}[H] % Use [H] to suppress floating and place the figure/table exactly where it is specified in the text
	\centering % Horizontally center the figure on the page
	\includegraphics[width=0.5\textwidth]{cards/history/franklin_play_howe.jpg} % Include the figure image
	\caption{Quiet Tension and Concentration}
	\label{fig:franklin} % Unique label used for referencing the figure in-text
\end{figure}
It began with a game of chess. In December 1774 Benjamin Franklin met Caroline Howe at the Royal Society in London. She challenged him to a game, which turned into a series of chess matches over several days.

On Christmas Day, she introduced the American to her brother, Lord Richard Howe, who told Franklin that some members of the British government ``were extremely well disposed to any reasonable accommodation'' between the British and the American colonists. The two men continued to use the chess matches as a front for a series of secret meetings to negotiate a piece. Nothing concrete resulted form the meetings, which ended in March 1775, but the Franklin and Howe concluded their talks with a mutual respect for one another.

The following year the admiral was appointed commander of the British navy in North America. Historian Walter Isaacson summarizes what happened just after the Continental Congress issued its Declaration of Independence. [Admiral Howe] carried a detailed proposal that offered a truce, pardons for the rebel leaders (with John Adams secretly exempted) and rewards for any American who helped restore peace. 

Because the British did not recognize the Continental Congress as a legitimate body, Lord Howe was unsure where to direct his proposals. So when he reached Sandy Hook, New Jersey, he sent a letter to Franklin, whom he addressed as ``my worthy friend.'' He had ``hopes of being serviceable'' Howe declared, ``in promoting the establishment of lasting peace and union with the colonies''.

Congress granted Franklin permission to reply, which he did on July 30. Franklin opened the following letter with a cordial statement, but the tone quickly sours as he rejected the offer with fury: ``it is impossible we should think of submission to a government'' whose ``atrocious injuries have extinguished avery remaining Spark of Affection for that Parent Coiuntry we once held so dear.''

\section{Steinitz's Tragedy}
Wilhelm Steinitz was more than just a great chess player-he wa a revolutionary. Born in Prague in 1836, Steinitz transformed chess from a romantic game of dashing attacks into a scientific discipline based on positional principles. In 1886, he became the first official World Chess Champion, a title he defended successfully for years. His theories about chess strategy formed the foundation of modern chess thinking, and for over a quarter-century, he was considered virtually invincible in match play. 

But the weight of maintaining supremacy in chess, a game that demands absolute mental precision and psychological fortitude, would eventually exact a terrible price.

In 1894, at age 58, Steinitz faced a young challenger named Emanuel Lasker in a match for the world championship. What followed was a psychological catastrophe that revealed the devastating toll competitive chess can take on human mind. 

Game 7 marked the start of five consecutive losses to Lasker. This was an unprecedented humiliation for a man who had been unbeaten in match play for over 25 years and had previously declared he would win without doubt.

After this crushing five-game streak, Steinitz asked for a week's rest-a telling admission of his shattered mental state. In his own words after one of these losses, he wrote: ``Mr. Lasker then broke into my game in the most woful manner and won a Pawn, blocking my pieces, and he had things almost all his own way.'' Steinitz attributed his collapse to poor physical condition, particularly his disability which prevented him from walking and exercising properly, cause ``insomnia, rushing of blood to the head, and general depression.''

He had lost his crown, but worse was yet to come.

\subsection*{The 1896-97 ematch: Complete Mewntal Disintegration}
Desperate to reclaim his title and prove his theories were still sound, Steinitz challenged Lasker to a rematch. The rematch was even more devastating. Steinitz's performance at the Nuremberg tournament before the match was subpar, finishing sixth place, which predetermined his bad result to an extent. In the rematch held in Moscow from November 1896 to January 1897, Steinitz won only 2 games, drawing 5 and losing 10-a crushing defeat.

The aftermath was tragic: Just four weeks after the match Steinitz lost his mind and had to seek psychiatric help. Shortly after the match, he had a mental breakdown and was confined for 40 days in Moscow sanatorium, where he played chess with the inmates. There, in his confinement, he would play chess against other patients, eventually even making claims that he was playiung chess games with God himself-believing he could give God pawn and move odds and still win.

His mental state never fully recovered. Two years later, he reportedly experienced delusions while returning by ship from the London 1899 tournament. Commitment to a series of mental hospitals followed beginning in Febrary 1900, and he died in the state mental hospital on Ward's Island on August 12, 1900. The death certificate listed ``chronic endocardia (mitral stenosis)'' and ``acute melancholia'' as causes of death. The first World Chess Champion died nearly penniless, his brilliant mind shattered by the very game he had mastered.

\section{Boris Spassky's Nirvana}
\subsection{Nirvana}
Nirvana is not an escape from life, but a transformation within it. It is the moment when the fire of desire and fear burns itself out, leaving behind a clear, still awareness. In Buddhist thought, this state is beyond joy and sorrow - it is freedom from the compulsions that make us cling to success and recoil from failure. What dies in Nirvana is not the self, but the illusion that life must always go our way.

Transformation toward Nirvana does not happen through achievement, but through exhaustion of striving. When a person suffers deeply and yet remains present to that suffering, something inside begins to shift. The mind sees that every victory fades and every defeat passes, and that both belong to the same cycle of attachment. In that seeing, a kind of quiet surrender arises - not resignation, but acceptance. 

This transformation is like a sword tempered by fire. The ego, once raw and restless, becomes refined through the heat of experience. Out of this refinement comes equanimity - a mind unshaken by fortune or loss. Nirvana, then, is not a distant heaven, but a change of vision; the realization that freedom lies not in controlling life, but in letting go of the need to control it.

\subsection{How Spassky became World Champion}
In the 1958 USSR Championship, Boris Spassky was leading the tournament at the beginning and needed only a draw in the last round against Mikhail Tal. After a dramatic struggle, he lost - from a winning position - and with it, his chance to qualify for the Interzonal, the gateway to the World Championship.

Three years later in 1961, Spassky was again leading the championship, undefeated. In his critical game against Lev Polugaevsky, he launched a brilliant sacrifice that could have secured victory, but he missed the win and eventually lost. 


The defeat broke his momentum; he collapsed in the remaining rounds and once more failed to reach Interzonal.

Commentators said Spassky would never become World Champion—that he was too emotional, too fragile at decisive moments.

In 1966, Spassky became the final challenger and lost to Petrosian. In 1969, he became the challenger again and this time he beat Petrosian to become World Champion.

\section{1984 Karpov - Kasparov World Championship}
The 1984 Karpov-Kasparov World Championship in Moscow was not just a chess match - it was a myth unfolding on a 64-square battlefield. It followed, almost eerily, Joseph Campbell's hero's journey: the call, the descent, the ordeal and the miraculous return

\subsection*{The call to Adventure}
Garry Kasparov was twenty-one, the youngest challenger in history. Across from him sat Anatoly Karpov - the reigning world champion, embodiment of Soviet control, precision, and restraint.

Karpov had inherited Fischer's vacant crown a decade earlier and rulesd chess with a surgeon's calmness. Kasparov, by contrast, was fire-creative, defiant, unorthodox. 

The Soviet establishment saw the match as a test of loyalty and order; the yong challenger symbolized chaos, energy and change. The gods had arranged their duel. 

\subsection*{Crossing the Threshold}
The first games were a diaster. Karpov struck like a cold wind: clinical, silent, inevitable. 

After nine games, Kasparov was down 4-0. In the mythic sense, this was the descent into the underwold- the young hero's illusions burned away.

Observers whispered that the match might end 6-0 within days.

Karpov seemed invincible, his eyes as expresisonless as stone. Kasparov stood on the edge of annihilation. Yet, instead of collapsing, he chose a new path-patience. He stopped fighting on Karpov's terms and began drawing, consolidating, waiting. It was not glory; it was survival. The hero refused to die.

\subsection*{The Road of Trials}
From game 10 onward, something shifted. Karpov pressed again and again - 17 draws followed- but the young challenger did not fall. He built a fortress of iron will. The Moscow crowd grew restless: the ``boy from Baku'' was not leaving. For months the match dragged on - the longest in history.

Kasparov trained his mind like a monk in the desert, meditating under pressure. Karpov, meanwhile, began to fade. His face grew pale, his hands trembled. The once unbreakable champion was aging before the audience's eyes.

This was the ordeal: two minds locked in endless struggle, endurance replacing brilliance.

\subsection*{The Abyss and the Miracle}
After five months, Karpov led 5-0. He needed only one more win to end the saga. But the last step proved impossible. Karsparov began to rise - cautiously, methodically. In game 32, he won his first victory. The audience gasped; life had returned to the hero. 

Then came another, and another. The impossible had begun to happen. The young hero was crawling out of the pit. Karpov's energy broke; his will cracked. His apotheosis was near - the moment of transcendence through suffering.

\subsection*{The Return without Victory}
Then came the twist: after 48 games, the match was adruptly stopped by the Soviet chess federation, officially ``for health reasons''. No winner was declared. Karpov retained his title; Kasparov kept his sould. In Campbell's terms, this was the refusal of the gods to grant closure - the hero returns not with a crown, but with knowledge. Kasparov had been through death and rebirth. He now understood the cost of greatness.

A year later, when they met again in 1985, he completed the journey - won the title, and began his own mythic reign.

\subsection*{Epilogue}
Karpov was order, Kasparov chaos. Their 1984 clash was not about pieces, but about two visions of intelligence: one shaped by control, the other by rebellion.

And in the ruins of that endless match, the audience witnessed something larger than chess- the moment when the new hero emerged from the ashes, unbroken, carrying the flame forward.

\section{Anony of a Winning Position}
\begin{figure}[H] % Use [H] to suppress floating and place the figure/table exactly where it is specified in the text
	\centering % Horizontally center the figure on the page
	\includegraphics[width=0.5\textwidth]{cards/history/ivanchuk_cry.jpg} % Include the figure image
	\caption{Heartbreak after a Game}
	\label{fig:placeholder} % Unique label used for referencing the figure in-text
\end{figure}


The clock was merciless. Three minutes for the entire game, two seconds added after each move-barely enough time to think, only enough to survive.

Round 11 of the 2024 World Blity Championship in New York. The stakes couldn't be higher. Eight players tied for first place on 9 points. Only eight could advance to the knockout stage. Vassyl Ivanchuk, the 55-year-old Ukrainian legend-a man who had danced with chass brilliance for decades-sat across from Daniel Naroditsky, fighting for his tournament life.

Naroditsky launched a sacrificial attack, pieces flying across the board in a blur of aggression. The pressure was suffocating. But Ivanchuk, drawing on a lifetime of tactical genius, defended brilliantly. He navigated the chaos, parried the threats, and emerged from the storm with a winning position. The board told a clear story-victory was his to claim. 

Then came move 40. Ivanchuk made an automatic move. The advantage that had been crystal clear moments ago began to evaporate like morning mist. The position had flipped entirely. What had been victory was now resignation.

Ivanchuk's hand moved to tip his king over. The gesture felt like it took forever-a king falling in slow motion, carrying with it decades of passion, frustration, and desperate love for this impossible game.

And then the tears came. The veteran grandmaster slumped over his board, shoulders shaking, tears streaming down his face. Now quiet dignified tears, but raw uncontrollable sobbing. The kind of grief that comes from watching victory slip through your fingers when you needed it most. 

\section{``I drank coffee''}
Life is like a game of chess, full of emotional highs and lows. Few moments capture this better than a world championship match, where pressure is immense-victory brings glory, while defeat can lead to silence and obscurity.

In Game 12 of the 2024 World Chess Championship, reigning champion Ding Liren was one point behind, standing at the edge of loss. His challenger needed only a draw- one of the hardest situations to overturn. Yet Ding played with clarity and calm, offering his opponent no real chances. The game became legend.

Afterward, the world asked how he prepared for such a performance under pressure. Ding smiled and said, ``it's simple. I changed my haircut alittle. I drank coffee.''

Simplicity, in a moment where the world expected depth, complexity or secret preparation. Ding reminded us: power doesn't always come from doing more, but from doing less-well. In life, as in chess, we often overthink. But clarity, presence and small shifts in mindset can carry us through.

\section{The Fatal Glance: The Story of Orpheus and Eurydice}
Orpheus was the greatest musician who ever lived. When he played his lyre, trees bent to listen, rivers changed their cource, and wild beasts grew gentle. His wife Eurydice was the love of his life, but their happiness was cut short when she died from a serpent's bite while fleeing an unwanted pursuer.

Consumed by grief, Orpheus did what no living mortal had dared: he descended into the underworld itself to bring her back.

His music softend even the hardest hearts in Hades. The tortured souls ceased their weeping to listen. Cerberus, the three-headed guard dog, lay down peacefully. Even the Furies, ancient goddesses of vengeance, wept at his songs. Finally, Orpheus stood before Hades and Persephone, the king and queen of the dead, and sang of his love and his lose.

The rulers of the underworld was moved. Hades, who had never granted such a request, made Orpheus an offer:``You may take Eurydice back to the world of the living. But you must lead the way, and you must not look back at her until you both stand in the sunlight. If you turn around even once before then, she will return to me forever.'' 

It seemed simple enough. Orpheus had already accomplished the impossible-he had descended to the underworld and persuaded Death itself to release his beloved. All he had to do now was walk the path back to the surface without looking behind him. The hard part was over. Victory was within his grasp.

\subsection*{The Ascent}
Orpheus began the long climb up through the dark passages toward the world above. Behind him, he could hear nothing-no footsteps, no breathing, no sign that Eurydice was following. The underworld was utterly silent except for his own footfalls echoing in the gloom.

As he climbed, doubt began to creep into his mind. Was she really there? Had Hades truly kept his word? Perhaps this was a cruel trick, and he was climbing alone. He strained to hear some sound of her presence, but the silence was complete.

The path seemed endless. Every step brought him closer to the surface, closer to success-but also amplified his uncertainty. He had taken such an enormous risk descending into Hades. He had gambled everything on this journey. Shouldn't he make sure his gamble had paid off? What if he emerged intosunlight alone, having trusted blindly when a single glance could have given him certainty?

He tried to reason with himself: I must trust. I've come this far. Just keep walking. But another voice argued back: But what if she's not there? Better to know now. Better to turn back and try again than to discover at the end that you've been walking alone.

Light began to appear ahead-the entrance to the world of the living was near. He was almost there. Just a few more steps and they would both be free. The ordeal was nearly over.

\subsection*{The Fatal Choice}
And it was precisely then, withj victory so close he could see the sunlight, that Orpheus faced his terrible choice. He could take the risk of trusting-of walking forward into that light without confirmation, gambling that Eurydice was behind him. Or he could play it safe, turn around, and verify that she was truly there before taking those final steps. The safe choice seemed so reasonable. Just one quick look to confirm. Then he could turn back around and complete the journey with confidence rather than doubt. Surely the gods would understand-he was so close to the exit anyway. What difference could one glance make when the journey was nearly complete?

Ah the very threshold, with sunlight on his face and living world just step away, Orpheus made his decision. He swould not take the final risk. He would not walk blindly into the light. He turned around to look. Eurydice was there. She had been there all along, following faithfully behind him, just as Hades had promised. Their eyes met for one brief, beautiful moment.

An then she began to fade. Her form grew transparent, dissolving like moning mist. She reached out toward him, but her hand passed through his like smoke. ``Farewell,'' she whispered, and the word echoed as she was pulled back down into the darkness, back to the realm of the dead-this time forever.

\section{The youngest World Champion}
\begin{figure}[H] % Use [H] to suppress floating and place the figure/table exactly where it is specified in the text
	\centering % Horizontally center the figure on the page
	\includegraphics[width=0.5\textwidth]{cards/history/ding_blunders.jpg} % Include the figure image
	\caption{Face palm after a Blunder.}
	\label{fig:placeholder} % Unique label used for referencing the figure in-text
\end{figure}
In 2024, the chess world was abuzz with excitement as Ding Liren, the reigning World Chess Champion, faced off against the young prodigy, Gukesh D. S. It was a match that had been building for months, a fierce clash of generations. Gukesh, the 18-year-old Indian grandmaster, had risen through the ranks with a remarkable blend of creativity and calculation, making him a formidable opponent for even the most seasoned players.

The final game of the World Chess Championship unfolded in a magnificent hall brimming with spectators, media representatives, and fans from around the globe. An electric tension permeated the atmosphere as both players took their seats, acutely aware of the stakes. With the score tied at 6.5 - 6.5, the pressure was palpable; a draw would lead to a rapid playoff, a scenario that favored Ding. As a result, Gukesh was determined to push aggressively for victory.

For Ding, the weight of the title rested heavily on his shoulders; he had fought hard to reach this point and was resolute in defending his title. Gukesh, on the other hand, was hungry for victory, poised to seize the moment and make history as the youngest World Chess Champion.

As the match commenced, Ding employed a conservative and familiar opening, relying on strategies that had historically served him well. In contrast, Gukesh opted for a more aggressive approach, choosing unconventional lines and applying early pressure on Ding’s position.

In the middle game, Ding found himself in a recognizable scenario—he was slightly better, with a solid position and minor advantages in pawn structure. However, as he pushed for a secure and strategic victory, he meticulously weighed every option, contemplating the risks associated with each move.

Gradually, Ding lost his advantage after a few moves. Now, he had to defend an endgame with one pawn down. The position was theoretically a draw, but concentration remained vital. Ultimately, Ding blundered, overlooking Gukesh's ability to simplify the position and win in a straightforward pawn endgame.

Perhaps Ding was too tired, or maybe he was overly cautious. Playing for a win when holding an advantage might have been a more viable path than aiming for a draw.

I still remember what Grandmaster Peter Leko commented during the match ``sometimes the biggest risk is not to take a risk.''  \index{Risk, Uncertainty}

\chapter{Chess and Life}
\section{Life as Chess Game}
Life, like chess, unfolds on a board of choices. Each move shapes the next, each mistake demands recovery, and every player faces the same truth - there is no rewinding the clock. Strategy, foresight, and patience are not luxuries; they are the means to survival. Consider Napoleon at Waterloo. For years he had played the board of Europe with unmathced brilliance, predicting opponents' response serveral moves ahead. But at Waterloo, he misjudged both weather and timing - much like a grandmaster who overestimates his position. The rain delayed his artillery, and his opponent, Wellington, played defensively, waiting for the right counter. Napoleon's final gambit, a desperate assault, collapsed like a miscalculated endgame - proof that even a master's plan fails when overconfidence blinds calculation. 

Even in personal history, the pattern repeats. When Marie Curie chose to pursue science agianst societal expectations, she opened with a bold gambit - sacrificing comfort for knowledge. Her discoveries reshaped the scientific board forever.

In chess and in life, the truth is the same: mastery is not about control, but about awareness. The wise player accepts uncertainty, plans without attachment and sees each move not as an end, but as part of a living strategy. Victory comes not from knowing the future, but from being present for the move that is now. 
\section{Present Moment}
When life rises and crashes like the sea, the only thing that saves you is focus - not on fortune's whim, but on this moment, the piece of deck beneath your feet. 

Focus is the hero's rudder. When Odysseus sailed past the Sirens, he did not control the sea; he controlled himself. The hero never stops the storm - he survives by keeping his hands on the rope and his eyes on the next move.

In mythic terms, every ``now'' is a test from the gods. To live fully in the present is not to deny fate; it's to 
dance with it. When the wave lifts, you ride. When it falls, you bend. The world will always rise and fall - but the one who stays awake in the middle of motion becomes like the calm eye of a storm: untouched, aware, alive. 

That's why focus matters. Because the present moment is the only place where the hero can steer.

\section{Meditation}
Meditations is an old as humanity's first awareness of its own thoughts. It arose not from religion, but from the simple realization that the minditslef can be observed. Long before temples or doctrines, humans noticed that pain, fear and desire were not fixed realities - they were movements inside consciousness. 

In India, meditaitons grew from the early Vedic seers who sought stillness beyond retual. They discoverd that by watching breath and thought, a deeper awareness emerged - something constant beneath the noise.

The Buddha later turned this into method: sit, breathe, observe. See that all things - sensations, thoughts, even ``I'' - arise and pass away. From China to Greece, similar insights appeard: Taoist spoke of wu wei - effortless awareness. Greek Stoics trined the mind to stay within what can be controlled - the present act. Christian mystics and Sufi dervishes, in their own language, spoke of silence, rememberance, presence before devine. Different forms, same discovery: the way out of suffering is through awareness.

Meditations begins when you stop chasing the next thought and simply see. Normally, the mind drifts between past and future - replaying wounds, anticipating gain. This movement creates anxiety, pride, regret. Meditation cuts the circuit: you return to now. When attention anchors in the present, time loosens. The noise of self fades. There is no ``me trying to win'' or ``me who failed''. There is just breathing, sound, sensation - the raw fabric of existence.

This state is not escape: it is contact. You finally meet reality without distortion. Connection to the Present Moment Meditation is training for the same clarity the hero needs amid chaos.


In life, fortune shifts, emotions surge. Without awareness, we react blindly - anger, fear, craving. Meditaiton strengthens the ability to notice first, act later. It builds an inner stillness that doesn't depend on circumstances. To live meditatively means to live in the only time that truly exists: the present.

The past is memoryu, the future imagination - both flicker in the mind. Only the present is real, and meditation is how you stand in it, fully awake. 

Religion made meditation sacred. Philosophy made it rational. Psychology made it practical. But its truth remains timeless - the quiet power to see, breathe and steer your life with open eyes. \index{Meditation}
Meditation and chess share the same mental discipline - the art of being fully present. In meditation, the goal is to center the mind on the current moment- not to dwell on past mistakes, nor to fear what is yet to come. In chess, this presence becomes a weapon. Every move demands complete awareness: calculation, intuition, and calm under pressure. A player who clings to a lost opportunity or worries about the next ten moves loses clarity - just as a distracted miditator loses balance. Both practices train the same muscle: the ability to return to now, again and again. Through meditation, the chess player sharpens focus, steadies emotions and finds stillness in chaos - the same stillness from which the best moves arises.


\section{Risk and Uncertainty}
Most people are confused with risk and uncertainty, often using the terms interchangeably. However, they represent different concepts that are crucial in understanding decision-making processes.

Risk refers to situations where future outcomes are known, and the probabilities of these outcomes can be quantified. For example, when you invest in the stock market, you can analyze historical data to determine the probability of potential gains or losses based on past performance, trends, and economic factors. In this scenario, you can assess the risk associated with your investment by understanding how much you could win or lose and what the chances are for each outcome.

On the other hand, uncertainty describes circumstances where the outcomes are unknown and the probabilities cannot be easily determined. This often occurs in novel situations or complex scenarios where there is insufficient information. For instance, the emergence of a new technology or a revolutionary business model introduces a level of uncertainty because there may be no historical data or established framework to predict its success or failure accurately.

The confusion between risk and uncertainty often stems from their interplay in real-life situations and the human tendency to attach emotions to decision-making. People might perceive risks through a lens of uncertainty, focusing on the unknowns rather than acknowledging the measurable probabilities of known risks. For example, someone might shy away from investing in a stock (perceived as risky) due to an unpredictable economy, where uncertainty tangentially influences their understanding of the risk involved.

Moreover, the psychological aspects of decision-making play a role in this confusion. People often gravitate toward quantifiable risks they can analyze while struggling to cope with the inherent uncertainty of their environment, leading to a misinterpretation of the two concepts.

Understanding the distinction between risk and uncertainty is essential for making informed choices, whether in finance, healthcare, or everyday life. By acknowledging this difference, individuals can better navigate their decision-making processes and develop strategies to mitigate potential negative outcomes. \index{Risk, Uncertainty}
\section{Tal's Play}
Tal's play shows perfectly how uncertainty differs from risk. He didn't simply gamble with bad moves hoping for luck. His sacrifices and daring ideas were not blind risks; they were calculated invitations into positions where ordinary logic no longer worked.

When he sacrificed a piece, he often couldn't prove that the combination was sound-but he sensed that the position contained possibilities his opponent would not fully grasp. That is uncertainty: stepping into a realm where the outcome cannot be measured, but creative intuition guides the way. He wasn't risking in the sense of playing unsoundly; he was venturing into uncharted territory where imagination had more value than accuracy.

If he had only taken risks-making moves with a clear probability of failure-he would never have reached the top. But by embracing uncertainty, Tal turned the game into an adventure of discovery His opponents tried to calculate; he invited them into a fog where calculation lost its meaning. That is why his chess, though dangerous, wsa not reckless-it was the art of living comfortably in the unknown.

Most chess players rely on familarity-they recognize recurring patterns, or ``chunks'', built from years of experience. These mental shortcuts let them navigate normal positions efficiently, uch like walking through a well-lit city why know by hear. Tal disrupted that comfort. He led his opponents into positions that broke all established patterns, where intuition had to replace memory.

Paradoxically, this reuced his own risk: in irrational, chaotic positions where others felt lost, Tal was at home. He turned confusion into his advantage, transforming the unknown into his territory. What looked dangerous to others was, for him, the natural landscape of creativity.\index{Risk, Uncertainty}


\section{O Fortuna}
\input{cards/life/o_fortuna}
\input{cards/life/o_fortuna_history}

\section{Stress} \index{Stress}
Stress rarely lives in the present moment. It comes from worrying about what might happen and regretting what already has. The mind prejects itself forward, imagining losses, failures or unfinished duties - or it driftsbackward, replaying mistakes and missed chances. In both directions, it fights against what cannot be chontrolled. When attention leaves the present, tension arises: muscles tighten, breathing shortens, thoughts loop. The body reacts as if danger were real, even when it exists only in imagination.

Peace returns the moment awareness anchors in now. The present contains effort but not anxiety, action but not regret. Letting go of past and future isn't passivity - it's clarity: doing what can be done here and now, and releasing everything else.

\section{Hamlet}
Hamlet's tragedy is overthinking - he worries about everything: the truth of the ghost, the morality of revenge, the fear of damnation, even the meaning of life itself. He delays action because every possible outcome troubles him. His famous line ``thus conscience does make cowards of us all'' is pure worry turned philosophical. 

In this image, Hamlet stands alone beneath a sky that seems to twist with his thoughts. His face is drawn tight, his eyes hollow with sleeplessness, and his hand grips his forehead as if trying to still the storm within. The skull he hold is not only a symbol of death but of his endless reflection - a mirror for the mind that cannot rest. Around him, the world bends and swirls, as though reality itslef shares his unrest. Every color hums with unease; every line trembles. It is not action that torments him, but the weight of imagining - the fear of what may come, the regret of what has passed, the paralysis of a mind that cannot stop worrying.

\section{Gordian's Knot}
\input{cards/mythology/gordians_knot}

\chapter{Chess and Thinking}

\section{Chess and Computer}
Chess, a game celebrated for its complexity and deep strategic nature, has found a formidable opponent in the realm of computers. The integration of computers into chess has revolutionized the way the game is played, analyzed, and understood. From enthusiast players seeking to improve their skills to grandmasters strategizing for championships, computer technology has become an indispensable tool in chess.


The development of chess-playing software involves a blend of algorithms and heuristic techniques. At its core, the tasks undertaken by these programs revolve around evaluating positions and predicting the most advantageous moves. Some of the primary techniques used include:

\begin{itemize}
	\item{Minimax Algorithm}

	This foundational approach in game theory entails that the computer evaluates all possible moves and their outcomes. The computer seeks to minimize the maximum loss (hence “minimax”) while considering potential responses from its opponent, thereby making the most strategic decisions.
	\item{Alpha-Beta Pruning}

	This optimization technique improves the efficiency of the minimax algorithm by eliminating branches in the move tree that won’t be chosen, thus reducing the number of positions to evaluate. This allows the program to search deeper in the same amount of time.
	\item{Evaluation Function}

	These are algorithms designed to assign a score to a given position based on various factors such as material count, piece mobility, control of the center, and king safety. The evaluation functions use heuristics developed from the vast knowledge of chess principles learned from both historical games and databases.
	\item{Machine Learning}

	In recent years, advanced machine learning techniques, particularly deep learning, have been introduced. Programs like AlphaZero utilize neural networks trained through self-play to discover novel strategies and approaches to the game that transcend traditional evaluations.
	\item{Databases and Opening Books}

	Top chess engines are often equipped with large databases of historical games and opening theory, enabling them to play with extensive opening knowledge and adapt to various styles of play.

\end{itemize}

While chess engines are incredibly powerful and capable of executing millions of calculations per second, their style of play differs significantly from that of human players. Computers approach the game with cold precision, calculating variations far beyond human capacity, often perceiving the game through a lens of probabilities rather than intuition. They focus on calculating concrete lines to determine the best possible move rather than relying on the generalized strategies or psychological aspects that humans frequently employ.

Humans, on the other hand, utilize a combination of experience, psychological insight, and instinctive judgment. They often derive strategies from patterns and familiar positions, relying on their emotional understanding of the game and their opponents. This contrast underscores the fact that while computers can play chess at an exceedingly high level, they do so with a fundamentally different methodology than human beings. For this reason, humans do not need to emulate computer strategies or techniques; instead, they should embrace the unique strengths and insights that a human mind brings to the ancient game of chess.

In conclusion, the interplay between computers and chess highlights not only the advancements in technology but also the intrinsic beauty and complexity of the game itself. As computers continue to evolve and enhance our understanding of chess, players are encouraged to appreciate both the analytical prowess of machines and the irreplaceable qualities that define human gameplay.

\section{Perception and Decision Stages}
Adrian de Groot was one of the first psychologists to study how chess masters actually think when they play-not how strong they are, but what happens in their mind when they look at a position.

In the 1940s, he asked players of different strengths-from amateurs to world champions- to think out loud while solving chess problems. He carefully recorded their words and reasoning step by step. The plays were told find the best move but not to move the pieces; they just had to explain what they were seeing and thinking. De Groot expected grandmasters to analyze many more moves and variations than weaker players. But that's not what he found. Everyone - masters and amateurs- looked about the same number of moves ahead, typically 3-5. The big difference wasn't in how far they calculated but in what they saw. Masters immediately recognized key features of a position almost at a glance. They didn't start from zero; they started from structure.

He described their process as having four stages:
Orientation: taking in the position and forming a first impression

Exploration: testing a few promising ideas. 

Investigation: calculating deeper in one or two lines.

Proof: checking the correctness and making a final choice.

De Groot's main conclusion: chess expertise is not mainly about brute-calculation, but about pattern recognition and structured thinking. Masters have internalized thousands of meaningful configuration, so they instantly focus on the right ideas- a finding that later inspired Simon and Chase's ``chunking'' theory.

In short, de Groot showed that genius in chess doesn't come from seeing further-it comes from seeing better. \index{Chess Perception}

\section{Chunk}
Herbert Simon, along with William Chase, studied chess expertise in the 1970s. They found that chess masters don't just momorize isolated moves - they momorize patterns, or ``chunks'' of pieces and positions. These chunks allow them to recognize familiar structures quickly, reducing the cognitive load when thinking about moves. 

For example, a master mighjt see a certain pawn structure and instantly recall a typical plan associated with it, instread of calculating every possible move from screatch.

Chunks encode meaningful patterns: masters can store and retrieve thousands of these patterns in long term memory. Expert memory is domain-specific. A master's memory advantage disappears when dealing with random chess positions because chunks are meaningful patterns, not random arrangements.

Chunks accelerate decision-making: They allow quick recognition of threats, opportunities, and strategic ideas. 

Life is full of complex situations that can be overwhelming if we try to analyze every detail individually. Chunking lets us organize information into meaningful patterns, so we can respond effectively. 

For example, an experienced doctor doesn't analyze every symptom in isilation-they recognize patterns from prior cases. A skilled negotiator sees the structure os a discussion and anticipates moves.

By consciously building ``chunks'' of experience-through deliberate practice and reflection- we can make faster, better decisions and reduce cognitive overload in complex, real-world tasks. \index{Chunks}
\subsection{Daedalus and the Labyrinth}
In the myth, Daedalus is a skilled craftsman who designed the Labyrinth to contain the Minotaur. The Labyrinth is complex and convoluted, representing a challenge of navigation and understanding. To find a way out, one needs to develop a strategy to avoid getting lost among its many pathways.

The Labyrinth can be seen as a representation of information complexity. When faced with a vast amount of information or a difficult problem (like navigating the Labyrinth), individuals can easily feel overwhelmed.

Daedalus, as the master architect, symbolizes the human ability to solve complex problems. He creates a structure that initially seems confusing, but it is designed to be navigable with some clever thinking.
Chunking as a Strategy:

In order to navigate through the Labyrinth successfully, one cannot remember all the paths and turns as individual, standalone locations. Instead, one would benefit from grouping or 'chunking' the pathways into segments or sections. By memorizing key junctures or landmarks within the Labyrinth, an individual can reduce the cognitive load and simplify the navigation process.
Efficiency:

Just as Daedalus provided a thread to Theseus to follow back through the Labyrinth, chunking provides a way to remember complex information more efficiently. By organizing bits of information into manageable "chunks," a person can recall and use them effectively without becoming lost in a maze of details.
Learning and Application:

Similar to how Theseus learns to navigate the Labyrinth with the help of Daedalus's design, chunking enables learners to build connections and organize knowledge systematically. This method facilitates not just better recall, but also deeper understanding, as relationships between the chunks can be recognized and leveraged for problem-solving.

\section{Satisficing}
Herbert Simon coined the term ``satisficing'' to describe a decision-making strategy where a person chooses an option that is good enough, rather than the absolute best. 

Instead of exhaustively searching for the optimal solution-which may be impossible or too costly-people settle for a solution that meets their criteria of acceptability.

Example: Choosing a restaurant that looks decent instead of analyzing every possible place in town to find the theoretically ``perfect'' one.

\subsection*{Why is it important}
In real life, we rarely have the time, information or computational power to find the perfect solution. Satisficing allows us to act effectively under constraints. Reduces stress and indecision: Trying to optimize everything leads to analysis paralysis. Accepting ``good enough'' decisions frees mental resources for other tasks. 

Simon showed that humans are boundedly rational-we make decisions under limitations, and satisficing is a rational strategy within those limits.

In short, satisficing is a powerful principle for dealing with complexity: it balances quality and efficiency, letting us move forward without getting stuck chasing perfection. \index{Satisficing}
\section{Heuristic} \index{Heuristics}
Herbert Simon introduced the concept of heuristics as mental shortcuts or rules of thumb that help people make decisions efficiently under uncertainty, tather than exhaustively analyzing every possibility. 

A heuristic is a simple strategy that guides problem-solving and decision-making. It doesn't guarantee the perfect solution but often gives a satisfactory one quickly.

\subsection*{Connection to Computer Science}
In earch algorithms like A*, heuristics estimate the ``distance'' to a goal, helping the algorithm focus on promising paths instead of exploring every possibility.

Life is too complex to calculate every possible outcome. Heuristics help us make practical decisions with limited time and information.

Gut feeling is a real-world manifestation of heuristics. It's your brain recognizing patterns and past experience subconsciously to point toward a reasonable choice. 

Simon's idea shows that humans are ``boundedly rational''. We don't need perfect knowledge; we just need heuristics to navigate complexity efficiently.

\subsection{Heuristic in Life} \index{Heuristics!Life}
Here are three vivid examples where gut feeling acts like a heuristic in life:

\begin{itemize}
	\item{Emergency Decision}

	You smell smoke and see a small flame. You don't calculate every possible exit route or estimate fire spread precisely. Your gut tells you ``Get out now!''. Heuristic: Recognize danger patterns from past experience or instinct-act quickly without full analysis.
	\item{Business Choice}

	You meet a potential business partner. Something about their tone, confidence or past behavior signals dishonesty. You decide to reject even without checking every reference. Heuristic: pattern recognition from prior interactions guides a rejection efficiently.
	\item{Medical Intuition}

	A seasoned doctor sees a patient with subtle signs of illness. Even if lab results are pending, they have a gut feeling about the diagnosis and start treatment. Heuristic: Experience forms chunks and mental shortcuts that guide immediate action.


\end{itemize}


\subsection{Heuristic in Chess} \index{Heuristics!Chess}
Herbert Simon's concept of heuristics is very clear in the way chess masters think. Rather than calculating every possible move, masters recognize familiar patterns, or ``chunks'' on the board. These chunks-weak squares, pawn structures or typical mating configurations-allow them to respond quickly. Their thinking is guided by heuristics: rule of thumb derived from experience. For example, seeing an isolated pawn on e6 might immediately suggest targeting it with rooks and the queen. This recognition is fast and almost automatic and it saves enormous mental effort compared to exhaustively calculating every possibility.

Chess masters also demonstrate satisficing behavior. They rarely search for the single best move; instead, they often choose a move that is sufficiently strong to maintain or increase their advantage. A simple heuristic like ``any move that improves piece activity or king safety is likely sufficient'' prevents the paralysi that comes from trying to optimize every decision. This approach shows that effective decision-making does not require perfection, only a practical solution that achieves the goal.

Even in evaluating the opponent's intentions, heuristics play a critical role. Masters can sense the opponent's plans and threats without analyzing every potential variation. A gut feeling might signal ``the opponent wants to attack my kingside; I need to defend here.'' based on prior experience and pattern recognition. This fast, intuitive judgment allows them to act effectively under time pressure and complixity, mirroring the same principles Simon observed in human decision-making more broadly.

\part*{Chess Play}
\chapter{Positional Play}
\section{Simple Chess} \index{Simple Chess}
We finally get down to business! Most chess books on the market are filled with endless variations. Yet for any human being, it is impossible to calculate all those lines, visualize every position, and then evaluate them accurately. Use such analysis as a standard only leads to frustration and stress.

A far more practical approach is to play from the present position, relying on one's own understanding - just as in life, where decisions are made with limited foresight. This method is more effective because it is human: it respects the natural limits of thought and emphasizes clarity over complexity.

World champions such as Capablanca, Petrosian and Karpov all advocated this style. In their writings, they valued intuition, simplicity and sound judgment over endless calculation. Their success shows that chess, like life is best played one clear move at a time. 
\section{Prophylaxis} \index{Prophylaxis}
Prophylaxis in chess refers to the strategic concept of anticipating and preventing an opponent's threats and plans before they materialize. It involves making moves that not only advance your own position but also disrupt or hinder the opponent's potential strategies. This proactive approach allows players to maintain control of the game and dictate the flow of play, rather than simply reacting to their opponent's moves.

The importance of prophylaxis in chess cannot be overstated. It helps players to:

\begin{itemize}
\item{Maintain Initiative}

By anticipating the opponent's plans, a player can make moves that keep them in control of the game, often leading to more favorable positions.
\item{Minimize Risks}

Prophylactic moves can help to diminish the threats posed by the opponent, thereby reducing the chances of falling into traps or losing material.
\item{Enhance Position}

By focusing on both one’s own plans and those of the opponent, players can improve their own position while simultaneously weakening the opponent’s setup.
\item{Create Strategic Plans}

Understanding the opponent’s potential moves allows a player to formulate a more cohesive and effective strategy, as they can plan based on what the opponent is likely to do.

\end{itemize}

In essence, prophylaxis is a key element of high-level chess play, as it embodies the principle of strategic foresight and the ever-important balance between attack and defense. By incorporating prophylactic thinking into their game, players can significantly enhance their overall performance and decision-making skills on the board.

\section{Schematic Thinking} \index{Thinking!Schematic Thinking}
Thinking schematically means that you don’t look at specific moves or even plan, but instead just imagine which position you want to reach. When you have found your dream position, you can then try to find the moves to reach it. This thinking technique is so useful since it prevents you from getting lost in many different variations and gives you a clear position you want to reach.

Many great endgame players are known for their schematic thinking skills and Capablanca is a great example. He often made chess look easy and effortless because he was thinking in schemes and focused on the important parts of the position.

\section{Thinking Template} \index{Thinking!Three Questions} \label{sec:thinking-template}

This chapter presents an array of positional chess problems designed to enhance your strategic thinking. Each problem can be approached using the thinking template outlined earlier. The aim is to cultivate efficient problem-solving skills while minimizing oversight in variations. As you work through each challenge, take the time to jot down your thoughts. This practice will allow you to identify any gaps in your reasoning by comparing your solutions with the provided answers.
Chess can quickly become overwhelming if you try to calculate every variation. Endless lines drain energy, cloud judgment, and increase mistakes. A simpler, structured approach keeps your mind clear and your decisions effective. Focus on three key questions \cite{Aagaard:2012}:
\begin{itemize}
	\item{Where are the weaknesses?}

	Look for vulnerable squares, weak pawns or exposed king positions. Identifying weaknesses gives direction and targets for your plan.
	\item{Which is the worst-placed piece?}

	Evaluate both your and your opponent's pieces. Misplaced pieces often limit mobility and coordination. Exploiting or improving these positions can turn the game in your favor without complex calculation. 
	\item{Whas is my opponent's idea?}

	Try to understand your opponent's plan. Knowing their intentions helps you defend efficiently and counterattack strategically.
\end{itemize}
This template works because it emphasizes pattern recognition and strategic understanding over brute-force calculation. Instead of burning energy on countless variations that rarely occur in full, you make practical decisions baswed on the position's reality. World champions like Capablanca, Petrosian and Karpov relied on this approach: they played the position, not the tree of possibilities.

By asking these three questions each turn, you conserve mental energy, reduce errors, and develop a style rooted in clarity, control and real understanding. Chess becomes more about thinking smartly. 
\subsection{Visualization the Three Questions}
We use a simple graphical system to illustrate the three questions as used by Aagaard in his book ``Grandmaster Preparations Positional Play''. 

We will use circle to identify weakness. We will use square to identify the worst-placed pieces. We will use arrows to illustrate the opponent's ideas.


Let us look at an example 

\subsection*{Hikaru Nakamura - Vladimir Kramnik}

 \chessboard[
        setfen=3r2k1/p4pbp/b3p3/npqpP1pP/8/2P3P1/P3QPBN/R3R1K1 w - - 0 1,
 	markstyle=circle,
 	linewidth=0.05em,
 	markfields={g5, f6},
	markstyle=border,
	markfields={g2},
	pgfstyle=straightmove,
 	markmove=h7-h6
        ]

Black has weakness on g5 and f6. All his pieces are bad. White's worst piece is the bishop on g2.

Black intends to play h6 to protect his biggest weakness: g5 pawn. To prevent this move White can play h6 himself and then \symqueen h5 attack the weak pawn on g5.






%\begin{multicols}{2}

\nocite{Dvoretsky:2016}
\nocite{Aagaard:2002}


\ifdefined\chessproblem
    % The command is already defined, do nothing.
\else
    \newcommand{\chessproblem}[3]{
        \subsection*{#1. #2}
        \chessboard[
            setfen=#3,
        ]
    }
\fi
\ifdefined\conquestNunn
    % The command is already defined, do nothing.
\else
    \newcommand{\conquestNunn}[1]{
	\chessproblem{#1}{Conquest - Nunn}{5rk1/pp3ppp/5b2/2p1pb2/3q4/2NP2P1/PPP3KP/R2QR3 b - - 0 1}

    }
\fi

\ifdefined\conquestNunnAnswer
    % The command is already defined, do nothing.
\else
    \newcommand{\conquestNunnAnswer}[1]{
		\subsection*{#1. Conquest - Nunn}
 \chessboard[
        setfen=5rk1/pp3ppp/5b2/2p1pb2/3q4/2NP2P1/PPP3KP/R2QR3 b - - 0 1,
 	markstyle=circle,
 	linewidth=0.05em,
 	markfields={g2},
	markstyle=border,
 	markfields={f5}
        ]
	\begin{itemize}
		\item{Where are the weaknesses?}

		White king as well as his main diagonal is weak.
		\item{Which is the worst-placed piece?}

		The light-squared bishop does nothing.
		\item{What is my opponent's idea?}

		He can only wait passively.
	\end{itemize}

	By answering the template question above, the correct move is \symbishop d7, improving the piece as well as exploiting White's weakness on the main diagonal.
    }
\fi





\ifdefined\fischerKeres
    % The command is already defined, do nothing.
\else
    \newcommand{\fischerKeres}[1]{
	\chessproblem{#1}{Fischer - Keres}{r5k1/1pq2ppp/2rb1n2/4n2P/p2pPB2/P2P2PB/R1P1Q3/1N3RK1 b - - 0 1}
    }
\fi

\ifdefined\fischerKeresAnswer
    % The command is already defined, do nothing.
\else
    \newcommand{\fischerKeresAnswer}[1]{
	\subsection*{#1. Fischer - Keres}
 \chessboard[
        setfen=r5k1/1pq2ppp/2rb1n2/4n2P/p2pPB2/P2P2PB/R1P1Q3/1N3RK1 b - - 0 1,
 	markstyle=circle,
 	linewidth=0.1em,
 	markfields={c2},
	markstyle=border,
 	markfields={a8}
        ]

	\begin{itemize}
		\item{Where are the weaknesses?}

		White has weakness on c2.
		\item{Which is the worst-placed piece?}

		The rook on a8 does nothing
		\item{Whas is my opponent's idea?}

		White has no active plan so black can choose his own maneuver.
	\end{itemize}

	\symrook c8 is not possible since white's bishop controls c8. \symrook a5 is natural because the 
	rook targets c5 aiming the weakness c2 in the next move.

	Serendipitously, the move also aims h5. White must figure out how to defend. At 
	the moment we don't need to concern how white will defend. Let's pass the ball to 
	white and decide then what to do.
    }
\fi





\ifdefined\petrosianBannik
    % The command is already defined, do nothing.
\else
    \newcommand{\petrosianBannik}[1]{
	\chessproblem{#1}{Petrosian - Bannik}{3r3r/ppk1b2p/1np2p2/4p1pP/2P1N3/1P2B1P1/P3PP2/2KR3R w - - 0 1}
    }
\fi

\ifdefined\petrosianBannikAnswer
    % The command is already defined, do nothing.
\else
    \newcommand{\petrosianBannikAnswer}[1]{
	\subsection*{#1. Petrosian - Bannik}
 \chessboard[
        setfen=3r3r/ppk1b2p/1np2p2/4p1pP/2P1N3/1P2B1P1/P3PP2/2KR3R w - - 0 1,
 	markstyle=circle,
 	linewidth=0.05em,
 	markfields={f6,e6},
	markstyle=border,
 	linewidth=0.05em,
 	markfields={e3},
        ]
	\begin{itemize}
		\item{Where are the weaknesses?}

		Black has weaknesses on e6 and f6.
		\item{Which is the worst-placed piece?}

		The bishop on e3 is inactive.
		\item{What is my opponent's idea?}

		Black has no active plan, so White can choose his own maneuver.
	\end{itemize}
	The best move is \symbishop c5 to exchange the inactive bishop and exploit Black's weakness on f6 by moving the king to e6. Let's see what Petrosian has to say about this position. More than the move, it's his comments that made a deep impression.
	\begin{quote}
		``In deciding on this move, it was imperative to weigh all the pros and cons thoroughly. The move looks illogical as White is voluntarily exchanging his good bishop for his opponent's bad one, instead of swapping the bishop for knight and securing his preponderance. However, if you probe into the position a little more deeply, it becomes obvious that after a possible exchange of rooks on the d-file and the transfer of king to e6, Black would cover his vulnerable points and create an impregnable formation. The role played in this by the bad bishop would be of no small importance. After 1.\symbishop c5 \symrook xd1 2.\symrook xd1 \symbishop xc5 3.\symknight xc5 White was threatening infiltration on e6 and after 3...\symrook e8 4.\symknight e4 \symrook e6 5.g4 He was clearly better as the f6 pawn is very weak.'' \cite{Petrosian:2015}
	\end{quote}
    }
\fi
\ifdefined\bebchukBakulin
    % The command is already defined, do nothing.
\else
    \newcommand{\bebchukBakulin}[1]{
	\chessproblem{#1}{Bebchuk - Bakulin}{r1b1k2r/bpp1nppp/p2p1q2/P2P4/7P/1N6/1PP1QPP1/R1B1KB1R w - - 0 1}
    }
\fi

\ifdefined\bebchukBakulinAnswer
    % The command is already defined, do nothing.
\else
    \newcommand{\bebchukBakulinAnswer}[1]{
	\subsection*{#1. Bebchuk - Bakulin}
 	\chessboard[
        	setfen=r1b1k2r/bpp1nppp/p2p1q2/P2P4/7P/1N6/1PP1QPP1/R1B1KB1R w - - 0 1,
 		markstyle=circle,
 		linewidth=0.05em,
 		markfields={e8,e7,f6},
 		markstyle=border,
 		linewidth=0.05em,
 		markfields={a1},
        ]

	\begin{itemize}
		\item{Where are the weaknesses?}

		Black needs to castle. His e-file is weak and his queen is in danger.
		\item{Which is the worst-placed piece?}

		The rook on a1 is inactive. 
		\item{What is my opponent's idea?}

		He wants to castle.
	\end{itemize}

	Trying to catch the queen doesn't work \variation[invar]{1.Bg5 Qe5} since White must exchange the queen. 
	\symrook a4 is a natural idea, intending \symrook e4 to catch the queen.
	
	\variation[level=1]{1. Ra4 O-O 2. Rf4 Bf5 3. g4 Rae8 4. Kd1 Qe5 5. Qxe5 dxe5 6. Rxf5 Nxf5 7. gxf5 Rd8 8. Bg2 Bxf2 9. Ke2} 
	
	White has some advantage. Black could have tried \symbishop f5. Let's play \symrook a4 and let Black find the right defense.
	
    }
\fi
\ifdefined\gelfandAnand
    % The command is already defined, do nothing.
\else
    \newcommand{\gelfandAnand}[1]{
	\chessproblem{#1}{Gelfand - Anand}{1r2k2r/pb1n1ppp/4p3/2q5/Q3B3/4P3/1P1N1PPP/R4RK1 b - - 0 18}
    }
\fi

\ifdefined\gelfandAnandAnswer
    % The command is already defined, do nothing.
\else
    \newcommand{\gelfandAnandAnswer}[1]{
        \subsection*{#1. Gelfand - Anand}
        \chessboard[
            setfen=1r2k2r/pb1n1ppp/4p3/2q5/Q3B3/4P3/1P1N1PPP/R4RK1 b - - 0 1,
            markstyle=circle,
            linewidth=0.05em,
            markfields={d7,d2},
            markstyle={border},
            markfields={h8}
                ]

        \begin{itemize}
            \item{Where are the weaknesses?}
            
            Both knight on d2 and d7 are weak.
            \item{Which is the worst-placed piece?}
            
            The rook on h8 does nothing.
            \item{What is my opponent's idea?}
            
            White wants to attack the king in the middle. Maybe with \symrook ac1, \symbishop xb7, \symrook c8. 
        \end{itemize}

        The position is sharp. Black must find a way to solve his problem of his weak knight. 

        The best way to solve the problem is to get rid of it.
        \variation{19... O-O 20. Qxd7 Rfd8 21. Bxh7 Kf8! 22. Qa4 Rxd2} Black has active pieces, which is enough
        for the pawn. All of White's pieces could find better squares.
        
        Aagaard wrote that if one could see both \variation{21. Bxh7} and \variation{21...Kf8}, he is ready for a tournament. Are you ready?
    }
\fi

\ifdefined\morovicKarpov
    % The command is already defined, do nothing.
\else
    \newcommand{\morovicKarpov}[1]{
	\chessproblem{#1}{Morovic - Karpov}{r4rk1/pp1qnpbp/2pb1p1/4p3/1PP1P3/P1N1P1P1/1B2QPBP/R4RK1 w - - 0 1}
    }
\fi

\ifdefined\morovicKarpovAnswer
    % The command is already defined, do nothing.
\else
    \newcommand{\morovicKarpovAnswer}[1]{
		\subsection*{#1. Morovic - Karpov}
		\chessboard[
			setfen=r4rk1/pp1qnpbp/2pb1p1/4p3/1PP1P3/P1N1P1P1/1B2QPBP/R4RK1 w - - 0 1,
			markstyle=circle,
			linewidth=0.05em,
			markfields={c4},
			markstyle=border,
			linewidth=0.05em,
			markfields={e7},
			pgfstyle=straightmove,
			markmove={e7-c8,c8-b6,b6-c4},
		]

		\begin{itemize}
			\item{Where are the weaknesses?}

			White has weakness on c4.
			\item{Which is the worst-placed piece?}

			The knight on e7 is inactive. 
			\item{Whas is my opponent's idea?}

			Not clear. He has difficulty to evolve his position.
		\end{itemize}

		White has weakness on c4. e7 knight is the worst piece. 
		So moving it aiming for c4 is a natural idea. \symknight c8 is the answer.
	
    }
\fi
\ifdefined\sanakoevLungdal
    % The command is already defined, do nothing.
\else
    \newcommand{\sanakoevLungdal}[1]{
	\chessproblem{#1}{Sanakoev - Lungdal}{2r1k2r/1b3ppp/p3p3/2qpP3/1p1Q1P2/2P5/PP2B1PP/R2R2K1 w - - 0 1}
    }
\fi

\ifdefined\sanakoevLungdalAnswer
    % The command is already defined, do nothing.
\else
    \newcommand{\sanakoevLungdalAnswer}[1]{
		\subsection*{#1. Sanakoev - Lungdal}
        \newchessgame[
            setfen=2r1k2r/1b3ppp/p3p3/2qpP3/1p1Q1P2/2P5/PP2B1PP/R2R2K1 w - - 0 1,
            moveid=1w
        ]
		\chessboard[		
			markstyle=circle,
			linewidth=0.05em,
			markfields={b7},
			markstyle=border,
			linewidth=0.05em,
			markfields={a1},
		]

		\begin{itemize}
			\item{Where are the weaknesses?}

			The bishop on b7 is unprotected.
			\item{Which is the worst-placed piece?}

			The rook on a1 is inactive. 
			\item{Whas is my opponent's idea?}

			Black may want to castle kingside and exchange the queens.
		\end{itemize}

		\symrook ab1 is the right move.

        According to Dvoretsky, the move deserves two exclamation marks. We can
        however find the answer with logic and simple calculation.

        White is leading in the development so he should have the advantage.
        The queen is pinned so there is no way to start an attack with a queen as usual when one leads in development.

        \variation{1. cxb4 Qxd4 2. Rxd4 Rc2 3. Bd3 Rxb2 4. Rc1 Kd7 5. Rc2 Kd7 6. Rc2 Rxc2 7. Rxc2} Black
        simplifies the position and equalizes. 

        White fails to protect the pawn on b2 after taking the pawn on b4. So the answer is simple:
        \mainline[level=1]{1. Rab1} threatening cxb4. 
        
        \mainline{1... Qxd4 2. Rxd4 bxc3 3. bxc3 } White wins a tempo.
        
        \mainline{3... Rc7 4.Rdb4} White has the only open file and has the advantage. Black still needs to castle. His bishop is hopeless.        
        
        \chessboard
    }
\fi
\ifdefined\zlotnikLopes
    % The command is already defined, do nothing.
\else
    \newcommand{\zlotnikLopes}[1]{
	\chessproblem{#1}{Zlotnik - Lopes}{r1b1kb1r/ppp3qp/3p1pp1/2nN4/2P2B2/6P1/PP2PPBP/R2QK2R w - - 0 1}
    }
\fi

\ifdefined\zlotnikLopesAnswer
    % The command is already defined, do nothing.
\else
    \newcommand{\zlotnikLopesAnswer}[1]{
		\subsection*{#1. Zlotnik - Lopes}
        \newchessgame[
            setfen=r1b1kb1r/ppp3qp/3p1pp1/2nN4/2P2B2/6P1/PP2PPBP/R2QK2R w - - 0 1,
            moveid=1w
        ]
		\chessboard[		
			markstyle=circle,
			linewidth=0.05em,
			markfields={f6, g7, h8, c7},
			markstyle=border,
			linewidth=0.05em,
			markfields={f4},
		]

		\begin{itemize}
			\item{Where are the weaknesses?}

			The pawns on f6, queen on g7 and rook on h8 are vulnerable. Pawn on c7 is also under attack. The black queen is busy with protecting two weaknesses c7 and f6.
			\item{Which is the worst-placed piece?}

			The bishop on f4 is inactive. 
			\item{Whas is my opponent's idea?}

			A plan is hard to plan. He may want to castle.
		\end{itemize}

		\symbishop d2 is the right move.

        The bishop on f4 does nothing and should be moved to c3, controlling the main diagonal and attacking
        the weaknesses.
    }
\fi
\ifdefined\carlsenDing
    % The command is already defined, do nothing.
\else
    \newcommand{\carlsenDing}[1]{
	\chessproblem{#1}{Carlsen - Ding}{2kr2r1/pp1b1pp1/1qn1pn1p/3p4/1P6/B1PB1N2/P3QPPP/R4RK1 b - - 0 15}
    }
\fi

\ifdefined\carlsenDingAnswer
    % The command is already defined, do nothing.
\else
    \newcommand{\carlsenDingAnswer}[1]{
	\subsection*{#1. Carlsen - Ding}
    \newchessgame[
        setfen=2kr2r1/pp1b1pp1/1qn1pn1p/3p4/1P6/B1PB1N2/P3QPPP/R4RK1 b - - 0 15,
        moveid=15b
    ]
 	\chessboard[     
        pgfstyle=straightmove,
        linewidth=0.05em,
        markmove={b4-b5},
        markstyle=circle,
        linewidth=0.05em,
        markfields={e5},
        markstyle=border,
        linewidth=0.05em,
        markfields={d7},
        ]

	\begin{itemize}
		\item{Where are the weaknesses?}

		e5 square will be weak after White plays b5 driving away the knight on c6.
		\item{Which is the worst-placed piece?}

		The bishop on d7 is inactive. 
		\item{What is my opponent's idea?}

		b5.
	\end{itemize}

	White threatens to play b5 to drive away the knight on c6 and then control the e5 square with his knight.
Once White achieves his goal, Black has a desperate position. Therefore he must react now!

Understanding White's idea, Black can play \variation[invar]{15... e5 16. b5 e4 17. bxc6 Qxc6 18. Ne5 Qc7 
19. Nxf7 Bg4 20. Qe3 Qxf7 21. Bc2} 

    In the actual game, Ding chose to play \mainline[level=1]{15... Kb8? 16. b5 \xskakcomment{ Of course!}} White has an
    advantage and wins the game in a few moves.
	
    }
\fi



\ifdefined\lelchukVoronova
    % The command is already defined, do nothing.
\else
    \newcommand{\lelchukVoronova}[1]{
	\chessproblem{#1}{Lelchuk - Voronova Analysis}{1b3q1r/3Q3p/pp1n1pp1/2pR4/P3pPPk/2P1N3/2P4P/7K w - - 0 10}
    }
\fi


\ifdefined\lelchukVoronovaAnswer
    % The command is already defined, do nothing.
\else
    \newcommand{\lelchukVoronovaAnswer}[1]{
		\subsection*{#1. Lelchuk - Voronova Analysis}
		\chessboard[
			setfen=1b3q1r/3Q3p/pp1n1pp1/2pR4/P3pPPk/2P1N3/2P4P/7K w - - 0 10,
			markstyle=circle,
			linewidth=0.05em,
			markfields={h4},
			markstyle=border,
			linewidth=0.05em,
			markfields={d7},
		]

		\begin{itemize}
			\item{Where are the weaknesses?}

			Black king on h4 is completely cut off.
			\item{Which is the worst-placed piece?}

			The White queen on d7.
			\item{What is my opponent's idea?}

			Not clear. 
		\end{itemize}

		The basic idea is to activate the White queen to attack the Black king.

        \variation{ 10. Qe6!!}

        This move has some very deep idea.    
        
        \begin{enumerate}
            \item{\variation[invar]{10... c4}}
        
            \chessboard[
                setfen=1b3q1r/7p/pp1nQpp1/3R4/P1p1pPPk/2P1N3/2P4P/7K w - - 0 11
            ]

            White intends to cut off the king completely and mate with \symqueen d5-d1 maneuver.

            \variation[invar]{11. Rh5+! gxh5 12. Ng2+ Kh3 13. g5+ f5 14. Qd5 h4 15. Qd1 Rg8 16. Ne3 Rxg5 17. Qf1+ Rg2 18. Qxg2# }
            
            \chessboard[setfen=1b3q2/7p/pp1n4/5p2/P1p1pP1p/2P1N2k/2P3QP/7K b - - 0 18]
        
            \item{\variation[invar]{10... f5}}

            \variation[invar]{
                10...f5 11. Rd3!}
                
                \chessboard[setfen=1b3q1r/7p/pp1nQ1p1/2p2p2/P3pPPk/2PRN3/2P4P/7K b - - 1 11]

                Black is doomed.

                \begin{enumerate}
                    \item {\variation[invar]{11...Nc4}}
                    
                    \variation{11...Nc4 12. Ng2+ Kxg4 13. Rg3+ Kh5 14. Rh3+ Kg4 15. Qxc4 Bxf4 16. Qf1 Be5 17. Nf4 Kg5 18. Ne6+ Kf6 }
                    \item {\variation[invar]{11...exd3}}

                    \variation{11... exd3 12. Ng2+ Kxg4 13. Qe1 Qe7 14. Qg3+ Kh5 15. Qh3+ Qh4 16. Qxh4# }
                \end{enumerate}
        \end{enumerate}
    }
\fi

\newpage
\section{Excercises}
\begin{multicols}{2}
	\conquestNunn{1}
	\fischerKeres{2}
	\petrosianBannik{3}
	\bebchukBakulin{4}
	\gelfandAnand{5}
	\morovicKarpov{6}
	\sanakoevLungdal{7}
	\zlotnikLopes{8}
	\carlsenDing{9}
	\lelchukVoronova{10}
\end{multicols}

\newpage
\section{Answers}
\begin{multicols}{2}
	\conquestNunnAnswer{1}
	\fischerKeresAnswer{2}
	\petrosianBannikAnswer{3}
	\bebchukBakulinAnswer{4}
	\gelfandAnandAnswer{5}
	\morovicKarpovAnswer{6}
	\sanakoevLungdalAnswer{7}
	\zlotnikLopesAnswer{8}
	\carlsenDingAnswer{9}
	\lelchukVoronovaAnswer{10}
\end{multicols}

\chapter{Process of Elimination} \index{Thinking!Process of Elimination}
\section{Process of Elimination}
\vocab{PoE}{Process of Elimination}{Ruling out bad moves, not immediately finding the perfect one. Commonly used in calculation, defense, and endgames}
 is the method instead of searching for a brilliant move directly, 
you narrow the field\cite{Dvoretsky:2015}.

How it works in practice:

\begin{itemize}
\item List candidate moves that are natural or forcing.
\item Eliminate bad moves by concrete reasons.
\item Compare the remaining moves and choose the one that best fits the position.
\end{itemize}
\index{Process of Elimination} 

\ifdefined\poechessproblem
    % The command is already defined, do nothing.
\else
    \newcommand{\poechessproblem
    }[3]{
        \subsection*{#1. #2}
        \chessboard[
            setfen=#3,
        ]
	\begin{itemize}
		\item{What are the candidate moves?}
		\item{Which move can be eliminated?}
		\item{Whas is the move?}
	\end{itemize}
    }
\fi
\ifdefined\karjakinCarlsenOne
    % The command is already defined, do nothing.
\else
    \newcommand{\karjakinCarlsenOne}[1]{
	\poechessproblem{#1}{Karjakin - Carlsen}{r4bk1/1bq2pp1/4rn1p/4N3/BPp1P3/2B4P/1Q3PP1/3RR1K1 w - - 1 29}
    }
\fi

\ifdefined\karjakinCarlsenTwo
    % The command is already defined, do nothing.
\else
    \newcommand{\karjakinCarlsenTwo}[1]{
	\poechessproblem{#1}{Karjakin - Carlsen}{4r1k1/1bq2pp1/3brn1p/4N3/1Pp1PP2/2B4P/1QB3P1/3RR1K1 w - - 1 31}
    }
\fi

\ifdefined\karjakinCarlsenThree
\else
    \newcommand{\karjakinCarlsenThree}[1]{
	\poechessproblem{#1}{Karjakin - Carlsen}{6k1/1b4p1/2q1r2p/4Bp2/1Pp5/6PP/1QB4K/6R1 b - - 2 39}
    }
\fi


\ifdefined\karjakinCarlsenOneAnswer
    % The command is already defined, do nothing.
\else
    \newcommand{\karjakinCarlsenOneAnswer}[1]{
		\subsection*{#1. Karjakin - Carlsen}
        \newchessgame[
            setfen=r4bk1/1bq2pp1/4rn1p/4N3/BPp1P3/2B4P/1Q3PP1/3RR1K1 w - - 1 29,
            moveid=29w
        ]
        \chessboard[
            setfen=r4bk1/1bq2pp1/4rn1p/4N3/BPp1P3/2B4P/1Q3PP1/3RR1K1 w - - 1 29,
            markstyle=circle,
            linewidth=0.05em,
            markfields={c4, e4, e5},
        ]
        \begin{itemize}
            \item{What are the candidate moves?}

            White's bishop is being attacked. He must play either \variation[invar]{29. Bb5} or \variation[invar]{29. Bc2}.
            \item{Which move can be eliminated?}
            \variation[invar]{29. Bc2 Rae8 30. f4 Bd6} 

            \chessboard

            White has weaknesses on e5 and e4. Black only needed to double his 
            rooks on the e-file and use his pieces to target the e5 square.

            However, it is difficult for White to find the right plan. He has 
            already weakened his king safety with his kingside pawn movements. He will
            also have to play g3 at some point to further weaken his king. 

            Therefore we eliminate \variation[invar]{29. Bc2}.
            \item{What is the move?}

            \variation[invar]{29. Bb5}. For example after 
            \variation[invar]{29. Bb5 Ba6 30. Ra1 Bb7 31. Rxa8 Bxa8 
                32. Bxc4 Rxe5 33. Bxf7+ Qxf7 34. Bxe5 Nxe4 } White has a playable position
        \end{itemize}
    }
\fi

\ifdefined\karjakinCarlsenTwoAnswer
    % The command is already defined, do nothing.
\else
    \newcommand{\karjakinCarlsenTwoAnswer}[1]{
		\subsection*{#1. Karjakin - Carlsen}
        \newchessgame[
            setfen=4r1k1/1bq2pp1/3brn1p/4N3/1Pp1PP2/2B4P/1QB3P1/3RR1K1 w - - 1 31,
            moveid=31w
        ]
        \chessboard[
            setfen=4r1k1/1bq2pp1/3brn1p/4N3/1Pp1PP2/2B4P/1QB3P1/3RR1K1 w - - 1 31,
            markstyle=circle,
            linewidth=0.05em,
            markfields={e4,e5},
            pgfstyle=straightmove,
            linewidth=0.05em,
            markmove={f6-h5},
        ]
        \begin{itemize}
            \item{What are the candidate moves?}

            \begin{itemize}
                \item{Where are the weaknesses?}
            
                e4 and e5.
                \item{Which is the worst-placed piece?}
               
                Not clear.
                \item{What is my opponent's idea?}
            
                He wants to play \symknight h5 to attack the f4 pawn.
            \end{itemize}

            When Black plays \symknight h5, White must play g3. As prophylaxis,
            it is a good idea to protect the g3 pawn first. There are two options
            \variation[invar]{31. Kh2} and \variation[invar]{31. Re3}.

            \item{Which move can be eliminated?}

            Karjakin should have eliminated \variation[invar]{31. Kh2} first. 
            His king is on the same diagonal as the Black queen and bishop.
            Therefore we eliminate \variation[invar]{31. Kh2}.
            \variation[invar]{31. Re3} is the right move.

            \variation[invar]{31. Re3 Nh5 32. g3 g5} 
            (\variation{32...f6 33. Nxc4 Bxf4 34. gxf4 Nxf4 35. Bb3 \xskakcomment{ Black king
            is now exposed, White has advantage.}}) \variation {33. Ba4 R8e7 34. Qe2} White is fine.
        
            \item{What is the move?}

            \variation[invar]{31. Re3} 
        \end{itemize}
    }
\fi


\ifdefined\karjakinCarlsenThreeAnswer
    % The command is already defined, do nothing.
\else
    \newcommand{\karjakinCarlsenThreeAnswer}[1]{
		\subsection*{#1. Karjakin - Carlsen}
        \newchessgame[
            setfen=6k1/1b4p1/2q1r2p/4Bp2/1Pp5/6PP/1QB4K/6R1 b - - 2 39,
            moveid=39w
        ]
        \chessboard
        \begin{itemize}
            \item{What are the candidate moves?}

            White must stop from being mated along the main diagonal.
            \variation[invar]{39. Kg1} and \variation[invar]{39. Be4} are the candidate moves.

            \item{Which move can be eliminated?}

            Karjakin should have eliminated \variation[invar]{39. Rg1} first. 
            \variation[invar]{39. Rg1 Qd5 40. Bd1 Rxe5 41. Qf2 c3 } His pieces 
            are passive. His pawn on b4 will fall soon while Black has a strong 
            passed pawn.
        
            \item{What is the move?}

            \variation[invar]{39. Be4}. After \variation[invar]{39. Be4 fxe4 40. Re3} 
            he can still defend.
        \end{itemize}
    }
\fi



\ifdefined\lelchukVoronovaPOE
    % The command is already defined, do nothing.
\else
    \newcommand{\lelchukVoronovaPOE}[1]{
	\poechessproblem{#1}{Lelchuk - Voronova}{1b1qnr2/4pQ1p/pp1r1ppk/2p1N3/P4P2/2P5/2P3PP/3R1R1K b - - 1 2}
    }
\fi

\ifdefined\lelchukVoronovaOnePOE
    % The command is already defined, do nothing.
\else
    \newcommand{\lelchukVoronovaOnePOE}[1]{
	\poechessproblem{#1}{Lelchuk - Voronova}{1b1q3r/4pQ1p/pp1n1pp1/2pR3k/P4P2/2P1N3/2P3PP/7K b - - 2 6}
    }
\fi


\ifdefined\lelchukVoronovaAnswerPOE
    % The command is already defined, do nothing.
\else
    \newcommand{\lelchukVoronovaAnswerPOE}[1]{
        \poechessanswer{#1}{Lelchuk - Voronova}{1b1qnr2/4pQ1p/pp1r1ppk/2p1N3/P4P2/2P5/2P3PP/3R1R1K b - - 1 2}
        \begin{itemize}
            \item{What are the candidate moves?}

            White threats to play \variation[invar]{3. Ng4+ Kh5 4. Qxh7+ Kxg4 5. h3}

        To parry the threat, Black may play \variation[invar]{2... fxe5}, \variation[invar]{2... Rxf7} or \variation[invar]{2...Rh8}.
            \item{Which move can be eliminated?}
            
            Further calculation eliminates the first two:
        
            \variation[invar]{2... fxe5 3. Qxf8+ Kh5 4. g4+ Kxg4 5. Rg1 Kf3 6. fxe5+} White has huge material advantage.
            \variation[invar]{2... Rxf7 3. Nxf7+ Kg7 4. Nxg8} White has the exchange up.
            \item{What is the move?}

            Therefore Black must play \variation[invar]{2... Rh8}.
        \end{itemize}
    }
\fi

\ifdefined\lelchukVoronovaAnswerOnePOE
    % The command is already defined, do nothing.
\else
    \newcommand{\lelchukVoronovaAnswerOnePOE}[1]{
        \poechessanswer{#1}{Lelchuk - Voronova}{1b1q3r/4pQ1p/pp1n1pp1/2pR3k/P4P2/2P1N3/2P3PP/7K b - - 2 6}
        \begin{itemize}
            \item{What are the candidate moves?}

            \variation[invar]{6... e5} and \variation[invar]{6... f5}.
            \item{Which move can be eliminated?}
            
            \variation[invar]{6... f5} loses immediately after \variation[invar]{6... f5 7. g4+ Kh6 8. Rxd6 exd6 9. Nxf5+  gxf5 10. g5+}.
            \item{What is the move?}

            Therefore Black must play \variation[invar]{6... e5}.
        \end{itemize}
    }
\fi

We will present some complex exercises to train your Process of Elimination skills. To solve these exercises, you may also use the thinking tools we have covered in the previous chapters: Prophylaxis, Schematic Thinking or Positional Thinking Template. You may need to calculate a little, but you don't need to calculate through the whole variation.

\newpage
\section{Exercises}
\begin{multicols}{2}
	\karjakinCarlsenOne{1}
	\karjakinCarlsenTwo{2}
	\karjakinCarlsenThree{3}
	\lelchukVoronovaPOE{4}
	\lelchukVoronovaOnePOE{5}
\end{multicols}

\newpage
\section{Answers}
\begin{multicols}{2}
	\karjakinCarlsenOneAnswer{1}
	\karjakinCarlsenTwoAnswer{2}
	\karjakinCarlsenThreeAnswer{3}
	\lelchukVoronovaAnswerPOE{4}
	\lelchukVoronovaAnswerOnePOE{5}
\end{multicols}

\chapter{Dynamic Play}
\ifdefined\dynamicchessproblem
    % The command is already defined, do nothing.
\else
    \newcommand{\dynamicchessproblem}[3]{
        \subsection*{#1. #2}
        \chessboard[
            setfen=#3,
        ]
	\begin{itemize}
		\item{Who has a static advantage?}
		\item{Which play shall I choose?}
		\item{Whas is the move?}
	\end{itemize}
    }
\fi
\section{Critical Position}

Dorfman suggests three criteria for the existence of a critical position\cite{Dorfman:2001} \index{Critical Position}. 

\begin{enumerate}
    \item{A position in which a decision has to be taken regarding a possible exchange. If the 
exchange is forced, there is no change compared with the previous critical position. }
    \item{A position in which a decision has to be taken regarding a possible change in the 
pawn formation. Especially of the central pawns. }
    \item{The end of a series of forced moves. Here one should not draw a parallel between 
forced moves and the moves relating to a combination. }
\end{enumerate}
To sense that a position is critical is already a great success.

\section{How to play a Critical Position} \label{sec:critical-position}
To choose the most effective move in a given position, it is essential to evaluate the static aspects of that position. This evaluation helps determine whether a player should adopt a static or dynamic approach in the subsequent phase of the game.

Method for Evaluating Static Balance \cite{Dorfman:2001}

To assess the static balance of the position, follow a structured analysis method. This involves breaking down the position into its fundamental static elements in a regressive order.

\begin{itemize}
    \item{Static Evaluation}
    \begin{itemize}
        \item{King Safety}

        Evaluate the positioning and safety of both kings. A secure king often dictates the strength of the overall position.
        \item{Material Correlation}

        Analyze the material balance between the two players (i.e., which player has more material advantage).
        \item{Post-Queen Exchange Position}

        Consider who has a stronger position after the potential exchange of queens. The impact of this exchange can significantly alter the dynamics of the game.
        \item{Pawn Structure}

        Assess the formation and distribution of pawns. A advantageous pawn structure can lead to long-term strategic benefits.
    \end{itemize}
    \item{Who can evolve independently?}

    There is a crude method, enabling an 
    immediate static evaluation of a position to 
    be obtained: - analyse whether it is possible for your 
    own position to evolve independently of the 
    opponent's; - analyse whether the opponent's position 
    can evolve independently of your own. 
    
    The position which is ready for evolution 
    is statically better.


    \item{Choosing Between Dynamic and Static Play}

    Once you have completed the static evaluation of the position, the next step is to decide whether to pursue a dynamic or static strategy. This decision will guide you in identifying candidate moves:    
\end{itemize}

If your evaluation indicates a strong static advantage (e.g., superior king safety or material correlation), opt for a static strategy. Focus on maintaining your advantage and consolidating your position.
Conversely, if the position suggests that dynamic opportunities exist (such as tactical motifs or generating counterplay), consider a dynamic strategy. Look for candidate moves that exploit these opportunities and create imbalances in the position.

\ifdefined\shortSokolov
    % The command is already defined, do nothing.
\else
    \newcommand{\shortSokolov}[1]{
	    \dynamicchessproblem{#1}{Short - Sokolov}{r2q1rk1/1b3ppp/p1nppb2/4P3/5P2/PNp1B3/1PP1B2P/3RQRK1 b - - 0 1}
    }
\fi

\ifdefined\shortSokolovAnswer
    % The command is already defined, do nothing.
\else
    \newcommand{\shortSokolovAnswer}[1]{
		\subsection*{#1. Short - Sokolov}
		\chessboard[
			setfen=r2q1rk1/1b3ppp/p1nppb2/4P3/5P2/PNp1B3/1PP1B2P/3RQRK1 b - - 0 1,
		]

        \begin{itemize}
            \item{Who has a static advantage?}

            White has static advantage because after normal play, he can simply play \symqueen xc3. 
            He can double his rooks on the d-file. If Black responds with d5, White can attack freely along 
            the g-file.
            \item{Which play shall I choose?}

            I must therefore choose a dynamic play. 
            \item{Whas is the move?}

            \symknight xe5.
        \end{itemize}

        Black has c3 pawn as an asset. White extends too much on the kingside and Black must exploit it. 

        After \variation[invar]{1...Nxe5 2.fxe5 Bxe5 3.bxc3 Rc8!}, Black has enough compensation.
    }
\fi
\ifdefined\dingGukeshGameTwelve
    % The command is already defined, do nothing.
\else
    \newcommand{\dingGukeshGameTwelve}[1]{
	\dynamicchessproblem{#1}{Ding - Gukesh}{1r1qr1k1/1ppnbpp1/2n4p/pN2pb2/2P5/P2PBNPP/1P1Q1PBK/3RR3 b - - 10 17}
    }
\fi

\ifdefined\dingGukeshGameTwelveAnswer
    % The command is already defined, do nothing.
\else
    \newcommand{\dingGukeshGameTwelveAnswer}[1]{
		\subsection*{#1. Ding - Gukesh}
		\chessboard[
			setfen=1r1qr1k1/1ppnbpp1/2n4p/pN2pb2/2P5/P2PBNPP/1P1Q1PBK/3RR3 b - - 10 17,
            pgfstyle=straightmove,
			markmove={d3-d4},
		]

        \begin{itemize}
            \item{Who has a static advantage?}

            White has static advantage because he can play d4 improving his position gradually while
            Black has no clear plan
            \item{Which play shall I choose?}

            I must therefore choose a dynamic play. 
            \item{Whas is the move?}

            \symknight c5.
        \end{itemize}

        After the move, White must sacrifice an exchange to keep his advantage: \variation[invar]{17... Nc5 18. d5 Nd3 19. d5 Nxe1 20. Qxe1 Nd4 21. Nfxd4 exd4 22. Nxd4 Bh7 23. Qxa5 Bg5 24. Bxg5 hxg5 25. Qd2} Black has more space to maneuver.

        In the actual game, Gukesh chose \variation[level=1]{17...Bg6} and had a passive decision. He lost the game quickly.
	
    }
\fi
\ifdefined\nielsenDreev
    % The command is already defined, do nothing.
\else
    \newcommand{\nielsenDreev}[1]{
	    \dynamicchessproblem{#1}{Nielsen - Dreev}{r4rk1/pp1nb1p1/2p1p2p/5q1P/3PQBN1/8/PPP2P2/1K1R3R w - - 0 1}
    }
\fi

\ifdefined\nielsenDreevAnswer
    % The command is already defined, do nothing.
\else
    \newcommand{\nielsenDreevAnswer}[1]{
		\subsection*{#1. Nielsen - Dreev}
		\chessboard[
			setfen=r4rk1/pp1nb1p1/2p1p2p/5q1P/3PQBN1/8/PPP2P2/1K1R3R w - - 0 1,
            markstyle=circle,
            linewidth=0.05em,
            markfields={h5, f2},
            pgfstyle=straightmove,
            linewidth=0.05em,
            markmove={e6-e5},
        ]

        \begin{itemize}
            \item{Who has a static advantage?}

            Black has the static advantage. His pawn structure is better because
            White has weaknesses on h5 and f2. 

            Black also intends to play e5 at some point
            to remove his weak e6 pawn. 

            \item{Which play shall I choose?}

            I must therefore choose a dynamic play. 

            \item{Whas is the move?}
            Note Black can neutralize White's aggression on the king side 
            since he has enough defenders. (queen, rook, bishop) 
            
            White can choose to exchange the queens and change the dynamic 
            of the game by playing

            \variation[invar]{19. Qxf5 Rxf5 20. Bxh6 gxh6 21. Nxh6+ Kh7 22. Nxf5 exf5 23. d5 
            \xskakcomment{ White has a rook and two pawns for the minor pieces. He also
            has a passed pawn}}
            (\variation{23... exd5 24. Rxd5 \xskakcomment{ Black loses his f5 pawn.}}) 
        \end{itemize}

        In the actual game, Nielsen chooses \variation{19. Qe2}, has to therefore
        defend an inferior position with no counter play and loses the game.
    }
\fi
\ifdefined\dingNepomniachtchiGameFour
    % The command is already defined, do nothing.
\else
    \newcommand{\dingNepomniachtchiGameFour}[1]{
	    \dynamicchessproblem{#1}{Ding - Nepomniachtchi}{4r1k1/p1p1r1pp/1p1nPp1q/2pP3b/2P2p2/2Q2B1P/P2NRPP1/4R1K1 b - - 2 25}
    }
\fi

\ifdefined\dingNepomniachtchiGameFourAnswer
    % The command is already defined, do nothing.
\else
    \newcommand{\dingNepomniachtchiGameFourAnswer}[1]{
		\subsection*{#1. Ding - Nepomniachtchi}
		\chessboard[
			setfen=4r1k1/p1p1r1pp/1p1nPp1q/2pP3b/2P2p2/2Q2B1P/P2NRPP1/4R1K1 b - - 2 25,
		]

        \begin{itemize}
            \item{Who has the static advantage?}

            White has the static advantage because his passed pawns in the center are strong.
            \item{Which play shall I choose?}
            I shall choose a static play

            The main asset of White is the passed pawns, which have been blocked by the Black rooks.

            The rooks on the e-file are not doing much because there is no open file. The knights are 
            important pieces. The Black knight keeps an eye on e4 so that the rooks cannot attack the 
            f4 pawn. 

            I have a holdable position because I have only on weakness and White must find a way
            to prove his advantage. Therefore I choose a static play without changing the position.
            \item{Whas is the move?}

            At some point, White may play \symrook e4 to attack the f4 pawn and exchange the Black knight.

            Alternatively, White may play \symknight e4 to exchange the Black knight. In either case, e4 is 
            an important square and must be protected in advance. \variation[invar]{25... Bg6} is a good move.
        \end{itemize}
	}
\fi 
\ifdefined\arnasonMiles
    % The command is already defined, do nothing.
\else
    \newcommand{\arnasonMiles}[1]{
	\dynamicchessproblem{#1}{Arnason - Miles}{5r2/8/p1B1qp2/2p1p1nk/1pP1P1R1/8/PP2Q3/6K1 w - - 0 36}
    }
\fi

\ifdefined\arnasonMilesAnswer
    % The command is already defined, do nothing.
\else
    \newcommand{\arnasonMilesAnswer}[1]{
		\subsection*{#1. Arnason - Miles}
        \newchessgame[
            setfen=5r2/8/p1B1qp2/2p1p1nk/1pP1P1R1/8/PP2Q3/6K1 w - - 0 36,
	        moveid=36w
        ]
        \chessboard

        \begin{itemize}
            \item{Who has a static advantage?}

            Black enjoys a static advantage thanks to his healthy extra pawn.
            With best play he should be able to convert.
            \item{Which plan should I choose?}

            I therefore need to look for dynamic resources. 
            \item{What is the move?}

            \symrook xg5!
        \end{itemize}

        After 
        \mainline{
        36. Rxg5+ Kxg5 37. Qd2+ Kg6 38. Qg2+ Kf7 39. Bd5 Qxd5 40. exd5 Rg8 41. Qxg8+ Kxg8 }

        \chessboard

        The resulting position is no worse than the initial one.
        White remains a pawn down, but his protected passed pawn grants him counterplay.
        Black still has to demonstrate a concrete route to victory. 
	
    }
\fi

\section{Excercises}
\begin{multicols}{2}
	\shortSokolov{1}
	\dingGukeshGameTwelve{2}
	\nielsenDreev{3}
	\dingNepomniachtchiGameFour{4}
	\arnasonMiles{5}
\end{multicols}

\section{Answers}
\begin{multicols}{2}
	\shortSokolovAnswer{1}
	\dingGukeshGameTwelveAnswer{2}
	\nielsenDreevAnswer{3}
	\dingNepomniachtchiGameFourAnswer{4}
	\arnasonMilesAnswer{5}
\end{multicols}

\chapter{Annotations}
I annotate chess games as a means of self-improvement and personal learning. The market offers a wealth of excellent chess literature, much of which is dense with variations. Additionally, there are countless chess games available, and uncovering hidden insights within these games provides a valuable experience for me.

My primary focus is on identifying critical moments and key moves using straightforward methods defined in \ref{sec:thinking-template} and \ref{sec:critical-position}, rather than getting lost in complex variations. I prefer to analyze games with fewer moves, often avoiding endgames, as they tend to require extensive calculations and are not my main interest.

After pinpointing these critical moments, I verify my findings using computer analysis. There are times when I miss key points, but other times I succeed in my assessments. This process is an integral part of my learning journey.

I intentionally refrain from using computers to evaluate the players, as I believe such an approach will not contribute to my growth as a chess player.

Ultimately, I aspire to discover effective strategies for playing good chess through simple methods, drawing inspiration from the styles of great players like Karpov and Petrosian.
% Define a macro for keywords, the keywords are displayed in italic
% show the text as: Keywords: London System, French Defence, Square Play, Diagonal Play
\newcommand{\keywords}[1]{Keywords: \textit{#1}}

\section{Anatoly Karpov - Garry Kasparov, FIDE World Championship Match 1984/85, Round 3}
\epigraph{An impulsive, nervous reaction, the preceding game would appear to have put me in a not altogether correct
frame of mind, such that I could solve all my problems immediately with the help of a `sharp' pawn sacrifice}{\textit{Garry Kasparov}}

\keywords{Maroczy Bind, Hedgehog Position, Impulsive Reaction}
% My first annotation without books and computers. I chose this game
% because it is quite short. The strategic win from Karpov is also instructive according
% to chessgames.com

% I am happy with the result that I found out 12... Nc5 was bad. 
% I didnt figure out 16... d5 was explicitly because my analysis was not thourough enough and I 
% turned to books too early..

% The winning of Karpov was instructive. I should have taken more time.
\newchessgame[
 id=main,
 event={FIDE World Championship Match 1984/85},
 white={Karpov, Anatoly},
 black={Kasparov, Garry},
 round={3}]

\mainline{1.e4 c5 2.Nf3 e6 3.d4 cxd4 4.Nxd4 Nc6 5.Nb5 d6 6.c4 Nf6 7.N1c3
a6 8.Na3 Be7 9.Be2 O-O 10.O-O b6 } 

\chessboard[
  markstyle=circle,
  linewidth=0.05em,
  markfields={b6},
]

Black has weakness on b6. The next moves of White are logical, attacking 
the weakness and developing his pieces.


\mainline{11.Be3 Bb7} 
Black has a so-called hedgehog position. Typically Black wants to play 
b6-b5 and d6-d5 to free himself. With White's Maroczy bind (c4 and e4 pawns),
he can stop Blacking from playing d5. With White's next move, he stops both b5 and d5 break.

\chessboard[
    pgfstyle=straightmove,
    linewidth=0.05em,
    markmove={b6-b5, d6-d5},
    markstyle=circle,
    linewidth=0.05em,
    markfields={b6},
]

\mainline{12. Qb3}

\chessboard

b6 pawn is under attack. \variation[invar]{12... Nd7} is a natural move.
after \variation[invar]{13. Rad1 \xskakcomment{ Preventing Nc5 with \symbishop xc5. The bishop on
b7 is unprotected..} Qc7 \xskakcomment{intending \symknight c5}} Black is still in the game.

\mainline[level=1]{12... Na5?} 

Kasparov chose to force the matter. After this move White has 3 vs 1 
on the queenside and Black has a backward pawn on the d-file.

\chessboard

\variation[invar]{13. Bxb6 Nxb3 14. Bxd8 Rfxd8 15. axb3 Nxe4 16. Nxe4 Bxe4} is not good enough because
Black's bishop pair gives him enough chance to defend.

\mainline[level=1]{ 13.Qxb6
Nxe4 14.Nxe4 Bxe4 15.Qxd8 Bxd8} (\variation[invar]{15... Rxd8 16. Bb6} This is by the way another
drawback of the move \variation{12... Na5}) 


\mainline[level=1]{ 16.Rad1} 

Quite often the most natural move is the best.

\chessboard

% I wrote "This position must be estimated carefully before Black's 12th move."
% Actually Kasparov has expected only 16. Rfd1 in his preparation and was instantaneously nervous after
% seeing this move. If he could not foresee such a move, I shouldnt expect myself
% to foresee it on the board. 


\mainline{16... d5?}

\quote{An impulsive, nervous reaction, the
preceding game would appear to have put
me in a not altogether correct frame of
mind, such that I could solve all my
problems immediately with the help of a
`sharp' pawn sacrifice} \cite{Kasparov:2008}



\variation[invar]{16... Be7 17. Nb1 \xskakcomment{ (improving his worst piece) } Rac8} Black can still hold on.

\chessboard

\mainline{ 17.f3 Bf5 }

\chessboard

\mainline{18.cxd5
exd5 19.Rxd5 Be6 20.Rd6 Bxa2 21.Rxa6 Rb8 }

\chessboard

\mainline{22.Bc5! Re8 23.Bb5! }
Notice how Karpov used bishops to win tempi and protect
the b2 pawn from the b8 rook. Also note how the Black knight on a5 is doomed by the b5 bishop. 


\mainline{23... Re6
24.b4 Nb7 25.Bf2 Be7 26.Nc2 Bd5 27.Rd1 Bb3 28.Rd7! Rd8} (\variation[invar]{28... Bxc2 29. Rxe6}) \mainline[level=1]{29.Rxe6
Rxd7 30.Re1 Rc7 31.Bb6} 1-0

\subsection*{Lessons Learned}
\begin{itemize}
  \item{Clear your mind, no impulsive, nervous reaction!}
  \item{Evaluate carefully when forcing the exchanges}
  \item{One can make simple natural moves has the advantage}
\end{itemize}
%\input{cards/chess/annotations/kasparov_karpov_1984_game_6}
\section{Boris Spassky - Tigran Vartanovich Petrosian, World Championship Match 1969 Game 5}


\newchessgame[
 id=main,
 event={Petrosian - Spassky World Championship Match},
 white={Boris Spassky},
 black={Tigran Petrosian},
 round={5}]

\mainline{
    1. c4 Nf6 2. Nc3 e6 3. Nf3 d5 4. d4 c5 5. cxd5 Nxd5 6. e4 Nxc3
    7. bxc3 cxd4 8. cxd4}
    
\chessboard [
    markstyle=border,
    linewidth=0.05em,
    markfields={c8},
]

I don't like Black's position. White already has a static advantage
because his pawn formation is better and Black has a passive
light square bishop.

On the other hand, White has some natural moves to develop:
\symbishop c4, O-O, \symrook ad1, \symrook fe1 etc.

It is however difficult to recommend any dynamic play for Black here.

\mainline{8... Bb4+ 9. Bd2 Bxd2+ 10. Qxd2 O-O 11. Bc4
    Nc6 12. O-O b6 13. Rad1 Bb7 14. Rfe1} 
    
\chessboard [
    pgfstyle=straightmove,
    linewidth=0.05em,
    markmove={d4-d5},
]

\mainline {14... Rc8 } 

White is planning the d5 thrust to create a central passed pawn.
Black could also tried
\variation[invar]{14... Na5 15. Bf1 Rc8 16. d5 Nc4 17. Bxc4 Rxc4}
    \variation{18. d6 Rxe4 19. Ne5 Rxe1 20. Rxe1 Bd5\xskakcomment{Black can defend the ending}}

% 2025. 11.5: I wrote the following during my own analysis. 
% the conclusion is premature because I miss error on Black's next move
% "Now White has a straight forward move, his light square bishop can
% exchange an important defender, Black's light square bishop 
% while the Black's knight can't help too much in the defense."

\chessboard [
	pgfstyle=straightmove,
    linewidth=0.05em,
 	markmove={d4-d5},
]


% 2025. 11.5: In my analysis, I saw the knight stuck at a5 for a long time.
% I failed to point out, that 16... Na5 is an error.
\mainline[level=1]{15. d5}

\chessboard
% 2025. 11.5: I missed the error completely. Alas, it is a critical position and I failed to 
% see it. 
Black should have played \varriation[invar]{15... Na5}
\variation{16. dxe6 Nxc4? 17. exf7+ Kh8 18. Qxd8 Rcxd8 19. Rxd8 Rxd8 20. e5} White wins.
\variation{16. dxe6 Qxd2 17. exf7+ Kh8 18. Nxd2 Nxc4 19. Nxc4 Rxc4 20. e5 Bc8 21. e6 Bxe6 22. Rxe6} equal.
Petrosian didn't like \variation{16. Bd3 exd5 17. e5! Nc4 18. Qf4} White has an attack. Indeed, Polugaevsky won a brilliant game
against Tal later. Black must find the only defense \symrook c6.

\mainline{15... exd5 16. Bxd5
    Na5 17. Qf4} 


\chessboard

Black has a desperate position. He cannot defend the d5 passed pawn.
White could also play \symqueen f5, \symknight g5 attack the h7 pawn.
    

\mainline{17... Qc7 18. Qf5 Bxd5 19. exd5 Qc2 20. Qf4 Qxa2 21. d6
    Rcd8 22. d7 Qc4 23. Qf5 h6 24. Rc1 Qa6 25. Rc7 b5}
   
\chessboard
    
\mainline{26. Nd4 Qb6
    27. Rc8 Nb7 28. Nc6 Nd6 29. Nxd8 Nxf5 30. Nc6 
}
1-0


\newpage
\section{Ding, Liren - Gukesh D, World Championship Match, Round 12}

\epigraph{I drank coffee. I changed my haircut a little}{\textit{Ding Liren}}

\keywords{Simple Play, Dynamic Play}

\begin{multicols}{2}
	\chessgameinfo{World Championship Match}{Ding, Liren}{Gukesh D}{12}{2025.02.03}{1-0}
It was the 12th game of a 14-game match, and Ding found himself trailing by one point after his recent loss. Playing with the white pieces, he knew he needed to make a strong move to regain momentum. In a strategic decision, he opted for a rather calm opening, setting the stage for the challenges that lay ahead. 

\newchessgame[
 id=main,
 storefen=example,
 event={Ding - Gukesh World Championship Match},
 white={Ding, Liren},
 black={Gukesh D},
 round={12}]
 \mainline[level=1]{1. c4 e6 2. g3 d5 3. Bg2 Nf6 4. Nf3 d4 5. O-O Nc6 6. e3 Be7 7. d3 dxe3 8. Bxe3 e5 9. Nc3 O-O 10. Re1 h6}

 \chessboard[
   	pgfstyle=straightmove,
	linewidth=0.05em,
 	markmove=d3-d4
 ]

 Ding Liren is known for his solid, strategic approach to chess. He excels in positional play and endgames, often outmaneuvering opponents with his patience and precision. Here the position is calm and it's all about piece maneuvering. 


He thought for a long time about his next move to provoke. 

 \mainline[level=1]{
11. a3 } 
In this position, White wants to push d3-d4. His knight on f3 is critical for the breakthrough and should not be exchanged.  \variation[invar]{11.h3} is a good alternative.

\mainline[level=1]{11...a5}

\chessboard[
   	pgfstyle=straightmove,
        linewidth=0.05em,
 	markmove=d3-d4,
 	markstyle=circle,
 	linewidth=0.05em,
 	markfields={b5},
 ]

It was understandable that Gukesh wanted to prevent b4 with this move. However, compared to the previous position,
he now has one serious weakness on b5 which White can easily exploit with his knight.

Black would love to move his knight on c6 to play c6, guarding the weakness on b5 and restricting the bishop on g2.
He cannot do so because the knight must guard e5. We can also understand Ding's next move, which keeps the 
pressure on e5 by preventing Black from playing \variation{11...Bg4}, exchanging the knight on f3 and releasing the pressure
on his e5 pawn.

\mainline[level=1]{12. h3} 

\resumechessgame
A direct \variation[invar]{12.d4 exd4 13. Nxd4 Nxd4 14. Bxd4 c6 15. c5} according to the engine. The position opens up and White has some advantage.

 \chessboard[
        setfen=r1bq1rk1/1p2bpp1/2p2n1p/p1P5/3B4/P1N3P1/1P3PBP/R2QR1K1 b - - 0 15
        ]

I don't know if the position favors White. After exchanging some pieces, Black should have more room to maneuver than 
several moves ago, where he had a cramped position. Chess law states it is better to avoid exchanging when one has a space advantage. This is exactly the case for White.

\chessboard[
   	pgfstyle=straightmove,
        linewidth=0.05em,
 	markmove=d3-d4,
 	markstyle=circle,
 	linewidth=0.05em,
 	markfields={b5},
 ]

It is instructive to observe the position here. White still has his trumps: d3-d4 and an outpost on b5, and Black can do nothing to parry. So there is no need for White to rush. With h3, White deprives Black of placing a bishop or knight on
g4. White slowly improves his position and Black has no counterattack. 

\resumechessgame[id=main]
\mainline[level=1]{12...Be6 13. Kh2} 

No need to rush! \vocab{prophylaxis}{Prophylaxis} against \symqueen d7 \index{Prophylaxis}.

\mainline{13...Rb8}

\chessboard[
   	pgfstyle=straightmove,
        linewidth=0.05em,
 	markmove={d3-d4,e5-e4},
 	markstyle=circle,
 	linewidth=0.05em,
 	markfields={b5},
 ]

Here White has more options. Again, \variation{14. d4 exd4 15. Nxd4 Nxd4 16. Bxd4 c6 17. c5} loses his advantage. White loses his trump b5 now because the c6 pawn guards it. 
\variation{14. Nb5 Nh7 15. Qd2 Ng5 16. Nxg5 hxg5 17. Rad1} 

\chessboard[
        setfen=1r1q1rk1/1pp1bpp1/2n1b3/pN2p1p1/2P5/P2PB1PP/1P1Q1PBK/3RR3 b - - 1 17,
	markstyle=circle,
 	linewidth=0.05em,
 	markfields={e4},
        ]
White keeps the advantage because Black now has a new weakness on e4.

Ding's next move is prophylaxis against any Black potential e4 thrust.
\mainline[level=1]{14. Qc2 Re8} 

\chessboard[
   	pgfstyle=straightmove,
        linewidth=0.05em,
 	markmove={d3-d4,e5-e4},
 	markstyle=circle,
 	linewidth=0.05em,
 	markfields={b5},
 ]

\mainline{ 15. Nb5}

\chessboard[
   	pgfstyle=straightmove,
        linewidth=0.05em,
 	markmove={d3-d4,e5-e4},
 ]

Finally the move! White achieves his goal on b5. Now White is preparing d4. 

\mainline{15... Bf5}

\chessboard[
   	pgfstyle=straightmove,
        linewidth=0.05em,
 	markmove={d3-d4,e5-e4},
 	markstyle=border,
 	linewidth=0.05em,
 	markfields={a1},
 ]

The rook on a1 is not so active. It can move to d1 to support the d3-d4 thrust. The next
move is logical. Even a World Chess Champion game can be so simple!

\mainline{16. Rad1 Nd7}

\chessboard[
   	pgfstyle=straightmove,
        linewidth=0.05em,
 	markmove={d3-d4,e7-g5},
 ]

White wants to play d4 and prevent Black from playing \variation[invar]{16...Bg5}. His next move is logical.

\mainline[level=1]{ 17. Qd2} 

\chessboard[
   	pgfstyle=straightmove,
        linewidth=0.05em,
 	markmove={d3-d4},
 ]

 Grandmaster Dorfman writes: ``If for one of the players the static balance is negative, he must without hesitation 
 employ dynamic means, and be ready to go in for extreme measures.''  \cite{Dorfman:2001}

 Here White has obviously a static advantage: he has a clear plan to improve his position while
 Black just has no clear plan. For this reason, Black must play dynamically!
\mainline{17... Bg6? } 

\variation[invar]{17... Nc5 18. d5 Nd3 19. d5 Nxe1 20. Qxe1 Nd4 21. Nfxd4 exd4 22. Nxd4 Bh7 23. Qxa5}

\chessboard[
   	setfen=1r1qr1k1/1pp2pp1/6bp/Q2P2b1/2PN4/P3B1PP/1P3PBK/3R4 w - - 1 24
 ]

White still has a static advantage, with the price of one exchange. Black has more
space to maneuver after exchanging some pieces. The position is much more playable
than in the actual game.

Go back to the actual game. White has prepared everything. It is now the time!

\chessboard[
   	pgfstyle=straightmove,
        linewidth=0.05em,
 	markmove={d3-d4},
 ]

\mainline[level=1]{18. d4 e4}

The only move. \variation[invar]{18...exd4 19. Bf4! Rc8 20. Nfxd4 Nxd4 21. Qxd4 Nc5 22. Nxc7!} White wins easily.

\chessboard[
 	markstyle=border,
 	linewidth=0.05em,
 	markfields={f3},
	pgfstyle=straightmove,
        linewidth=0.05em,
 	markmove={f3-g1, g1-e2, e2-f4},
 ]

\mainline[level=1]{ 19. Ng1 Nb6 20. Qc3! }

\chessboard[
	pgfstyle=straightmove,
        linewidth=0.05em,
 	markmove={d4-d5},
 ]

It is quite an instructive move. The queen deprives the knight on c6 of the a5 and e5 squares.
Meanwhile, d5 is coming.

\mainline{20...Bf6 21. Qc2 a4 22. Ne2 Bg5} 

\chessboard[
	pgfstyle=straightmove,
        linewidth=0.05em,
 	markmove={e2-f4},
 ]

\mainline{ 23. Nf4 Bxf4 24. Bxf4 Rc8 25. Qc3 Nb8}

\chessboard[
	pgfstyle=straightmove,
        linewidth=0.05em,
 	markmove={d4-d5},
 ]

\mainline{26. d5 Qd7 27. d6 c5 28. Nc7 Rf8 29. Bxe4 Nc6} 
(\variation[invar]{29...Bxe4 30. Rxe4 Nc6 31. Bxh6 gxh6 32. Rg4+ Kh7 33. Qg7#})

\mainline[level=1]{ 30. Bg2 Rcd8
31. Nd5 Nxd5 32. cxd5 Nb8 } 

\chessboard

Gukesh could have resigned here.

\mainline{33. Qxc5 Rc8 34. Qd4 Na6 35. Re7 Qb5
36. d7 Rc4 37. Qe3 Rc2 38. Bd6 f6} 

\chessboard

The live commentator said:``This is the only time Ding has calculated during the game!''

\mainline{39. Rxg7+} Black resigned.

 \chessboard

 Certainly this was not Gukesh's best game. Besides the game,
 it was more instructive to hear how Ding prepared for the game after the defeat the previous day:
 ``I drank coffee, I changed my haircut a little bit.''

\end{multicols}
 \subsection*{Lessons Learned}
 \begin{itemize}
    \item{Simplicity is often the best strategy. ``I drank coffee, I changed my haircut a little bit.''}
	\item{Play dynamically when one has a static disadvantage!}
\end{itemize}

\section{Magnus Carlsen - Ding Liren, Magnus Carlsen Chess Tour Finals}
% \keywords{London System, French Defence, Square Play, Diagonal Play}

\epigraph{ In difficult positions I make moves that do 
not lose by force.}{\textit{Anatoly Karpov}}

\keywords{London System, French Defence, Square Play, Diagonal Play}

\begin{multicols}{2}
\newchessgame[
 id=main,
 event={Magnus Carlsen Chess Tour Finals},
 white={Magnus Carlsen},
 black={Ding Liren},
 round={1}]

 \mainline[level=1]{
    1. d4 Nf6 2. Nf3 d5 3. Bf4 c5 4. e3 e6 5. c3 Bd6 6. Nbd2}
    
\chessboard[
    markstyle=border,
    linewidth=0.05em,
    markfields={c8},
]

A typical London System opening: with a hidden cunning idea. We 
reach the first critical moment of the game. White offers a pawn sacrifice. Black 
has more options to choose from. 

\begin{enumerate}
    \item{Decline the sacrifice}

    Black has a pawn stucture similar
    to French Defence. His light squared bishop is not active. He could
    choose to proceed with a normal French Defence setup:
    \variation[invar]{6... Bxf4 7. exf4 Qb6 8. Qc2 Qc7 9. g3 b6 10. Bg2 Bg7}

    \chessboard[
        setfen=rn2k2r/pbq2ppp/1p2pn2/2pp4/3P1P2/2P2NP1/PPQN1PBP/R3K2R w KQkq - 2 11
    ]
    \item{Accept the sacrifice, take on c3}

    \variation[invar]{6... cxd4 7. Bxd6 dxc3} White has compensation and can play \variation[invar]
    {8.Qa4+} or \variation[invar]{8.Ba3} as in the game.

    \chessboard[
        setfen=rnbqk2r/pp3ppp/3Bpn2/3p4/8/2p1PN2/PP1N1PPP/R2QKB1R w KQkq - 0 8
    ]

    \item{Accept the sacrifice, take on e3}

    As in the current game:    
\end{enumerate}
    
\mainline[level=1]{6... cxd4 7. Bxd6 dxe3 8. Ba3 exd2+ 9. Qxd2}

\chessboard[
    pgfstyle=straightmove,
    linewidth=0.05em,
    markmove={d2-g5},
    markstyle=border,
    linewidth=0.05em,
    markfields={c8},
    markstyle=circle,
    linewidth=0.05em,
    markfields={g7},
]


Here another critical moment.

Let us first list the static advantages for White:
\begin{itemize}
    \item{Bishop pair}
    \item{Better King safety}
    \item{Black has an inactive bishop on c8.}
    \item{Black has weakness on the dark squares.}
\end{itemize}
Dynamically White has lead in development for a sacrificed pawn.

Black has however a complete pawn structure. He can castle after some moves. So his
position is still defendable.

White intends to play \variation[invar]{10. Qg5} threatens the g7 pawn. One move is to play
\variation[invar]{10... Ne4 11. Qe3 Qb6 12. Nd4 f6}.
Black can move his king to f7:

\chessboard[
    setfen=r1b4r/pp3kpp/1qn1pp2/3p4/3Nn3/B1PBQ3/PP3PPP/R4RK1 w - - 4 14
]

In the game, Ding chose to play 9... \symknight c6. This move is per se not bad. 
He must however play accurately afterwards

\mainline[level=1]{9... Nc6 10. Qg5 Rg8 11. Bd3 h6
12. Qe3 Qb6 13. Qe2 Bd7 14. O-O O-O-O 15. b4} 

\chessboard[
    pgfstyle=straightmove,
    linewidth=0.05em,
    markmove={b4-b5},
    markstyle=circle,
    linewidth=0.05em,
    markfields={e5},
    markstyle=border,
    linewidth=0.05em,
    markfields={d7},
]

Another critical moment.
White threatens to play b5 to drive away the knight on c6 and then controls the e5 square with his knight:
Once White achieves his goal, Black has a desperate position. Therefore he must react now!

Understanding White's idea, Black can play \variation[invar]{15... e5 16. b5 e4 17. bxc6 Qxc6 18. Ne5 Qc7 
19. Nxf7 Bg4 20. Qe3 Qxf7 21. Bc2}. This move also activates the bishop on d7 as a bonus.

\chessboard[
    setfen=2kr2r1/pp3qp1/5n1p/3p4/4p1b1/B1P1Q3/P1B2PPP/R4RK1 b - - 1 21
]

Black equalizes.

In the game, Ding's move is a serious mistake. 
\mainline[level=1]{15... Kb8? 16. b5 \xskakcomment{ Of course!} Na5
17. Ne5 Be8 18. Bb4 Rc8 19. a4} 

\chessboard

Black is passive but has no weaknesses. He should have waited (e.g. \variation[invar]{19... Qc7}) and
let White to prove his advantage. Here it is important
to remember what Karpov said: in difficult positions I make moves that do 
not lose by force.

After the text move, White wins quickly.
\mainline[level=1]{19... Ne4 20. Bxe4 dxe4 21. Qxe4 f6
22. Qh7 Nb3 23. Qxg8 Nxa1 24. Qxe8 \xskakcomment{ 1-0}
 }

 \chessboard

\end{multicols}
 % 2025. 11.6: I am happy to find the critical positions on my own: 15... Kb8
 \subsection*{Lessons Learned}
 \begin{itemize}
    \item{Knowledge in other openings with similar pawn structure
 can help to find the right idea.}
    \item{Don't make moves that lose by force in difficult positions.}
 \end{itemize}
\section{Ding, Liren - Ian Nepomniachtchi 2023 World Championship Game 4}

\epigraph{
    Ian Nepomniachtchi is a great dynamic player. Such players often
    find it difficult to sit and defend passively. And it seems that
    this position requires exactly that.
}{Grandmaster David Navara}

\keywords{Pawn Sacrifice, Patience in Defense, Practical Play, Passed Pawns}

\begin{multicols}{2}
    \newchessgame[      
        id=main,
        event={FIDE World Championship 2023},
        white={Ding, Liren},
        black={Nepomniachtchi, Ian},
        round={4}]
    \mainline[level=1]{
        1. c4 Nf6 2. Nc3 e5 3. Nf3 Nc6 4. e3 Bb4 5. Qc2 Bxc3 6. bxc3 d6 7. e4 O-O 8. Be2 Nh5 9. d4}

    \chessboard

    There is a fierce fight in the center around d4. An alternative is \variation[invar]{9... Qf6 \xskakcomment{ a natural idea.}
    10. d5 Na5 11. g3 Bg4 } 

    \chessboard[
        setfen=r4rk1/ppp2ppp/3p1q2/n2Pp2n/2P1P1b1/2P2NP1/P1Q1BP1P/R1B1K2R w KQ - 1 12
    ]

    Black has a solid but passive position. It is probably not to Nepo's taste. 

    \mainline[level=1]{9...Nf4 10. Bxf4 exf4 11. O-O Qf6 12. Rfe1 Re8}
    
    \chessboard

    In the middlegame, it is often more about choosing moves and positions according to the 
    players' style than finding the objectively best move.

    White has a solid center and Black has a compact position. Continuing in this manner 
    would suit Ding better because Nepo, as a dynamic player, cannot sit and wait passively.

    \variation[invar]{13. c5 dxc5 14. e5 Qh6 15. Rad1 Bg4 16. Qb3} would be a good idea but not a good practical decision:

    \chessboard[
        setfen=r3r1k1/ppp2ppp/2n4q/2p1P3/3P1pb1/1QP2N2/P3BPPP/3RR1K1 b - - 4 16
    ]

    The position is sharp. White may have some advantage. However, the position 
    is open and Black has counterplay and open lines. What is the point of allowing
    a dynamic player tactical opportunities?

    Ding chooses a natural move, improving his position slowly and not allowing Nepo any counterplay.
    
    \mainline[level=1]{ 13. Bd3 Bg4 14. Nd2}
    
    \chessboard
    
    \mainline[level=1]{14...Na5?}
    
    A very strange move. The knight on a5 has no future. \variation[invar]{14...Rad8} would be a natural move.

    \chessboard
    
    \mainline[level=1]{ 15. c5! \xskakcomment{ With his sacrifice White activates his pieces.} dxc5 16. e5 Qh6 17. d5 Rad8 18. c4}
    
    \chessboard

    White has some advantage here. His center is strong and Black has 
    a passive knight on a5. 

    While the position may still be equal according to the computer, for human players White has a much easier position to play. He controls the center, while his opponent, as a dynamic player, can only sit and wait. At some point, his opponent would lose patience while defending and make mistakes, as happened in this game. 

    \mainline[level=1]{18... b6 19. h3 Bh5}

    \chessboard[
        pgfstyle=straightmove,
        linewidth=0.05em,
        markmove=f4-f3,
        markstyle=circle,
        linewidth=0.05em,
        markfields={e5},
    ]

    \begin{itemize}
        \item Where are the weaknesses?
    
        The e5 pawn is the pivot of the position and must be protected.
        \item Which is the worst-placed piece?
       
        The rooks must be activated.
        \item What is my opponent's idea?
    
        He wants to play \symknight f3, creating some counterplay.
    \end{itemize}
    
    By answering the questions above, White can find the next few moves:
    \begin{itemize}
        \item Move his bishop to e4 then f3 (if Black exchanges the bishop, White has a knight on f3, which further strengthens the e5 pawn).
        \item Move his queen to c3 to protect the e5 pawn.
        \item Double his rooks on the e-file to protect the e5 pawn.
    \end{itemize}

    Black, however, must play move by move.

    \mainline[level=1]{20. Be4 Re7 21. Qc3 Rde8 22. Bf3 Nb7 23. Re2} 
    
    As mentioned above, White has a clear plan and needs only to execute it, without thinking too much. 

    \chessboard

    Black makes a difficult decision here, allowing White to create a passed pawn but gaining 
    a good square on d6 for his knight.
    
    \mainline[level=1]{23... f6 24. e6 Nd6 25. Rae1} 
    
    \chessboard[
        setfen=4r1k1/p1p1r1pp/1p1nPp1q/2pP3b/2P2p2/2Q2B1P/P2NRPP1/4R1K1 b - - 2 25
    ]

    White has executed his plan and Black has defended well. How should Black defend next?

    I believe the question can be answered logically without calculating a lot.
    The main asset of White is the passed pawns, which have been blocked by the Black rooks.

    The rooks on the e-file are not doing much because there is no open file. The knights are 
    important pieces. The Black knight keeps an eye on e4 so that the rooks cannot attack the 
    f4 pawn. 
    
    At some point, White may play \symrook e4 to attack the f4 pawn and exchange the Black knight.

    Alternatively, White may play \symknight e4 to exchange the Black knight. In either case, e4 is 
    an important square and must be protected in advance. \variation[invar]{25... Bg6} is a good move.

    It is unclear how White can make progress here. In the actual game, Black chooses to defend
    actively and soon makes a severe mistake.
    
    \mainline[level=1]{25... Nf5?!} 
    
    \chessboard

    White's bishop cannot improve White's position, while Black's bishop can defend the important e4
    square. It is therefore logical to exchange the bishops and then occupy the e4 square with the rook.
    
    \mainline[level=1]{ 26. Bxh5 Qxh5 27. Re4 Qh6 }
    
    \chessboard
    
    \mainline{28. Qf3}

    Again, a very logical move, attacking the weak f4 pawn. Ding's moves 
    are natural, although not the perfect computer moves. By playing these moves,
    he sets problems for Nepo that cannot be solved using dynamic play---a 
    very practical choice!
    
    \chessboard

    After \variation[invar]{28... g5 29. g4 Nd6}, the position is still defensible for Black. 

    \mainline[level=1]{28...Nd4?? } 
    
    \chessboard

    I am not sure whether Nepo sees a trap that backfires because Ding has set a deeper one. After \variation[invar]{29. Qxf4 Qxf4 30. Rxf4 c6 31. Nf3 Nxf3 32. Rxf3 cxd5 33. cxd5 Rd8 34. Rd3 Rd6} 
    
    \chessboard[setfen=6k1/p3r1pp/1p1rPp2/2pP4/8/3R3P/P4PP1/4R1K1 w - - 3 35]

    White has no advantage.

    More plausibly, Nepo loses his patience in defense and wants to force a draw. 
    
    \chessboard

    \mainline[level=1]{29. Rxd4! } 
    
    Of course, the knight is much more valuable than the rook.

    \mainline[level=1]{29...cxd4 30. Nb3 g5 31. Nxd4 Qg6 32. g4 fxg3 33. fxg3 h5 34. Nf5 Rh7 35. Qe4 Kh8 36. e7 Qf7 37. d6 cxd6 38. Nxd6 Qg8 39. Nxe8 Qxe8 40. Qe6 Kg7 41. Rf1 Rh6 42. Rd1 f5 43. Qe5+ Kf7 44. Qxf5+ Rf6 45. Qh7+ Ke6 46. Qg7 Rg6 47. Qf8}

    \chessboard
\end{multicols}

\subsection*{Lessons Learned}

In the middlegame, it is often more important to choose positions that match your playing style than to find the objectively best move. Ding chose a solid, positional approach that suited him better than sharp tactical lines that would favor his dynamic opponent. He correctly avoided opening the position unnecessarily, which would have given Nepomniachtchi counterplay and tactical chances, even though it might have been objectively good. By choosing positions where he felt comfortable and his opponent would struggle, Ding created opportunities for mistakes.

A well-timed pawn sacrifice can activate your pieces and create a strong center. Sometimes material is less important than piece activity and positional control. When your pieces become active and you gain strategic advantages, a pawn sacrifice can be a powerful tool.

When defending a difficult position, patience is essential. Black's position was still defensible, but an impatient move led to immediate defeat. Dynamic players often struggle with passive defense, and maintaining patience can be the difference between holding the position and losing.

Ding's moves were natural and practical, even if not always the computer's top choice. By setting problems that his opponent couldn't solve with dynamic play, he achieved a practical advantage. Sometimes the best move is not the objectively strongest one, but the one that creates the most problems for your opponent in a practical game.

\section{Ding, Liren - Ian Nepomniachtchi 2023 World Championship Game 6}

\keywords{London System, Knight Maneuvering}
\begin{multicols}{2}
    \newchessgame[      
        id=main,
        event={FIDE World Championship 2023},
        white={Ding, Liren},
        black={Nepomniachtchi, Ian},
        round={6}]
    \mainline[level=1]{1. d4 Nf6 2. Nf3 d5 3. Bf4 c5 4. e3 Nc6 5. Nbd2 cxd4
    6. exd4 Bf5 7. c3 e6} 
    
    \chessboard[
    ]
    
    Modern chess is full of subtleties. The position is actually a reversed
    Queen's Gambit Declined Exchange Variation. Typically after 
    \variation[invar]{1. d4 d5 2. c4 e6 3. Nc3 Nf6 4. cxd5 exd5 5. Bg5 Be7 6. e3 c6}
    
    \chessboard[
        setfen=r1bqk2r/pp1nbppp/2p2n2/3p2B1/3P4/2NBP3/PP3PPP/R2QK1NR w KQkq - 2 8
    ]
    
    We have a so-called Karlsbad structure. White has the following typical ideas:

    \begin{itemize}
        \item{Minority Attack}

        White advances his queenside pawns with the main purpose of creating weaknesses in black’s structure.

        \chessboard[
            setfen=8/pp3ppp/2p5/3p4/3P4/4P3/PP3PPP/8 w KQkq - 2 8,
            pgfstyle=straightmove,
            linewidth=0.05em,
            markmove={b2-b4, b4-b5, b5-c6, b7-c6},
        ]
        
        \item{Playing for the e3-e4 push}

        This can be done with or without the support of the f pawn (by pushing f3) then e4.
    
        \chessboard[
            setfen=8/pp3ppp/2p5/3p4/3P4/4P3/PP3PPP/8 w KQkq - 2 8,
            pgfstyle=straightmove,
            linewidth=0.05em,
            markmove={e3-e4},
        ]   
    \end{itemize}
    
    Black must defend accordingly. His ideas are
    \begin{itemize}
        \item{Moving the knight from g8 to e4}
        \item{Moving the knight from b8 to c4 via d7 and b6.}
    \end{itemize}
    
    We will see in this game that Ding adopts both of Black's ideas from the Queen's Gambit Declined Exchange Variation.  
    Personally I am not a fan of Black's opening. If in the Queen's Gambit
    Declined Exchange Variation, Black can defend well. With one more
    tempo and colors reversed, White should have some advantage. How can Black still
    defend? 

    Let's go back to the current game.

    \chessboard[
        pgfstyle=straightmove,
        linewidth=0.05em,
        markmove={e6-e5},
    ]

    The position is static and slow, meaning both players must
    seek plans episode after episode to improve their positions.

    \begin{itemize}
        \item{Where are the weaknesses?}
    
        It is still too hard to tell.
        \item{Which is the worst-placed piece?}
       
        The bishop on f1 must be developed, then castling.
        \item{What is my opponent's idea?}
    
        He wants to play e5 to free himself.
    \end{itemize}

    Understanding these, White should develop his bishop and then castle.
    He should also pay attention to Black's e5 thrust. So \symrook e1 is natural.

    \mainline[level=1]{ 8. Bb5}

    \chessboard[
        pgfstyle=straightmove,
        linewidth=0.05em,
        markmove={e6-e5},
    ]

    \begin{itemize}
        \item{Where are the weaknesses?}
    
        It is still too hard to tell.
        \item{Which is the worst-placed piece?}
       
        The bishop on f8 must be developed, then castling.
        \item{What is my opponent's idea?}
    
        Not clear yet.
    \end{itemize}

    White has a strong bishop outside its pawn chain on f4. Exchanging it
    with \symbishop d6 and developing is logical. The next few moves are natural.

    \mainline[level=1]{ 8...Bd6 9. Bxd6 Qxd6 10. O-O O-O
    11. Re1 h6} 
    
    \chessboard [
        markstyle=circle,
        linewidth=0.05em,
        markfields={c5, e5},
        markstyle=border,
        linewidth=0.05em,
        markfields={a1, d2},
        pgfstyle=straightmove,
        markmove={e6-e5},
    ]
    
    \begin{itemize}
        \item{Where are the weaknesses?}
    
        c5 and e5 are weak. White may have knights to occupy these
        squares.
        \item{Which is the worst-placed piece?}
       
        Not clear, neither the rook on a1 nor the knight on d2.
        \item{What is my opponent's idea?}
    
        e5 thrust
    \end{itemize}
    
    \mainline{ 12. Ne5} 
    
    \begin{itemize}
        \item{Where are the weaknesses?}
    
        c5 and e5 are weak. White may have knights to occupy these
        squares.
        \item{Which is the worst-placed piece?}
       
        Not clear, either the rook on a1 or the knight on d2.
        \item{What is my opponent's idea?}
    
        e5 thrust
    \end{itemize}
    
    \chessboard [
        markstyle=circle,
        linewidth=0.05em,
        markfields={c5, e5},
    ]

    \begin{itemize}
        \item{Where are the weaknesses?}
    
        c5 and e5 are weak squares.
        \item{Which is the worst-placed piece?}
       
        Not clear.
        \item{What is my opponent's idea?}
    
        \symbishop xc6, then occupy c5 with his knight.
    \end{itemize}

    \mainline{12... Ne7 } 
    
    % TODO:
    % r4rk1/pp2npp1/3qpn1p/1B1pNb2/3P4/2P5/PP1N1PPP/R2QR1K1 w - - 2 13
    % play out!


    \mainline{13. a4 a6}
    This move is okay in itself. However, Black starts a wrong plan that
    causes his position to deteriorate soon.

    \mainline{ 14. Bf1}
    
    \chessboard [
        markstyle=circle,
        linewidth=0.05em,
        markfields={c5},
        markstyle=border,
        linewidth=0.05em,
        markfields={a1, d2},
        pgfstyle=straightmove,
        markmove={a4-a5},
    ]

    \begin{itemize}
        \item{Where are the weaknesses?}
    
        c5 is weak.
        \item{Which is the worst-placed piece?}
       
        Not clear.
        \item{What is my opponent's idea?}
    
        a5 fixing the pawn structure.
    \end{itemize}

    Focus on the position only, \variation[invar]{14... a5} is the right move.
    Ironically, the opponent's plan \variation[invar]{15. a5} exists
    only since Black's last move. It is also psychologically difficult to play
    \variation[invar]{14... a5}, admitting \variation[invar]{13... a6} was a wrong
    plan. 

    \mainline[level=1]{14...  Nd7?} 
    
    \mainline{ 15. Nxd7 Qxd7} Natural moves!
    \mainline{16. a5!} We already know White's plan.
    
    
    \mainline{16... Qc7} 
    
    \chessboard

    White intends to install his knight on c5. A direct \variation[invar]{17. Nb3}
    would be answered with \variation[invar]{17... Nc6}. Black can
    also assault with \variation[invar]{18... Bc2}. White cannot improve his position.

    The queen on c7 is a defender of White's idea. So he tries to exchange it.
    Ding does this skillfully.

    \mainline[level=1]{ 17. Qf3 Rfc8 18. Ra3!} 
    
    \mainline[level=1]{18... Bg6 19. Nb3 Nc6} \mainline[level=1]{ 20. Qg3} 
    
    \chessboard

    Here Black could have defended patiently by playing \variation[invar]{20... Qxg3 21. hxg3 Bf5 22. Nc5 Rc7}.
    His position would be quite solid.
    
    However, Nepo chooses an active plan by countering with e5 in the center. His pieces
    are too passive for such a dynamic plan. White soon gets the upper hand.
    \mainline[level=1]{20... Qe7
    21. h4!}  A typical pawn move to remove back rank weakness in a better position
    before launching an attack.
    
    
    \mainline{ 21... Re8 22. Nc5 e5?} 
    Result of a wrong plan.
    
    \mainline[level=1]{23. Rb3 Nxa5 24. Rxe5 Qf6 25. Ra3 Nc4
    26. Bxc4 dxc4 } 
    
    \chessboard

    Black should have foreseen and estimated this position before he started 
    his plan of countering in the center on his 20th move. His moves 
    are forced while White may still have some improvements.

    Who has a better position? Obviously White. He has full control of 
    the center. Had Nepo estimated this position correctly, he would have 
    chosen a different plan. 
    
    \chessboard
    \mainline{27. h5?  } 
    
    \variation[invar]{27. Nxb7 Rxe5 28. dxe5 Qb6 29. Nd6 Qxb2 30. Nxc4 } would be winning.
    Note White should keep his strong knight to attack the Black king.

    Missed an opportunity!
    \variation{27... Rxe5 28. dxe5 Qd8 \xskakcomment{ This move is hard to see.} 29. Qf3 Qd2 30. hxg6 Qe1+ 31. Kh2 Qxe5+ 32. Kg1 Qe1+ 33. Kh2 Qe5+ 34. g3 \xskakcomment{ Otherwise perpetual check} Qxc5 35. Qxf7+ Kh8 }
    would have saved the game.
    \mainline[level=1]{ 27... Bc2? 28. Nxb7 Qb6 29. Nd6 Rxe5 30. Qxe5 Qxb2}

    \chessboard

    In this position, White can use process of elimination to find the best move.

    \variation[invar]{31. Nxc4?! Qc1+ 32. Kh2 Bd3 33. Qe3 Qd1 34. Ne5 Qxh5+ 35. Qh3 Qxh3+ 36. gxh3 }
    Black has a good chance to defend. 

    \chessboard [
        markstyle=circle,
        linewidth=0.05em,
        markfields={f7},
        markstyle=border,
        linewidth=0.05em,
        markfields={a3}
    ]

    \begin{itemize}
        \item{Where are the weaknesses?}
    
        f7 pawn.
        \item{Which is the worst-placed piece?}
       
        The rook on a3.
        \item{What is my opponent's idea?}
    
        Not clear
    \end{itemize}

    Black has a weakness on f7. \symrook a5-c5-c7 is natural to attack it.
    By the way, White's rook is attacked. There is no other way to parry the threat.

    \mainline[level=1]{ 31. Ra5! Kh7 32. Rc5 Qc1+ 33. Kh2 f6 34. Qg3 a5 35. Nxc4 a4
    36. Ne3 Bb1 37. Rc7 Rg8 38. Nd5 Kh8 39. Ra7 a3 40. Ne7 Rf8
    41. d5 a2 42. Qc7 Kh7 43. Ng6 Rg8 44. Qf7 \xskakcomment{ 1-0}}
    \chessboard
    \end{multicols}


\subsection*{Lessons Learned}
A very deep game played by Ding. He gains an advantage in the middlegame and 
waits patiently until Nepo takes immature measures.

The London system can become a Queen's Gambit Exchange Variation reversed.
This suggests the question of how to handle a Queen's Gambit Exchange Variation.

One example of an answer is that both sides have chances, both sides have their 
own prospects for attack and it is the player who has the greater knowledge and 
skill (and so who can, for example, employ resources that are less obvious than 
the resources which the opponent can employ) who will get the better of it.

When defending, one must choose between passive and active defense.
One can only use active defense when one has practical chances to defend.
Otherwise, an active defense is just suicide.
\section{S.Karjakin - Magnus Carlsen 2013}

\keywords{Ruy Lopez, Prophylaxis, Process of Elimination, Playability}

\begin{multicols}{2}
\newchessgame

\mainline[level=1]{1. e4 e5 2. Nf3 Nc6 3. Bb5 a6 4. Ba4 Nf6 5. O-O Be7 6. Re1 b5 7. Bb3 d6 8. c3 O-O
9. h3 Nb8 10. d4 Nbd7 11. Nbd2 Bb7} 

\chessboard

A typical idea in Ruy Lopez. Black intends to play \symrook e8, \symbishop f8, exd4 and 
then take the e4 pawn. White must protect his e4 pawn sooner or later.

\mainline[level=1]{ 12. Bc2 Re8 13. a4 Bf8} 

\chessboard

White usually maneuvers his d2 knight to f5 via f1, e3 or g3. At the moment
this maneuver is not possible because Black can play exd4 then take the e4 pawn.

\mainline{ 14. Bd3 c6 15. Qc2 Rc8
16. axb5 axb5 17. b4 Qc7 18. Bb2 Ra8 19. Rad1 Bb6 20. c4 bxc4 21. Nxc4 Nxc4 22. Bxc4 h6
23. dxe5 dxe5 24. Bc3 Ba6 25. Bb3 c5 26. Qb2 c4 27. Ba4 Re6 28. Nxe5 Bb7}

\chessboard[
    markstyle=circle,
    linewidth=0.05em,
    markfields={c4, e4, e5},
]

A critical position! The position is full of tactics so that
it is difficult to estimate statically. At the moment both kings
are safe. Piece balance and pawn structure will change dramatically.

White's bishop is being attacked. Black has a weakness
on c4. A natural idea is \variation[invar]{29. Bb5 Ba6 30. Ra1 Bb7 31. Rxa8 Bxa8 
32. Bxc4 Rxe5 33. Bxf7+ Qxf7 34. Bxe5 Nxe4 }.

After a more or less forced line we have the following position. 

\chessboard[
    setfen=b4bk1/5qp1/7p/4B3/1P2n3/7P/1Q3PP1/4R1K1 w - - 0 35
]

White has one rook and two pawns for two pieces. His king is safe. 
The position should be playable for him.

Alternatively, White moves his bishop to c2. He is forced to weaken his 
king safety to protect his weak e4 pawn. 

\chessboard[
    setfen=r4bk1/1bq2pp1/4rn1p/4N3/BPp1P3/2B4P/1Q3PP1/3RR1K1 w - - 1 29
]

\mainline[level=1]{29. Bc2 Rae8 
30. f4 Bd6} 

\chessboard

It was quite easy for Black to find the previous moves. 
White has weaknesses on e5 and e4. Black only needed to double his 
rooks on the e-file and use his pieces to target the e5 square.

However, it is difficult for White to find the right plan. He has 
already weakened his king safety with his kingside pawn movements. He will
also have to play g3 at some point to further weaken his king. 
From this development, we can conclude that White's 29th move was wrong. \variation[invar]{29. Bb5}
would have been better.

Looking forward, White still must find his next move.

\chessboard[
    setfen=4r1k1/1bq2pp1/3brn1p/4N3/1Pp1PP2/2B4P/1QB3P1/3RR1K1 w - - 1 31,
    markstyle=circle,
    linewidth=0.05em,
    markfields={e4,e5},
    pgfstyle=straightmove,
    linewidth=0.05em,
    markmove={f6-h5},
]

\begin{itemize}
    \item{Where are the weaknesses?}

    e4 and e5.
    \item{Which is the worst-placed piece?}
   
    Not clear.
    \item{What is my opponent's idea?}

    He wants to play \symknight h5 to attack the f4 pawn.
\end{itemize}

When Black plays \symknight h5, White must play g3. As prophylaxis,
it is a good idea to protect the g3 pawn first. There are two options
\variation[invar]{31. Kh2} and \variation[invar]{31. Re3}.

Karjakin should have eliminated \variation[invar]{31. Kh2} first. 
His king is on the same diagonal as the Black queen and bishop. 
\variation[invar]{31. Re3} is the right move.
\variation[invar]{31. Re3 Nh5 32. g3 g5} 
(\variation{32...f6 33. Nxc4 Bxf4 34. gxf4 Nxf4 35. Bb3 \xskakcomment{ Black king
is now exposed, White has advantage.}}) \variation {33. Ba4 R8e7 34. Qe2} White is fine.

\mainline[level=1]{ 31. Kh2? Nh5 32. g3 f6 33. Ng6 Nxf4}

\chessboard

\variation[invar]{34. gxf4 Bxf4+ 35. Kh1 Rxe4 36. Bxe4 Rxe4 37. Kg1 Be3+ 38. Rxe3 Rxe3 39. Qh2 Qxh2+ 40. Kxh2 Rxc3 41. Qh3}
Black is winning. Karjakin finds the only defense.

\mainline[level=1]{34. Rxd6 Nxg6 35. Rxe6 Rxe6 36. Bd4 f5
37. e5} 

\chessboard

White is still paying debts for weakening his own king safety voluntarily.
Black can attack along the main diagonal. Actually he only needs to make
natural moves and again White must come up with good defense.

\mainline[level=1]{37... Nxe5! 38. Bxe5 Qc6}

\chessboard

Here again, White must find the right defense. 

\variation[invar]{39. Be4 fxe4 40. Re3} is his last chance. After the text move, he
has no more chances.

\mainline[level=1]{39. Rg1? Qd5 40. Bxf5 Rxd5 41. Bg4 h5 42. Bd1 c3 43. Qf2 Rf5 44. Qe3 Qf7 45. g4 Re5 46. Qd4 Qc7 \xskakcomment{ 0-1}}

\end{multicols}

\subsection*{Lessons Learned}
Practically a player should choose to play playable positions.
We see again and again during the game, White struggles
to find the right move while Black only needs to make
logical moves. Such play is of course tiring for White and 
at some point, he makes decisive mistakes and then loses the game.

King safety is always the most important thing. It is unwise 
to weaken king safety.

Prophylaxis and Process of Eliminations are important
tools to find the right move without calculating too much. 
At move 31, White finds the right idea with prophylaxis.
However, he fails to choose the right move with Process of 
Elimination.



\chapter{Side Lines}
\section{Ding - Gukesh Game 12 Side Lines}

\begin{multicols}{2}
It is instructive and even entertaining to play against a computer, even in a tactical winning position. Lots of new ideas can be learned.
\subsection*{Position 0}

\newchessgame[
        setfen=1nrqr1k1/1pp2pp1/1n4bp/1N1P4/p1P1pB2/P1Q3PP/1P3PBK/3RR3 b - - 0 26,
	moveid=26b
        ]
\chessboard

Instead of \variation{26...Qd7}, the computer plays \mainline{26... Na6 27.d6 c6 28. Nc7 Nxc7 29. dxc7 Qe7}

\chessboard

\subsection*{Position 1}


\mainline{30. Rd4! c5 31. Rdxe4 Bxe4 32. Rxe4 Qd7}

It is amazing to see the whole game including this side line in Petrosian style. 
First we see a positional play, then we see an exchange sacrifice!

\subsection*{Position 2}
There are at least two alternatives, both leads to winning.

\chessboard

\mainline{33. Bxh6! gxh6 \xskakcomment{ Here some calculation is required.}} 

 \chessboard[
 	markstyle=circle,
 	linewidth=0.05em,
 	markfields={h6,b6},
        ]

Watch how a queen and a rook attack together and collect material. Black is defenceless. 

\mainline{34. Rg4 Kf8 35. Rg7 Ke7 36. Rf4 Kd6 37. Rxf7 Re7 38. Qxh6 Kxc7 39. Qf4 Kd8 40. Rf8 Re8 41. Qf6 Qe7 42. Qxb6 Qc7 43. Rxe8 Rxe8}

\chessboard

First White plays like Petrosian with patient maneuver and exchange sacrifice. Then he sacrifices a biship for an attack like Kasparov! Who would play like this? Anyway, we are only studying.

White has a clear material advantage. 

\subsection*{Position 3}

Instead of a sacrifice, we choose to win ``normally'' by exchanging the rooks from Position 2:

\newchessgame[
        setfen=4r1k1/1pPq1pp1/1n5p/2p5/p1P2B2/P1Q3PP/1P3PBK/8 w - - 0 34,
	moveid=34w,
        ]
 \chessboard[
 	markstyle=circle,
 	linewidth=0.05em,
 	markfields={f2},
        ]

\mainline{34. g4!}

It is important to keep in mind not to rush. White wants to play g4, \symbishop g3 to protect the pawns around his king.

\mainline{34... Nc6 35. Bg3 b6 36. Qf3 Rf8 37. Qd5 Qf8} 

 \chessboard[
 	markstyle=circle,
 	linewidth=0.05em,
 	markfields={h7},
        ]

Watch how the queen and the bishops attack the castled king to force an exchange favorable to White.

\mainline{38. Be4 Ne7 39. Qe5 Nc8}

 \chessboard[
 	markstyle=circle,
 	linewidth=0.05em,
 	markfields={h7},
        ]

\mainline{40. Qf5 g6 41. Qf6 Qe6 42. Qxe6 fxe6 43. Bxg6}

 \chessboard[
        ]

White has material advantage.

\end{multicols}
\section{Combinative Firework}

\epigraph{
    I do not believe that I have ever seen a position under analysis, 
    filled with such an enormous varity of combinative content. \cite{Dvoretsky:2008}
}{Mark Dvoretsky}

\begin{multicols}{2}
\newchessgame[
    setfen=1b1qnr2/4p1kp/pp1rQpp1/2p1N1B1/P4P2/2P5/2P3PP/3R1R1K w - - 0 1,
    moveid=1w
]

Lelchuk - Voronova, USSR 1983

\chessboard

White has a pawn down, and she has four pieces en prise. 
Playing normally is completely hopeless. So she starts an attack.
\mainline[level=1]{1. Bh6+ Kxh6 2. Qf7} 

\chessboard

A warm-up exercise. What must Black play? We can use Process of Elimination here.
White threatens to play \variation[invar]{3. Ng4+ Kh5 4. Qxh7+ Kxg4 5. h3}

To parry the threat, Black may play \variation[invar]{2... fxe5}, \variation[invar]{2... Rxf7} or \variation[invar]{2...Rh8}.
Further calculation eliminates the first two:

\variation[invar]{2... fxe5 3. Qxf8+ Kh5 4. g4+ Kxg4 5. Rg1 Kf3 6. fxe5+} White has huge material advantage.
\variation[invar]{2... Rxf7 3. Nxf7+ Kg7 4. Nxg8} White has the exchange up.
Therefore Black must play \variation[invar]{2... Rh8}.

\mainline[level=1]{2... Rh8 3. Ng4+ Kh5} 

\chessboard

How to continue the attack here? Black is threatening to take the 
rook on d1. Moving the knight to e3 to protect the rook and at 
the same time making room on g4 for her pawn is natural.

\mainline{ 4. Ne3 } 

\chessboard

Again, Black must decide how to defend here. The candidate moves 
are \variation[invar]{4... f5}, \variation[invar]{4... e5} and \variation[invar]{4... Rxd1}. 

Let's check \variation[invar]{4... f5} first. 

Black wants to trap White's queen, intending to play \variation[invar]{5. Rf6}.
White can however answer with \variation[invar]{5. Nxf5! Rxd1 6. g4 Kxg4 7. Ne3 \xskakcomment{We know this maneuver already!} Kh5
8. Rxd1 Qc8 9. Qd5+ e5 10. Qg2 Kh6 11. Qg5+ Kg7 12. Qe7+ Kh6 13. Rd7 \xskakcomment{White is crushing}}

There is no direct line refuting \variation[invar]{4... e5}.

\variation[invar]{4...e5 5. fxe5 Rxd1 6. Rxd1 Qc8 7. Rf1 Kh6 8. exf6 Be5} 

\chessboard[setfen=2q1n2r/5Q1p/pp3Ppk/2p1b3/P7/2P1N3/2P3PP/5R1K w - - 1 9]

\variation[invar]{9. h3! \xskakcomment{Such a quiet move is not easy to predict on move 4 when Black
chooses how to defend.}  Qc7 10. Ng4+ Kg5 11. Qd5 Qd6 12. Qxe5+ Qxe5 13. Nxe5 } White is winning.

Now we have only \variation[invar]{4... Rxd1} left.
\mainline{4...Rxd1 }
    
    \chessboard

    White can force a draw here: \variation{5. g4+ Kh6 6. Nf5+ gxf5 7. g5+ fxg5 8. fxg5+ Kxg5 9. Qxf5+ Kh6 10. Qh3+ Kg5 11. Qf5+ Kh6 12. Qh3+ Kg5 13. Qf5+ Kh6 
    }. It is often advantageous to find these forced draws while attacking as an emergency plan.

    \mainline[level=1]{5. Rxd1} 
    
    \chessboard

    Choosing the right move is still not easy here. Understanding White has \symknight g4 as resource,
    Black can use her queen to defend the square and go to an unbalanced endgame:
    \variation[invar]{5... Qc8 6. Rd5+ Kh6 7. h3 Nc7 8. Ng4+ Qxg4 9. hxg4 Nxd5 10. g5 fxg5 11. fxg5+ Kxg5 12. Qxd5})
    
    \chessboard[setfen=1b5r/4p2p/pp4p1/2pQ2k1/P7/2P5/2P3P1/7K b - - 0 12]

    The material is roughly equal. White has an active queen. She will 
    take another pawn but her pawn structure is inferior. Maybe she has some small advantage.

    We go back to the mainline. In the game Voronova decided to develop her knight.
    \mainline[level=1]{5... Nd6 6. Rd5+} 
    
    \chessboard

    We arrive at the deciding moment of the game. How should Black defend using Process of Elimination?

    Candidate moves are \variation[invar]{6...e5} and \variation[invar]{6...f5}. The latter loses 
    immediately after \variation[invar]{6... f5 7. g4+ Kh6 8. Rxd6 exd6 9. Nxf5+  gxf5 10. g5+}.
    
    In the game, Black Voronova was unable to bear the tension, played \variation[invar]{6...f5} and lost
    quickly. What would happen if she played \variation[invar]{6...e5}?

    \chessboard

    \mainline[level=1]{6... e5} 
    
    \chessboard

    How should White continue the attack?
    
    \mainline{ 7. Qg7} While chasing a wandering king, it is important
    to cut off his retreat. 
    
    \chessboard

    What is White's threat here?

    White also threatens \variation[invar]{8. g4 Kh4 9. Qh6#}. The defense is natural.

    \mainline[level=1]{7...  Qf8 8. g4+ Kh4} 
    
    \chessboard

    Black wants to exchange the queens and then cash out his material advantage.

    White must move her queen. d7 is a good square.

    \mainline[level=1]{9. Qd7}

    \chessboard

    What is White's threat here?

    White threatens to play \variation[invar]{10. Ng2+ Kh3 11. Rd3#}. To defend the 
    threat, Black can play \variation[invar]{9... e4} or \variation[invar]{9... f5}
    
    \mainline{ 9... e4} 
    
    \chessboard[
        setfen=1b3q1r/3Q3p/pp1n1pp1/2pR4/P3pPPk/2P1N3/2P4P/7K w - - 0 10
    ]
    
    \mainline{ 10. Qe6!!}

    This move has a very deep idea.    
    
    \begin{enumerate}
        \item{\variation[invar]{10... c4}}
     
        \chessboard[
            setfen=1b3q1r/7p/pp1nQpp1/3R4/P1p1pPPk/2P1N3/2P4P/7K w - - 0 11
        ]

        White intends to cut off the king completely and mate with a \symqueen d5-d1 maneuver.

        \variation[invar]{11. Rh5+! gxh5 12. Ng2+ Kh3 13. g5+ f5 14. Qd5 h4 15. Qd1 Rg8 16. Ne3 Rxg5 17. Qf1+ Rg2 18. Qxg2# }
        
        \chessboard[setfen=1b3q2/7p/pp1n4/5p2/P1p1pP1p/2P1N2k/2P3QP/7K b - - 0 18]
    
        \item{\variation[invar]{10... f5}}

        \variation[invar]{
            10...f5 11. Rd3!}
            
            \chessboard[setfen=1b3q1r/7p/pp1nQ1p1/2p2p2/P3pPPk/2PRN3/2P4P/7K b - - 1 11]

            Black is doomed.

            \begin{enumerate}
                \item {\variation[invar]{11...Nc4}}
                
                \variation{11...Nc4 12. Ng2+ Kxg4 13. Rg3+ Kh5 14. Rh3+ Kg4 15. Qxc4 Bxf4 16. Qf1 Be5 17. Nf4 Kg5 18. Ne6+ Kf6 }
                \item {\variation[invar]{11...exd3}}

                \variation{11... exd3 12. Ng2+ Kxg4 13. Qe1 Qe7 14. Qg3+ Kh5 15. Qh3+ Qh4 16. Qxh4# }
            \end{enumerate}
    \end{enumerate}

\end{multicols}


%----------------------------------------------------------------------------------------

\stopcontents[part] % Manually stop the 'part' table of contents here so the previous Part page table of contents doesn't list the following chapters

%----------------------------------------------------------------------------------------
%	BIBLIOGRAPHY
%----------------------------------------------------------------------------------------

\chapterimage{} % Chapter heading image
\chapterspaceabove{2.5cm} % Whitespace from the top of the page to the chapter title on chapter pages
\chapterspacebelow{2cm} % Amount of vertical whitespace from the top margin to the start of the text on chapter pages

%------------------------------------------------

\chapter*{Bibliography}
\markboth{\sffamily\normalsize\bfseries Bibliography}{\sffamily\normalsize\bfseries Bibliography} % Set the page headers to display a Bibliography chapter name
\addcontentsline{toc}{chapter}{\textcolor{ocre}{Bibliography}} % Add a Bibliography heading to the table of contents

%\section*{Articles}
%\addcontentsline{toc}{section}{Articles} % Add the Articles subheading to the table of contents

\printbibliography[heading=bibempty, type=article] % Output article bibliography entries

\section*{Books}
\addcontentsline{toc}{section}{Books} % Add the Books subheading to the table of contents

\printbibliography[heading=bibempty, type=book] % Output book bibliography entries

%----------------------------------------------------------------------------------------
%	INDEX
%----------------------------------------------------------------------------------------

\cleardoublepage % Make sure the index starts on an odd (right side) page
\phantomsection
\addcontentsline{toc}{chapter}{\textcolor{ocre}{Index}} % Add an Index heading to the table of contents
\printindex % Output the index

%----------------------------------------------------------------------------------------
%	APPENDICES
%----------------------------------------------------------------------------------------

\chapterimage{orange2.jpg} % Chapter heading image
\chapterspaceabove{6.75cm} % Whitespace from the top of the page to the chapter title on chapter pages
\chapterspacebelow{7.25cm} % Amount of vertical whitespace from the top margin to the start of the text on chapter pages

\begin{appendices}

\renewcommand{\chaptername}{Appendix} % Change the chapter name to Appendix, i.e. "Appendix A: Title", instead of "Chapter A: Title" in the headers

%------------------------------------------------


%------------------------------------------------

\end{appendices}

%----------------------------------------------------------------------------------------

\end{document}
