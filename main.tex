%%%%%%%%%%%%%%%%%%%%%%%%%%%%%%%%%%%%%%%%%
% The Legrand Orange Book
% LaTeX Template
% Version 3.1 (February 18, 2022)
%
% This template originates from:
% https://www.LaTeXTemplates.com
%
% Authors:
% Vel (vel@latextemplates.com)
% Mathias Legrand (legrand.mathias@gmail.com)
%
% License:
% CC BY-NC-SA 4.0 (https://creativecommons.org/licenses/by-nc-sa/4.0/)
%
% Compiling this template:
% This template uses biber for its bibliography and makeindex for its index.
% When you first open the template, compile it from the command line with the 
% commands below to make sure your LaTeX distribution is configured correctly:
%
% 1) pdflatex main
% 2) makeindex main.idx -s indexstyle.ist
% 3) biber main
% 4) pdflatex main x 2
%
% After this, when you wish to update the bibliography/index use the appropriate
% command above and make sure to compile with pdflatex several times 
% afterwards to propagate your changes to the document.
%
%%%%%%%%%%%%%%%%%%%%%%%%%%%%%%%%%%%%%%%%%

%----------------------------------------------------------------------------------------
%	PACKAGES AND OTHER DOCUMENT CONFIGURATIONS
%----------------------------------------------------------------------------------------

\documentclass[
	11pt, % Default font size, select one of 10pt, 11pt or 12pt
	fleqn, % Left align equations
	a4paper, % Paper size, use either 'a4paper' for A4 size or 'letterpaper' for US letter size
	%oneside, % Uncomment for oneside mode, this doesn't start new chapters and parts on odd pages (adding an empty page if required), this mode is more suitable if the book is to be read on a screen instead of printed
]{LegrandOrangeBook}

% Book information for PDF metadata, remove/comment this block if not required 
\hypersetup{
	pdftitle={Finding Inner Peace Through Chess, A PracticalGuide}, % Title field
	pdfauthor={Ming Lu}, % Author field
	pdfsubject={Peace}, % Subject field
	pdfkeywords={chess, meditation, myths...}, % Keywords
	pdfcreator={LaTeX}, % Content creator field
}

\addbibresource{sample.bib} % Bibliography file

\definecolor{ocre}{RGB}{243, 102, 25} % Define the color used for highlighting throughout the book

\chapterimage{zen-stone.jpg} % Chapter heading image
\chapterspaceabove{6.5cm} % Default whitespace from the top of the page to the chapter title on chapter pages
\chapterspacebelow{6.75cm} % Default amount of vertical whitespace from the top margin to the start of the text on chapter pages

\usepackage{chessboard}
\usepackage{skak}
\usepackage{sectsty}
\usepackage{tocloft}

%----------------------------------------------------------------------------------------

\begin{document}

%----------------------------------------------------------------------------------------
%	TITLE PAGE
%----------------------------------------------------------------------------------------

\titlepage % Output the title page
	{\includegraphics[width=\paperwidth]{background.pdf}} % Code to output the background image, which should be the same dimensions as the paper to fill the page entirely; leave empty for no background image
	{ % Title(s) and author(s)
		\centering\sffamily % Font styling
		{\Huge\bfseries Finding Inner Peace Through Chess\par} % Book title
		\vspace{16pt} % Vertical whitespace
		{\LARGE A Practical Guide\par} % Subtitle
		\vspace{24pt} % Vertical whitespace
		{\huge\bfseries Ming Lu\par} % Author name
	}

%----------------------------------------------------------------------------------------
%	COPYRIGHT PAGE
%----------------------------------------------------------------------------------------

\thispagestyle{empty} % Suppress headers and footers on this page

~\vfill % Push the text down to the bottom of the page

\noindent Copyright \copyright\ 2022 Goro Akechi\\ % Copyright notice

\noindent \textsc{Published by Publisher}\\ % Publisher

\noindent \textsc{\href{https://www.latextemplates.com/template/legrand-orange-book}{book-website.com}}\\ % URL


%----------------------------------------------------------------------------------------
%	TABLE OF CONTENTS
%----------------------------------------------------------------------------------------

\pagestyle{empty} % Disable headers and footers for the following pages


\newcommand{\Color}[1]{\hypersetup{linkcolor=#1}\color{#1}}
% Optionally change font style
\renewcommand{\cftchapfont}{\Large\bfseries\Color{ocre}} 
\renewcommand{\cftsecfont}{\large}  % Bold section font
\renewcommand{\cftsubsecfont}{\large\itshape} % Italic subsection font
\renewcommand{\cftsecpagefont}{\large}   % Page numbers for sections
\renewcommand{\cftsubsecpagefont}{\large} % Page numbers for subsections

\begingroup % start a TeX group
\color{ocre}% or whatever color you wish to use
\tableofcontents % Output the table of contents
\listoffigures % Output the list of figures, comment or remove this command if not required
\listoftables % Output the list of tables, comment or remove this command if not required
\endgroup


\pagestyle{fancy} % Enable default headers and footers again

\cleardoublepage % Start the following content on a new page

% Define section and subsection colors
\chapterfont{\color{ocre}}
\sectionfont{\color{ocre}}
\subsectionfont{\color{ocre}}

%----------------------------------------------------------------------------------------
%	Main Content
%----------------------------------------------------------------------------------------
%\part{Life}
\chapter{Chess History}
\section{Diplomacy with Chess}
\begin{figure}[H] % Use [H] to suppress floating and place the figure/table exactly where it is specified in the text
	\centering % Horizontally center the figure on the page
	\includegraphics[width=0.5\textwidth]{cards/history/franklin_play_howe.jpg} % Include the figure image
	\caption{Quiet Tension and Concentration}
	\label{fig:franklin} % Unique label used for referencing the figure in-text
\end{figure}
It began with a game of chess. In December 1774 Benjamin Franklin met Caroline Howe at the Royal Society in London. She challenged him to a game, which turned into a series of chess matches over several days.

On Christmas Day, she introduced the American to her brother, Lord Richard Howe, who told Franklin that some members of the British government ``were extremely well disposed to any reasonable accommodation'' between the British and the American colonists. The two men continued to use the chess matches as a front for a series of secret meetings to negotiate a piece. Nothing concrete resulted form the meetings, which ended in March 1775, but the Franklin and Howe concluded their talks with a mutual respect for one another.

The following year the admiral was appointed commander of the British navy in North America. Historian Walter Isaacson summarizes what happened just after the Continental Congress issued its Declaration of Independence. [Admiral Howe] carried a detailed proposal that offered a truce, pardons for the rebel leaders (with John Adams secretly exempted) and rewards for any American who helped restore peace. 

Because the British did not recognize the Continental Congress as a legitimate body, Lord Howe was unsure where to direct his proposals. So when he reached Sandy Hook, New Jersey, he sent a letter to Franklin, whom he addressed as ``my worthy friend.'' He had ``hopes of being serviceable'' Howe declared, ``in promoting the establishment of lasting peace and union with the colonies''.

Congress granted Franklin permission to reply, which he did on July 30. Franklin opened the following letter with a cordial statement, but the tone quickly sours as he rejected the offer with fury: ``it is impossible we should think of submission to a government'' whose ``atrocious injuries have extinguished avery remaining Spark of Affection for that Parent Coiuntry we once held so dear.''

\section{Steinitz's Tragedy}
Wilhelm Steinitz was more than just a great chess player-he wa a revolutionary. Born in Prague in 1836, Steinitz transformed chess from a romantic game of dashing attacks into a scientific discipline based on positional principles. In 1886, he became the first official World Chess Champion, a title he defended successfully for years. His theories about chess strategy formed the foundation of modern chess thinking, and for over a quarter-century, he was considered virtually invincible in match play. 

But the weight of maintaining supremacy in chess, a game that demands absolute mental precision and psychological fortitude, would eventually exact a terrible price.

In 1894, at age 58, Steinitz faced a young challenger named Emanuel Lasker in a match for the world championship. What followed was a psychological catastrophe that revealed the devastating toll competitive chess can take on human mind. 

Game 7 marked the start of five consecutive losses to Lasker. This was an unprecedented humiliation for a man who had been unbeaten in match play for over 25 years and had previously declared he would win without doubt.

After this crushing five-game streak, Steinitz asked for a week's rest-a telling admission of his shattered mental state. In his own words after one of these losses, he wrote: ``Mr. Lasker then broke into my game in the most woful manner and won a Pawn, blocking my pieces, and he had things almost all his own way.'' Steinitz attributed his collapse to poor physical condition, particularly his disability which prevented him from walking and exercising properly, cause ``insomnia, rushing of blood to the head, and general depression.''

He had lost his crown, but worse was yet to come.

\subsection*{The 1896-97 ematch: Complete Mewntal Disintegration}
Desperate to reclaim his title and prove his theories were still sound, Steinitz challenged Lasker to a rematch. The rematch was even more devastating. Steinitz's performance at the Nuremberg tournament before the match was subpar, finishing sixth place, which predetermined his bad result to an extent. In the rematch held in Moscow from November 1896 to January 1897, Steinitz won only 2 games, drawing 5 and losing 10-a crushing defeat.

The aftermath was tragic: Just four weeks after the match Steinitz lost his mind and had to seek psychiatric help. Shortly after the match, he had a mental breakdown and was confined for 40 days in Moscow sanatorium, where he played chess with the inmates. There, in his confinement, he would play chess against other patients, eventually even making claims that he was playiung chess games with God himself-believing he could give God pawn and move odds and still win.

His mental state never fully recovered. Two years later, he reportedly experienced delusions while returning by ship from the London 1899 tournament. Commitment to a series of mental hospitals followed beginning in Febrary 1900, and he died in the state mental hospital on Ward's Island on August 12, 1900. The death certificate listed ``chronic endocardia (mitral stenosis)'' and ``acute melancholia'' as causes of death. The first World Chess Champion died nearly penniless, his brilliant mind shattered by the very game he had mastered.

\section{Boris Spassky's Nirvana}
\subsection{Nirvana}
Nirvana is not an escape from life, but a transformation within it. It is the moment when the fire of desire and fear burns itself out, leaving behind a clear, still awareness. In Buddhist thought, this state is beyond joy and sorrow - it is freedom from the compulsions that make us cling to success and recoil from failure. What dies in Nirvana is not the self, but the illusion that life must always go our way.

Transformation toward Nirvana does not happen through achievement, but through exhaustion of striving. When a person suffers deeply and yet remains present to that suffering, something inside begins to shift. The mind sees that every victory fades and every defeat passes, and that both belong to the same cycle of attachment. In that seeing, a kind of quiet surrender arises - not resignation, but acceptance. 

This transformation is like a sword tempered by fire. The ego, once raw and restless, becomes refined through the heat of experience. Out of this refinement comes equanimity - a mind unshaken by fortune or loss. Nirvana, then, is not a distant heaven, but a change of vision; the realization that freedom lies not in controlling life, but in letting go of the need to control it.

\subsection{How Spassky became World Champion}
In the 1958 USSR Championship, Boris Spassky was leading the tournament at the beginning and needed only a draw in the last round against Mikhail Tal. After a dramatic struggle, he lost - from a winning position - and with it, his chance to qualify for the Interzonal, the gateway to the World Championship.

Three years later in 1961, Spassky was again leading the championship, undefeated. In his critical game against Lev Polugaevsky, he launched a brilliant sacrifice that could have secured victory, but he missed the win and eventually lost. 


The defeat broke his momentum; he collapsed in the remaining rounds and once more failed to reach Interzonal.

Commentators said Spassky would never become World Champion—that he was too emotional, too fragile at decisive moments.

In 1966, Spassky became the final challenger and lost to Petrosian. In 1969, he became the challenger again and this time he beat Petrosian to become World Champion.

\section{1984 Karpov - Kasparov World Championship}
The 1984 Karpov-Kasparov World Championship in Moscow was not just a chess match - it was a myth unfolding on a 64-square battlefield. It followed, almost eerily, Joseph Campbell's hero's journey: the call, the descent, the ordeal and the miraculous return

\subsection*{The call to Adventure}
Garry Kasparov was twenty-one, the youngest challenger in history. Across from him sat Anatoly Karpov - the reigning world champion, embodiment of Soviet control, precision, and restraint.

Karpov had inherited Fischer's vacant crown a decade earlier and rulesd chess with a surgeon's calmness. Kasparov, by contrast, was fire-creative, defiant, unorthodox. 

The Soviet establishment saw the match as a test of loyalty and order; the yong challenger symbolized chaos, energy and change. The gods had arranged their duel. 

\subsection*{Crossing the Threshold}
The first games were a diaster. Karpov struck like a cold wind: clinical, silent, inevitable. 

After nine games, Kasparov was down 4-0. In the mythic sense, this was the descent into the underwold- the young hero's illusions burned away.

Observers whispered that the match might end 6-0 within days.

Karpov seemed invincible, his eyes as expresisonless as stone. Kasparov stood on the edge of annihilation. Yet, instead of collapsing, he chose a new path-patience. He stopped fighting on Karpov's terms and began drawing, consolidating, waiting. It was not glory; it was survival. The hero refused to die.

\subsection*{The Road of Trials}
From game 10 onward, something shifted. Karpov pressed again and again - 17 draws followed- but the young challenger did not fall. He built a fortress of iron will. The Moscow crowd grew restless: the ``boy from Baku'' was not leaving. For months the match dragged on - the longest in history.

Kasparov trained his mind like a monk in the desert, meditating under pressure. Karpov, meanwhile, began to fade. His face grew pale, his hands trembled. The once unbreakable champion was aging before the audience's eyes.

This was the ordeal: two minds locked in endless struggle, endurance replacing brilliance.

\subsection*{The Abyss and the Miracle}
After five months, Karpov led 5-0. He needed only one more win to end the saga. But the last step proved impossible. Karsparov began to rise - cautiously, methodically. In game 32, he won his first victory. The audience gasped; life had returned to the hero. 

Then came another, and another. The impossible had begun to happen. The young hero was crawling out of the pit. Karpov's energy broke; his will cracked. His apotheosis was near - the moment of transcendence through suffering.

\subsection*{The Return without Victory}
Then came the twist: after 48 games, the match was adruptly stopped by the Soviet chess federation, officially ``for health reasons''. No winner was declared. Karpov retained his title; Kasparov kept his sould. In Campbell's terms, this was the refusal of the gods to grant closure - the hero returns not with a crown, but with knowledge. Kasparov had been through death and rebirth. He now understood the cost of greatness.

A year later, when they met again in 1985, he completed the journey - won the title, and began his own mythic reign.

\subsection*{Epilogue}
Karpov was order, Kasparov chaos. Their 1984 clash was not about pieces, but about two visions of intelligence: one shaped by control, the other by rebellion.

And in the ruins of that endless match, the audience witnessed something larger than chess- the moment when the new hero emerged from the ashes, unbroken, carrying the flame forward.

\section{Anony of a Winning Position}
\begin{figure}[H] % Use [H] to suppress floating and place the figure/table exactly where it is specified in the text
	\centering % Horizontally center the figure on the page
	\includegraphics[width=0.5\textwidth]{cards/history/ivanchuk_cry.jpg} % Include the figure image
	\caption{Heartbreak after a Game}
	\label{fig:placeholder} % Unique label used for referencing the figure in-text
\end{figure}


The clock was merciless. Three minutes for the entire game, two seconds added after each move-barely enough time to think, only enough to survive.

Round 11 of the 2024 World Blity Championship in New York. The stakes couldn't be higher. Eight players tied for first place on 9 points. Only eight could advance to the knockout stage. Vassyl Ivanchuk, the 55-year-old Ukrainian legend-a man who had danced with chass brilliance for decades-sat across from Daniel Naroditsky, fighting for his tournament life.

Naroditsky launched a sacrificial attack, pieces flying across the board in a blur of aggression. The pressure was suffocating. But Ivanchuk, drawing on a lifetime of tactical genius, defended brilliantly. He navigated the chaos, parried the threats, and emerged from the storm with a winning position. The board told a clear story-victory was his to claim. 

Then came move 40. Ivanchuk made an automatic move. The advantage that had been crystal clear moments ago began to evaporate like morning mist. The position had flipped entirely. What had been victory was now resignation.

Ivanchuk's hand moved to tip his king over. The gesture felt like it took forever-a king falling in slow motion, carrying with it decades of passion, frustration, and desperate love for this impossible game.

And then the tears came. The veteran grandmaster slumped over his board, shoulders shaking, tears streaming down his face. Now quiet dignified tears, but raw uncontrollable sobbing. The kind of grief that comes from watching victory slip through your fingers when you needed it most. 


\chapter{Chess and Life}
\section{Life as Chess Game}
Life, like chess, unfolds on a board of choices. Each move shapes the next, each mistake demands recovery, and every player faces the same truth - there is no rewinding the clock. Strategy, foresight, and patience are not luxuries; they are the means to survival. Consider Napoleon at Waterloo. For years he had played the board of Europe with unmathced brilliance, predicting opponents' response serveral moves ahead. But at Waterloo, he misjudged both weather and timing - much like a grandmaster who overestimates his position. The rain delayed his artillery, and his opponent, Wellington, played defensively, waiting for the right counter. Napoleon's final gambit, a desperate assault, collapsed like a miscalculated endgame - proof that even a master's plan fails when overconfidence blinds calculation. 

Even in personal history, the pattern repeats. When Marie Curie chose to pursue science agianst societal expectations, she opened with a bold gambit - sacrificing comfort for knowledge. Her discoveries reshaped the scientific board forever.

In chess and in life, the truth is the same: mastery is not about control, but about awareness. The wise player accepts uncertainty, plans without attachment and sees each move not as an end, but as part of a living strategy. Victory comes not from knowing the future, but from being present for the move that is now. 
\section{Present Moment}
When life rises and crashes like the sea, the only thing that saves you is focus - not on fortune's whim, but on this moment, the piece of deck beneath your feet. 

Focus is the hero's rudder. When Odysseus sailed past the Sirens, he did not control the sea; he controlled himself. The hero never stops the storm - he survives by keeping his hands on the rope and his eyes on the next move.

In mythic terms, every ``now'' is a test from the gods. To live fully in the present is not to deny fate; it's to 
dance with it. When the wave lifts, you ride. When it falls, you bend. The world will always rise and fall - but the one who stays awake in the middle of motion becomes like the calm eye of a storm: untouched, aware, alive. 

That's why focus matters. Because the present moment is the only place where the hero can steer.

\section{Meditation}
Meditations is an old as humanity's first awareness of its own thoughts. It arose not from religion, but from the simple realization that the minditslef can be observed. Long before temples or doctrines, humans noticed that pain, fear and desire were not fixed realities - they were movements inside consciousness. 

In India, meditaitons grew from the early Vedic seers who sought stillness beyond retual. They discoverd that by watching breath and thought, a deeper awareness emerged - something constant beneath the noise.

The Buddha later turned this into method: sit, breathe, observe. See that all things - sensations, thoughts, even ``I'' - arise and pass away. From China to Greece, similar insights appeard: Taoist spoke of wu wei - effortless awareness. Greek Stoics trined the mind to stay within what can be controlled - the present act. Christian mystics and Sufi dervishes, in their own language, spoke of silence, rememberance, presence before devine. Different forms, same discovery: the way out of suffering is through awareness.

Meditations begins when you stop chasing the next thought and simply see. Normally, the mind drifts between past and future - replaying wounds, anticipating gain. This movement creates anxiety, pride, regret. Meditation cuts the circuit: you return to now. When attention anchors in the present, time loosens. The noise of self fades. There is no ``me trying to win'' or ``me who failed''. There is just breathing, sound, sensation - the raw fabric of existence.

This state is not escape: it is contact. You finally meet reality without distortion. Connection to the Present Moment Meditation is training for the same clarity the hero needs amid chaos.


In life, fortune shifts, emotions surge. Without awareness, we react blindly - anger, fear, craving. Meditaiton strengthens the ability to notice first, act later. It builds an inner stillness that doesn't depend on circumstances. To live meditatively means to live in the only time that truly exists: the present.

The past is memoryu, the future imagination - both flicker in the mind. Only the present is real, and meditation is how you stand in it, fully awake. 

Religion made meditation sacred. Philosophy made it rational. Psychology made it practical. But its truth remains timeless - the quiet power to see, breathe and steer your life with open eyes. \index{Meditation}
Meditation and chess share the same mental discipline - the art of being fully present. In meditation, the goal is to center the mind on the current moment- not to dwell on past mistakes, nor to fear what is yet to come. In chess, this presence becomes a weapon. Every move demands complete awareness: calculation, intuition, and calm under pressure. A player who clings to a lost opportunity or worries about the next ten moves loses clarity - just as a distracted miditator loses balance. Both practices train the same muscle: the ability to return to now, again and again. Through meditation, the chess player sharpens focus, steadies emotions and finds stillness in chaos - the same stillness from which the best moves arises.

\section{O Fortuna}
\input{cards/life/o_fortuna}
\input{cards/life/o_fortuna_history}

\section{Stress} \index{Stress}
Stress rarely lives in the present moment. It comes from worrying about what might happen and regretting what already has. The mind prejects itself forward, imagining losses, failures or unfinished duties - or it driftsbackward, replaying mistakes and missed chances. In both directions, it fights against what cannot be chontrolled. When attention leaves the present, tension arises: muscles tighten, breathing shortens, thoughts loop. The body reacts as if danger were real, even when it exists only in imagination.

Peace returns the moment awareness anchors in now. The present contains effort but not anxiety, action but not regret. Letting go of past and future isn't passivity - it's clarity: doing what can be done here and now, and releasing everything else.

\section{Hamlet}
Hamlet's tragedy is overthinking - he worries about everything: the truth of the ghost, the morality of revenge, the fear of damnation, even the meaning of life itself. He delays action because every possible outcome troubles him. His famous line ``thus conscience does make cowards of us all'' is pure worry turned philosophical. 

In this image, Hamlet stands alone beneath a sky that seems to twist with his thoughts. His face is drawn tight, his eyes hollow with sleeplessness, and his hand grips his forehead as if trying to still the storm within. The skull he hold is not only a symbol of death but of his endless reflection - a mirror for the mind that cannot rest. Around him, the world bends and swirls, as though reality itslef shares his unrest. Every color hums with unease; every line trembles. It is not action that torments him, but the weight of imagining - the fear of what may come, the regret of what has passed, the paralysis of a mind that cannot stop worrying.


\chapter{Chess and Thinking}

\section{Chess and Computer}
Chess, a game celebrated for its complexity and deep strategic nature, has found a formidable opponent in the realm of computers. The integration of computers into chess has revolutionized the way the game is played, analyzed, and understood. From enthusiast players seeking to improve their skills to grandmasters strategizing for championships, computer technology has become an indispensable tool in chess.


The development of chess-playing software involves a blend of algorithms and heuristic techniques. At its core, the tasks undertaken by these programs revolve around evaluating positions and predicting the most advantageous moves. Some of the primary techniques used include:

\begin{itemize}
	\item{Minimax Algorithm}

	This foundational approach in game theory entails that the computer evaluates all possible moves and their outcomes. The computer seeks to minimize the maximum loss (hence “minimax”) while considering potential responses from its opponent, thereby making the most strategic decisions.
	\item{Alpha-Beta Pruning}

	This optimization technique improves the efficiency of the minimax algorithm by eliminating branches in the move tree that won’t be chosen, thus reducing the number of positions to evaluate. This allows the program to search deeper in the same amount of time.
	\item{Evaluation Function}

	These are algorithms designed to assign a score to a given position based on various factors such as material count, piece mobility, control of the center, and king safety. The evaluation functions use heuristics developed from the vast knowledge of chess principles learned from both historical games and databases.
	\item{Machine Learning}

	In recent years, advanced machine learning techniques, particularly deep learning, have been introduced. Programs like AlphaZero utilize neural networks trained through self-play to discover novel strategies and approaches to the game that transcend traditional evaluations.
	\item{Databases and Opening Books}

	Top chess engines are often equipped with large databases of historical games and opening theory, enabling them to play with extensive opening knowledge and adapt to various styles of play.

\end{itemize}

While chess engines are incredibly powerful and capable of executing millions of calculations per second, their style of play differs significantly from that of human players. Computers approach the game with cold precision, calculating variations far beyond human capacity, often perceiving the game through a lens of probabilities rather than intuition. They focus on calculating concrete lines to determine the best possible move rather than relying on the generalized strategies or psychological aspects that humans frequently employ.

Humans, on the other hand, utilize a combination of experience, psychological insight, and instinctive judgment. They often derive strategies from patterns and familiar positions, relying on their emotional understanding of the game and their opponents. This contrast underscores the fact that while computers can play chess at an exceedingly high level, they do so with a fundamentally different methodology than human beings. For this reason, humans do not need to emulate computer strategies or techniques; instead, they should embrace the unique strengths and insights that a human mind brings to the ancient game of chess.

In conclusion, the interplay between computers and chess highlights not only the advancements in technology but also the intrinsic beauty and complexity of the game itself. As computers continue to evolve and enhance our understanding of chess, players are encouraged to appreciate both the analytical prowess of machines and the irreplaceable qualities that define human gameplay.

\section{Perception and Decision Stages}
Adrian de Groot was one of the first psychologists to study how chess masters actually think when they play-not how strong they are, but what happens in their mind when they look at a position.

In the 1940s, he asked players of different strengths-from amateurs to world champions- to think out loud while solving chess problems. He carefully recorded their words and reasoning step by step. The plays were told find the best move but not to move the pieces; they just had to explain what they were seeing and thinking. De Groot expected grandmasters to analyze many more moves and variations than weaker players. But that's not what he found. Everyone - masters and amateurs- looked about the same number of moves ahead, typically 3-5. The big difference wasn't in how far they calculated but in what they saw. Masters immediately recognized key features of a position almost at a glance. They didn't start from zero; they started from structure.

He described their process as having four stages:
Orientation: taking in the position and forming a first impression

Exploration: testing a few promising ideas. 

Investigation: calculating deeper in one or two lines.

Proof: checking the correctness and making a final choice.

De Groot's main conclusion: chess expertise is not mainly about brute-calculation, but about pattern recognition and structured thinking. Masters have internalized thousands of meaningful configuration, so they instantly focus on the right ideas- a finding that later inspired Simon and Chase's ``chunking'' theory.

In short, de Groot showed that genius in chess doesn't come from seeing further-it comes from seeing better. \index{Chess Perception}

\section{Chunk}
Herbert Simon, along with William Chase, studied chess expertise in the 1970s. They found that chess masters don't just momorize isolated moves - they momorize patterns, or ``chunks'' of pieces and positions. These chunks allow them to recognize familiar structures quickly, reducing the cognitive load when thinking about moves. 

For example, a master mighjt see a certain pawn structure and instantly recall a typical plan associated with it, instread of calculating every possible move from screatch.

Chunks encode meaningful patterns: masters can store and retrieve thousands of these patterns in long term memory. Expert memory is domain-specific. A master's memory advantage disappears when dealing with random chess positions because chunks are meaningful patterns, not random arrangements.

Chunks accelerate decision-making: They allow quick recognition of threats, opportunities, and strategic ideas. 

Life is full of complex situations that can be overwhelming if we try to analyze every detail individually. Chunking lets us organize information into meaningful patterns, so we can respond effectively. 

For example, an experienced doctor doesn't analyze every symptom in isilation-they recognize patterns from prior cases. A skilled negotiator sees the structure os a discussion and anticipates moves.

By consciously building ``chunks'' of experience-through deliberate practice and reflection- we can make faster, better decisions and reduce cognitive overload in complex, real-world tasks. \index{Chunks}
\subsection{Daedalus and the Labyrinth}
In the myth, Daedalus is a skilled craftsman who designed the Labyrinth to contain the Minotaur. The Labyrinth is complex and convoluted, representing a challenge of navigation and understanding. To find a way out, one needs to develop a strategy to avoid getting lost among its many pathways.

The Labyrinth can be seen as a representation of information complexity. When faced with a vast amount of information or a difficult problem (like navigating the Labyrinth), individuals can easily feel overwhelmed.

Daedalus, as the master architect, symbolizes the human ability to solve complex problems. He creates a structure that initially seems confusing, but it is designed to be navigable with some clever thinking.
Chunking as a Strategy:

In order to navigate through the Labyrinth successfully, one cannot remember all the paths and turns as individual, standalone locations. Instead, one would benefit from grouping or 'chunking' the pathways into segments or sections. By memorizing key junctures or landmarks within the Labyrinth, an individual can reduce the cognitive load and simplify the navigation process.
Efficiency:

Just as Daedalus provided a thread to Theseus to follow back through the Labyrinth, chunking provides a way to remember complex information more efficiently. By organizing bits of information into manageable "chunks," a person can recall and use them effectively without becoming lost in a maze of details.
Learning and Application:

Similar to how Theseus learns to navigate the Labyrinth with the help of Daedalus's design, chunking enables learners to build connections and organize knowledge systematically. This method facilitates not just better recall, but also deeper understanding, as relationships between the chunks can be recognized and leveraged for problem-solving.

\section{Satisficing}
Herbert Simon coined the term ``satisficing'' to describe a decision-making strategy where a person chooses an option that is good enough, rather than the absolute best. 

Instead of exhaustively searching for the optimal solution-which may be impossible or too costly-people settle for a solution that meets their criteria of acceptability.

Example: Choosing a restaurant that looks decent instead of analyzing every possible place in town to find the theoretically ``perfect'' one.

\subsection*{Why is it important}
In real life, we rarely have the time, information or computational power to find the perfect solution. Satisficing allows us to act effectively under constraints. Reduces stress and indecision: Trying to optimize everything leads to analysis paralysis. Accepting ``good enough'' decisions frees mental resources for other tasks. 

Simon showed that humans are boundedly rational-we make decisions under limitations, and satisficing is a rational strategy within those limits.

In short, satisficing is a powerful principle for dealing with complexity: it balances quality and efficiency, letting us move forward without getting stuck chasing perfection. \index{Satisficing}
\section{Heuristic} \index{Heuristics}
Herbert Simon introduced the concept of heuristics as mental shortcuts or rules of thumb that help people make decisions efficiently under uncertainty, tather than exhaustively analyzing every possibility. 

A heuristic is a simple strategy that guides problem-solving and decision-making. It doesn't guarantee the perfect solution but often gives a satisfactory one quickly.

\subsection*{Connection to Computer Science}
In earch algorithms like A*, heuristics estimate the ``distance'' to a goal, helping the algorithm focus on promising paths instead of exploring every possibility.

Life is too complex to calculate every possible outcome. Heuristics help us make practical decisions with limited time and information.

Gut feeling is a real-world manifestation of heuristics. It's your brain recognizing patterns and past experience subconsciously to point toward a reasonable choice. 

Simon's idea shows that humans are ``boundedly rational''. We don't need perfect knowledge; we just need heuristics to navigate complexity efficiently.

\subsection{Heuristic in Life} \index{Heuristics!Life}
Here are three vivid examples where gut feeling acts like a heuristic in life:

\begin{itemize}
	\item{Emergency Decision}

	You smell smoke and see a small flame. You don't calculate every possible exit route or estimate fire spread precisely. Your gut tells you ``Get out now!''. Heuristic: Recognize danger patterns from past experience or instinct-act quickly without full analysis.
	\item{Business Choice}

	You meet a potential business partner. Something about their tone, confidence or past behavior signals dishonesty. You decide to reject even without checking every reference. Heuristic: pattern recognition from prior interactions guides a rejection efficiently.
	\item{Medical Intuition}

	A seasoned doctor sees a patient with subtle signs of illness. Even if lab results are pending, they have a gut feeling about the diagnosis and start treatment. Heuristic: Experience forms chunks and mental shortcuts that guide immediate action.


\end{itemize}


\subsection{Heuristic in Chess} \index{Heuristics!Chess}
Herbert Simon's concept of heuristics is very clear in the way chess masters think. Rather than calculating every possible move, masters recognize familiar patterns, or ``chunks'' on the board. These chunks-weak squares, pawn structures or typical mating configurations-allow them to respond quickly. Their thinking is guided by heuristics: rule of thumb derived from experience. For example, seeing an isolated pawn on e6 might immediately suggest targeting it with rooks and the queen. This recognition is fast and almost automatic and it saves enormous mental effort compared to exhaustively calculating every possibility.

Chess masters also demonstrate satisficing behavior. They rarely search for the single best move; instead, they often choose a move that is sufficiently strong to maintain or increase their advantage. A simple heuristic like ``any move that improves piece activity or king safety is likely sufficient'' prevents the paralysi that comes from trying to optimize every decision. This approach shows that effective decision-making does not require perfection, only a practical solution that achieves the goal.

Even in evaluating the opponent's intentions, heuristics play a critical role. Masters can sense the opponent's plans and threats without analyzing every potential variation. A gut feeling might signal ``the opponent wants to attack my kingside; I need to defend here.'' based on prior experience and pattern recognition. This fast, intuitive judgment allows them to act effectively under time pressure and complixity, mirroring the same principles Simon observed in human decision-making more broadly.

%\part{Let's play some Chess}
\chapter{Chess Play}
\section{Simple Chess} \index{Simple Chess}
We finally get down to business! Most chess books on the market are filled with endless variations. Yet for any human being, it is impossible to calculate all those lines, visualize every position, and then evaluate them accurately. Use such analysis as a standard only leads to frustration and stress.

A far more practical approach is to play from the present position, relying on one's own understanding - just as in life, where decisions are made with limited foresight. This method is more effective because it is human: it respects the natural limits of thought and emphasizes clarity over complexity.

World champions such as Capablanca, Petrosian and Karpov all advocated this style. In their writings, they valued intuition, simplicity and sound judgment over endless calculation. Their success shows that chess, like life is best played one clear move at a time. 
\section{Thinking Template} \index{Three Questions}

This chapter presents an array of positional chess problems designed to enhance your strategic thinking. Each problem can be approached using the thinking template outlined earlier. The aim is to cultivate efficient problem-solving skills while minimizing oversight in variations. As you work through each challenge, take the time to jot down your thoughts. This practice will allow you to identify any gaps in your reasoning by comparing your solutions with the provided answers.
Chess can quickly become overwhelming if you try to calculate every variation. Endless lines drain energy, cloud judgment, and increase mistakes. A simpler, structured approach keeps your mind clear and your decisions effective. Focus on three key questions \cite{Aagaard:2012}:
\begin{itemize}
	\item{Where are the weaknesses?}

	Look for vulnerable squares, weak pawns or exposed king positions. Identifying weaknesses gives direction and targets for your plan.
	\item{Which is the worst-placed piece?}

	Evaluate both your and your opponent's pieces. Misplaced pieces often limit mobility and coordination. Exploiting or improving these positions can turn the game in your favor without complex calculation. 
	\item{Whas is my opponent's idea?}

	Try to understand your opponent's plan. Knowing their intentions helps you defend efficiently and counterattack strategically.
\end{itemize}
This template works because it emphasizes pattern recognition and strategic understanding over brute-force calculation. Instead of burning energy on countless variations that rarely occur in full, you make practical decisions baswed on the position's reality. World champions like Capablanca, Petrosian and Karpov relied on this approach: they played the position, not the tree of possibilities.

By asking these three questions each turn, you conserve mental energy, reduce errors, and develop a style rooted in clarity, control and real understanding. Chess becomes more about thinking smartly. 
\subsection{Visualization the Three Questions}
We use a simple graphical system to illustrate the three questions as used by Aagaard in his book ``Grandmaster Preparations Positional Play''. 

We will use circle to identify weakness. We will use square to identify the worst-placed pieces. We will use arrows to illustrate the opponent's ideas.


Let us look at an example 

\subsection*{Hikaru Nakamura - Vladimir Kramnik}

 \chessboard[
        setfen=3r2k1/p4pbp/b3p3/npqpP1pP/8/2P3P1/P3QPBN/R3R1K1 w - - 0 1,
 	markstyle=circle,
 	linewidth=0.05em,
 	markfields={g5, f6},
	markstyle=border,
	markfields={g2},
	pgfstyle=straightmove,
 	markmove=h7-h6
        ]

Black has weakness on g5 and f6. All his pieces are bad. White's worst piece is the bishop on g2.

Black intends to play h6 to protect his biggest weakness: g5 pawn. To prevent this move White can play h6 himself and then \symqueen h5 attack the weak pawn on g5.






\section{Excercises}
\nocite{Dvoretsky:2016}
\ifdefined\chessproblem
    % The command is already defined, do nothing.
\else
    \newcommand{\chessproblem}[3]{
        \subsection*{#1. #2}
        \chessboard[
            setfen=#3,
        ]
    }
\fi
\ifdefined\conquestNunn
    % The command is already defined, do nothing.
\else
    \newcommand{\conquestNunn}[1]{
	\chessproblem{#1}{Conquest - Nunn}{5rk1/pp3ppp/5b2/2p1pb2/3q4/2NP2P1/PPP3KP/R2QR3 b - - 0 1}

    }
\fi

\ifdefined\conquestNunnAnswer
    % The command is already defined, do nothing.
\else
    \newcommand{\conquestNunnAnswer}[1]{
		\subsection*{#1. Conquest - Nunn}
 \chessboard[
        setfen=5rk1/pp3ppp/5b2/2p1pb2/3q4/2NP2P1/PPP3KP/R2QR3 b - - 0 1,
 	markstyle=circle,
 	linewidth=0.05em,
 	markfields={g2},
	markstyle=border,
 	markfields={f5}
        ]
	\begin{itemize}
		\item{Where are the weaknesses?}

		White king as well as his main diagonal is weak.
		\item{Which is the worst-placed piece?}

		The light-squared bishop does nothing.
		\item{What is my opponent's idea?}

		He can only wait passively.
	\end{itemize}

	By answering the template question above, the correct move is \symbishop d7, improving the piece as well as exploiting White's weakness on the main diagonal.
    }
\fi





\ifdefined\fischerKeres
    % The command is already defined, do nothing.
\else
    \newcommand{\fischerKeres}[1]{
	\chessproblem{#1}{Fischer - Keres}{r5k1/1pq2ppp/2rb1n2/4n2P/p2pPB2/P2P2PB/R1P1Q3/1N3RK1 b - - 0 1}
    }
\fi

\ifdefined\fischerKeresAnswer
    % The command is already defined, do nothing.
\else
    \newcommand{\fischerKeresAnswer}[1]{
	\subsection*{#1. Fischer - Keres}
 \chessboard[
        setfen=r5k1/1pq2ppp/2rb1n2/4n2P/p2pPB2/P2P2PB/R1P1Q3/1N3RK1 b - - 0 1,
 	markstyle=circle,
 	linewidth=0.1em,
 	markfields={c2},
	markstyle=border,
 	markfields={a8}
        ]

	\begin{itemize}
		\item{Where are the weaknesses?}

		White has weakness on c2.
		\item{Which is the worst-placed piece?}

		The rook on a8 does nothing
		\item{Whas is my opponent's idea?}

		White has no active plan so black can choose his own maneuver.
	\end{itemize}

	\symrook c8 is not possible since white's bishop controls c8. \symrook a5 is natural because the 
	rook targets c5 aiming the weakness c2 in the next move.

	Serendipitously, the move also aims h5. White must figure out how to defend. At 
	the moment we don't need to concern how white will defend. Let's pass the ball to 
	white and decide then what to do.
    }
\fi





\ifdefined\petrosianBannik
    % The command is already defined, do nothing.
\else
    \newcommand{\petrosianBannik}[1]{
	\chessproblem{#1}{Petrosian - Bannik}{3r3r/ppk1b2p/1np2p2/4p1pP/2P1N3/1P2B1P1/P3PP2/2KR3R w - - 0 1}
    }
\fi

\ifdefined\petrosianBannikAnswer
    % The command is already defined, do nothing.
\else
    \newcommand{\petrosianBannikAnswer}[1]{
	\subsection*{#1. Petrosian - Bannik}
 \chessboard[
        setfen=3r3r/ppk1b2p/1np2p2/4p1pP/2P1N3/1P2B1P1/P3PP2/2KR3R w - - 0 1,
 	markstyle=circle,
 	linewidth=0.05em,
 	markfields={f6,e6},
	markstyle=border,
 	linewidth=0.05em,
 	markfields={e3},
        ]
	\begin{itemize}
		\item{Where are the weaknesses?}

		Black has weaknesses on e6 and f6.
		\item{Which is the worst-placed piece?}

		The bishop on e3 is inactive.
		\item{What is my opponent's idea?}

		Black has no active plan, so White can choose his own maneuver.
	\end{itemize}
	The best move is \symbishop c5 to exchange the inactive bishop and exploit Black's weakness on f6 by moving the king to e6. Let's see what Petrosian has to say about this position. More than the move, it's his comments that made a deep impression.
	\begin{quote}
		``In deciding on this move, it was imperative to weigh all the pros and cons thoroughly. The move looks illogical as White is voluntarily exchanging his good bishop for his opponent's bad one, instead of swapping the bishop for knight and securing his preponderance. However, if you probe into the position a little more deeply, it becomes obvious that after a possible exchange of rooks on the d-file and the transfer of king to e6, Black would cover his vulnerable points and create an impregnable formation. The role played in this by the bad bishop would be of no small importance. After 1.\symbishop c5 \symrook xd1 2.\symrook xd1 \symbishop xc5 3.\symknight xc5 White was threatening infiltration on e6 and after 3...\symrook e8 4.\symknight e4 \symrook e6 5.g4 He was clearly better as the f6 pawn is very weak.'' \cite{Petrosian:2015}
	\end{quote}
    }
\fi

\conquestNunn{1}
\fischerKeres{2}
\petrosianBannik{3}


\section{Answers}
\conquestNunnAnswer{1}
\fischerKeresAnswer{2}
\petrosianBannikAnswer{3}

%----------------------------------------------------------------------------------------

\stopcontents[part] % Manually stop the 'part' table of contents here so the previous Part page table of contents doesn't list the following chapters

%----------------------------------------------------------------------------------------
%	BIBLIOGRAPHY
%----------------------------------------------------------------------------------------

\chapterimage{} % Chapter heading image
\chapterspaceabove{2.5cm} % Whitespace from the top of the page to the chapter title on chapter pages
\chapterspacebelow{2cm} % Amount of vertical whitespace from the top margin to the start of the text on chapter pages

%------------------------------------------------

\chapter*{Bibliography}
\markboth{\sffamily\normalsize\bfseries Bibliography}{\sffamily\normalsize\bfseries Bibliography} % Set the page headers to display a Bibliography chapter name
\addcontentsline{toc}{chapter}{\textcolor{ocre}{Bibliography}} % Add a Bibliography heading to the table of contents

\section*{Articles}
\addcontentsline{toc}{section}{Articles} % Add the Articles subheading to the table of contents

\printbibliography[heading=bibempty, type=article] % Output article bibliography entries

\section*{Books}
\addcontentsline{toc}{section}{Books} % Add the Books subheading to the table of contents

\printbibliography[heading=bibempty, type=book] % Output book bibliography entries

%----------------------------------------------------------------------------------------
%	INDEX
%----------------------------------------------------------------------------------------

\cleardoublepage % Make sure the index starts on an odd (right side) page
\phantomsection
\addcontentsline{toc}{chapter}{\textcolor{ocre}{Index}} % Add an Index heading to the table of contents
\printindex % Output the index

%----------------------------------------------------------------------------------------
%	APPENDICES
%----------------------------------------------------------------------------------------

\chapterimage{orange2.jpg} % Chapter heading image
\chapterspaceabove{6.75cm} % Whitespace from the top of the page to the chapter title on chapter pages
\chapterspacebelow{7.25cm} % Amount of vertical whitespace from the top margin to the start of the text on chapter pages

\begin{appendices}

\renewcommand{\chaptername}{Appendix} % Change the chapter name to Appendix, i.e. "Appendix A: Title", instead of "Chapter A: Title" in the headers

%------------------------------------------------


%------------------------------------------------

\end{appendices}

%----------------------------------------------------------------------------------------

\end{document}
