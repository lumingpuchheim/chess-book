It is an illusion that chess masters carry a fully formed plan in their minds and then simply maneuver their pieces until the game is won.

David Bronstein once remarked:
``From Tarrasch was born the impression, which exists to this day, of so-called consistent games, in which one of the players carries out, from beginning to end, a logical plan like the demonstration of a geometric theorem. I think that such games between equal opponents do not happen, and commentators—who are often the winners—pass off the desirable for the actual.''

It is unlikely that real chess thinking works this way. Even when one player is much stronger, the weaker side still influences the course of the game, sometimes by making mistakes that simplify the task, sometimes by resisting in unexpected ways. A winning plan cannot be fixed in advance; it is a moving target shaped by the opponent's choices.

More often, a game unfolds as a sequence of local objectives. A player pursues a concrete idea until it succeeds or is neutralized. Then the position changes, a new situation arises, and both players reassess, form new ideas, and continue. Chess is not the execution of a single grand plan, but an ongoing process of adaptation.