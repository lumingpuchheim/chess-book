Herbert Simon introduced the concept of heuristics as mental shortcuts or rules of thumb that help people make decisions efficiently under uncertainty, tather than exhaustively analyzing every possibility. 

A heuristic is a simple strategy that guides problem-solving and decision-making. It doesn't guarantee the perfect solution but often gives a satisfactory one quickly.

\subsection*{Connection to Computer Science}
In earch algorithms like A*, heuristics estimate the ``distance'' to a goal, helping the algorithm focus on promising paths instead of exploring every possibility.

Life is too complex to calculate every possible outcome. Heuristics help us make practical decisions with limited time and information.

Gut feeling is a real-world manifestation of heuristics. It's your brain recognizing patterns and past experience subconsciously to point toward a reasonable choice. 

Simon's idea shows that humans are ``boundedly rational''. We don't need perfect knowledge; we just need heuristics to navigate complexity efficiently.