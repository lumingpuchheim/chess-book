The song O Fortuna thunders with the voice of fate itself. Written in the 13th century as part of the edieval manuscript Carmina Burana and later set to music by Carl Orff in 1935, it begins with a cry against Fortune - a wheel that raises some to glory and casts others into ruin. Its words speak of the turning of life's wheel: ``Fortune, like the moon, ever changing, waxing and waning.'' The monks who wrote it saw life as a game played under the shadow of chance - where strength, virtue, and intellect could all be undone by one twist of fate.

Orff's music captures that merciless rhythm. The pounding drums, the rising chorus, the sudden silence - all mirror how power builds and collapses. One moment triumphant, the next defeated. It is not a song of despair, but of awareness - a reminder that no one commands the board forever.

The same truth governs chess. A game may begin with perfect logic, yet one careless move can reverse fortune entirely. A pawn once ignored becomes a queen; a king once safe stands suddently exposed. 

Life, too, is this wheel. One can play skillfully, calculate deeply, yet must bow to the element of chance - the weather, the timing, the unseen move of another mind. To play well is not to control fortune, but to dance with it - as Orff's chorus rises and falls, as the chessboard shifts, as life itself spins onward beneath the hand of Fortuna.