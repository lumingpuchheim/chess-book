In the myth, Daedalus is a skilled craftsman who designed the Labyrinth to contain the Minotaur. The Labyrinth is complex and convoluted, representing a challenge of navigation and understanding. To find a way out, one needs to develop a strategy to avoid getting lost among its many pathways.

The Labyrinth can be seen as a representation of information complexity. When faced with a vast amount of information or a difficult problem (like navigating the Labyrinth), individuals can easily feel overwhelmed.

Daedalus, as the master architect, symbolizes the human ability to solve complex problems. He creates a structure that initially seems confusing, but it is designed to be navigable with some clever thinking.
Chunking as a Strategy:

In order to navigate through the Labyrinth successfully, one cannot remember all the paths and turns as individual, standalone locations. Instead, one would benefit from grouping or 'chunking' the pathways into segments or sections. By memorizing key junctures or landmarks within the Labyrinth, an individual can reduce the cognitive load and simplify the navigation process.
Efficiency:

Just as Daedalus provided a thread to Theseus to follow back through the Labyrinth, chunking provides a way to remember complex information more efficiently. By organizing bits of information into manageable "chunks," a person can recall and use them effectively without becoming lost in a maze of details.
Learning and Application:

Similar to how Theseus learns to navigate the Labyrinth with the help of Daedalus's design, chunking enables learners to build connections and organize knowledge systematically. This method facilitates not just better recall, but also deeper understanding, as relationships between the chunks can be recognized and leveraged for problem-solving.