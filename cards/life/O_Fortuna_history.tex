``O Fortuna'' is a medieval Latin Goliardic poem written in the 13th century of uncertain authorship. 
It is a complaint against the goddess of fortune, contained in the collection known as the Carmina Burana. Scott Horton wrote in Harper's that the text of the poem highlights how few people, at the time it was written, ``felt any control over their own destiny'' while at the same time it ``rings with a passion for life, a demand to seize and treasure the sweet moments that pitiful human existence affords''.