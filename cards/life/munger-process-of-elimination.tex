In chess, strong players often use the \vocab{PoE}{Process of Elimination}{}: instead of trying to find the best move directly, they first rule out the clearly inferior ones. Moves that lose material for no reason, weaken the king, or violate basic principles are discarded. What remains is a much smaller set of candidate moves that deserve serious calculation. By systematically asking, “What can I safely throw away?”, the player frees mental energy to focus on the few moves that really matter.

Charlie Munger uses the same method in life and investing. Rather than starting with the question “What should I do?”, he often begins with “What must I avoid?”. He looks for obvious mistakes: businesses with dishonest management, fragile balance sheets, or economics he cannot understand. By eliminating these “bad moves” first, he narrows the field to a handful of sensible opportunities. This negative approach is not pessimism; it is disciplined filtering.

Just as a chess player eliminates blunders before analyzing subtle options, Munger eliminates stupidity before seeking brilliance. In both domains, the gains come not from genius, but from consistently avoiding disasters. A grandmaster may not always find the absolute best move, but he almost never plays a losing one. Likewise, Munger’s success comes less from predicting the future than from refusing to do obviously dumb things.

You can apply this mindset in your own decisions. When facing a difficult choice, start by listing everything that would clearly make the situation worse: lying, taking on unmanageable risk, ignoring important constraints, or acting in anger. Eliminate those options first. What remains will rarely be perfect, but it will be safer and more robust—much like choosing between several reasonable moves in a complex chess position. Over time, the habit of systematic elimination turns chaos into structure, both on the board and in life.


