In life, we constantly face situations we have never seen before. From the outside they may look simple, even ``easy''—a project that should be straightforward, a relationship that seems obvious, a decision that others handle without visible effort. Only from the inside do we feel the pressure, the uncertainty, and the limits of our current strength. It is tempting, afterwards, to ask: ``How could I fail at something like this?'' and to judge ourselves more harshly than we would ever judge a friend.

Wilhelm Steinitz, the first World Chess Champion, offers a powerful example. For decades he was almost unbeatable in match play, a pioneer who transformed chess into a scientific game. Yet late in life, facing Emanuel Lasker, he suddenly found himself in a new kind of battle. His opponent's style belonged more to the 20th century than to Steinitz's own romantic era—a style that would later be refined by players like Tal, decades before Tal was even born. On the board, this sometimes looked ``simple'': positions where he was two pawns up, positions that commentators would later say ``should be won''. From the outside, it is easy to ask: how could such a great player lose this?

From the inside, it was different. The positions were anything but easy. Even Garry Kasparov, analyzing these games with the help of computers, failed to find Steinitz's decisive mistakes. The complexity was hidden beneath the surface: what appeared to be a winning advantage was actually a minefield of subtle tactical and strategic challenges. Steinitz was facing positions and ideas that had never been seen before in his era. The ``mistakes'' he made were not signs of stupidity, but signs of a human being navigating genuinely new and unforgiving territory.

We often do the same to ourselves. We treat every failure as proof that we ``should have known better'', especially when the problem looks easy in hindsight. But like Steinitz, we are always facing new situations. The person facing today's challenge is encountering circumstances that may never have been seen before in exactly this form. Judging ourselves as if we should have perfect knowledge of situations we have never encountered is both unrealistic and unfair.

Accepting oneself means recognising this humanity. When you look back at a painful loss—on the board or in life—try to see the full position, not just the final blunder. What pressures were you under? What was new or unfamiliar? What aspects of the situation had you never encountered before? Instead of asking ``How could I lose this?'', ask ``Given who I was and what I knew then, was it understandable that I struggled?''

This does not mean ignoring responsibility or refusing to improve. Steinitz's legacy is not only his tragic decline, but also his willingness to rethink chess from the ground up. We can still learn, adjust, and grow. But we do so more wisely when we start from compassion rather than contempt. Just as we would never call Steinitz a fool for losing positions that even Kasparov with computers could not fully understand, we should not call ourselves fools for failing in situations that were, in truth, new, complex, and genuinely difficult.

In chess, it is easy to say ``two pawns up—this should be winning''. In life, it is easy to say ``this should have been simple''. In both cases, the real story is more complex. Accepting yourself means allowing that complexity to exist, and granting yourself the same understanding you would gladly offer to Steinitz—or to anyone else you admire.


