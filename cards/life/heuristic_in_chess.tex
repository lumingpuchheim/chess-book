Herbert Simon's concept of heuristics is very clear in the way chess masters think. Rather than calculating every possible move, masters recognize familiar patterns, or ``chunks'' on the board. These chunks-weak squares, pawn structures or typical mating configurations-allow them to respond quickly. Their thinking is guided by heuristics: rule of thumb derived from experience. For example, seeing an isolated pawn on e6 might immediately suggest targeting it with rooks and the queen. This recognition is fast and almost automatic and it saves enormous mental effort compared to exhaustively calculating every possibility.

Chess masters also demonstrate satisficing behavior. They rarely search for the single best move; instead, they often choose a move that is sufficiently strong to maintain or increase their advantage. A simple heuristic like ``any move that improves piece activity or king safety is likely sufficient'' prevents the paralysi that comes from trying to optimize every decision. This approach shows that effective decision-making does not require perfection, only a practical solution that achieves the goal.

Even in evaluating the opponent's intentions, heuristics play a critical role. Masters can sense the opponent's plans and threats without analyzing every potential variation. A gut feeling might signal ``the opponent wants to attack my kingside; I need to defend here.'' based on prior experience and pattern recognition. This fast, intuitive judgment allows them to act effectively under time pressure and complixity, mirroring the same principles Simon observed in human decision-making more broadly.