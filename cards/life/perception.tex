Adrian de Groot was one of the first psychologists to study how chess masters actually think when they play-not how strong they are, but what happens in their mind when they look at a position.

In the 1940s, he asked players of different strengths-from amateurs to world champions- to think out loud while solving chess problems. He carefully recorded their words and reasoning step by step. The plays were told find the best move but not to move the pieces; they just had to explain what they were seeing and thinking. De Groot expected grandmasters to analyze many more moves and variations than weaker players. But that's not what he found. Everyone - masters and amateurs- looked about the same number of moves ahead, typically 3-5. The big difference wasn't in how far they calculated but in what they saw. Masters immediately recognized key features of a position almost at a glance. They didn't start from zero; they started from structure.

He described their process as having four stages:
Orientation: taking in the position and forming a first impression

Exploration: testing a few promising ideas. 

Investigation: calculating deeper in one or two lines.

Proof: checking the correctness and making a final choice.

De Groot's main conclusion: chess expertise is not mainly about brute-calculation, but about pattern recognition and structured thinking. Masters have internalized thousands of meaningful configuration, so they instantly focus on the right ideas- a finding that later inspired Simon and Chase's ``chunking'' theory.

In short, de Groot showed that genius in chess doesn't come from seeing further-it comes from seeing better.