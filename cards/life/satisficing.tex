Herbert Simon coined the term ``satisficing'' to describe a decision-making strategy where a person chooses an option that is good enough, rather than the absolute best. 

Instead of exhaustively searching for the optimal solution-which may be impossible or too costly-people settle for a solution that meets their criteria of acceptability.

Example: Choosing a restaurant that looks decent instead of analyzing every possible place in town to find the theoretically ``perfect'' one.

\subsection*{Why is it important}
In real life, we rarely have the time, information or computational power to find the perfect solution. Satisficing allows us to act effectively under constraints. Reduces stress and indecision: Trying to optimize everything leads to analysis paralysis. Accepting ``good enough'' decisions frees mental resources for other tasks. 

Simon showed that humans are boundedly rational-we make decisions under limitations, and satisficing is a rational strategy within those limits.

In short, satisficing is a powerful principle for dealing with complexity: it balances quality and efficiency, letting us move forward without getting stuck chasing perfection.