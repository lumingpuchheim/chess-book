\ifdefined\sanakoevLungdal
    % The command is already defined, do nothing.
\else
    \newcommand{\sanakoevLungdal}[1]{
	\chessproblem{#1}{Sanakoev - Lungdal}{2r1k2r/1b3ppp/p3p3/2qpP3/1p1Q1P2/2P5/PP2B1PP/R2R2K1 w - - 0 1}
    }
\fi

\ifdefined\sanakoevLungdalAnswer
    % The command is already defined, do nothing.
\else
    \newcommand{\sanakoevLungdalAnswer}[1]{
		\subsection*{#1. Sanakoev - Lungdal}
        \newchessgame[
            setfen=2r1k2r/1b3ppp/p3p3/2qpP3/1p1Q1P2/2P5/PP2B1PP/R2R2K1 w - - 0 1,
            moveid=1w
        ]
		\chessboard[		
			markstyle=circle,
			linewidth=0.05em,
			markfields={b7},
			markstyle=border,
			linewidth=0.05em,
			markfields={a1},
		]

		\begin{itemize}
			\item{Where are the weaknesses?}

			The bishop on b7 is unprotected.
			\item{Which is the worst-placed piece?}

			The rook on a1 is inactive. 
			\item{Whas is my opponent's idea?}

			Black may want to castle kingside and exchange the queens.
		\end{itemize}

		\symrook ab1 is the right move.

        According to Dvoretsky, the move deserves two exclamation marks. We can
        however find the answer with logic and simple calculation.

        White is leading in the development so he should have the advantage.
        The queen is pinned so there is no way to start an attack with a queen as usual when one leads in development.

        \variation{1. cxb4 Qxd4 2. Rxd4 Rc2 3. Bd3 Rxb2 4. Rc1 Kd7 5. Rc2 Kd7 6. Rc2 Rxc2 7. Rxc2} Black
        simplifies the position and equalizes. 

        White fails to protect the pawn on b2 after taking the pawn on b4. So the answer is simple:
        \mainline[level=1]{1. Rab1} threatening cxb4. 
        
        \mainline{1... Qxd4 2. Rxd4 bxc3 3. bxc3 } White wins a tempo.
        
        \mainline{3... Rc7 4.Rdb4} White has the only open file and has the advantage. Black still needs to castle. His bishop is hopeless.        
        
        \chessboard
    }
\fi