\ifdefined\laskerSteinitz
    % The command is already defined, do nothing.
\else
    \newcommand{\laskerSteinitz}[1]{
	\chessproblem{#1}{Lasker - Steinitz}{4r2n/1p2qpkP/p1p3p1/3p4/8/3Br3/PPPQ4/2K2R1R w - - 1 29}
    }
\fi

\ifdefined\laskerSteinitzAnswer
    % The command is already defined, do nothing.
\else
    \newcommand{\laskerSteinitzAnswer}[1]{
	\subsection*{#1. Lasker - Steinitz}
 	\chessboard[
        setfen=4r2n/1p2qpkP/p1p3p1/3p4/8/3Br3/PPPQ4/2K2R1R w - - 1 29,
    ]

	\begin{itemize}
		\item{Where are the weaknesses?}

		White has potential back-rank weakness before starting an attack.

		\item{What is my opponent's idea?}

		He wants to activate his knight at the corner.
	\end{itemize}

    \variation{29. a3} is the right move. 

    White intends to play both \symking b1 and a3, but the order matters critically. 
    Calculating the optimal sequence through brute force is practically impossible here. 
    However, by comparing the consequences of each move, one can deduce that a3 must be played first, 
    because \symking b1 creates a potential back-rank weakness that restricts White's options.

    }
