\ifdefined\carlsenDing
    % The command is already defined, do nothing.
\else
    \newcommand{\carlsenDing}[1]{
	\chessproblem{#1}{Carlsen - Ding}{2kr2r1/pp1b1pp1/1qn1pn1p/3p4/1P6/B1PB1N2/P3QPPP/R4RK1 b - - 0 15}
    }
\fi

\ifdefined\carlsenDingAnswer
    % The command is already defined, do nothing.
\else
    \newcommand{\carlsenDingAnswer}[1]{
	\subsection*{#1. Carlsen - Ding}
    \newchessgame[
        setfen=2kr2r1/pp1b1pp1/1qn1pn1p/3p4/1P6/B1PB1N2/P3QPPP/R4RK1 b - - 0 15,
        moveid=15b
    ]
 	\chessboard[     
        pgfstyle=straightmove,
        linewidth=0.05em,
        markmove={b4-b5},
        markstyle=circle,
        linewidth=0.05em,
        markfields={e5},
        markstyle=border,
        linewidth=0.05em,
        markfields={d7},
        ]

	\begin{itemize}
		\item{Where are the weaknesses?}

		e5 square will be weak after White plays b5 driving away the knight on c6.
		\item{Which is the worst-placed piece?}

		The bishop on d7 is inactive. 
		\item{What is my opponent's idea?}

		b5.
	\end{itemize}

	White threatens to play b5 to drive away the knight on c6 and then control the e5 square with his knight.
Once White achieves his goal, Black has a desperate position. Therefore he must react now!

Understanding White's idea, Black can play \variation[invar]{15... e5 16. b5 e4 17. bxc6 Qxc6 18. Ne5 Qc7 
19. Nxf7 Bg4 20. Qe3 Qxf7 21. Bc2} 

    In the actual game, Ding chose to play \mainline[level=1]{15... Kb8? 16. b5 \xskakcomment{ Of course!}} White has an
    advantage and wins the game in a few moves.
	
    }
\fi


