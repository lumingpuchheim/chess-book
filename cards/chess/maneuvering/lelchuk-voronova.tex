\ifdefined\lelchukVoronova
    % The command is already defined, do nothing.
\else
    \newcommand{\lelchukVoronova}[1]{
	\chessproblem{#1}{Lelchuk - Voronova Analysis}{1b3q1r/3Q3p/pp1n1pp1/2pR4/P3pPPk/2P1N3/2P4P/7K w - - 0 10}
    }
\fi


\ifdefined\lelchukVoronovaAnswer
    % The command is already defined, do nothing.
\else
    \newcommand{\lelchukVoronovaAnswer}[1]{
		\subsection*{#1. Lelchuk - Voronova Analysis}
		\chessboard[
			setfen=1b3q1r/3Q3p/pp1n1pp1/2pR4/P3pPPk/2P1N3/2P4P/7K w - - 0 10,
			markstyle=circle,
			linewidth=0.05em,
			markfields={h4},
			markstyle=border,
			linewidth=0.05em,
			markfields={d7},
		]

		\begin{itemize}
			\item{Where are the weaknesses?}

			Black king on h4 is completely cut off.
			\item{Which is the worst-placed piece?}

			The White queen on d7.
			\item{What is my opponent's idea?}

			Not clear. 
		\end{itemize}

		The basic idea is to activate the White queen to attack the Black king.

        \variation{ 10. Qe6!!}

        This move has some very deep idea.    
        
        \begin{enumerate}
            \item{\variation[invar]{10... c4}}
        
            \chessboard[
                setfen=1b3q1r/7p/pp1nQpp1/3R4/P1p1pPPk/2P1N3/2P4P/7K w - - 0 11
            ]

            White intends to cut off the king completely and mate with \symqueen d5-d1 maneuver.

            \variation[invar]{11. Rh5+! gxh5 12. Ng2+ Kh3 13. g5+ f5 14. Qd5 h4 15. Qd1 Rg8 16. Ne3 Rxg5 17. Qf1+ Rg2 18. Qxg2# }
            
            \chessboard[setfen=1b3q2/7p/pp1n4/5p2/P1p1pP1p/2P1N2k/2P3QP/7K b - - 0 18]
        
            \item{\variation[invar]{10... f5}}

            \variation[invar]{
                10...f5 11. Rd3!}
                
                \chessboard[setfen=1b3q1r/7p/pp1nQ1p1/2p2p2/P3pPPk/2PRN3/2P4P/7K b - - 1 11]

                Black is doomed.

                \begin{enumerate}
                    \item {\variation[invar]{11...Nc4}}
                    
                    \variation{11...Nc4 12. Ng2+ Kxg4 13. Rg3+ Kh5 14. Rh3+ Kg4 15. Qxc4 Bxf4 16. Qf1 Be5 17. Nf4 Kg5 18. Ne6+ Kf6 }
                    \item {\variation[invar]{11...exd3}}

                    \variation{11... exd3 12. Ng2+ Kxg4 13. Qe1 Qe7 14. Qg3+ Kh5 15. Qh3+ Qh4 16. Qxh4# }
                \end{enumerate}
        \end{enumerate}
    }
\fi