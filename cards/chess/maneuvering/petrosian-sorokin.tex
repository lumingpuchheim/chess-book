\ifdefined\petrosianSorokin
    % The command is already defined, do nothing.
\else
    \newcommand{\petrosianSorokin}[1]{
	\chessproblem{#1}{Petrosian - Sorokin}{2r1kb1r/pp3ppp/1q2pn2/n2p4/3P1B2/P1NQPN2/1P3PPP/R4RK1 w - - 0 1}
    }
\fi

\ifdefined\petrosianSorokinAnswer
    % The command is already defined, do nothing.
\else
    \newcommand{\petrosianSorokinAnswer}[1]{
	\subsection*{#1. Petrosian - Sorokin}
 \chessboard[
        setfen=2r1kb1r/pp3ppp/1q2pn2/n2p4/3P1B2/P1NQPN2/1P3PPP/R4RK1 w - - 0 1,
 	markstyle=circle,
 	linewidth=0.05em,
 	markfields={e8},
	pgfstyle=straightmove,
 	markmove=e3-e4
        ]

	Let's see what Petrosian has to say about this position. More than the move, it's his comments that made a deep impression.
	\begin{quote}
		``Now the plan of b2-b4 and Na4 is not at all dangerous to Black, since his own knight will land on c4. Nonetheless Black's scheme has a major defect: he has `shelved' the task of developing his kingside pieces and castling. At that time I had already mastered one of the important laws of chess strategy: if one side has fallen behind in development, the game must be opened up to punish the offender.'' \cite{Petrosian:2015}
	\end{quote}
    }
\fi
