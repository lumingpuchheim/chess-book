\ifdefined\bebchukBakulin
    % The command is already defined, do nothing.
\else
    \newcommand{\bebchukBakulin}[1]{
	\chessproblem{#1}{Bebchuk - Bakulin}{r1b1k2r/bpp1nppp/p2p1q2/P2P4/7P/1N6/1PP1QPP1/R1B1KB1R w - - 0 1}
    }
\fi

\ifdefined\bebchukBakulinAnswer
    % The command is already defined, do nothing.
\else
    \newcommand{\bebchukBakulinAnswer}[1]{
	\subsection*{#1. Bebchuk - Bakulin}
 	\chessboard[
        	setfen=r1b1k2r/bpp1nppp/p2p1q2/P2P4/7P/1N6/1PP1QPP1/R1B1KB1R w - - 0 1,
 		markstyle=circle,
 		linewidth=0.05em,
 		markfields={e8,e7,f6},
 		markstyle=border,
 		linewidth=0.05em,
 		markfields={a1},
        ]

	\begin{itemize}
		\item{Where are the weaknesses?}

		Black needs to castle. His e-file is weak and his queen is in danger.
		\item{Which is the worst-placed piece?}

		The rook on a1 is inactive. 
		\item{What is my opponent's idea?}

		He wants to castle.
	\end{itemize}

	Trying to catch the queen doesn't work \variation[invar]{1.Bg5 Qe5} since White must exchange the queen. 
	\symrook a4 is a natural idea, intending \symrook e4 to catch the queen.
	
	\variation[level=1]{1. Ra4 O-O 2. Rf4 Bf5 3. g4 Rae8 4. Kd1 Qe5 5. Qxe5 dxe5 6. Rxf5 Nxf5 7. gxf5 Rd8 8. Bg2 Bxf2 9. Ke2} 
	
	White has some advantage. Black could have tried \symbishop f5. Let's play \symrook a4 and let Black find the right defense.
	
    }
\fi