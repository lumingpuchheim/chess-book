\ifdefined\fischerKeres
    % The command is already defined, do nothing.
\else
    \newcommand{\fischerKeres}[1]{
	\chessproblem{#1}{Fischer - Keres}{r5k1/1pq2ppp/2rb1n2/4n2P/p2pPB2/P2P2PB/R1P1Q3/1N3RK1 b - - 0 1}
    }
\fi

\ifdefined\fischerKeresAnswer
    % The command is already defined, do nothing.
\else
    \newcommand{\fischerKeresAnswer}[1]{
	\subsection*{#1. Fischer - Keres}
 \chessboard[
        setfen=r5k1/1pq2ppp/2rb1n2/4n2P/p2pPB2/P2P2PB/R1P1Q3/1N3RK1 b - - 0 1,
 	markstyle=circle,
 	linewidth=0.1em,
 	markfields={c2},
	markstyle=border,
 	markfields={a8}
        ]

	\begin{itemize}
		\item{Where are the weaknesses?}

		White has weakness on c2.
		\item{Which is the worst-placed piece?}

		The rook on a8 does nothing
		\item{Whas is my opponent's idea?}

		White has no active plan so black can choose his own maneuver.
	\end{itemize}

	\symrook c8 is not possible since white's bishop controls c8. \symrook a5 is natural because the 
	rook targets c5 aiming the weakness c2 in the next move.

	Serendipitously, the move also aims h5. White must figure out how to defend. At 
	the moment we don't need to concern how white will defend. Let's pass the ball to 
	white and decide then what to do.
    }
\fi




