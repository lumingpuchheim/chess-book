\ifdefined\dingAronian
    % The command is already defined, do nothing.
\else
    \newcommand{\dingAronian}[1]{
        \chessproblem{#1}{Ding - Aronian}{1r1q1rk1/3nbppp/2p1pn2/ppP5/1PpP4/P1N1P3/RB2QPPP/4NRK1 w - - 0 16}
    }
\fi

\ifdefined\dingAronianAnswer
    % The command is already defined, do nothing.
\else
    \newcommand{\dingAronianAnswer}[1]{
        \subsection*{#1. Ding - Aronian}
        \newchessgame[
            setfen=1r1q1rk1/3nbppp/2p1pn2/ppP5/1PpP4/P1N1P3/RB2QPPP/4NRK1 w - - 0 16,
            moveid=16w
        ]
        \chessboard[
            markstyle=circle,
            linewidth=0.05em,
            markfields={d6},
            pgfstyle=straightmove,
            markmove={e4-e5, e3-e4},
        ]

        \begin{itemize}
            \item{Where are the weaknesses?}
    
            d6 square is weak.
            \item{Which is the worst-placed piece?}
    
            Here we are talking about how to regroup the pieces. 
            White must find a way to exploit the weakness on d6.
            This can be done by placing the bishop on f4, the knight on e4. 
            The pawn on e3 must be placed on e5 at some point.

            \item{What is my opponent's idea?}
            
            He may look for counterplay along the a-file.
            He may also attack the d4 pawn. Therefore both of them must be protected.
        \end{itemize}

        It is always important to remember, don't rush!

        \mainline[level=1]{16. e4 Rb7 17. Nc2 Nb8 18. Raa1 Qc8 19. Rad1 Rd8 20. Bc1 Na6 21. Bf4 Rbd7 22. h3 Ne8 23. Qe3 Bf6 24. e5! Be7 25. Ne4!
 Nac7 26. Nd6}
 
        \chessboard

    }
\fi