\ifdefined\gelfandAnand
    % The command is already defined, do nothing.
\else
    \newcommand{\gelfandAnand}[1]{
	\chessproblem{#1}{Gelfand - Anand}{1r2k2r/pb1n1ppp/4p3/2q5/Q3B3/4P3/1P1N1PPP/R4RK1 b - - 0 18}
    }
\fi

\ifdefined\gelfandAnandAnswer
    % The command is already defined, do nothing.
\else
    \newcommand{\gelfandAnandAnswer}[1]{
        \subsection*{#1. Gelfand - Anand}
        \chessboard[
            setfen=1r2k2r/pb1n1ppp/4p3/2q5/Q3B3/4P3/1P1N1PPP/R4RK1 b - - 0 1,
            markstyle=circle,
            linewidth=0.05em,
            markfields={d7,d2},
            markstyle={border},
            markfields={h8}
                ]

        \begin{itemize}
            \item{Where are the weaknesses?}
            
            Both knight on d2 and d7 are weak.
            \item{Which is the worst-placed piece?}
            
            The rook on h8 does nothing.
            \item{What is my opponent's idea?}
            
            White wants to attack the king in the middle. Maybe with \symrook ac1, \symbishop xb7, \symrook c8. 
        \end{itemize}

        The position is sharp. Black must find a way to solve his problem of his weak knight. 

        The best way to solve the problem is to get rid of it.
        \variation{19... O-O 20. Qxd7 Rfd8 21. Bxh7 Kf8! 22. Qa4 Rxd2} Black has active pieces, which is enough
        for the pawn. All of White's pieces could find better squares.
        
        Aagaard wrote that if one could see both \variation{21. Bxh7} and \variation{21...Kf8}, he is ready for a tournament. Are you ready?
    }
\fi
