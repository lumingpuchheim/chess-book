\section{Combinative Firework}

\epigraph{
    I do not believe that I have ever seen a position under analysis, 
    filled with such an enormous varity of combinative content. \cite{Dvoretsky:2008}
}{Mark Dvoretsky}

\begin{multicols}{2}
\newchessgame[
    setfen=1b1qnr2/4p1kp/pp1rQpp1/2p1N1B1/P4P2/2P5/2P3PP/3R1R1K w - - 0 1,
    moveid=1w
]

Lelchuk - Voronova, USSR 1983

\chessboard

White has a pawn down, and she has four pieces en prise. 
Playing normally is completely hopeless. So she starts an attack.
\mainline[level=1]{1. Bh6+ Kxh6 2. Qf7} 

\chessboard

A warm-up exercise. What must Black play? We can use Process of Elimination here.
White threatens to play \variation[invar]{3. Ng4+ Kh5 4. Qxh7+ Kxg4 5. h3}

To parry the threat, Black may play \variation[invar]{2... fxe5}, \variation[invar]{2... Rxf7} or \variation[invar]{2...Rh8}.
Further calculation eliminates the first two:

\variation[invar]{2... fxe5 3. Qxf8+ Kh5 4. g4+ Kxg4 5. Rg1 Kf3 6. fxe5+} White has huge material advantage.
\variation[invar]{2... Rxf7 3. Nxf7+ Kg7 4. Nxg8} White has the exchange up.
Therefore Black must play \variation[invar]{2... Rh8}.

\mainline[level=1]{2... Rh8 3. Ng4+ Kh5} 

\chessboard

How to continue the attack here? Black is threatening to take the 
rook on d1. Moving the knight to e3 to protect the rook and at 
the same time making room on g4 for her pawn is natural.

\mainline{ 4. Ne3 } 

\chessboard

Again, Black must decide how to defend here. The candidate moves 
are \variation[invar]{4... f5}, \variation[invar]{4... e5} and \variation[invar]{4... Rxd1}. 

Let's check \variation[invar]{4... f5} first. 

Black wants to trap White's queen, intending to play \variation[invar]{5. Rf6}.
White can however answer with \variation[invar]{5. Nxf5! Rxd1 6. g4 Kxg4 7. Ne3 \xskakcomment{We know this maneuver already!} Kh5
8. Rxd1 Qc8 9. Qd5+ e5 10. Qg2 Kh6 11. Qg5+ Kg7 12. Qe7+ Kh6 13. Rd7 \xskakcomment{White is crushing}}

There is no direct line refuting \variation[invar]{4... e5}.

\variation[invar]{4...e5 5. fxe5 Rxd1 6. Rxd1 Qc8 7. Rf1 Kh6 8. exf6 Be5} 

\chessboard[setfen=2q1n2r/5Q1p/pp3Ppk/2p1b3/P7/2P1N3/2P3PP/5R1K w - - 1 9]

\variation[invar]{9. h3! \xskakcomment{Such a quiet move is not easy to predict on move 4 when Black
chooses how to defend.}  Qc7 10. Ng4+ Kg5 11. Qd5 Qd6 12. Qxe5+ Qxe5 13. Nxe5 } White is winning.

Now we have only \variation[invar]{4... Rxd1} left.
\mainline{4...Rxd1 }
    
    \chessboard

    White can force a draw here: \variation{5. g4+ Kh6 6. Nf5+ gxf5 7. g5+ fxg5 8. fxg5+ Kxg5 9. Qxf5+ Kh6 10. Qh3+ Kg5 11. Qf5+ Kh6 12. Qh3+ Kg5 13. Qf5+ Kh6 
    }. It is often advantageous to find these forced draws while attacking as an emergency plan.

    \mainline[level=1]{5. Rxd1} 
    
    \chessboard

    Choosing the right move is still not easy here. Understanding White has \symknight g4 as resource,
    Black can use her queen to defend the square and go to an unbalanced endgame:
    \variation[invar]{5... Qc8 6. Rd5+ Kh6 7. h3 Nc7 8. Ng4+ Qxg4 9. hxg4 Nxd5 10. g5 fxg5 11. fxg5+ Kxg5 12. Qxd5})
    
    \chessboard[setfen=1b5r/4p2p/pp4p1/2pQ2k1/P7/2P5/2P3P1/7K b - - 0 12]

    The material is roughly equal. White has an active queen. She will 
    take another pawn but her pawn structure is inferior. Maybe she has some small advantage.

    We go back to the mainline. In the game Voronova decided to develop her knight.
    \mainline[level=1]{5... Nd6 6. Rd5+} 
    
    \chessboard

    We arrive at the deciding moment of the game. How should Black defend using Process of Elimination?

    Candidate moves are \variation[invar]{6...e5} and \variation[invar]{6...f5}. The latter loses 
    immediately after \variation[invar]{6... f5 7. g4+ Kh6 8. Rxd6 exd6 9. Nxf5+  gxf5 10. g5+}.
    
    In the game, Black Voronova was unable to bear the tension, played \variation[invar]{6...f5} and lost
    quickly. What would happen if she played \variation[invar]{6...e5}?

    \chessboard

    \mainline[level=1]{6... e5} 
    
    \chessboard

    How should White continue the attack?
    
    \mainline{ 7. Qg7} While chasing a wandering king, it is important
    to cut off his retreat. 
    
    \chessboard

    What is White's threat here?

    White also threatens \variation[invar]{8. g4 Kh4 9. Qh6#}. The defense is natural.

    \mainline[level=1]{7...  Qf8 8. g4+ Kh4} 
    
    \chessboard

    Black wants to exchange the queens and then cash out his material advantage.

    White must move her queen. d7 is a good square.

    \mainline[level=1]{9. Qd7}

    \chessboard

    What is White's threat here?

    White threatens to play \variation[invar]{10. Ng2+ Kh3 11. Rd3#}. To defend the 
    threat, Black can play \variation[invar]{9... e4} or \variation[invar]{9... f5}
    
    \mainline{ 9... e4} 
    
    \chessboard[
        setfen=1b3q1r/3Q3p/pp1n1pp1/2pR4/P3pPPk/2P1N3/2P4P/7K w - - 0 10
    ]
    
    \mainline{ 10. Qe6!!}

    This move has a very deep idea.    
    
    \begin{enumerate}
        \item{\variation[invar]{10... c4}}
     
        \chessboard[
            setfen=1b3q1r/7p/pp1nQpp1/3R4/P1p1pPPk/2P1N3/2P4P/7K w - - 0 11
        ]

        White intends to cut off the king completely and mate with a \symqueen d5-d1 maneuver.

        \variation[invar]{11. Rh5+! gxh5 12. Ng2+ Kh3 13. g5+ f5 14. Qd5 h4 15. Qd1 Rg8 16. Ne3 Rxg5 17. Qf1+ Rg2 18. Qxg2# }
        
        \chessboard[setfen=1b3q2/7p/pp1n4/5p2/P1p1pP1p/2P1N2k/2P3QP/7K b - - 0 18]
    
        \item{\variation[invar]{10... f5}}

        \variation[invar]{
            10...f5 11. Rd3!}
            
            \chessboard[setfen=1b3q1r/7p/pp1nQ1p1/2p2p2/P3pPPk/2PRN3/2P4P/7K b - - 1 11]

            Black is doomed.

            \begin{enumerate}
                \item {\variation[invar]{11...Nc4}}
                
                \variation{11...Nc4 12. Ng2+ Kxg4 13. Rg3+ Kh5 14. Rh3+ Kg4 15. Qxc4 Bxf4 16. Qf1 Be5 17. Nf4 Kg5 18. Ne6+ Kf6 }
                \item {\variation[invar]{11...exd3}}

                \variation{11... exd3 12. Ng2+ Kxg4 13. Qe1 Qe7 14. Qg3+ Kh5 15. Qh3+ Qh4 16. Qxh4# }
            \end{enumerate}
    \end{enumerate}

\end{multicols}
