To choose the most effective move in a given position, it is essential to evaluate the static aspects of that position. This evaluation helps determine whether a player should adopt a static or dynamic approach in the subsequent phase of the game.

Method for Evaluating Static Balance \cite{Dorfman:2001}

To assess the static balance of the position, follow a structured analysis method. This involves breaking down the position into its fundamental static elements in a regressive order.

\begin{itemize}
    \item{Static Evaluation}
    \begin{itemize}
        \item{King Safety}

        Evaluate the positioning and safety of both kings. A secure king often dictates the strength of the overall position.
        \item{Material Correlation}

        Analyze the material balance between the two players (i.e., which player has more material advantage).
        \item{Post-Queen Exchange Position}

        Consider who has a stronger position after the potential exchange of queens. The impact of this exchange can significantly alter the dynamics of the game.
        \item{Pawn Structure}

        Assess the formation and distribution of pawns. A advantageous pawn structure can lead to long-term strategic benefits.
    \end{itemize}
    \item{Who can evolve independently?}

    There is a crude method, enabling an 
    immediate static evaluation of a position to 
    be obtained: - analyse whether it is possible for your 
    own position to evolve independently of the 
    opponent's; - analyse whether the opponent's position 
    can evolve independently of your own. 
    
    The position which is ready for evolution 
    is statically better.


    \item{Choosing Between Dynamic and Static Play}

    Once you have completed the static evaluation of the position, the next step is to decide whether to pursue a dynamic or static strategy. This decision will guide you in identifying candidate moves:    
\end{itemize}

If your evaluation indicates a strong static advantage (e.g., superior king safety or material correlation), opt for a static strategy. Focus on maintaining your advantage and consolidating your position.
Conversely, if the position suggests that dynamic opportunities exist (such as tactical motifs or generating counterplay), consider a dynamic strategy. Look for candidate moves that exploit these opportunities and create imbalances in the position.