Chess can quickly become overwhelming if you try to calculate every variation. Endless lines drain energy, cloud judgment, and increase mistakes. A simpler, structured approach keeps your mind clear and your decisions effective. Focus on three key questions \cite{Aagaard:2012}:
\begin{itemize}
	\item{Where are the weaknesses?}

	Look for vulnerable squares, weak pawns or exposed king positions. Identifying weaknesses gives direction and targets for your plan.
	\item{Which is the worst-placed piece?}

	Evaluate both your and your opponent's pieces. Misplaced pieces often limit mobility and coordination. Exploiting or improving these positions can turn the game in your favor without complex calculation. 
	\item{Whas is my opponent's idea?}

	Try to understand your opponent's plan. Knowing their intentions helps you defend efficiently and counterattack strategically.
\end{itemize}
This template works because it emphasizes pattern recognition and strategic understanding over brute-force calculation. Instead of burning energy on countless variations that rarely occur in full, you make practical decisions baswed on the position's reality. World champions like Capablanca, Petrosian and Karpov relied on this approach: they played the position, not the tree of possibilities.

By asking these three questions each turn, you conserve mental energy, reduce errors, and develop a style rooted in clarity, control and real understanding. Chess becomes more about thinking smartly. 