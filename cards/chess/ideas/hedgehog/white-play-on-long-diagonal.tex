\newpage
\section{Hedgehog Position: White Play on Long Diagonal}

The bishop on b7 plays a crucial role for Black in the Hedgehog position.
If White can exchange it, he can have a strategic advantage by playing on the long diagonal.


\begin{multicols}{2}
    \newchessgame
    \chessgameinfo{Phillips & Drew Kings}{Ulf Andersson}{Yasser Seirawan}{}{1982.04.21}{1-0}
    \mainline[level=1]{
        1.Nf3 Nf6 2.c4 c5 3.g3 b6 4.Bg2 Bb7 5.O-O e6 6.Nc3 Be7 7.d4
        cxd4 8.Qxd4 d6 9.Bg5 a6}
    
    \chessboard

    We know that Black's main plan is to play ...b5 and ...d5 to free himself.
    White will prevent the idea of ...d5 by exchanging the f6 knight with his bishop.
    His knight on c3 also controls the essential b5 and d5 squares.
    
    \mainline[level=1]{10.Bxf6 Bxf6 11.Qf4 O-O 12.Rfd1 Be7
        13.Ne4 Bxe4 14.Qxe4 Ra7 15.Nd4 Rc7 16.b3 Rc5 }
    
    \chessboard

    Black has also exchanged White's knight on e4 with his light-squared bishop.
    Therefore, he has holes on his light squares which White's bishop can exploit. 

    \mainline[level=1]{17.a4}

    First things first, Black wants to play ...b5 at some point. 
    White stops this idea by playing a4. This is usually a good \vocab{prophylaxis}{prophylaxis}. We will also see later that this 
    move frees the square for his queen. \index{Prophylaxis}

    Next White will overprotect his c4 pawn before playing b4, pushing back 
    Black's rook on c5.

    \mainline[level=1]{17... Qc7 18.Qb1
        Rc8 19.Ra2 Bf8 20.e3 Qe7 21.Rc2 g6 22.Qa2!}

    \chessboard

    Andersson's trademark rearrangement. The queen proves to be more useful not in 
    the centre, but in the corner behind the pawns! The white pieces defend the c4–square, freeing the 
    b3–pawn for active play. \cite{griffin:2021}

    \mainline[level=1]{22... Qg5 23.h4 Qf6}
    
    \chessboard
    
    White has prepared for everything. Now it is time to set up 
    an outpost on c6. \index{Outpost}
    \mainline[level=1]{24.b4!
        R5c7 25.b5! a5 26.Nc6!}
        
    \chessboard

    White now has a strategic advantage. His pieces are more active while 
    Black has a cramped position and doesn't have any counterattack.
    White wins the game eventually.

    \mainline[level=1]{26... Nd7 27.Rcd2 Nc5 28.Qc2 Qg7 29.f4 Kh8
        30.Bf3 Re8 31.Kg2 f5 32.e4 e5 33.fxe5 dxe5 34.Rd8 Rxd8 35.Rxd8
        fxe4 36.Bxe4 Qf6 37.Bd5 Kg7 38.Re8 e4 39.Bxe4 Rf7 40.Qe2 Nxa4
        41.Bf3 Rd7 42.Bg4 Rd6 43.Ne5 Nc5 44.Ra8 Rd8 45.Ra7+ Kg8 46.Bf3
        Bg7 47.Bd5+ Rxd5 48.Ng4 Qd8 49.cxd5 Qxd5+ 50.Kh2 Ne4 51.Re7
    } 1-0

\end{multicols}