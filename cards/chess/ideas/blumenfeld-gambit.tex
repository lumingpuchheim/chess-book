\newpage

\section{Blumenfeld Gambit}
\newchessgame

The Blumenfeld Countergambit is a chess opening characterised by the moves 3...e6 4.Nf3 b5 in the Benoni Defense arising after:

\mainline[level=1]{
1. d4 Nf6
2. c4 c5
3. d5 e6
4. Nf3 b5}

\chessboard

Black sacrifices a wing pawn to establish an imposing centre with pawns on c5, d5 and e6. The natural development of the bishops to b7 and d6, combined with the half-open f-file for a rook, tend to facilitate Black's play on the kingside.


\begin{multicols}{2}
\newchessgame
\chessgameinfo{Bad Pistyan}{Siegbert Tarrasch}{Alexander Alekhine}{1}{1922.04.07}{0-1}
\mainline[level=1]{
    1. d4 Nf6 2. Nf3 e6 3. c4 c5 4. d5 b5 5. dxe6 fxe6 6. cxb5 d5}

\chessboard

Black has a very intimidating center.

\mainline[level=1]{
7. e3?!}

This move is too passive. White must try to destroy Black's center before Black
starts an attack.  Tarrasch wanted to prevent
Black from playing ...d4, but this was not necessary. He should have played \symknight c3 first.
Black cannot play \variation[invar]{7... d4 8. Na4 Bd6 9. e3} because his 
center is destroyed completely. 
While after \variation[invar]{7... Nbd7 8. e4 d4 9. Na4 Bb7 10. Bd3 Bxe4 11. Bxe4 Nxe4 12. O-O Ndf6 13. b4 cxb4 14. Nxd4 }
White has nothing to fear. 

\mainline[level=1]{7...Bd6 8. Nc3 O-O 9. Be2 Bb7 10. b3 Nbd7 11. Bb2 Qe7
12. O-O Rad8 13. Qc2 e5 }

Black starts an attack. ...e4 is in the air.

\mainline[level=1]{14. Rfe1? e4 15. Nd2 Ne5 16. Nd1 Nfg4
17. Bxg4 Nxg4 18. Nf1 Qg5 19. h3 Nh6 20. Kh1 Nf5 21. Nh2 d4
22. Bc1 d3 23. Qc4+ Kh8 24. Bb2 Ng3+ 25. Kg1 Bd5 26. Qa4 Ne2+
27. Kh1 Rf7 28. Qa6 h5 29. b6 Ng3+ 30. Kg1 axb6 31. Qxb6 Ne2+
32. Kh1 Ng3+ 33. Kg1 d2 34. Rf1 Nxf1 35. Nxf1 Be6 36. Kh1 Bxh3
37. gxh3 Rf3 38. Ng3 h4 39. Bf6 Qxf6 40. Nxe4 Rxh3+
} 0-1

\end{multicols}