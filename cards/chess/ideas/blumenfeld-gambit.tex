\newpage

\section{Blumenfeld Gambit}
\newchessgame

The Blumenfeld Countergambit is a chess opening characterised by the moves 3...e6 4.Nf3 b5 in the Benoni Defense arising after:

\mainline[level=1]{
1. d4 Nf6
2. c4 c5
3. d5 e6
4. Nf3 b5}

\chessboard

Black sacrifices a wing pawn to establish an imposing centre with pawns on c5, d5 and e6. The natural development of the bishops to b7 and d6, combined with the half-open f-file for a rook, tend to facilitate Black's play on the kingside.


\begin{multicols}{2}
\newchessgame
\chessgameinfo{Bad Pistyan}{Siegbert Tarrasch}{Alexander Alekhine}{1}{1922.04.07}{0-1}
\mainline[level=1]{
    1. d4 Nf6 2. Nf3 e6 3. c4 c5 4. d5 b5 5. dxe6 fxe6 6. cxb5 d5}

\chessboard

Black has a very intimidating center.

\mainline[level=1]{
7. e3?!}

This move is too passive. White must try to destroy Black's center before Black
starts an attack.  Tarrasch wanted to prevent
Black from playing ...d4, but this was not necessary. He should have played \symknight c3 first.
Black cannot play \variation[invar]{7... d4 8. Na4 Bd6 9. e3} because his 
center is destroyed completely. 
While after \variation[invar]{7... Nbd7 8. e4 d4 9. Na4 Bb7 10. Bd3 Bxe4 11. Bxe4 Nxe4 12. O-O Ndf6 13. b4 cxb4 14. Nxd4 }
White has nothing to fear. 

\mainline[level=1]{7...Bd6 8. Nc3 O-O 9. Be2 Bb7 10. b3 Nbd7 11. Bb2 Qe7
12. O-O Rad8 13. Qc2 e5 }

Black starts an attack. ...e4 is in the air.

\mainline[level=1]{14. Rfe1? e4 15. Nd2 Ne5 16. Nd1 Nfg4
17. Bxg4 Nxg4 18. Nf1 Qg5 19. h3 Nh6 20. Kh1 Nf5 21. Nh2 d4
22. Bc1 d3 23. Qc4+ Kh8 24. Bb2 Ng3+ 25. Kg1 Bd5 26. Qa4 Ne2+
27. Kh1 Rf7 28. Qa6 h5 29. b6 Ng3+ 30. Kg1 axb6 31. Qxb6 Ne2+
32. Kh1 Ng3+ 33. Kg1 d2 34. Rf1 Nxf1 35. Nxf1 Be6 36. Kh1 Bxh3
37. gxh3 Rf3 38. Ng3 h4 39. Bf6 Qxf6 40. Nxe4 Rxh3+
} 0-1



\newchessgame
\chessgameinfo{World Championship Match}{Ding, Liren}{Gukesh D}{11}{2024.12.08}{1-0}

\mainline[level=1]{1. Nf3 d5 2. c4 d4 3. b4 c5 4. e3}

\chessboard

Blumenfeld Gambit in World Championship Match, with color reversed! 
Taking the pawn with \variation[invar]{4... dxe3 5. fxe3 cxb4 6. d4}
will be risky because comparing to normal Blumenfeld Gambit,
White has one more tempo. Who knows what kind surprise has team Gukesh 
prepared for Black? Ding declined the gambit.

\mainline[level=1]{4...Nf6 5. a3 Bg4 6. exd4 cxd4 7. h3 Bxf3 8. Qxf3 Qc7 9. d3}

\chessboard

Strictly the only move to give White a comfortable edge is 9.c5!, 
but instead Gukesh played 9.d3?, having thought for just five minutes in a position 
where he had over an hour's lead on the clock. Why so fast? 
He revealed in the press conference that he'd thought he had the position 
on the board in the morning and that d3 was the move, but his guess was that 
he'd actually been looking at the position after \variation[invar]{8...Nc6}.

\mainline[level=1]{9...a5 10. b5 Nbd7 11. g3 }

\chessboard

With White's e-pawn exchanged with Black's c-pawn on d4, we have a (reversed) Benoni-like structure again. As in Benoni structure, White's d3 pawn is weak and 
Black's plan is often attacking the d3 pawn with his knight on c5. The next move is quite logical. 

\mainline{11...Nc5}! e5 is also good, a standard move to a standard problem (Benoni), what can go wrong? \mainline{12. Bg2 Nfd7} Move the knight to better e5 square. A typical Benoni play \variation{12... e5 13. O-O Bd6 } is also good since white needs to seek equality as in Benoni Defence. \mainline{ 13. O-O Ne5 14. Qf4 Rd8 15. Rd1}

\chessboard

Black has a much better position here: his pieces are better developed while White is still struggling to develop his queen side pieces because his d3 pawn is attacked.

With such a strategic advantage, it is useful to think in prophylaxis. Finding out what White tries to achieve in such desperate position helps to find the next move.

\begin{itemize}
    \item{White would love to get rid of the c5 knight with a4, \symbishop{} a3}
    \item{b6 would be a resource}
    \item{Black queen on c7 hangs}
\end{itemize}

Another method to find the next move is think schematically. Black wants to develop his bishop. The bishop would love to be on d6 to protect the Black queen and/or attack the White queen. e6 and then \symbishop d6 is logical.

\mainline{15...g6}?  

Ding lost the thread. \variation{15...e6} should have been played. For example \variation{16. b6 Qd6 17. Nd2 Nexd3 18. Qxd6 Rxd6} Black is up a pawn while White has little compensation.

Eventually Ding loses the important game. 
\mainline{16. a4 h5 17. b6  Qd6 18. Ba3 Bh6
19. Bxc5 Qxc5 20. Qe4 Nc6 21. Na3 Rd7 22. Nc2 Qxb6 23. Rab1 Qc7 24. Rb5 O-O 25.
Na1 Rb8 26. Nb3 e6 27. Nc5 Re7 28. Rdb1 Qc8 29. Qxc6 } 1-0

\end{multicols}