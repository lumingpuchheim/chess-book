\newpage
\section{Prophylactic King Move}

Sometimes the opponnent needs a crucial check to achieve his 
goal. It is helpful to move the king to a safe square to 
eliminate the opponent's check.

\nocite{Dvoretsky:1996}

\begin{multicols}{2}
\chessgameinfo{Linares}{J.Nunn}{A.Jusupov}{}{1988}{0-1}
\newchessgame[
    setfen=r4rk1/p3npp1/4p2p/n2pP3/R7/2qBR3/2PN1PPP/3Q2K1 b - - 0 18,
    moveid=18b
]
\chessboard

White threatens to play \symbishop h7+, winning the queen. 

If Black moves his queen away, for example \variation[invar]{18... Qc7} White can play 19. \symqueen c5 
and then \symrook g3, maintaining the attack.

\mainline[level=1]{18... Kh8!}

``Despite looking rather awkward on c3 the queen at least attacks the d2-knight and so
limits the movement of the White queen.'' (Dvoretsky)

\mainline[level=1]{19. g4 Nac6 20. Nf3 Rab8 21. Bc4 Qb2 22. Bb3 Ng6 23. Ra2 Rxb3 24. Rxb2 Rxb2 }

\chessboard

Black has compensation for his queen. He wins the game eventually.


\chessgameinfo{Mar del Plata}{J.Timman}{B.Larsen}{}{1982}{0-1}
\newchessgame[
    setfen=1r2rbk1/2q3p1/p1p1bn1p/1pP1p3/8/1PN3P1/PBQ1PPB1/2RR2K1 b - - 0 23,
    moveid=23b
]

\chessboard

\mainline[level=1]{23... Kh8!}

\variation[invar]{23... Bxc5 24. Ne4 Nxe4 25. Qxe4} White will also gain 
the c6 pawn and have advantage.

\mainline[level=1]{24. e3}

Now \variation[24. Ne4 Bf5]. White cannot take the f6 knight since there is 
no check and his queen is hanging.

The game continues and Black wins eventually.
\end{multicols}