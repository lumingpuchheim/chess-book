\epigraph{
    Magnus Carlsen's exceptional memory and pattern recognition skills are central to his chess mastery. In a remarkable demonstration, he accurately recreated a 26-piece position after just a two-second glance at the board. This position was from Bobby Fischer's 1956 game against Donald Byrne, and Carlsen not only reconstructed it but also recalled the subsequent moves with near-perfect accuracy.
}{}

When Magnus Carlsen plays chess, he searches for similar positions in his brain so that he doesn't need to calculate too much. This ability is rooted in the cognitive process known as ``chunking,'' where individuals group information into meaningful clusters to enhance memory retention and recall. In chess, chunking allows players to perceive complex positions as manageable patterns, facilitating quick decision-making. 

Understanding and utilizing chunking is crucial in chess. By recognizing and recalling familiar patterns, players can reduce the need for extensive calculation, leading to more efficient and effective play.

Chess is about finding chunks that are familiar to you. It is therefore practical to collect these chunks for future use. In this part, we are going to list some ideas and hopefully you can use them in your future games.

