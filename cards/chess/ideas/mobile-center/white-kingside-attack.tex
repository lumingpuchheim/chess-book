\newpage
\section{Mobile Center: White Kingside Attack}
\begin{multicols}{2}

\chessgameinfo{USSR Championship}{L.Polugaevsky}{M.Tal}{2}{1969.9.7}{1-0}
\newchessgame
\mainline[level=1]{
    1. c4 Nf6 2. Nc3 e6 3. Nf3 d5 4. d4 c5 5. cxd5 Nxd5 6. e4 Nxc3 7. bxc3 cxd4 8. cxd4 Bb4+ 9. Bd2 Bxd2+ 10. Qxd2 O-O 11. Bc4 Nc6 12. O-O b6 13. Rad1 Bb7 14. Rfe1}

    \chessboard

    The same position also appeared in the World Chess Championship game between
    Spassky and Petrosian one month ago. Petrosian chose \variation{14... Rc8}. Tal 
    chose another variation in this game.
    \index{Mobile Center!White Kingside Attack}

    \mainline[level=1]{14... Na5 15. Bd3 Rc8}
    
    \chessboard

    White will thrust his pawn to e5. Before doing so, he sacrifices his d-pawn first.
    Otherwise, Black can exchange the knight with his bishop, removing an important attacker.

    \mainline[level=1]{16. d5 exd5 17. e5 Nc4 18. Qf4 Nb2? }
    
    \chessboard

    Here we will see a Greek Gift! \index{Attack!Greek Gift}
    \mainline[level=1]{19. Bxh7+ Kxh7 20. Ng5+ Kg6 21. h4!}
    
    Bring a further attacker!
    
    \mainline[level=1]{21...Rc4}
    
    \chessboard

    \mainline[level=1]{22. h5+?!}
    
    Too slow! \variation[invar]{22. Qg3 Kh6 23. e6 Qf6 24. exf7 Nxd1 25. Re6 } White wins the 
    game. White wins after an endgame after the text move.
    
    \mainline[level=1]{22... Kh6 23. Nxf7+ Kh7 24. Qf5+ Kg8 25. e6 Qf6 26. Qxf6 gxf6 27. Rd2 Rc6 28. Rxb2 Re8 29. Nh6+ Kh7 30. Nf5 Rexe6 31. Rxe6 Rxe6 32. Rc2 Rc6 33. Re2 Bc8 34. Re7+ Kh8 35. Nh4 f5 36. Ng6+ Kg8 37. Rxa7 
} 1-0
\end{multicols}