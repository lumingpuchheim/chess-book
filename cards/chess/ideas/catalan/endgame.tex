\section{Catalan Opening: Endgame}

White could steer the game toward a favorable endgame, as Ulf Andersson demonstrated in this instructive example.

\begin{multicols}{2}
\chessgameinfo{Swedish Team Championship}{Ulf Andersson}{Sergey Vladimirovich Ivanov}{}{2000}{1-0}
\newchessgame
\mainline[level=1]{1. Nf3 d5 2. d4 Nf6 3. c4 e6 4. g3 dxc4 5. Qa4+ Nbd7 6. Bg2 a6
7. Nc3 Rb8 8. Qxc4 b5 9. Qd3 Bb7 10. O-O c5 11. dxc5 Bxc5
12. Bf4 Rc8 13. Rad1 O-O 14. Ne5}

\chessboard

Many pieces were exchanged, and Andersson deliberately headed for an endgame
where he could press with a persistent but risk-free advantage.

\mainline[level=1]{14... Bxg2 15. Kxg2 Nxe5 16. Bxe5
Qxd3 17. Rxd3 Rfd8 18. Rxd8+ Rxd8 19. Bxf6 gxf6}

\chessboard

This was the position Andersson had in mind when he played his 14th move.
How should we evaluate this endgame? A good starting point is to examine
the pawn structure.

\chessboard[
    setfen=8/5p1p/p3pp2/1p6/8/P5P1/1P2PP1P/8 w HAha - 0 1,
    showmover=false,
    markstyle=circle,
    linewidth=0.05em,
    markfields={a6, h7},
]

Black had two clear weaknesses: the pawns on a6 and h7.
As the game continued, White used his king to attack these pawns directly.
Also notice Black's doubled pawns on the f-file; they gave White enough time
to bring his king into Black's position.

A similar pawn endgame was played by Rubinstein in 1909. 
In that game, the defender had exactly the same pawn structure!
I am sure Andersson was aware of this endgame when he played the opening.
From that position, Rubinstein was winning. We will examine that endgame 
after this game. In Rubinstein's game, the attacking king marched to the kingside. 
After liquidating all the pawns there, the king dominated because it was closer 
to the queenside pawns.

White had no real weaknesses in his camp. He would place his pawns on light
squares so that Black's dark-squared bishop could not attack them.

Compared to that game, one key difference made Black's current position still 
defendable: his king was closer to the kingside weakness. We will see 
how Andersson lured the Black king away in the actual game.

We also had a knight versus bishop endgame (with the rooks). 
After the rooks were exchanged, White's pieces were actually too slow to attack
both weaknesses simultaneously, because the bishop could defend in time, so Black had decent practical chances to hold.
However, White could simply improve his position and wait for a mistake, so his
winning chances remained very good.

\mainline[level=1]{20. Rc1 Be7}

\chessboard

One of the most neglected principles in chess is \emph{don't rush}. 
Before pushing forward, White had to first prevent Black's rook from invading
on d2. His next move was highly instructive.

\mainline[level=1]{
21. Nb1! f5 22. e3 Bf6 23. b3 Kf8 24. Kf3 Ke7 25. h3 Rd5
26. Ke2 Kd7 27. Nd2 Be7 28. Nf3 Bf6 29. Ne1 Rd6 30. g4 fxg4
31. hxg4 Rc6 32. Rxc6 Kxc6 33. Nd3!}

\chessboard

White had to stop Black's king from invading. The d5 square was also
temporarily defended because after \variation[invar]{33...Kd5? 34. Nb4+}
Black would simply lose a pawn.

\mainline[level=1]{33...a5 34. e4!}

\chessboard 

Now d5 was also defended permanently. White was ready to march his king toward
Black's kingside while Black was reduced to waiting.
\mainline[level=1]{34...a4 35. Ke3 axb3
36. axb3 Kd6 37. Kf4 Bd8 }

\chessboard

\mainline[level=1]{38. g5!}

White now sacrificed a pawn temporarily to open lines for a decisive king invasion. \index{Pawn Sacrifice!Infiltration}

\mainline[level=1]{38...Ke7 39. Kg4 Kf8 40. f3 f6
41. Kh5 fxg5 42. Kh6 Kg8 43. Nc5 Kf7 44. Kxh7}

\chessboard

The first mission was accomplished: White's king had infiltrated Black's camp.
Next he would march toward the queenside to attack the remaining weakness.

\mainline[level=1]{44...Bb6 45. Nd3 Kf6
46. Kg8 Bg1 47. Kf8 e5 48. Ke8 Ke6 49. Kd8 Kd6 50. Kc8 Be3
51. Kb7}

\chessboard

\mainline[level=1]{51...Bd4?}

Until this point Black had defended well. After the move in the game, however,
White placed him in zugzwang, won a pawn, and with it the game. Black should
have played \variation[invar]{51... Bg1}
\mainline[level=1]{52. b4 Kd7 53. Nc5+ Kd6 54. Kb6 Bc3 55. Kxb5 Bd2
56. Nb7+ Kc7 57. Na5 Be3 58. Ka6} 1-0

Now let's examine the endgame played by Rubinstein in 1909.

\chessgameinfo{St. Petersburg}{Erich Cohn}{Akiba Rubinstein}{}{1909.02.28}{0-1}
\newchessgame[
    setfen=8/pp2kppp/4p3/8/1P6/P3PP2/5P1P/2K5 b - - 0 25,
    moveid=25b
]
\chessboard

White had exactly the same pawn structure as in the previous game with 
color reversed. 

The Black king started his invasion as in the previous game.
\mainline[level=1]{
    25... Kf6 26. Kg1 Kg5 27. Kh1 Kh4}

\chessboard   

Chess is a tragedy of one tempo. White would draw if he could play \symking g2 now, 
but he lacks that crucial move!

\mainline[level=1]{28. Kf1 Kh3 29. Kg1 e5 30. Kh1 b5 31. Kg1 f5 32. Kh1 g5 33. Kg1 h5 34. Kh1 g4 }

\chessboard

\mainline[level=1]{35. e4 }

\variation[invar]{35. fxg4 fxg4 36. Kg1 h4 37. Kh1 g3 38. hxg3 hxg3 39. f3 g2+ 
40. Kg1 Kg3 41. f4 exf4 42. exf4 Kxf4 43. Kxg2 Ke3} Black is winning (John Nunn).

\chessboard[
    setfen=8/p7/8/1p6/1P6/P3k3/6K1/8 w - - 1 44
]

We go back to the Rubinstein's game.

\chessboard

He won the game soon.

\mainline[level=1]{35... fxe4 36. fxe4 h4 37. Kg1 g3 38. hxg3 hxg3 0-1
}

It is remarkable that Andersson played against the same pawn structure with
the same king maneuver as Rubinstein did about 100 years ago. As the saying goes, ``knowledge is power''---Andersson's 
familiarity with this classic endgame undoubtedly influenced his opening choice and subsequent play. 
\end{multicols}