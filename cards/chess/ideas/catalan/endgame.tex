\section{Catalan Opening: Endgame}

White can steer the game toward a favorable endgame, as Ulf Andersson demonstrated in this instructive example.

\begin{multicols}{2}
\chessgameinfo{Swedish Team Championship}{Ulf Andersson}{Sergey Vladimirovich Ivanov}{}{2000}{1-0}
\newchessgame
\mainline[level=1]{1. Nf3 d5 2. d4 Nf6 3. c4 e6 4. g3 dxc4 5. Qa4+ Nbd7 6. Bg2 a6
7. Nc3 Rb8 8. Qxc4 b5 9. Qd3 Bb7 10. O-O c5 11. dxc5 Bxc5
12. Bf4 Rc8 13. Rad1 O-O 14. Ne5}

\chessboard

Many pieces will be exchanged, and Andersson deliberately heads for an endgame
where he can press with a persistent but risk-free advantage.

\mainline[level=1]{14... Bxg2 15. Kxg2 Nxe5 16. Bxe5
Qxd3 17. Rxd3 Rfd8 18. Rxd8+ Rxd8 19. Bxf6 gxf6}

\chessboard

This is the position Andersson had in mind when he played his 14th move.
How should we evaluate this endgame? A good starting point is to examine
the pawn structure.

\chessboard[
    setfen=8/5p1p/p3pp2/1p6/8/P5P1/1P2PP1P/8 w HAha - 0 1,
    showmover=false,
    markstyle=circle,
    linewidth=0.05em,
    markfields={a6, h7},
]

White has no real weaknesses in his camp. He will place his pawns on light
squares so that Black's dark-squared bishop cannot attack them.

On the other hand, Black has two clear weaknesses: the pawns on a6 and h7.
As the game continues, White will use his king to attack these pawns directly.
Also notice Black's doubled pawns on the f-file; they give White enough time
to bring his king into Black's position.

After the rooks are exchanged, White's pieces are actually too slow to attack
both weaknesses simultaneously, so Black has decent practical chances to hold.
However, White can simply improve his position and wait for a mistake, so his
winning chances remain very good.

\mainline[level=1]{20. Rc1 Be7}

\chessboard

One of the most neglected principles in chess is \emph{don't rush}. 
Before pushing forward, White must first prevent Black's rook from invading
on d2. His next move is highly instructive.

\mainline[level=1]{
21. Nb1! f5 22. e3 Bf6 23. b3 Kf8 24. Kf3 Ke7 25. h3 Rd5
26. Ke2 Kd7 27. Nd2 Be7 28. Nf3 Bf6 29. Ne1 Rd6 30. g4 fxg4
31. hxg4 Rc6 32. Rxc6 Kxc6 33. Nd3!}

\chessboard

White must now stop Black's king from invading. The d5 square is also
temporarily defended because after \variation[invar]{33...Kd5? 34. Nb4+}
Black would simply lose a pawn.

\mainline[level=1]{33...a5 34. e4!}

\chessboard 

Now d5 is also defended permanently. White is ready to march his king toward
Black's kingside while Black is reduced to waiting.
\mainline[level=1]{34...a4 35. Ke3 axb3
36. axb3 Kd6 37. Kf4 Bd8 }

\chessboard

\mainline[level=1]{38. g5!}

White now sacrifices a pawn temporarily to open lines for a decisive king invasion.

\mainline[level=1]{38...Ke7 39. Kg4 Kf8 40. f3 f6
41. Kh5 fxg5 42. Kh6 Kg8 43. Nc5 Kf7 44. Kxh7}

\chessboard

The first mission is accomplished: White's king has infiltrated Black's camp.
Next he will march toward the queenside to attack the remaining weakness.

\mainline[level=1]{44...Bb6 45. Nd3 Kf6
46. Kg8 Bg1 47. Kf8 e5 48. Ke8 Ke6 49. Kd8 Kd6 50. Kc8 Be3
51. Kb7}

\chessboard

\mainline[level=1]{51...Bd4?}

Until this point Black has defended well. After the move in the game, however,
White places him in zugzwang, wins a pawn, and with it the game. Black should
have played \variation[invar]{51... Bg1}
\mainline[level=1]{52. b4 Kd7 53. Nc5+ Kd6 54. Kb6 Bc3 55. Kxb5 Bd2
56. Nb7+ Kc7 57. Na5 Be3 58. Ka6} 1-0

\end{multicols}