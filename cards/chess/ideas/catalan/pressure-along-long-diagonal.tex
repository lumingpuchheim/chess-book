\newpage
\section{Catalan Opening: Pressure Along the Long Diagonal}

In the Catalan, White fianchettoes the king's bishop to g2, placing it on the
long diagonal where it exerts strong pressure on the central squares d5 and e4,
and later on the queenside.

In the following game Black struggles to develop on the queenside. White
wins by following a clear strategic plan:
\begin{itemize}
    \item Pressing against the weakness on b7, forcing all of Black's pieces
          into increasingly passive positions.
    \item Creating a second weakness on g6.
    \item Infiltrating with his king to exploit both weaknesses and decide the game.
\end{itemize}

\begin{multicols}{2}

\newchessgame
\chessgameinfo{Training Match}{Garry Kasparov}{Ulf Andersson}{}{1985.06}{1-0}
\mainline[level=1]{1.d4 Nf6 2.c4 e6 3.g3 d5 4.Bg2 Be7 5.Nf3 O-O 6.Qc2 dxc4 7.Qxc4 a6 8.Bf4 Bd6 9.Ne5}

\chessboard

Because of White's bishop on the long diagonal, Black cannot develop his
queenside pieces naturally. The bishop cannot move, as it must defend the b7
pawn, and the b-pawn itself cannot advance because the rook on a8 would be
left hanging. Even \variation[invar]{9...Nbd7} does not really help Black's
development, since White can simply ignore it.

\mainline[level=1]{9...Nd5 10.Nc3 Nxf4 11.gxf4 Nd7 12.e3 Qe7
13.O-O Rb8 14.Ne4 Nf6 15.Nc5 c6}

\chessboard

Black finally completes queenside development. White now has a simple mission:
play a4 and b4, then b5, exchange on b5, and keep targeting b7 to drive Black's
pieces into passive defence.

\mainline[level=1]{16.b4 Kh8 17.a4 Nd5 18.b5 f6
19.Ned3 axb5 20.axb5 cxb5 21.Qxb5 Rd8 22.Ra7 Bxc5 23.Nxc5 Bd7
24.Nxd7 Rxd7 25.f5 g6 26.fxe6 Qxe6 27.Rfa1 Kg7 28.Qb3}

\chessboard

White is threatening winning Black's knight. Black must exchange the queens.

\mainline[level=1]{28...Nf4
29.Qxe6 Nxe6 30.Rb1 Nd8 31.Bf3 Rc8 32.Ra5 Rcc7 33.Rab5 f5}

\chessboard

Now Black is completely passive. Here the \emph{Principle of Two Weaknesses}
applies: to win a better position, especially in the endgame, it is often not
enough to attack a single weakness—you need to create a second target so your
opponent's pieces are overstretched. \index{Principle of Two Weaknesses}

White therefore creates a second weakness on g6 on the kingside.

\mainline[level=1]{
34.h4 Kf7 35.h5 Kg7 36.Kg2 Re7 37.Rb6 Rf7 38.Bd5 Rfd7 39.R1b5
Re7 40.Kg3 Red7 41.hxg6 hxg6}

\chessboard

White's king will now march over to join the attack and convert his advantage.
\mainline[level=1]{42.Kf4 Rc2 43.Kg5 Rxf2 44.Rxg6+
Kf8 45.Bb3 Nf7+ 46.Kf6 f4 47.e4 Rb2 48.e5 f3 49.e6 f2 50.Bc4
} 1-0
\end{multicols}