\newpage
\section{Catalan Opening}
The Catalan Opening is a chess opening that begins with 1.d4 and 2.c4, followed by g3 and Bg2. It often arises via the move order 1.d4 Nf6 2.c4 e6 3.g3, combining ideas from the Queen’s Gambit and the Réti.

\chessboard[
    setfen=rnbqkb1r/pppp1ppp/4pn2/8/2PP4/6P1/PP2PP1P/RNBQKBNR b KQkq - 0 3
]

The main idea of the Catalan is long-term positional pressure rather than immediate tactics. White fianchettoes the king’s bishop to g2, placing it on the long diagonal where it exerts strong pressure on the central squares d5 and e4, and later on the queenside. White usually allows tension in the center and does not rush to occupy it with pawns. \index{Endgame!Knight vs. Bishop}

A key theme is that Black may capture the pawn on c4. White often accepts this temporarily, using faster development and better piece activity to pressure Black until the pawn is recovered or Black is forced into concessions. This is known as the Open Catalan. If Black keeps the center closed and does not take on c4, the game becomes a Closed Catalan, where White slowly builds pressure against Black’s solid but somewhat passive structure.

Overall, the Catalan aims for stable central control, strong piece coordination, and a small but persistent advantage that often becomes clearer in the middlegame and endgame.

\section{Catalan Opening: Endgame}

White can steer the game toward a favorable endgame, as Ulf Andersson demonstrated in this instructive example.

\begin{multicols}{2}
\chessgameinfo{Swedish Team Championship}{Ulf Andersson}{Sergey Vladimirovich Ivanov}{}{2000}{1-0}
\newchessgame
\mainline[level=1]{1. Nf3 d5 2. d4 Nf6 3. c4 e6 4. g3 dxc4 5. Qa4+ Nbd7 6. Bg2 a6
7. Nc3 Rb8 8. Qxc4 b5 9. Qd3 Bb7 10. O-O c5 11. dxc5 Bxc5
12. Bf4 Rc8 13. Rad1 O-O 14. Ne5}

\chessboard

Many pieces will be exchanged, and Andersson deliberately heads for an endgame
where he can press with a persistent but risk-free advantage.

\mainline[level=1]{14... Bxg2 15. Kxg2 Nxe5 16. Bxe5
Qxd3 17. Rxd3 Rfd8 18. Rxd8+ Rxd8 19. Bxf6 gxf6}

\chessboard

This is the position Andersson had in mind when he played his 14th move.
How should we evaluate this endgame? A good starting point is to examine
the pawn structure.

\chessboard[
    setfen=8/5p1p/p3pp2/1p6/8/P5P1/1P2PP1P/8 w HAha - 0 1,
    showmover=false,
    markstyle=circle,
    linewidth=0.05em,
    markfields={a6, h7},
]

White has no real weaknesses in his camp. He will place his pawns on light
squares so that Black's dark-squared bishop cannot attack them.

On the other hand, Black has two clear weaknesses: the pawns on a6 and h7.
As the game continues, White will use his king to attack these pawns directly.
Also notice Black's doubled pawns on the f-file; they give White enough time
to bring his king into Black's position.

After the rooks are exchanged, White's pieces are actually too slow to attack
both weaknesses simultaneously, so Black has decent practical chances to hold.
However, White can simply improve his position and wait for a mistake, so his
winning chances remain very good.

\mainline[level=1]{20. Rc1 Be7}

\chessboard

One of the most neglected principles in chess is \emph{don't rush}. 
Before pushing forward, White must first prevent Black's rook from invading
on d2. His next move is highly instructive.

\mainline[level=1]{
21. Nb1! f5 22. e3 Bf6 23. b3 Kf8 24. Kf3 Ke7 25. h3 Rd5
26. Ke2 Kd7 27. Nd2 Be7 28. Nf3 Bf6 29. Ne1 Rd6 30. g4 fxg4
31. hxg4 Rc6 32. Rxc6 Kxc6 33. Nd3!}

\chessboard

White must now stop Black's king from invading. The d5 square is also
temporarily defended because after \variation[invar]{33...Kd5? 34. Nb4+}
Black would simply lose a pawn.

\mainline[level=1]{33...a5 34. e4!}

\chessboard 

Now d5 is also defended permanently. White is ready to march his king toward
Black's kingside while Black is reduced to waiting.
\mainline[level=1]{34...a4 35. Ke3 axb3
36. axb3 Kd6 37. Kf4 Bd8 }

\chessboard

\mainline[level=1]{38. g5!}

White now sacrifices a pawn temporarily to open lines for a decisive king invasion.

\mainline[level=1]{38...Ke7 39. Kg4 Kf8 40. f3 f6
41. Kh5 fxg5 42. Kh6 Kg8 43. Nc5 Kf7 44. Kxh7}

\chessboard

The first mission is accomplished: White's king has infiltrated Black's camp.
Next he will march toward the queenside to attack the remaining weakness.

\mainline[level=1]{44...Bb6 45. Nd3 Kf6
46. Kg8 Bg1 47. Kf8 e5 48. Ke8 Ke6 49. Kd8 Kd6 50. Kc8 Be3
51. Kb7}

\chessboard

\mainline[level=1]{51...Bd4?}

Until this point Black has defended well. After the move in the game, however,
White places him in zugzwang, wins a pawn, and with it the game. Black should
have played \variation[invar]{51... Bg1}
\mainline[level=1]{52. b4 Kd7 53. Nc5+ Kd6 54. Kb6 Bc3 55. Kxb5 Bd2
56. Nb7+ Kc7 57. Na5 Be3 58. Ka6} 1-0

\end{multicols}
\newpage
\section{Catalan Opening: Pressure Along the Long Diagonal}

In the Catalan, White fianchettoes the king's bishop to g2, placing it on the
long diagonal where it exerts strong pressure on the central squares d5 and e4,
and later on the queenside.

In the following game Black struggles to develop on the queenside. White
wins by following a clear strategic plan:
\begin{itemize}
    \item Pressing against the weakness on b7, forcing all of Black's pieces
          into increasingly passive positions.
    \item Creating a second weakness on g6.
    \item Infiltrating with his king to exploit both weaknesses and decide the game.
\end{itemize}

\begin{multicols}{2}

\newchessgame
\chessgameinfo{Training Match}{Garry Kasparov}{Ulf Andersson}{}{1985.06}{1-0}
\mainline[level=1]{1.d4 Nf6 2.c4 e6 3.g3 d5 4.Bg2 Be7 5.Nf3 O-O 6.Qc2 dxc4 7.Qxc4 a6 8.Bf4 Bd6 9.Ne5}

\chessboard

Because of White's bishop on the long diagonal, Black cannot develop his
queenside pieces naturally. The bishop cannot move, as it must defend the b7
pawn, and the b-pawn itself cannot advance because the rook on a8 would be
left hanging. Even \variation[invar]{9...Nbd7} does not really help Black's
development, since White can simply ignore it.

\mainline[level=1]{9...Nd5 10.Nc3 Nxf4 11.gxf4 Nd7 12.e3 Qe7
13.O-O Rb8 14.Ne4 Nf6 15.Nc5 c6}

\chessboard

Black finally completes queenside development. White now has a simple mission:
play a4 and b4, then b5, exchange on b5, and keep targeting b7 to drive Black's
pieces into passive defence.

\mainline[level=1]{16.b4 Kh8 17.a4 Nd5 18.b5 f6
19.Ned3 axb5 20.axb5 cxb5 21.Qxb5 Rd8 22.Ra7 Bxc5 23.Nxc5 Bd7
24.Nxd7 Rxd7 25.f5 g6 26.fxe6 Qxe6 27.Rfa1 Kg7 28.Qb3}

\chessboard

White is threatening winning Black's knight. Black must exchange the queens.

\mainline[level=1]{28...Nf4
29.Qxe6 Nxe6 30.Rb1 Nd8 31.Bf3 Rc8 32.Ra5 Rcc7 33.Rab5 f5}

\chessboard

Now Black is completely passive. Here the \emph{Principle of Two Weaknesses}
applies: to win a better position, especially in the endgame, it is often not
enough to attack a single weakness—you need to create a second target so your
opponent's pieces are overstretched. \index{Principle of Two Weaknesses}

White therefore creates a second weakness on g6 on the kingside.

\mainline[level=1]{
34.h4 Kf7 35.h5 Kg7 36.Kg2 Re7 37.Rb6 Rf7 38.Bd5 Rfd7 39.R1b5
Re7 40.Kg3 Red7 41.hxg6 hxg6}

\chessboard

White's king will now march over to join the attack and convert his advantage.
\mainline[level=1]{42.Kf4 Rc2 43.Kg5 Rxf2 44.Rxg6+
Kf8 45.Bb3 Nf7+ 46.Kf6 f4 47.e4 Rb2 48.e5 f3 49.e6 f2 50.Bc4
} 1-0
\end{multicols}
\newpage
\section{Catalan Opening: Setup Pawn Center}
If the circumstance allows, White can also setup a pawn center.

\begin{multicols}{2}
    \chessgameinfo{Amsterdam IBM Tournament}{B.Spassky}{D.Ciric}{}{1970}{1-0}

    \newchessgame
    \mainline[level=1]{
        1. d4 d5 2. c4 e6 3. Nf3 Nf6 4. g3 Be7 5. Bg2 O-O 6. O-O c6 7. b3 Nbd7 8. Bb2 b6 9. Nbd2 Bb7 10. Rc1 Rc8 11. e3}
    
        ``White is planning to place the queen on e2 and not on the usual c2 square, where
    it would be `exposed' to the c8 rook'' (Najdorf).

    \mainline[level=1]{11... c5 12. Qe2}
    
    \chessboard

    ``White is preparing \symrook fd1 and an eventual \symknight e5'' (Spassky).

    \mainline[level=1]{12... Rc7 13. cxd5}
    
    \chessboard
    
    Critical moment! The next move decides the character of the game.

    \variation[invar]{13... exd5 14. dxc5 bxc5 15. Ne5}

    \chessboard[
        setfen=3qr1k1/pbrnbppp/5n2/2ppN3/8/1P2P1P1/PB1NQPBP/2RR2K1 b - - 3 16
    ]

    Black has \vocab{hanging-pawns}{Hanging pawns}{two side-by-side pawns on adjacent files (here the c- and d-pawns) with no friendly pawns on neighboring files behind them to support them. They grant space and dynamic piece activity but can become long-term weaknesses if blockaded or forced to advance} on the c- and d-files. The position is playable for both sides.
    
    The text move allows White to setup a pawn center.
    \mainline[level=1]{13... Bxd5 14. e4 Bb7 15. e5 Nd5 16. Nc4}

    White has made some natural moves. 

    \mainline[level=1]{16...Qa8 }
    
    \chessboard 

    \mainline[level=1]{17. Nd6!}
    
    White sacrifices a pawn to activate his pieces (his pawn will fall in 
    the next a few moves). \index{Pawn Sacrifice!Piece Activity}

    \mainline[level=1]{17...Bxd6 18. exd6 Rc6 19. dxc5 bxc5 20. Ng5 Rxd6 21. Rfd1!}
    
    ``The strongest move, pinning the knight and bringing his last inactive piece into play'' (Bernard Cafferty).
    It is also good practical play, if one cannot find a decisive blow, just bring new 
    pieces into play.
    
    The double sacrifice on h7 and g7 is too hasty: \variation[invar]{21. Nxh7 Kxh7 22. Qh5+ Kg8 23. Bxg7 Kxg7 24. Qg5+ Kh8 25.Rc4
    Nf4! 26. Rxf4 Rd4!}

    \mainline[level=1]{21... Ra6}
    
    \chessboard

    \mainline[level=1]{22. Qe4}
    
    Amusingly, now the double sacrifice mentioned in the last move works because the Black rook has moved 
    to a6! Sometimes the opponent will help us. \index{Attack!Double Sacrifice}
    \variation[invar]{22. Nxh7 Kxh7 23. Qh5+ Kg8 24. Bxg7 Kxg7 25. Qg5+ Kh8 26.Rc4} White mate soon
    
    \mainline[level=1]{22...f5 23. Qc4 Qe8 24. Re1 Rxa2 25. Rxe6 Qa8 26. Bxd5 Bxd5 27. Qh4 h6}
    
    \chessboard

    \mainline[level=1]{28. Qxh6! Nf6 29. Rxf6}
    
    1-0. There is no defense after \variation[invar]{29... Rxf6 30. Qh7+ Kf8 31. Qh8+ Bg8 32. Bxf6}.
\end{multicols}