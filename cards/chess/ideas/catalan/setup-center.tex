\newpage
\section{Catalan Opening: Setup Pawn Center}
If the circumstance allows, White can also setup a pawn center.

\begin{multicols}{2}
    \chessgameinfo{Amsterdam IBM Tournament}{B.Spassky}{D.Ciric}{}{1970}{1-0}

    \newchessgame
    \mainline[level=1]{
        1. d4 d5 2. c4 e6 3. Nf3 Nf6 4. g3 Be7 5. Bg2 O-O 6. O-O c6 7. b3 Nbd7 8. Bb2 b6 9. Nbd2 Bb7 10. Rc1 Rc8 11. e3}
    
        ``White is planning to place the queen on e2 and not on the usual c2 square, where
    it would be `exposed' to the c8 rook'' (Najdorf).

    \mainline[level=1]{11... c5 12. Qe2}
    
    \chessboard

    ``White is preparing \symrook fd1 and an eventual \symknight e5'' (Spassky).

    \mainline[level=1]{12... Rc7 13. cxd5}
    
    \chessboard
    
    Critical moment! The next move decides the character of the game.

    \variation[invar]{13... exd5 14. dxc5 bxc5 15. Ne5}

    \chessboard[
        setfen=3qr1k1/pbrnbppp/5n2/2ppN3/8/1P2P1P1/PB1NQPBP/2RR2K1 b - - 3 16
    ]

    Black has \vocab{hanging-pawns}{Hanging pawns}{two side-by-side pawns on adjacent files (here the c- and d-pawns) with no friendly pawns on neighboring files behind them to support them. They grant space and dynamic piece activity but can become long-term weaknesses if blockaded or forced to advance} on the c- and d-files. The position is playable for both sides.
    
    \mainline[level=1]{13... Bxd5 14. e4 Bb7 15. e5 Nd5 16. Nc4 Qa8 17. Nd6 Bxd6 18. exd6 Rc6 19. dxc5 bxc5 20. Ng5 Rxd6 21. Rfd1 Ra6 22. Qe4 f5 23. Qc4 Qe8 24. Re1 Rxa2 25. Rxe6 Qa8 26. Bxd5 Bxd5 27. Qh4 h6 28. Qxh6 Nf6 29. Rxf6}
    1-0
\end{multicols}