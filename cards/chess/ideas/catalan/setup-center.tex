\newpage
\section{Catalan Opening: Setup Pawn Center}
If the circumstance allows, White can also setup a pawn center.

\begin{multicols}{2}
    \chessgameinfo{Amsterdam IBM Tournament}{B.Spassky}{D.Ciric}{}{1970}{1-0}

    \newchessgame
    \mainline[level=1]{
        1. d4 d5 2. c4 e6 3. Nf3 Nf6 4. g3 Be7 5. Bg2 O-O 6. O-O c6 7. b3 Nbd7 8. Bb2 b6 9. Nbd2 Bb7 10. Rc1 Rc8 11. e3}
    
        ``White is planning to place the queen on e2 and not on the usual c2 square, where
    it would be `exposed' to the c8 rook'' (Najdorf).

    \mainline[level=1]{11... c5 12. Qe2}
    
    \chessboard

    ``White is preparing \symrook fd1 and an eventual \symknight e5'' (Spassky).

    \mainline[level=1]{12... Rc7 13. cxd5}
    
    \chessboard
    
    Critical moment! The next move decides the character of the game.

    \variation[invar]{13... exd5 14. dxc5 bxc5 15. Ne5}

    \chessboard[
        setfen=3qr1k1/pbrnbppp/5n2/2ppN3/8/1P2P1P1/PB1NQPBP/2RR2K1 b - - 3 16
    ]

    Black has \vocab{hanging-pawns}{Hanging pawns}{two side-by-side pawns on adjacent files (here the c- and d-pawns) with no friendly pawns on neighboring files behind them to support them. They grant space and dynamic piece activity but can become long-term weaknesses if blockaded or forced to advance} on the c- and d-files. The position is playable for both sides.
    
    The text move allows White to setup a pawn center.
    \mainline[level=1]{13... Bxd5 14. e4 Bb7 15. e5 Nd5 16. Nc4}

    White has made some natural moves. 

    \mainline[level=1]{16...Qa8 }
    
    \chessboard 

    \mainline[level=1]{17. Nd6!}
    
    White sacrifices a pawn to activate his pieces (his pawn will fall in 
    the next a few moves). \index{Pawn Sacrifice!Piece Activity}

    \mainline[level=1]{17...Bxd6 18. exd6 Rc6 19. dxc5 bxc5 20. Ng5 Rxd6 21. Rfd1!}
    
    ``The strongest move, pinning the knight and bringing his last inactive piece into play'' (Bernard Cafferty).
    It is also good practical play, if one cannot find a decisive blow, just bring new 
    pieces into play.
    
    The double sacrifice on h7 and g7 is too hasty: \variation[invar]{21. Nxh7 Kxh7 22. Qh5+ Kg8 23. Bxg7 Kxg7 24. Qg5+ Kh8 25.Rc4
    Nf4! 26. Rxf4 Rd4!}

    \mainline[level=1]{21... Ra6}
    
    \chessboard

    \mainline[level=1]{22. Qe4}
    
    Amusingly, now the double sacrifice mentioned in the last move works because the Black rook has moved 
    to a6! Sometimes the opponent will help us. \index{Attack!Double Sacrifice}
    \variation[invar]{22. Nxh7 Kxh7 23. Qh5+ Kg8 24. Bxg7 Kxg7 25. Qg5+ Kh8 26.Rc4} White mate soon
    
    \mainline[level=1]{22...f5 23. Qc4 Qe8 24. Re1 Rxa2 25. Rxe6 Qa8 26. Bxd5 Bxd5 27. Qh4 h6}
    
    \chessboard

    \mainline[level=1]{28. Qxh6! Nf6 29. Rxf6}
    
    1-0. There is no defense after \variation[invar]{29... Rxf6 30. Qh7+ Kf8 31. Qh8+ Bg8 32. Bxf6}.
\end{multicols}