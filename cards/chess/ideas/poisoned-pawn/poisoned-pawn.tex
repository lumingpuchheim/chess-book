\newpage
\section{Poisoned Pawn}

The Poisoned Pawn Variation is any of several series of opening moves in chess in which a pawn is said to be "poisoned" because its capture can result in a positional loss of time or a loss of material. This commonly refers to a capture of the b-pawn by the queen, especially by Black.

In chess, a \vocab{poisoned-pawn}{Poisoned pawn}{
    A pawn that seems to be hanging or undefended, enticing an opponent to capture it. However, doing so often results in a significant positional or material loss due to hidden tactical threats or traps set by the opponent.
} is a pawn that seems to be hanging or undefended, enticing an opponent to capture it. However, doing so often results in a significant positional or material loss due to hidden tactical threats or traps set by the opponent.

Being aware of the concept of a poisoned pawn can protect you from a lot of trouble. If you spot a pawn hanging, it is always a good idea to double-check to see if it is poisoned. Not only that, but you can also use a poisoned pawn to set a trap for your opponents.