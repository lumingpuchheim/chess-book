\newpage
\section{Poisoned Pawn: Torre Attack}
\begin{multicols}{2}
 Spassky seized the initiative by sacrificing a pawn in the next game. His
 attack developed and the Black king never escaped. 
 \newchessgame
 \chessgameinfo{USSR Championship}{B.Spassky}{V.Osnos}{}{1963}{1-0}   
 \mainline[level=1]{
 1. d4 Nf6 2. Nf3 e6 3. Bg5 c5 4. e3 Qb6 5. Nbd2 Qxb2 }
 
 \chessboard

 Here we have another poisoned pawn position. Taking the pawn is playable for Black
 since there are no clear weaknesses and the position is closed. White, on the other hand, 
 has some lead in development. The struggle continues.

 \mainline[level=1]{6. Bd3 cxd4?!}
 
 This move opens the position. White may have the open c-file in the future.
 \variation[invar]{6... d5} is better.

 \mainline[level=1]{7. exd4 Qc3 8. O-O d5 9. Re1 Be7 10. Re3}
 
 \chessboard

 Black cannot castle now because of \variation[invar]{10...O-O 11. Bxh7}, which loses the queen.
 The natural move is to remove the queen from danger.
 \mainline[level=1]{10... Qc7 11. Ne5}
 
 \chessboard

 Black still cannot castle because of the ``Greek Gift'':
 \variation[invar]{11...O-O 12. Bxf6 Bxf6 13. Bxh7+ Kxh7 14. Qh5+ Kg8 15. Rh3} Black will be mated soon.

 \mainline[level=1]{11... Nc6 12. c3}
 
 \chessboard

 \mainline[level=1]{12... Nxe5?!}
 
 A strange move, as the knight must retreat immediately.
 Since Black cannot castle on the kingside due to the sacrifice on h7 
 as mentioned, he must try to castle on the queenside. \variation[invar]{12...Bd7} is better.

 \mainline[level=1]{13. dxe5 Ng8 }
 
 Forced. \variation[invar]{13... Nd7 14. Bxe7 Kxe7 15. Qh5} Black's king is stuck in the center. 
 
 \mainline[level=1]{14. Nf3 h6 15. Bf4 Bd7 16. Nd4}
 
 \chessboard

 It is logical for Black to exchange pieces. The question is which one.
 White's knight is much more dangerous than the bishop. Therefore, Black should play 
 \variation[invar]{16...Bc5}.

 Note that \variation[invar]{16...Qxc3 17. Nxe6} destroys Black's center.
 
 \mainline[level=1]{16... Bg5 17. Bxg5 hxg5 18. Qg4 Qxc3 19. Nb3 Nh6?}
 
 \variation[invar]{19... Ne7} is better. After the text move, White is completely winning.

 \mainline[level=1]{20. Qxg5 Qb4 21. Rg3 Qf8 }
 
 \chessboard

 \mainline[level=1]{22. Rc1!}
 
 A very logical move, bringing all pieces into the attack!

 \mainline[level=1]{22... f6 23. Qe3 f5 24. Nc5 f4 }
 
 \chessboard
 
 \mainline[level=1]{25. Bg6+! Ke7 26. Qa3} 1-0
 
\end{multicols}