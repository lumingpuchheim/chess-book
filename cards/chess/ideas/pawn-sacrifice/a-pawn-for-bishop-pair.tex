%\newpage
\section{Pawn Sacrifice: Bishop Pair}

The bishop pair is a powerful strategic advantage in chess, allowing control over both color complexes and enhancing board coverage.

Sometimes it is a good idea to sacrifice a pawn for a bishop pair. \index{Pawn Sacrifice!Bishop Pair}

\begin{multicols}{2}
    \chessgameinfo{Linares 26th}{V.Anand}{L.Aronian}{4}{2009.02.20}{0-1}
    \newchessgame
    \mainline[level=1]{
        1. d4 d5 2. c4 c6 3. Nf3 Nf6 4. Nc3 e6 5. e3 Nbd7 6. Bd3 dxc4 7. Bxc4 b5 8. Bd3 Bd6 9. O-O O-O 10. Qc2 Bb7 11. a3 a6 12. Ng5}
    
    \chessboard
    
    White sacrificed a pawn for a bishop pair and a pawn center, expecting the ``Greek Gift'' from Black.

    \mainline[level=1]{12...Bxh2+ 13. Kxh2 Ng4+ 14. Kg1 Qxg5 15. f3 Ngf6 16. e4}
    
    \chessboard

    Black must stop White to double rook on the h-file
    \mainline[level=1]{16...Qh4!}
    
    The battle continued. White got an advantage but loses the game due to a blunder.

    \mainline[level=1]{17. Be3 e5 18. Ne2 Nh5 19. Qd2 h6 20. Rfd1 Rae8 21. Bc2 Re6 22. Bf2 Qe7 23. g4 Rg6 24. Kf1 Nhf6 25. Ng3 Nxg4 26. fxg4 Qh4 27. Nf5 Qxg4 28. Qc3 Re8 29. Qg3 Qh5 30. Qh4 Qf3 31. Rd3 Qg2+ 32. Ke2 exd4 33. Rg3 Rxg3 34. Qxg3 Rxe4+ 35. Kd2 Rg4 36. Qxg2 Rxg2 37. Ke2 c5 38. Rg1 Ne5 39. Rxg2 Bxg2 40. Kd2 h5 41. b4 Nc4+ 42. Kc1 Nxa3 43. Bd1 cxb4 44. Bxh5 g6 45. Ne7+ Kf8 46. Nxg6+ fxg6 47. Bxg6 Ke7 48. Bxd4 Kd6 49. Bd3 Nc4 50. Bg7 a5 51. Be2 Be4 52. Bf6 a4 53. Bg7 Kd5
    }

\end{multicols}