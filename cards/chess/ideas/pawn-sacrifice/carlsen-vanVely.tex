\newpage
\section{Pawn Sacrifice: Infiltration}
In many endgames the guiding principle is clear: piece activity is more important than material.
A rook is especially powerful on the seventh rank, where it attacks pawns and restricts the enemy king.
It is therefore quite natural to consider sacrificing a pawn in order to infiltrate with the rook.
\index{Pawn Sacrifice!Infiltration}

\begin{multicols}{2}
\newchessgame[
    setfen=2rr2k1/pb1nbpp1/1qp1p2p/1p6/3PN3/4PN2/PPQ1BPPP/2RR2K1 w - - 8 17,
    moveid=17w
]
\chessboard

In this position, Carlsen sacrificed a pawn to infiltrate with his rook. \index{Pawn Sacrifice!Infiltration}
\index{Endgame!Infiltration}

\mainline[level=1]{17. Nc5 Nxc5 18. dxc5 Rxd1+ 19. Rxd1 Qxc5 20. Qxc5 Bxc5 21. Rd7 }

\chessboard

\mainline{21... Ba8 22. Ne5 Bb6 23. Nxf7 Rc7}

\chessboard

Carlsen could have played \variation[invar]{24. Nxh6}, winning a pawn.
Instead, he preferred to keep the initiative and the activity of his rook.


\chessgameinfo{Candidates qf2}{Tigran Petrosian}{Lajos Portisch}{13}{1974}{1-0}
\newchessgame[
    setfen=3r2k1/p4pp1/1p2n2p/3p4/3P1q1P/1P1Q1NP1/P4PK1/2R5 w - - 1 22,
    moveid=22w
]

\chessboard

Black has just set up a ``trap'' for White.

\mainline[level=1]{
22. gxf4!}

White went for it, sacrificing a pawn. As we will see, his rook soon
infiltrated the enemy camp.

\mainline[level=1]{22... Nxf4+ 23. Kg3 Nxd3 24. Rc3 Nb4 25. a3 Na6 
26. b4 Nb8 27. Rc7}

\chessboard

This was the position Petrosian was aiming for. White had infiltrated his rook to 
the seventh rank while Black was totally passive.

\mainline[level=1]{27... a5 28. b5 Nd7 29. Kf4 h5 30. Ne5!}

\chessboard

White pushed Black's knight even further back. Exchanging the knights would be 
disastrous for Black: \variation[invar]{
    30...  Nxe5 31. Kxe5 f6+ 32. Kf5 Kh7 33. Ke6 Kg6 34. Rc6 
}

White eventually converted his advantage and won the game.

\mainline[level=1]{30... Nf8 31. Rb7 f6 32. Nc6 Ng6+ 33. Kg3 Rd6 34. Rxb6 Re6 35. Rb8+ Nf8 36. Ra8 Re1 37. Nd8 Kh7 38. b6 Rb1 39. b7 Nd7 40. Rxa5
}
\end{multicols}