\newpage
\section{Carlsbad Structure}
\newchessgame


The Carlsbad pawn structure can be reached via many openings but is best 
known from the Queen's Gambit Declined Exchange Variation, which occurs after 

\mainline[level=1]{1.d4 d5 2.c4 e6 3.Nc3 Nf6 4.cxd5 exd5 5.Bg5 Be7 6.e3 c6}

\chessboard[
    setfen=r1bqk2r/pp1nbppp/2p2n2/3p2B1/3P4/2NBP3/PP3PPP/R2QK1NR w KQkq - 2 8
]

The Carlsbad pawn structure is important because it is full of rich possibilities 
for both sides and because the structure that can be reached via many openings, 
including the Nimzo-Indian, Caro-Kann, Scandinavian, and multiple other queen's 
pawn openings.

\begin{itemize}
    \item{Minority Attack}

    White advances his queenside pawns with the main purpose of creating weaknesses in black’s structure.

    \chessboard[
        setfen=8/pp3ppp/2p5/3p4/3P4/4P3/PP3PPP/8 w KQkq - 2 8,
        pgfstyle=straightmove,
        linewidth=0.05em,
        markmove={b2-b4, b4-b5, b5-c6, b7-c6},
    ]
    
    \item{Playing for the e3-e4 push}

    This can be done with or without the support of the f pawn (by pushing f3) then e4.

    \chessboard[
        setfen=8/pp3ppp/2p5/3p4/3P4/4P3/PP3PPP/8 w KQkq - 2 8,
        pgfstyle=straightmove,
        linewidth=0.05em,
        markmove={e3-e4},
    ]   
\end{itemize}

Black must defend accordingly. His ideas are
\begin{itemize}
    \item{Moving the knight from g8 to e4}
    \item{Moving the knight from b8 to c4 via d7 and b6.}
    \item{Idea from Petrosian, moving the knight from b8 to d6 via d7, b6 and c4, in anticipation of
    the standard minority attack b2-b4-b5. \footnote{See the game Bobotsov-Petrosian, Lugano Olympiad 1968 \cite{kasparov:2004}}}
\end{itemize}

\newpage

\section{Carlsbad Structure: White Center Break}

\begin{multicols}{2}
    In this section, we focus on the center-break plan for White.
    Typically, White needs to complete these steps to break in the center:

    \begin{itemize}
        \item{f2-f3}
        \item{e3-e4}
        \item{e4-e5}
        \item{An attack on the king side}
    \end{itemize}

    We will illustrate the idea by a famous game. \index{Carlsbad Structure!White Center Break}

    \chessgameinfo{USSR Championship}{M.Botvinnik}{P.Keres}{}{1952.12.09}{1-0}
    \newchessgame
    \mainline[level=1]{
        1.d4 Nf6 2.c4 e6 3.Nc3 d5 4.cxd5 exd5 5.Bg5 Be7 6.e3 O-O 7.Bd3
        Nbd7 8.Qc2 Re8 9.Nge2 Nf8 10.O-O c6 }
    
    \chessboard

    We reach a typical Queen's Gambit Declined Exchange Variation position.
    One typical idea for White is the minority attack.

    \mainline[level=1]{11.Rab1}
    
    \chessboard

    \mainline[level=1]{11...Bd6}

    This move doesn't have a clear plan. The text move threats \symbishop xh2.
    However the threat is easy to defend and we will see soon that the bishop 
    will go back.  

    \mainline[level=1]{12.Kh1 Ng6}

    \chessboard

    \mainline[level=1]{
        13.f3}
    
    First mission accomplished. Now White aims for e4 and directs all pieces toward that square.
        
    \mainline{13...Be7 14.Rbe1! Nd7?}
    
    Black loses two tempi because the knight will return to f6. \variation[invar]{14... h6} is the best move.

    \mainline[level=1]{15.Bxe7 Rxe7 16.Ng3! Nf6 17.Qf2}
    
    Note how White prepared for the e4 break. In the last move, he protected the d4 
    pawn with his queen, since it will be loose after e4.
    
    \mainline[level=1]{17...Be6
        18.Nf5 Bxf5 19.Bxf5 Qb6 }
    
    \chessboard

    Now the second mission is achieved. It is instructive to notice the piece maneuvers:
    \symknight g1 - e2 - g3, 
    \symrook a1 - b1 - e1,
    \symqueen d1 - f2 
    
    All the pieces are helping for the e4 break.
    \mainline[level=1]{20.e4! dxe4 21.fxe4 Rd8}
    
    \chessboard

    Now for the third mission: White launches a kingside attack and wins.
    \mainline[level=1]{22.e5 Nd5 23.Ne4 Nf8 24.Nd6 Qc7 25.Be4 Ne6 26.Qh4 g6 27.Bxd5 cxd5 28.Rc1 Qd7 29.Rc3 Rf8 30.Nf5 Rfe8 31.Nh6+ Kf8 32.Qf6 Ng7 33.Rcf3 Rc8 34.Nxf7 Re6 35.Qg5 Nf5 36.Nh6 Qg7 37.g4} Black resigned.

    Actually such a plan doesn't need to be restricted in Carlsbad structure, it 
    can also be used in Nimzo-Indian defense. The next game is a classic.

    \chessgameinfo{AVRO}{M.Botvinnik}{J.Capablanca}{11}{1938.11.22}{1-0}
    \newchessgame

    \mainline[level=1]{1.d4 Nf6 2.c4 e6 3.Nc3 Bb4 4.e3 d5 5.a3 Bxc3+ 6.bxc3 c5 7.cxd5
    exd5 8.Bd3 O-O 9.Ne2 b6 10.O-O Ba6 11.Bxa6 Nxa6 12.Bb2 Qd7
    13.a4 Rfe8 14.Qd3 c4}
    
    \chessboard

    White is preparing for f3 and e4. Notice the piece maneuvers, we know them already. 
    Certainly Botvinnik did remember his plan and just played them during the game.
    
    \mainline[level=1]{15.Qc2 Nb8 16.Rae1! Nc6 17.Ng3! Na5 18.f3! Nb3 19.e4! Qxa4}
    
    Mission accomplished: f3 and e4. The next is e5 winning a tempo.

    \mainline[level=1]{20.e5 Nd7 21.Qf2 g6 22.f4 f5 23.exf6 Nxf6 24.f5
    Rxe1 25.Rxe1 Re8 26.Re6 Rxe6 27.fxe6 Kg7 28.Qf4 Qe8 29.Qe5 Qe7}

    \chessboard

    This is the classic finish:
    \mainline[level=1]{30.Ba3 Qxa3 31.Nh5+ gxh5 32.Qg5+ Kf8 33.Qxf6+ Kg8 34.e7 Qc1+
    35.Kf2 Qc2+ 36.Kg3 Qd3+ 37.Kh4 Qe4+ 38.Kxh5 Qe2+ 39.Kh4 Qe4+
    40.g4 Qe1+ 41.Kh5} Black resigned.

\end{multicols}
\newpage
\section{Minority Attack}
In chess, a minority attack \index{Minority Attack} is the advancement of one's pawns on the side of the board where one has fewer pawns than their opponent, intending to use their minority to strategically provoke a weakness (i.e, an isolated or backward pawn) in the opponent's pawn structure. The minority attack is a common middlegame plan that can be played in many pawn structures. The name might be misleading, as the "attack" does not involve tactics planned to produce checkmate or significant material gain, but rather a strategical and structural advantage for the attacking player.

The minority attack can be strengthened by the moving of one or both rooks to the files where the attacking player intends to advance their pawns, planning prophylactically for the opening of the files. Common openings that result in pawn structures where a minority attack is effective include the Queen's Gambit Declined and the Caro-Kann Defense. The minority attack occurs most commonly on the queenside, as players commonly castle kingside in openings where a minority attack is effective, and the advancement of the pawns on the side of the castled king is widely considered to severely weaken the king's safety.

\begin{multicols}{2}
\subsection*{Basic Form}
White thrusts a- and b-pawns to create a weakness for black on c6.


\newchessgame[
id=main,
moveid=23w,print,
showmover,
mover=b,% has no effect
castling=Q,enpassant=a3,
setwhite={pa2,pb2,pd4,pe3,pf2,pg2,ph2},
addblack={pa7,pb7,pc6,pd5,pf7,pg7,ph7}]


White has the following moves to start a minority attack:
\begin{itemize}
    \item{1. b4}
    \item{1. a4}
    \item{1. b5}
\end{itemize}

\chessgameinfo{}{Capablanca}{Golombek}{}{1939}{1-0}
%\centering
\newchessgame[
id=main,
moveid=23w,print,
showmover,
mover=b,% has no effect
castling=Q,enpassant=a3,
setwhite={pa2,pb4,pd4,pe3,pf2,pg2,ph3,qd3,kg1,rb1,rc1,na4},
addblack={pa7,pb7,pc6,pd5,pf7,pg6,ph6,kg8,qd6,ra8,re8,ng7}]

\mainline[level=1]{23.b5 cxb5}

White continues the minority attack and favorably changes the pawn structure. If black allows white to capture on c6, then he will have a backward c6-pawn. If black captures on b5, white will recapture with the queen and can target the isolated d5 and b7 pawn.

\mainline{24.Qxb5 Ne6}

\chessboard

\mainline{25. Nc3!}

Much better than 
\variation{25. Qxb7} which gives black chances after \variation{25... Reb8 26. Qc6 Qxc6 27. Rxb8+ Rxb8 28. Rxc6 Rb1+}.

\mainline{25... Red8 26. Qxb7 Qa3 27.Nxd5 Qxa2}

\chessboard

\mainline{28.Nb4 Qa4 29. Nc6}
1-0

For white not only threatens the rook but also the queen with \variation{30. Ra1}.

\end{multicols}