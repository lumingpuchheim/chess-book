\newpage
\section{Carlsbad Structure: Black Kingside Offense}

If Black can suppress White's minority attack, he can launch a kingside attack himself.

\chessgameinfo{Chess Olympiad Final-A}{Milko Bobotsov}{Tigran Vartanovich Petrosian}{}{1968.10.26}{0-1}
\newchessgame
\mainline[level=1]{
1. d4 Nf6 2. c4 e6 3. Nf3 d5 4. cxd5 exd5 5. Nc3 c6 6. Bg5 Be7
7. Qc2 g6 8. e3 Bf5 9. Bd3 Bxd3 10. Qxd3 Nbd7 11. Bh6 Ng4
12. Bf4 O-O 13. O-O Re8 14. h3 Ngf6 15. Ne5}

\chessboard

We will see how Black can suppress White's minority attack with his mighty knight.
The knight will move to d6 via b6 and c4. \index{Carlsbad Structure!Black Knight on d6}

Note Black can succeed in his knight maneuver because White has no light 
square bishop. Otherwise, he can play \symbishop xc4 when Black plays ...\symknight c4.

\mainline[level=1]{15...Nb6! 16. Bg5 Ne4
17. Bxe7 Qxe7 18. Qc2 Nd6! 19. Na4 Nbc4 20. Nxc4 Nxc4 21. Nc5
Nd6!}

\chessboard

Black has completely suppressed White's minority attack. Now he can launch a kingside attack.
\index{Carlsbad Structure!Black Kingside Attack}
His plan is quite simple:
\begin{itemize}
    \item{Move the knight to e4}
    \item{Move the rooks behind g-pawn}
    \item{Thrust his g- and h-pawn}
\end{itemize}

On the other hand, White doesn't have a plan at all, because his position is difficult.



\mainline[level=1]{22. Rac1 Qg5 23. Qd1 h5 24. Kh1 Re7 25. Nd3 Ne4 26. Nc5
Nd6 27. Nd3 Qf5 28. Ne5 f6 29. Nf3 Rg7 30. Nh2 Re8 31. Kg1 Ne4
32. Qf3 Qe6 33. Rfd1 g5}

\chessboard

Here Black sacrifices a pawn
to start his attack. Soon a strong attack develops and Black wins quickly.

\mainline[level=1]{34. Qxh5 f5 35. Re1 g4 36. hxg4 fxg4
37. f3 gxf3 38. Nxf3 Rh7 39. Qe5 Qc8 40. Qf4 Rf8 41. Qe5 Rf5}