\newpage

\section{Carlsbad Structure: White Center Break}

\begin{multicols}{2}
    In this section, we focus on the center-break plan for White.
    Typically, White needs to complete these steps to break in the center:

    \begin{itemize}
        \item{f2-f3}
        \item{e3-e4}
        \item{e4-e5}
        \item{An attack on the king side}
    \end{itemize}

    We will illustrate the idea by a famous game. \index{Carlsbad Structure!White Center Break}

    \chessgameinfo{USSR Championship}{M.Botvinnik}{P.Keres}{}{1952.12.09}{1-0}
    \newchessgame
    \mainline[level=1]{
        1.d4 Nf6 2.c4 e6 3.Nc3 d5 4.cxd5 exd5 5.Bg5 Be7 6.e3 O-O 7.Bd3
        Nbd7 8.Qc2 Re8 9.Nge2 Nf8 10.O-O c6 }
    
    \chessboard

    We reach a typical Queen's Gambit Declined Exchange Variation position.
    One typical idea for White is the minority attack.

    \mainline[level=1]{11.Rab1}
    
    \chessboard

    \mainline[level=1]{11...Bd6}

    This move doesn't have a clear plan. The text move threats \symbishop xh2.
    However the threat is easy to defend and we will see soon that the bishop 
    will go back.  

    \mainline[level=1]{12.Kh1 Ng6}

    \chessboard

    \mainline[level=1]{
        13.f3}
    
    First mission accomplished. Now White aims for e4 and directs all pieces toward that square.
        
    \mainline{13...Be7 14.Rbe1! Nd7?}
    
    Black loses two tempi because the knight will return to f6. \variation[invar]{14... h6} is the best move.

    \mainline[level=1]{15.Bxe7 Rxe7 16.Ng3! Nf6 17.Qf2}
    
    Note how White prepared for the e4 break. In the last move, he protected the d4 
    pawn with his queen, since it will be loose after e4.
    
    \mainline[level=1]{17...Be6
        18.Nf5 Bxf5 19.Bxf5 Qb6 }
    
    \chessboard

    Now the second mission is achieved. It is instructive to notice the piece maneuvers:
    \symknight g1 - e2 - g3, 
    \symrook a1 - b1 - e1,
    \symqueen d1 - f2 
    
    All the pieces are helping for the e4 break.
    \mainline[level=1]{20.e4! dxe4 21.fxe4 Rd8}
    
    \chessboard

    Now for the third mission: White launches a kingside attack and wins.
    \mainline[level=1]{22.e5 Nd5 23.Ne4 Nf8 24.Nd6 Qc7 25.Be4 Ne6 26.Qh4 g6 27.Bxd5 cxd5 28.Rc1 Qd7 29.Rc3 Rf8 30.Nf5 Rfe8 31.Nh6+ Kf8 32.Qf6 Ng7 33.Rcf3 Rc8 34.Nxf7 Re6 35.Qg5 Nf5 36.Nh6 Qg7 37.g4} Black resigned.

    Actually such a plan doesn't need to be restricted in Carlsbad structure, it 
    can also be used in Nimzo-Indian defense. The next game is a classic.

    \chessgameinfo{AVRO}{M.Botvinnik}{J.Capablanca}{11}{1938.11.22}{1-0}
    \newchessgame

    \mainline[level=1]{1.d4 Nf6 2.c4 e6 3.Nc3 Bb4 4.e3 d5 5.a3 Bxc3+ 6.bxc3 c5 7.cxd5
    exd5 8.Bd3 O-O 9.Ne2 b6 10.O-O Ba6 11.Bxa6 Nxa6 12.Bb2 Qd7
    13.a4 Rfe8 14.Qd3 c4}
    
    \chessboard

    White is preparing for f3 and e4. Notice the piece maneuvers, we know them already. 
    Certainly Botvinnik did remember his plan and just played them during the game.
    
    \mainline[level=1]{15.Qc2 Nb8 16.Rae1! Nc6 17.Ng3! Na5 18.f3! Nb3 19.e4! Qxa4}
    
    Mission accomplished: f3 and e4. The next is e5 winning a tempo.

    \mainline[level=1]{20.e5 Nd7 21.Qf2 g6 22.f4 f5 23.exf6 Nxf6 24.f5
    Rxe1 25.Rxe1 Re8 26.Re6 Rxe6 27.fxe6 Kg7 28.Qf4 Qe8 29.Qe5 Qe7}

    \chessboard

    This is the classic finish:
    \mainline[level=1]{30.Ba3 Qxa3 31.Nh5+ gxh5 32.Qg5+ Kf8 33.Qxf6+ Kg8 34.e7 Qc1+
    35.Kf2 Qc2+ 36.Kg3 Qd3+ 37.Kh4 Qe4+ 38.Kxh5 Qe2+ 39.Kh4 Qe4+
    40.g4 Qe1+ 41.Kh5} Black resigned.

\end{multicols}