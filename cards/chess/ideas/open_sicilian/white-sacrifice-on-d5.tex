\newpage
\section{Najdorf Sicilian: White Sacrifice on d5}

\nocite{franco:2015}
When the Black king remains in the center, White often has the tactical option of 
 sacrificing on d5. If Black accepts the sacrifice, White opens the e-file with 
 exd5, gaining time and lines against the uncastled king and 
 obtaining a dangerous attack. Black must be very careful not to allow such a sacrifice 
 under favorable circumstances for White.

\begin{multicols}{2}
\chessgameinfo{USSR Chess Championship}{B.Spassky}{L.Polugaevsky}{}{1958}{}
\newchessgame
\mainline[level=1]{
    1. e4 c5 2. Nf3 d6 3. d4 cxd4 4. Nxd4 Nf6 5. Nc3 a6 6. Bg5 Nbd7 
    7. Bc4 Qa5 8. Qd2 e6 9. O-O-O b5 
}

\chessboard 

White can try the sacrifice and obtain an advantage if Black accepts: \variation[invar]{10. Bd5?! exd5?} 

\chessboard[
    setfen=r1b1kb1r/3n1ppp/p2p1n2/qp1p2B1/3NP3/2N5/PPPQ1PPP/2KR3R w kq - 0 11
]

After \variation[invar]{11. Nc6 Qb6 12. exd5} followed by \variation[invar]{13.Rhe1}, White 
 has a clear advantage, with the e-file opened and Black's king exposed in the center.

However, after the more precise \variation[invar]{10... b4 11. Bxa8 bxc3 12. bxc3 Qa2}, 
 Black has strong counterplay in the following position. \index{Najdorf Sicilian!White Sacrifice on d5}

\chessboard[
    setfen=B1b1kb1r/3n1ppp/p2ppn2/6B1/3NP3/2P5/q1PQ1PPP/2KR3R w k - 0 13
]

We now return to the main line.
\mainline[level=1]{10. Bb3}


 First, let's look at the natural but mistaken pawn thrust \variation[invar]{10... b4?}.

\chessboard[
    setfen=r1b1kb1r/3n1ppp/p2ppn2/q5B1/1p1NP3/1BN5/PPPQ1PPP/2KR3R w kq - 0 11
]

 Now White can play \variation[]{11. Nd5 exd5 12. exd5 Qb5 13. Rhe1}, obtaining 
  a strong, long-term attack along the e-file and against the black king.

 We now return to the actual game: \mainline[level=1]{10... Bb7 11. Rhe1 Be7 12. f4 Nc5}.

 Once again, if Black plays too ambitiously with \variation[invar]{12... b4 13. Nd5! exd5 14. exd5 Kf8} follows, and the king is driven into danger.

\chessboard[
    setfen=r4k1r/1b1nbppp/p2p1n2/q2P2B1/1p1N1P2/1B6/PPPQ2PP/2KRR3 w - - 1 15
]

 After the brilliant sequence 
 \variation[]{15. Rxe7! Kxe7 16. Nc6+ Bxc6 17. dxc6 Qc5 18. Qe2+ Kd8 19. cxd7 Kxd7 20. Bxf7}, 
  White obtains a powerful and enduring attack.

 The full game also features a later sacrifice on e6, which we will examine 
  in detail in a later section.
\end{multicols}