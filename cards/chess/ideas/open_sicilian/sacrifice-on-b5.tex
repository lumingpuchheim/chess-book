\newpage
\section{Najdorf Sicilian: White Sacrifice/Exchange on b5}

A sacrifice on b5 is another common theme in the Najdorf Sicilian.
White exchanges a piece (typically a bishop) for pawns, aiming 
for rolling passed pawns on the queenside or targeting Black's
king in the center.

\begin{multicols}{2}    
    \chessgameinfo{Tilburg rapid}{Vassily Ivanchuk}{Erik Gustaf Ferdinand Hellers}{}{1993.11.18}{1-0}
    \newchessgame
    \mainline[level=1]{1. e4 c5 2. Nf3 d6 3. d4 cxd4 4. Nxd4 Nf6 5. Nc3 Nc6 6. Bg5 e6 7. Qd2 a6 8.
    O-O-O h6 9. Be3 Bd7 10. f3 b5 11. Bd3 Qc7}
    
    \chessboard


    \mainline[level=1]{12. Bxb5 axb5 13. Ndxb5 Qb8 14. Nxd6+ Bxd6 15. Qxd6 Qxd6 16. Rxd6}
    
    \chessboard

    Here we see the typical exchange. After giving up a bishop, White's
    knight targets both the queen (which must retreat) and the pawn on d6.

    This is not really a sacrifice. White exchanges a bishop for three pawns. In return 
    he obtains more coordinated pieces, and his pawns on the queenside will roll at some point.

    White eventually wins.
    \mainline[level=1]{16...Na5 17. b3 Rc8 18. Kb2 Bc6 19. Rhd1 O-O 20. Bb6 Nb7 21. R6d2 Ra8 22. a4 Rfc8 23. Nb5 Ne8 24. Na7 Rxa7 25. Bxa7 Nc7 26. c4 Kf8 27. Kc3 Ke8 28. Bb6 Na6 29. a5 f6 30. g4 e5 31. h4 Nb8 32. g5 hxg5 33. hxg5 Nd7 34. gxf6 gxf6 35. Rh1 Kf7 36. Rh7+ Ke6 37. Rg2 Ra8 38. Rgg7 Kd6 39. b4 Nxb6 40. axb6 Ra3+ 41. Kb2 Rxf3 42. Rxb7 Bxb7 43. Rxb7 Kc6 44. Rb8 Rf4 45. b5+ Kc5 46. b7 Kb6 47. c5+ Kc7 48. Rf8 Kxb7 49. Rf7+ Kb8 50. b6 Rxe4 51. Kb3 Re1 52. Kc4}

When discussing this sacrifice, we must also mention Mikhail Tal. His sacrifice 
idea has withstood the test of time, as it was even used in the year 2000.

\chessgameinfo{Hastings 1973/74}{Mikhail Tal}{Michael Francis Stean}{}{1974.01.09}{1-0}
\newchessgame
\mainline[level=1]{
    1. e4 c5 2. Nf3 d6 3. d4 cxd4 4. Nxd4 Nf6 5. Nc3 a6 6. Bg5 e6 7. f4 Nbd7 8. Qf3 Qc7 9. O-O-O b5 }
    
    \chessboard

    Again we see the same theme: White gives up his bishop for two pawns, and his knight
    hits Black's queen with a tempo. In this game, Tal aims his attack against Black's king in 
    the center.

    \mainline[level=1]{10. Bxb5 axb5 11. Ndxb5 Qb8}
    
    \chessboard

    \mainline[level=1]{12. e5 }
    
    Again we have an e4-f4 center, and White thrusts with e5. Unlike 
    exchanging pawns to open the f-file, this time White 
    sacrifices his pawn to open the d-file.
    \mainline[level=1]{12... Bb7 13. Qe2 dxe5 14. Qc4 Bc5 15. Bxf6 gxf6 }

    \chessboard
    
    White must open the d-file to continue his attack; otherwise his 
    material deficit will tell after Black consolidates.
    \mainline[level=1]{16. Rxd7 Be3+ 17. Kb1 Kxd7 18. Rd1+ }
    
    \chessboard

    \mainline[level=1]{18... Bd4?}

    Until now Stean has defended well, but here he falters. He must play \variation[invar]{18... Ke7}.
    After the text move, White continues his attack by exchanging Black's important 
    dark-squared bishop. Tal eventually wins the game.

    \mainline[level=1]{19. fxe5 fxe5 20. Nxd4 exd4 21. Qxd4+ Ke7 22. Qc5+ Kf6 23. Rf1+ Kg6 24. Qe7 f5 25. Qxe6+ Kg7 26. Qe7+ Kg6 27. h4 Ra5 28. h5+ Kxh5 29. Qf7+ Kh4 30. Qf6+ Kg3 31. Qg5+ Kh2 32. Qh4+ Kxg2 33. Rf2+ Kg1 34. Ne2# }
    
    
\end{multicols}