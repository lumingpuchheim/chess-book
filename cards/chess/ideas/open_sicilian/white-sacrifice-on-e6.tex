\newpage
\section{Najdorf Sicilian: White Sacrifice on e6}

Quite often Black protects his e6 pawn with f7 pawn only, 
white White targets the pawn with a light-squared bishop and a 
knight on d4. White can sacrifice on e6 when he has enough compensation,
for example, the Black king must stay in the center for a long time.

\begin{multicols}{2}

\chessgameinfo{Bled-Zagreb-Belgrade Candidates}{Mikhail Tal}{Tigran Vartanovich Petrosian}{}{1959.10.25}{1/2-1/2}
\newchessgame
\mainline[level=1]{
    1. e4 c5 2. Nf3 d6 3. d4 cxd4 4. Nxd4 Nf6 5. Nc3 Nbd7 6. Bc4 a6 7. Bg5 Qa5?! 8. Qd2 e6 9. O-O h6 10. Bh4 g5 11. Bg3 Nh5 }
    
\chessboard

Here White has two attackers towards the e6 pawn while Black has only one. So 
a sacrifice is possible. Which piece should White sacrifice? Sacrificing a 
knight would be incorrect since White has not enough open lines for the bishop.
On the other hand, sacrificing a bishop allows him to lock the Black king in 
the center.

    \mainline[level=1]{12. Bxe6 fxe6 13. Nxe6 }
    
    \chessboard
    
    \mainline[level=1]{13...Nxg3?}
    
    It is understandable that Black want to exchange pieces to 
    defend. After all, White's bishop is targeting the d6 pawn.

    It is however difficult for a human being to find the right defence:
    \variation[invar]{13...Ndf6 14. Nxf8 Kxf8 15. Bxd6+ Kg7 16. e5 Nh7 17. b4 Qd8 18. Nd5 Be6 19. c4 Re8 20. Rac1 }

    After the text move, Black's king is still in the center and he cannot develop his queenside pieces.
    
    \mainline[level=1]{14. fxg3 Ne5 15. Rxf8+ Rxf8 16. Qxd6 Rf6}
    
    \chessboard

    \mainline[level=1]{17. Nc7+?}
    
    Losing all the advantages! \variation[invar]{17. Qc7 Qxc7 18. Nxc7+ Kd8 19. Nxa8} is better.
    After the queens are exchanged, White's material advantage is palpable. \index{Najdorf Sicilian!White Sacrifice on e6}
    
    \mainline[level=1]{17...Kf7 18. Rf1 Rxf1+ 19. Kxf1 Nc4 20. Qxh6 Qc5? }

    After \variation[invar]{20... Qe5}, the position is equal.
    White cannot check on h8 to get advantage as we will see because it is covered by the Black queen.
    
    \chessboard 

    \mainline[level=1]{21. Nxa8? }

    Allows Black to draw with perpetual check.
    \variation[invar]{21. Qh7+ Kf8 22. Qh8+ Kf7 23. N3d5 Ne3+ 24. Nxe3 Qxe3 25. Qe8+ Kg7 26. Qe5+ Kg6 27. Nxa8} gives
    White chances to win.
    
    \mainline[level=1]{21...Nd2+ 22. Ke2 Bg4+ 23. Kd3 Qc4+ 24. Ke3 Qc5+ 25. Kd3} draw with perpetual check.
    
\end{multicols}