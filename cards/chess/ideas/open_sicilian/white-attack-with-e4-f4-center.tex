\newpage
\section{Najdorf Sicilian: White Attack with e4 and f4 Center}

Often in Najdorf Sicilian positions, White sets up a center with e4 and f4:

\chessboard[
    setfen=8/pp3ppp/3pp3/8/4PP2/8/PPP3PP/8 w HAha - 0 1,
    pgfstyle=straightmove,
    markmove={e4-e5, f4-f5},
]

White can either play e5 to open the f-file for the rook or sometimes f5
to open the a2--g8 diagonal for the bishop.

\begin{multicols}{2}
    \chessgameinfo{USSR Chess Championship}{B.Spassky}{L.Polugaevsky}{}{1958}{}
    \newchessgame
    \mainline[level=1]{
        1. e4 c5 2. Nf3 d6 3. d4 cxd4 4. Nxd4 Nf6 5. Nc3 a6 6. Bg5 Nbd7 
        7. Bc4 Qa5 8. Qd2 e6 9. O-O-O b5 10. Bb3 Bb7 11. Rhe1 Be7 12. f4 Nc5}.
    
    \chessboard

    White sets up a center with e4 and f4 and then thrusts with e5 to open the f-file.
 
\mainline[level=1]{13. e5}

Good idea, but the move order is wrong. The correct sequence is 
\variation[invar]{13. Bxf6 Bxf6 14. e5}.

\mainline[level=1]{13... dxe5 14. Bxf6 Bxf6?}

\variation[invar]{14... gxf6} is more accurate.

\mainline[level=1]{15. fxe5 Bh4 16. g3 Be7}

\chessboard

\mainline[level=1]{17. Bxe6!}

Here we see this thematic attack on e6 again. It is not a true sacrifice, 
since Black cannot safely accept it.

\chessboard

Accepting the sacrifice is a mistake: \variation[invar]{17... fxe6 18. Nxe6 Rd8 19. Nxg7+ \xskakcomment{This is a typical idea after the sacrifice: the g7-pawn is unprotected and White captures it with check.} Kf7 20. Qh6 \xskakcomment{ White has a crushing attack.} }

\mainline[level=1]{17... O-O 18. Bb3 Rad8 19. Qf4 b4 20. Na4 h6 21. Nxc5 Qxc5 22. h4 }

\chessboard

White emerges a pawn up and with the initiative, and he eventually wins the game.

\mainline[level=1]{22... Bd5 23. Nf5 Bxb3 24. axb3 Rxd1+ 25. Rxd1 Rc8 26. Qe4 Bf8 27. e6 fxe6 28. Qxe6+ Kh8 29. Qe4 Qc6 30. Qd3 Re8 31. h5 Be7 32. Nxe7 Rxe7 33. Qg6 Qe8 34. g4 Re1 35. Qxe8+ Rxe8 36. Rd4 a5 37. Kd2 Re5 38. c3 bxc3+ 39. bxc3 Rg5 40. c4 Kg8 41. Rf4 g6}

\chessgameinfo{Solidarity Tournament }{Robert Fischer}{Efim Geller}{}{1967.08.01}{0-1}
\newchessgame
\mainline[level=1]{
    1. e4 c5 2. Nf3 d6 3. d4 cxd4 4. Nxd4 Nf6 5. Nc3 Nc6 6. Bc4 e6 7. Be3 Be7 8. Bb3 O-O 9. Qe2 Qa5 10. O-O-O Nxd4 11. Bxd4 Bd7}
    
    \chessboard
    
    \mainline[level=1]{12. Kb1! } We see another prophylactic king move. 
    Such a move is often useful, because it overprotects the a2-pawn.

    \variation[invar]{12. f4} is too hasty: \variation{12... e5! 13. Be3 Bg4} and White 
    loses an exchange. White must play \variation{13. Nd5 Nxd5 14. exd5 Bb5} and the position is 
    equal.
    
\mainline[level=1]{12... Bc6?!}

\variation[invar]{12... b5} is more direct and better. The text move allows White to thrust 
his f-pawn to f4, since there is no \symbishop g4 resource.

\mainline[level=1]{13. f4 Rad8 14. Rhf1 b5}

\chessboard

``The die is cast. I didn't want to lose a tempo playing it safe with 15. a3'' (Fischer) 

Interestingly, White also has other typical ideas with some amusing variations:
\variation[invar]{15. e5 dxe5 16. fxe5 Nd7 17. Qg4 b4? 18. Rxf7! Rxf7 19. Qxe6 Rf8 20. Qxe7 bxc3 21. Bxf7+ Rxf7 22. Qe8+ Rf8 23. Qe6+ Kh8 24. Qxc6 } White is winning;

\variation[invar]{15. g4 b4 16. g5 Nd7 17. f5 bxc3 18. Bxc3 Qb5 19. Bc4 Qc5 20. f6 }.
With the help of the g-pawn, White's f-pawn manages to march to f6.

\chessboard[
    setfen=3r1rk1/p2nbppp/2bppP2/2q3P1/2B1P3/2B5/PPP1Q2P/1K1R1R2 b - - 0 20
]

We now return to the main line.
\mainline[level=1]{15. f5 b4 16. fxe6 bxc3 17. exf7+ Kh8 18. Rf5 Qb4}

``Reminding White that he too has a vulnerable point on b2. In some cases Black has in mind ...\symrook xf7'' (Geller)
\chessboard

\mainline[level=1]{19. Qf1!}

``A hard move to find - it took around 45 minutes. The threat 
of \variation[invar]{20. Rxf6} must be attended to'' (Fischer)
After \variation{21... Bxf6 22. Qxf6 gxf6 23. Bxf6#}

\mainline[level=1]{19... Nxe4}

\chessboard

\mainline[level=1]{20. a3?}

20 \symqueen f4 is the winning move. For example, \variation[invar]{
    20. Qf4 cxb2 21. Rh5 \xskakcomment{ (threatening \symbishop xg7)} Nc3+ 
    \xskakcomment{ (21... \symbishop f6 22 \symqueen f5 h6 23 \symrook xh6!! gxh6 24 \symqueen g6 with mate)}
    22. Kxb2 Nxd1+ 23. Kc1 Rxf7 24. Bxf7 Bd7 25. Rxh7+ Kxh7 26. Qe4+ Kh8 27. Bxg7+ Kxg7 28. Qg6+ Kh8 29. Qh6# 
},
or \variation[invar]{ 20. Qf4 Nd2+ 21. Rxd2 cxd2 22. c3!! Qc5 23. Kc2 Qe5 24. Rxe5 dxe5 25. Qxe5 Bf6 26. Qc5 Bxd4 27. cxd4 Rc8 28. Kxd2 Ba4 29. Qe7 Bxb3 30. axb3 Rcd8 31. Ke3 } (Geller)

``This is the truth, established after many years of painstaking analysis. The 
number of moves with two exclamation marks demanded of White shows how difficult it was to find all
this at the board. A calculation of all the variations was not possible, and intuition in sharp
situations was not Fischer's strongest weapon.'' (Geller)

\mainline[level=1]{20... Qb7 }

\chessboard

\mainline[level=1]{21. Qf4} 

White can still pull the emergency brake with
\variation[invar]{21. Rh5 Nd2+ 22. Rxd2 cxd2 23. Rxh7+ Kxh7 24. Qf5+ Kh6 25. Qe6+ Kh7 26. Qf5+ } perpetual check.

\mainline[level=1]{21... Ba4 22. Qg4 Bf6 23. Rxf6 Bxb3 } 

White resigns because his attack has been parried and he no longer
has enough material. 

If White's rook moves, then \variation[invar]{23... Ba2! 24. Kxa2 Qxb2# }

``It is not enough to be a good player, observed Dr. Tarrasch; you must also play well.'' (Fischer)

\end{multicols}