\newpage
\section{e5 for White in the Modern Benoni Pawn Structure}
\begin{multicols}{2}
\chessgameinfo{Leipzig Olympiad 1960}{J.Penrose}{M.Tal}{11}{1960.11.18}{1-0}
At the time, Tal was unstoppable. He was about to win the World Championship in 1960,
while Penrose was a strong but less renowned player. However, Penrose crushed Tal in this game.

\newchessgame

\mainline[level=1]{1. d4 Nf6 2. c4 e6 3. Nc3 c5 4. d5 exd5 5. cxd5 d6 6. e4 g6 7. Bd3 Bg7 8. Nge2 O-O 9. O-O a6}

\chessboard

We have arrived at a typical position in the Modern Benoni. White 
must play a prophylaxis a4 to stop Black from playing ...b5. Then he 
can setup an e4-f4 center to control the e5 square.  \index{Modern Benoni!e5 Thrust}

It is hard to find a good plan for Black because the opening is dubious.
In the game, Black tries ...c4, ...\symknight c5, attempting ...\symknight d3. However, White's kingside 
attack comes much faster.

\mainline[level=1]{10. a4 Qc7 11. h3 Nbd7 12. f4 Re8 13. Ng3 c4 14. Bc2 Nc5 15. Qf3 Nfd7 16. Be3 b5 17. axb5 Rb8 18. Qf2 axb5 }

\chessboard[
    markstyle=circle,
    linewidth=0.05em,
    markfields={f7},
]

f7 is a clear weakness for Black. White plans to open the f-file and then break through.
The move 19. e5 is necessary because otherwise Black can play ...\symknight e5, gaining some counterplay.

\mainline[level=1]{19. e5 dxe5 20. f5 Bb7 21. Rad1 Ba8 22. Nce4 Na4 23. Bxa4 bxa4 24. fxg6 fxg6 }

\chessboard

White has a crushing position with a powerful attack on the kingside.

\mainline[level=1]{25. Qf7+ Kh8 26. Nc5 Qa7 27. Qxd7 Qxd7 28. Nxd7 Rxb2 29. Nb6 Rb3 30. Nxc4 Rd8 31. d6 Rc3 32. Rc1 Rxc1 33. Rxc1 Bd5 34. Nb6 Bb3 35. Ne4 h6 36. d7 Bf8 37. Rc8 Be7 38. Bc5 Bh4 39. g3} 1-0

\end{multicols}
