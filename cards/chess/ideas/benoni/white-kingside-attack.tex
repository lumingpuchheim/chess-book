\newpage
\section{Kingside Piece Attack for White in the Modern Benoni Pawn Structure}

\begin{multicols}{2}
In the next game, we show a Modern Benoni pawn structure coming 
from the Zaitsev variation of the Ruy Lopez. \index{Modern Benoni!Kingside Piece Attack}

\chessgameinfo{FIDE World Championship}{V.Anand}{M.Adams}{5}{2005.9.30}{1-0}

\newchessgame
\mainline[level=1]{
1. e4 e5 2. Nf3 Nc6 3. Bb5 a6 4. Ba4 Nf6 5. O-O Be7 6. Re1 b5 7. Bb3 d6 8. c3 O-O 9. h3
Bb7 10. d4 Re8 11. Nbd2 Bf8 12. a4 h6 13. Bc2 exd4 14. cxd4 Nb4 15. Bb1 c5 16. d5}

\chessboard

Here we have a Modern Benoni pawn structure coming from Ruy Lopez. The position 
looks intimidating for Black because all the White pieces are targeting the Black king 
(White can move the a1 rook to g3 via a3). Black tries to stop the attack with 
...c4, \symknight d7-c5-d3, blocking the important b1-h7 diagonal.

\mainline[level=1]{16...Nd7 17. Ra3 c4}

\chessboard

\mainline[level=1]{18. axb5?!}

This move is not good because it gives Black an additional resource: exchanging the rook on a3.
The Black rook can also attack on a1 when the White rook moves to g3.
\variation[invar]{18. Nd4} is better.

\mainline[level=1]{18...axb5 19. Nd4 Qb6 20. Nf5 Ne5 21. Rg3 g6 22. Nf3 Ned3}

\chessboard

White is starting a decisive attack while Black has deserted White's queenside and blocked White's light-squared bishop.

\mainline[level=1]{23. Qd2}

\chessboard

\mainline[level=1]{23...Bxd5?}

The most dangerous piece for Black is the knight on f3, which can move to h4 and sacrifice on g6.
Black must play \symknight xe1,
for example: \variation[invar]{23...Nxe1 24. Nxe1 Ra1 25. Nxh6+ Bxh6 26. Qxh6 Nxd5 27. Be3 Nxe3 28. Rxg6+ fxg6 29. Qxg6+ \xskakcomment{ perpetual check}};
\variation[invar]{23...Nxe1 24. Nxh6+ Bxh6 25. Qxh6 Nxf3+ 26. gxf3 Nd3 27. Be3 Qxe3 28. fxe3 Ra1 29. Kg2 Rxb1 \xskakcomment{ the position is equal}}

\mainline[level=1]{24. Nxh6+ Bxh6 25. Qxh6 Qxf2+ 26. Kh2 Nxe1}

\chessboard

\mainline[level=1]{27. Nh4! Ned3 28. Nxg6 Qxg3+ 29. Kxg3 fxg6 30. Qxg6+ Kf8 31. Qf6+ Kg8 32. Bh6} 1-0


\end{multicols}