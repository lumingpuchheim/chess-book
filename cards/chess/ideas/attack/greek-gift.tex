%\newpage
\section{Middlegame Attack: Greek Gift}

The Greek Gift is a classic attacking motif where White sacrifices a bishop on h7 
(or h2 for Black) to rip open the enemy king's shelter. \index{Attack!Greek Gift}
After Bxh7+, the attacker usually follows with Ng5+ and Qh5, 
aiming for a direct mating attack. It works when the defending king lacks key 
defenders—especially the knight on f6 or sufficient control of g5. When the 
conditions are right, the sacrifice is not speculative; it's a forced assault 
that often leads to a decisive attack or mate.

\begin{multicols}{2}

In the first round of my first tournament (25 minutes, no increment), I played White as the underdog against 
a strong opponent. I was fourteen and he was eleven. Feeling I had little chance 
in a slow strategic game, I decided to attack.

Before the tournament I had learned how Botvinnik beat Capablanca by setting up 
a center with e4 and starting an attack on the kingside. 

The game started. He played his moves instantly.

\newchessgame 

\mainline[level=1]{
    1. d4 Nf6 2. c4 e6 3. Nc3 Bb4 4. e3 O-O 5. Bd3 c5 6. Nf3 Nc6 7. a3 Bxc3+ 8. bxc3 b6
     9. O-O Ba6}

     \chessboard

     During the game I had no idea what he was doing. (Only years later did I understand that his moves b6, \symbishop a6, \symknight c6, \symknight a5
     are typical against White's c4 pawn. The idea was not bad in itself, if only he had protected his pawn.)
     
     
     I just developed my center.

     \mainline[level=1]{ 10. e4 }
     
     \chessboard

     My heart started to beat faster. My entire attacking plan depended on his next move.

     \mainline[level=1]{10...Na5?}
     
     I was relieved. He should have played \variation[invar]{10... d5 11. e5 Ne4} blocking my 
     bishop to stop the attack.

     \mainline[level=1]{11. e5 Ne8}
     
     \chessboard

     Now we reached a typical Greek Gift position.

     \mainline[level=1]{12. Bxh7+ }
     
     \chessboard

     No matter what Black plays now, White has a standard solution.

     \variation[invar]{12... Kxh7 13. Ng5+ Kg8 14. Qh5};
     
     \variation[invar]{12... Kxh7 13. Ng5+ Kg6 14. h4! Rh8 15. h5+! Rxh5 16. Qd3+ f5 17. exf6+ Kxf6 18. Qf3+ Ke7 19. Qf7+ Kd6 20. dxc5+ bxc5 21. Qxh5 }
     
     \mainline[level=1]{12...Kh8 13. Ng5 g6 14. Qf3 Ng7 15. Qh3 Nh5 16. Nxf7+ Rxf7 17. Bxg6 Qe7 18. Qxh5+ 
     } Black resigned.

     When I visited my middle-school math teacher after being admitted to university, I saw 
     his name in the class register and told my teacher how I had beaten him in the tournament.


\end{multicols}