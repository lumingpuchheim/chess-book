\newpage
\section{Benko Gambit}

The Benko Gambit (or Volga Gambit) is a chess opening characterised by the move 3...b5 in the Benoni Defence arising after:

\begin{multicols}{2}
\newchessgame
\mainline[level=1]{
1. d4 Nf6
2. c4 c5
3. d5 b5}

\chessboard

Black sacrifices a pawn for enduring queenside pressure. White can accept or decline the gambit.

When the sacrifice is accepted, White, who is already behind in development, must solve the problem of developing the f1-bishop.
An early e4 will lead to an exchange on f1, losing the right to castle.

Black also obtains fast development, good control of the a1–h8 diagonal, and can exert pressure down the half-open a- and b-files, where White's a- and b-pawns can become vulnerable. These benefits can last well into the endgame, and so, unusually for a gambit, Black does not generally mind if queens are exchanged.



\chessgameinfo{9th Canadian Open}{Gerald Aspler}{Pal Benko}{6}{1971.08.29}{0-1}
\newchessgame
\mainline[level=1]{
    1. d4 Nf6 2. c4 c5 3. d5 b5 4. cxb5 a6 5. bxa6 Bxa6}
    
\chessboard

This is a typical position in the Benko Gambit Accepted. Black has a clear plan:
attack along the a- and b-files. Note that the light-squared bishop on a6 prevents
White's normal development e4 (after exchanging on f1, White loses the right to castle).

\mainline[level=1]{6. Nc3 d6
7. Nf3 g6 8. g3 Bg7 9. Bg2 O-O 10. O-O Nbd7 11. Re1 Qb6 12. e4
Ng4 13. Qc2 Rfb8}

\chessboard

It is clear that Black is attacking along the a- and b-files. 
Black has a pawn structure similar to the Benoni structure with pawns on c5 and d6.
This means Black will also have a typical Benoni plan: 

\begin{itemize}
    \item{Advance the pawn to c4}
    \item{Move the knight to d3}
\end{itemize}

Note that the knight on d3 will be well protected by the c4 pawn and the a6 bishop.

\mainline[level=1]{14. h3 Nge5 15. Nxe5 Nxe5 16. b3 Nd3 17. Rd1}

\chessboard

\mainline[level=1]{17...c4! 18. Be3 Qb4 19. Bd2 Qc5 20. Rf1 cxb3 21. axb3 Nb4 22. Qb2
Rc8 23. Qa3 Bxc3 24. Bxc3 Qxc3 25. Rfc1 Qd4 26. Bf1 Rc2
} 0-1

We will also see that the great Capablanca played the idea of the Benko Gambit 
in a Ruy Lopez before the Benko Gambit was even invented.

\chessgameinfo{St. Petersburg}{Aron Nimzowitsch}{Jose Raul Capablanca}{1}{1914.04.21}{0-1}
\newchessgame

\mainline[level=1]{
    1. e4 e5 2. Nf3 Nc6 3. Nc3 Nf6 4. Bb5 d6 5. d4 Bd7 6. Bxc6
Bxc6 7. Qd3 exd4 8. Nxd4 g6}

\chessboard

The last move loses a pawn, yet the position is still playable. 
I believe Capablanca didn't play the gambit intentionally. 

\mainline[level=1]{9. Nxc6 bxc6 10. Qa6 Qd7 11. Qb7
Rc8 12. Qxa7 Bg7 13. O-O O-O}

\chessboard[
    markstyle=circle,
    linewidth=0.05em,
    markfields={e4},
]

\mainline[level=1]{14. Qa6?!}

This move simply loses one tempo. White's e4 pawn needs protection, 
one obvious idea is playing \variation[invar]{14. f3}. After 
the text move, Black can play some natural moves to develop while 
White can only react passively.

\mainline[level=1]{14...Rfe8 15. Qd3 Qe6 16. f3 Nd7 17. Bd2}

\chessboard[
    markstyle=circle,
    linewidth=0.05em,
    markfields={c3, b2},
]

Black's bishop targets all of White's weaknesses along the long diagonal.
White's d2 pawn is an important defender. A logical decision is to trade it.
Afterwards, Black can exert pressure on the a- and b-files with the rooks.

\mainline[level=1]{17...Ne5! 18. Qe2 Nc4! 19. Rab1 Ra8 20. a4 Nxd2! 21. Qxd2}

\chessboard[
    pgfstyle=straightmove,
    linewidth=0.05em,
    markmove={b2-b3},
]

White intends to play b3 and then move his knight away. 
If he succeeds, he will get rid of his weak a- and b-pawns, and these pawns will
even become strengths as passed pawns. Black must stop it.

\mainline[level=1]{21...Qc4! 22. Rfd1 Reb8}

\chessboard

Now Black has finally achieved the Benko Gambit idea without playing 
the opening. White's knight is tied down. Black will double his rooks on the b-file in 
the next few moves.

\mainline[level=1]{23. Qe3 Rb4}

\chessboard

\mainline[level=1]{24. Qg5? Bd4+ 25. Kh1 Rab8
26. Rxd4 Qxd4 27. Rd1 Qc4 28. h4 Rxb2 29. Qd2 Qc5 30. Re1 Qh5
31. Ra1 Qxh4+ 32. Kg1 Qh5 33. a5 Ra8 34. a6 Qc5+ 35. Kh1 Qc4
36. a7 Qc5 37. e5 Qxe5 38. Ra4 Qh5+ 39. Kg1 Qc5+ 40. Kh2 d5
41. Rh4 Rxa7 42. Nd1
} 0-1


\newchessgame
\chessgameinfo{USSR Championship 1961a}{Tigran Vartanovich Petrosian}{Eduard Gufeld}{11}{1961.01.27}{1-0}
\mainline[level=1]{
    1. c4 g6 2. d4 Bg7 3. Nc3 Nf6 4. e4 O-O 5. Bg5 d6 6. Qd2 c5
7. d5 Qa5 8. Bd3 a6 9. Nge2 e5?}

\chessboard

Black could have played \variation[invar]{9...b5 10. cxb5 Nbd7 11. bxa6 Ne5 12. O-O Nxd3 13. Qxd3 h6 14. Bd2 Bxa6},
playing the idea of the Benko Gambit. He has enough compensation on the a- and b-files for the 
pawn, as in the Benko Gambit.

\chessboard[
    setfen=r4rk1/4ppb1/b2p1npp/q1pP4/4P3/2NQ4/PP1BNPPP/R4RK1 w - - 0 15
]

Black has enough compensation

\begin{itemize}
    \item{he has his bishop on a6 controlling the c4 square}
    \item{c5 or e5 are perfect squares for his knight}
    \item{his rooks are ready for a- and b-file, exerting pressure on the a- and b-pawns}
\end{itemize}
the c4 square and c5 or e5 are perfect squares for his knight. 

In the game, Gufeld lost the thread and soon lost.

\mainline[level=1]{10. O-O Nbd7 11. a3 Nh5 12. f3
Bf6 13. Bh6 Ng7 14. g3 Rb8 15. Kh1 Qc7 16. b3 Be7 17. Rab1 Kh8
18. Rb2 Nf6 19. b4 Ng8 20. Be3 f5 21. bxc5 dxc5 22. Rfb1 Nf6
23. Rb6 Bd6 24. Bh6 Rf7 25. Ng1 f4 26. gxf4 Nd7 27. fxe5 Bxe5
28. Re6 b5 29. cxb5 c4 30. Rc6 Qd8 31. Bxc4 Qh4 32. Rc1 Nh5
33. Bg5 Ng3+ 34. Kg2 Nxe4 35. Nxe4 Qxh2+ 36. Kf1 Rxf3+
37. Nxf3 Qh1+ 38. Kf2 } 1-0

\end{multicols}