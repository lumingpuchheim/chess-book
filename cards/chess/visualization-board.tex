We use a simple graphical system to illustrate the three questions as used by Aagaard in his book ``Grandmaster Preparations Positional Play''. 

We will use circle to identify weakness. We will use square to identify the worst-placed pieces. We will use arrows to illustrate the opponent's ideas.


Let us look at an example 

\subsection*{Hikaru Nakamura - Vladimir Kramnik}

 \chessboard[
        setfen=3r2k1/p4pbp/b3p3/npqpP1pP/8/2P3P1/P3QPBN/R3R1K1 w - - 0 1,
 	markstyle=circle,
 	linewidth=0.05em,
 	markfields={g5, f6},
	markstyle=border,
	markfields={g2},
	pgfstyle=straightmove,
 	markmove=h7-h6
        ]

Black has weakness on g5 and f6. All his pieces are bad. White's worst piece is the bishop on g2.

Black intends to play h6 to protect his biggest weakness: g5 pawn. To prevent this move White can play h6 himself and then \symqueen h5 attack the weak pawn on g5.



