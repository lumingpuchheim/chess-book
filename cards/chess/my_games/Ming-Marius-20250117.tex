\newpage
\section{Exciting Fight}

\multicols{2}
    \chessgameinfo{Correspondence Chess 960}{Ming}{Marius}{}{2025.01.17-2025.02.26}{1-0}

    \newchessgame[
        setfen=rnnkqbbr/pppppppp/8/8/8/8/PPPPPPPP/RNNKQBBR w HAha - 0 1,
    ]
    \chessboard

    The position is quite similar to the standard opening position:
    queen and king are in the middle while the rooks are on the edges.


    \mainline[level=1]{1. e4 e5 2. Bc4?!} 
    
    I dislike this move now that I am annotating. During the game, I was 
    thinking of a setup similar to the London System:

    \chessboard[setfen=8/8/8/8/2B1P3/2NPBP2/PPP1N1PP/R2KQ2R ]

    Schematic thinking in itself is not bad. However, such a setup that requires many tempi can be
    easily disturbed. 

    Observing the opening position: it is necessary to move the e- and f-pawns to develop the 
    bishops. \variation[invar]{2. f4} in the spirit of the Queen's Gambit is called for. 
    The engine recommends \variation[invar]{2. f4 f6 3. Nc3 Nc6 4. Nd3 Nd6 5. O-O-O O-O-O 6. g3 h5 7. Kb1 Kb8 8. Bf2 }

    \mainline[level=1]{2... Nd6! 3. Bb3 f6!}
    
    \chessboard

    Marius played natural moves that refuted my idea. My bishop will eventually 
    be exchanged. During the game, I was not happy with this development. The game must go on, 
    so we continued with development moves.

    \mainline{4. Nc3 Nc6 5. f3 Be6 6. N1e2 }

    \chessboard[
        pgfstyle=straightmove,
        linewidth=0.05em,
        markmove={d2-d4},
        markstyle=circle,
        linewidth=0.05em,
        markfields={g2},
        markstyle=border,
        linewidth=0.05em,
        markfields={f8},   
    ]

    \begin{itemize}
		\item{Where are the weaknesses?}

		g2 pawn.
		\item{Which is the worst-placed piece?}

		f8 bishop is inactive. 
		\item{What is my opponent's idea?}

		He wants to play d4.
	\end{itemize}
    \mainline[level=1]{6...O-O-O?!} 
    
    I was happy to see this move during the game because I saw that I might start 
    an attack along the a7-g1 diagonal with a battery. At some point, my knight can move
    to b5. 
    \variation[invar]{6... Qg6} would have been more energetic.  
    \mainline[level=1]{ 7. d3 }
    
    I cannot explain why I did not play d4 directly. The reason may be that I was so obsessed 
    with the attack on the a7-g1 diagonal that I wanted to place my queen in front of my bishop.
    
    \mainline[level=1]{7... Qg6 8. Qf2 Kb8} 
    
    \chessboard

    It is difficult to formulate a concrete plan for White. The problem is that the queen and 
    bishop battery has no future for the attack. \variation[invar]{9.Nb5} is and will remain impossible 
    as long as there is a Black knight on d6. The h1 rook cannot be activated because 
    of the battery. 
    
    \mainline[level=1]{ 9. d4?!}
    
    This move is too committal. White is not ready for an attack because Black's defenders 
    are well placed. A move like d4 will open the center and therefore change the game completely from 
    slow maneuvering to rapid piece contact. White is not ready for this yet. 
    
    A waiting move like \variation[invar] {9. h4} would have been better.
    
    \mainline[invar]{9...exd4 10. Nxd4 Nxd4 11. Qxd4 b6 12. Qf2 Nc4 13. O-O-O}
    
    
    \mainline[level=1]{13...Bc5 }
    
    Finally, I can forget about my battery and the attack. Ironically, this move forces 
    me to make a good move. Then Black must decide what to do with his attacked knight 
    on c4. In a few moves, we will see that Black makes a mistake.
    
    \mainline{14. Qe2 Qg5+ 15. Kb1}
    
    \chessboard

    \mainline[level=1]{15...Ne3?}
    
    Marius overlooked the fact that the knight on e3 was in danger.
    I also overlooked it during the game. \variation[invar]{15... Bxg1} would have been better.
    
    \chessboard[
        setfen=1k1r3r/p1pp2pp/1p2bp2/2b3q1/4P3/1BN1nP2/PPP1Q1PP/1K1R2BR w - - 8 16
    ]
    
    \mainline[level=1]{16. Qa6?!}
    
    During the game, I was happy to find this route/rout. Black cannot take the rook because of
    \variation[invar]{17. Nb5}, leading to mate in several moves. 

    The problem with this move was that it helped Black solve his e3 knight problem. 
    
    \variation[invar]{16. Re1} would have been better. Black cannot defend two weaknesses 
    simultaneously: the knight on e3 and the weakness around his king. For example:

    \variation[invar]{16. Re1 Bxb3 17. axb3 Nxg2 18. h4 Qg6 19. Qa6 }

    \chessboard[
        setfen=1k1r3r/p1pp2pp/Qp3pq1/2b5/4P2P/1PN2P2/1PP3n1/1K2R1BR b - - 2 19
    ]

    White threatens \variation[invar]{20. Nb5}. Therefore \variation[invar]{19... c6 20. Bxc5 bxc5 21. Reg1 Qg3 22. Qe2 Nf4 23. Qf1 }. 
    
    \chessboard[
        setfen=1k1r3r/p2p2pp/2p2p2/2p5/4Pn1P/1PN2Pq1/1PP5/1K3QRR b - - 5 23
    ]
    
    The Black queen is trapped.

    We go back to the mainline.
    \mainline[level=1]{16...c6 17. Bxe3 Bxe3 18. Rd3 Bxb3 19. axb3 }
    
    \chessboard

    During the game, I was looking for something like \symknight d5 at some point. 
    The idea was ambitious. However, I could play \symknight a4, threatening to sacrifice on b6.
    Here we can also see that \variation[invar]{15... Ne3} helped White solve his h1 rook
    problem. \symrook hd1 is now in play. 

    \mainline[level=1]{19...d6}

    Another possibility would be \variation[invar]{
        19...Qxg2 20. Rhd1 Bf4 21. Na4 Qg5 22. Nxb6! axb6 23. Qxb6+ Kc8 24. Qa6+ Kc7 25. Rxd7+ Rxd7 26. Qa7+ Kc8 27. Rxd7 Qb5 28. Rxg7 c5 29. Qa8+ Qb8 
    }

    \chessboard[
        setfen=2k4r/Q5Rp/5p2/1qp5/4Pb2/1P3P2/1PP4P/1K6 w - - 0 29
    ]
    White makes no sacrifices here because the material is equal. His pieces are more active 
    and his king is safer. It is unclear whether White can convert this advantage.

    \mainline[level=1]{20. Rhd1 Bc5}
    
    \chessboard[
        setfen=1k1r3r/p5pp/Qppp1p2/2b3q1/4P3/1PNR1P2/1PP3PP/1K1R4 w - - 2 21
    ]
    
    Here is a critical position. White has a significant advantage here. Black's bishop is stuck on c5 and must protect the d6 pawn.
    White would love to play b4. \variation[invar]{21. Na2} is hard to see. White threatens 
    to play b4. For example: \variation[invar]{21. Na2 Qxg2 22. b4 Bg1 23. Nc3 Rd7 24. b5 c5 25. R3d2 Qg6 26. Rxd6 Rxd6 27. Rxd6 Qe8 }

    \chessboard[
        setfen=1k2q2r/p5pp/Qp1R1p2/1Pp5/4P3/2N2P2/1PP4P/1K4b1 w - - 1 28
    ]

    Compared to the actual game, White has achieved more: his queen is active, and his rook and 
    knight will be dominant. We return to the actual game.
    \mainline[level=1]{21. Qa4 } The correct idea, though the execution could have been better.
    
    \mainline[level=1]{21...Kb7 22. b4 b5!}
    
    I overlooked this move and soon afterward made a mistake. 
    \mainline[level=1]{23. Qa5?!} This wastes a tempo! \variation[invar]{23. Qb3} would have been better.
    
    \mainline[level=1]{23... Bb6 24. Qa2 Rd7}
    
    \chessboard[
        7r/pk1r2pp/1bpp1p2/1p4q1/1P2P3/2NR1P2/QPP3PP/1K1R4 w - - 4 25
    ]

    \mainline[level=1]{25. Qe6?}

    I missed an opportunity! Perhaps I was afraid of losing too many pawns on the kingside. \variation[invar]{25. Rxd6 Rxd6 26. Qf7+} would have been better. 

    \mainline[level=1]{25... Rhd8 26. g3 Qe5!}
    
    \chessboard

    That is it. Now the position is equal again.

    \mainline[level=1]{27. Qxe5 fxe5 28. Ne2 Kc7}
    
    \chessboard

    
    \mainline[level=1]{29. f4?}
    Feeling that I had some advantage earlier, I felt I must try for a win. A waiting move 
    would have been better.
    
    \mainline[level=1]{29...d5! 30. exd5}
    
    \chessboard

    \mainline[level=1]{30...Rxd5?}
    
    Black actually has some advantage here. \variation[invar]{30... e4} would 
    have been better. His passed pawn would be too strong.

    \chessboard

    \mainline[level=1]{31. Rxd5?}
    Another missed opportunity. \variation[invar]{31. fxe5} would have been better. In the endgame,
    both of us made too many automatic bad moves.

    \mainline[level=1]{31...Rxd5 32. Rxd5 cxd5 33. fxe5}
    
    \chessboard
    
    The position is still equal, although White has one more pawn. His e5 pawn is too weak.

    \mainline{33...a5??}
    It is understandable that Black wants to exchange some pawns to secure a draw, but this move is a blunder. 
    
    \mainline{34. bxa5 Bxa5 35. Nd4 b4 36. Ne6+ Kd7 37. Nxg7 Bc7 38. e6+ Ke7 39. Nf5+ Kxe6 40. Nd4+ Kf6 41. Ka2 Bd6 42. Kb3 h5 43. Nc6 Kg5 44. Nxb4 Bxb4 45. Kxb4 Kg4 46. Kc5 Kh3 47. b4 } Black resigned.

    It was an exciting game with many mistakes from both sides. Both of us 
    tried hard to play good chess. I had some good ideas, but I was too eager to play them without 
    trying to find better moves.

    I tried hard to create a non-existent attack in the opening. Marius parried 
    it easily. Later in the middlegame, I had some attacking chances without trying hard, but I missed several opportunities.

    The endgame was fortunate for me. 
\end{multicols}