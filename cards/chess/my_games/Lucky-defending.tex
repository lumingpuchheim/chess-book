\newpage
\section{Lucky Defending}
\begin{multicols}{2}
    \chessgameinfo{chess.com}{Ming}{Francis-Bot}{}{2024.09.02}{1-0}
    \newchessgame

    \mainline[level=1]{1. d4 f5 2. Bf4 e6 3. Nf3 Nf6 4. e3 Bd6}
    
    \chessboard

    Black chose the Dutch Defense, which is not very solid, and White already had 
    an advantage. Here I faced a decision: how to exchange the bishops? 
    \variation{5. Bxd6 cxd6 6. c4} was a good alternative. White would still keep a lead 
    in development and his bishop would be much better than Black's.

    In the game I forgot this alternative entirely and chose to move the bishop.
    I wanted to open the h-file and create an attack along it.
    
    \mainline[level=1]{5. Bg3 O-O }
    
    \chessboard

    \mainline[level=1]{6. Bc4?}
    
    This move threw away the advantage because it would sooner or later
    lose a tempo. The bishop on c4 was not active. It had no 
    target and Black could later play the natural d5 to drive it away.
    
    \mainline[level=1]{6...Nc6 7. c3 Bxg3 8. hxg3 d5 9. Bd3 Qe7 }
    
    \chessboard
    
    \mainline[level=1]{10. Ne5?}
    
    One of my sins was impatience. What was the point of this move? After \variation[invar]{10...Nxe5 11. f4 Ng4} White 
    lost a pawn and Black had a clear advantage. A solid \variation[invar]{11. Nbd2} finishing the 
    London System would have been better.

    \mainline[level=1]{10...Bd7?} As we already know, \variation[invar]{10... Nxe5} would have been better.
    
    \chessboard
    
    \mainline[level=1]{11. f4?}
    
    \variation[invar]{11. Nxc6} would have been better; White would keep the better bishop. 
    In the game I was obsessed with the so-called Pillsbury setup (White placing a knight on e5 supported by pawns on d4 and f4).

    \mainline[level=1]{11...Nxe5 12. dxe5 Ng4}

    \chessboard 

    Oops. I had to defend the e3 pawn now. 
    
    \mainline[level=1]{13. Qe2 Qc5 }
    
    \chessboard
    
    \mainline[level=1]{14. Kd2?!}
    
    I needed to defend my e3 pawn. What else could I do? It turned out White 
    didn't need to defend that pawn. Losing it would not make the position inferior
    because the bishop was still much better. White needed to keep the pieces active.

    \variation[invar]{14. b4 Qxe3 15. Qxe3 Nxe3 16. Nbd2} The position is still equal. 
    Besides his better bishop, he can also hope for the active play along the h-file.
    \mainline[level=1]{14...Qb6 15. b3 a5 16. Na3}
    
    \chessboard

    Black had a small edge now because his pieces were more active. He needed to 
    act quickly because once White consolidated, White would have counterplay on 
    the h-file. 
    
    \mainline[level=1]{16...Qc5?!}
    
    Now White just needed to make a natural move. \variation{16... c5} targeting d4 would have been better.
    
    \mainline[level=1]{17. Nc2}
    
    \chessboard
    
    \mainline[level=1]{17...h6?}
    
    Now White had the advantage again. Black should have played prophylactic g6, removing any potential
    sacrifice for White with \symrook xh7.

    \mainline[level=1]{18. Rh4 Rfd8 19. Rah1 Re8 }
    
    \chessboard 

    White now only needed to play the natural sacrifice.
    
    \mainline[level=1]{20. Rxg4! fxg4 21. Qxg4 Kf8}
    
    \chessboard
    
    \mainline[level=1]{22. Qg6}
    
    \variation[invar]{22. Bh7} intending \variation{23. Rxh6 gxh6? 24. Qg6} mate in a few moves would have been stronger.
    
    \chessboard

    \mainline[level=1]{22...a4?} Now mate is inevitable.

    \variation[invar]{22...Qe7 23. Qh7 Qf7 24. Bg6 Qg8 25. Qxg8+ Kxg8 26. Bxe8 Bxe8 27. Nd4 } White has advantage while 
    Black can still play.
    
    \chessboard
    
    The sacrifice is again quite natural.

    \mainline[level=1]{23. Rxh6 gxh6 24. Qxh6+ Kf7 25. Qh7+ Kf8 26. Bg6 Qxc3+ 27. Kxc3 d4+ 28. exd4 axb3 29. Qf7# }

    A lucky win for me indeed. From an inferior position, I only needed to play natural moves.
    After my opponent missed a difficult move, I just followed through with the natural sacrifices. 
    To be honest, there was not much calculation involved. I am, however, proud that I took the opportunity.
\end{multicols}