\ifdefined\karjakinCarlsenOne
    % The command is already defined, do nothing.
\else
    \newcommand{\karjakinCarlsenOne}[1]{
	\poechessproblem{#1}{Karjakin - Carlsen}{r4bk1/1bq2pp1/4rn1p/4N3/BPp1P3/2B4P/1Q3PP1/3RR1K1 w - - 1 29}
    }
\fi

\ifdefined\karjakinCarlsenTwo
    % The command is already defined, do nothing.
\else
    \newcommand{\karjakinCarlsenTwo}[1]{
	\poechessproblem{#1}{Karjakin - Carlsen}{4r1k1/1bq2pp1/3brn1p/4N3/1Pp1PP2/2B4P/1QB3P1/3RR1K1 w - - 1 31}
    }
\fi

\ifdefined\karjakinCarlsenThree
\else
    \newcommand{\karjakinCarlsenThree}[1]{
	\poechessproblem{#1}{Karjakin - Carlsen}{6k1/1b4p1/2q1r2p/4Bp2/1Pp5/6PP/1QB4K/6R1 b - - 2 39}
    }
\fi


\ifdefined\karjakinCarlsenOneAnswer
    % The command is already defined, do nothing.
\else
    \newcommand{\karjakinCarlsenOneAnswer}[1]{
		\subsection*{#1. Karjakin - Carlsen}
        \newchessgame[
            setfen=r4bk1/1bq2pp1/4rn1p/4N3/BPp1P3/2B4P/1Q3PP1/3RR1K1 w - - 1 29,
            moveid=29w
        ]
        \chessboard[
            setfen=r4bk1/1bq2pp1/4rn1p/4N3/BPp1P3/2B4P/1Q3PP1/3RR1K1 w - - 1 29,
            markstyle=circle,
            linewidth=0.05em,
            markfields={c4, e4, e5},
        ]
        \begin{itemize}
            \item{What are the candidate moves?}

            White's bishop is being attacked. He must play either \variation[invar]{29. Bb5} or \variation[invar]{29. Bc2}.
            \item{Which move can be eliminated?}
            \variation[invar]{29. Bc2 Rae8 30. f4 Bd6} 

            \chessboard

            White has weaknesses on e5 and e4. Black only needed to double his 
            rooks on the e-file and use his pieces to target the e5 square.

            However, it is difficult for White to find the right plan. He has 
            already weakened his king safety with his kingside pawn movements. He will
            also have to play g3 at some point to further weaken his king. 

            Therefore we eliminate \variation[invar]{29. Bc2}.
            \item{What is the move?}

            \variation[invar]{29. Bb5}. For example after 
            \variation[invar]{29. Bb5 Ba6 30. Ra1 Bb7 31. Rxa8 Bxa8 
                32. Bxc4 Rxe5 33. Bxf7+ Qxf7 34. Bxe5 Nxe4 } White has a playable position
        \end{itemize}
    }
\fi

\ifdefined\karjakinCarlsenTwoAnswer
    % The command is already defined, do nothing.
\else
    \newcommand{\karjakinCarlsenTwoAnswer}[1]{
		\subsection*{#1. Karjakin - Carlsen}
        \newchessgame[
            setfen=4r1k1/1bq2pp1/3brn1p/4N3/1Pp1PP2/2B4P/1QB3P1/3RR1K1 w - - 1 31,
            moveid=31w
        ]
        \chessboard[
            setfen=4r1k1/1bq2pp1/3brn1p/4N3/1Pp1PP2/2B4P/1QB3P1/3RR1K1 w - - 1 31,
            markstyle=circle,
            linewidth=0.05em,
            markfields={e4,e5},
            pgfstyle=straightmove,
            linewidth=0.05em,
            markmove={f6-h5},
        ]
        \begin{itemize}
            \item{What are the candidate moves?}

            \begin{itemize}
                \item{Where are the weaknesses?}
            
                e4 and e5.
                \item{Which is the worst-placed piece?}
               
                Not clear.
                \item{What is my opponent's idea?}
            
                He wants to play \symknight h5 to attack the f4 pawn.
            \end{itemize}

            When Black plays \symknight h5, White must play g3. As prophylaxis,
            it is a good idea to protect the g3 pawn first. There are two options
            \variation[invar]{31. Kh2} and \variation[invar]{31. Re3}.

            \item{Which move can be eliminated?}

            Karjakin should have eliminated \variation[invar]{31. Kh2} first. 
            His king is on the same diagonal as the Black queen and bishop.
            Therefore we eliminate \variation[invar]{31. Kh2}.
            \variation[invar]{31. Re3} is the right move.

            \variation[invar]{31. Re3 Nh5 32. g3 g5} 
            (\variation{32...f6 33. Nxc4 Bxf4 34. gxf4 Nxf4 35. Bb3 \xskakcomment{ Black king
            is now exposed, White has advantage.}}) \variation {33. Ba4 R8e7 34. Qe2} White is fine.
        
            \item{What is the move?}

            \variation[invar]{31. Re3} 
        \end{itemize}
    }
\fi


\ifdefined\karjakinCarlsenThreeAnswer
    % The command is already defined, do nothing.
\else
    \newcommand{\karjakinCarlsenThreeAnswer}[1]{
		\subsection*{#1. Karjakin - Carlsen}
        \newchessgame[
            setfen=6k1/1b4p1/2q1r2p/4Bp2/1Pp5/6PP/1QB4K/6R1 b - - 2 39,
            moveid=39w
        ]
        \chessboard
        \begin{itemize}
            \item{What are the candidate moves?}

            White must stop from being mated along the main diagonal.
            \variation[invar]{39. Kg1} and \variation[invar]{39. Be4} are the candidate moves.

            \item{Which move can be eliminated?}

            Karjakin should have eliminated \variation[invar]{39. Rg1} first. 
            \variation[invar]{39. Rg1 Qd5 40. Bd1 Rxe5 41. Qf2 c3 } His pieces 
            are passive. His pawn on b4 will fall soon while Black has a strong 
            passed pawn.
        
            \item{What is the move?}

            \variation[invar]{39. Be4}. After \variation[invar]{39. Be4 fxe4 40. Re3} 
            he can still defend.
        \end{itemize}
    }
\fi


