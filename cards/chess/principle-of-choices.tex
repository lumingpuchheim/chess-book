\section{Principle of Choices}

If you believe you are in a winning position, you must avoid 
making moves that give your opponent choices, as different 
outcomes may arise depending on an important tempo. The same 
principle applies when you are defending and your position is drawn.

On the other hand, if you are playing when the outcome is unclear, it is practical 
to give your opponent a choice, as they may make a mistake. \index{Principle of Choices}

Combining Principle of Choices and Process of Elimination makes it easier to find the correct 
move.

\begin{multicols}{2}
\newchessgame[
    setfen=6k1/8/5PP1/3p4/3Kp3/8/8/8  w - - 0 1
]

\chessboard

White believes he is winning (it is a chess problem, draw can be easily achieved. Obviously
White must find the winning move.)

The candidate moves are
\begin{itemize}
    \item{f7}
    \item{g7}
    \item{\symking e3}
    \item{Other king moves}
\end{itemize}

Other king moves are bad because Black can play ...e3 and then ...d4. He improves 
his position.

g7 is bad because White cannot make any further progress. He must move 
his king to e6 which can be easily parried by ...\symking f7. Not to mention when 
Black queens at e1, he gives a check.

To eliminate f7 we can use the Principle of Choices. After f7, Black 
can either play ...\symking f8 or ...\symking g7 and they lead to different result.
This means f7 leads to a draw when Black responds correctly.
Note we have eliminated the f7 without any calculation. Such a strategy saves 
energy and time in real games since we are only interested in finding the good move.

This leaves \symking e3. The game continues with 
\mainline[level=1]{1. Ke3 Kf8}

\chessboard

Here again we can use Process of Elimination.
Candidate moves are f7, g7, \symking d4 and other king moves.

Other king moves are bad because of the same reason as before.

\symking d4 is bad because Black can repeat the position. 

g7 is bad because after ...\symking Kf7, White has no further threats.

Therefore f7 is the move.

\mainline[level=1]{2. f7 Ke7}

\chessboard 

Of course \symking d4 is the only move.

\mainline[level=1]{3. Kd4 Kf8}

\chessboard

Even now we can use Process of Elimination! \symking e3 only repeats the position.
Certainly White doesnt want to sacrifice with g7 yet. Which leaves only \symking xd5.

\mainline[level=1]{4. Kxd5! e3}

\chessboard

The game finishes with a mate.

\mainline[level=1]{5. Ke6! e2 6. Kf6 e1=Q 7. g7#} 
\end{multicols}