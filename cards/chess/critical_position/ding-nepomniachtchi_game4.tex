\ifdefined\dingNepomniachtchiGameFour
    % The command is already defined, do nothing.
\else
    \newcommand{\dingNepomniachtchiGameFour}[1]{
	    \dynamicchessproblem{#1}{Ding - Nepomniachtchi}{4r1k1/p1p1r1pp/1p1nPp1q/2pP3b/2P2p2/2Q2B1P/P2NRPP1/4R1K1 b - - 2 25}
    }
\fi

\ifdefined\dingNepomniachtchiGameFourAnswer
    % The command is already defined, do nothing.
\else
    \newcommand{\dingNepomniachtchiGameFourAnswer}[1]{
		\subsection*{#1. Ding - Nepomniachtchi}
		\chessboard[
			setfen=4r1k1/p1p1r1pp/1p1nPp1q/2pP3b/2P2p2/2Q2B1P/P2NRPP1/4R1K1 b - - 2 25,
		]

        \begin{itemize}
            \item{Who has the static advantage?}

            White has the static advantage because his passed pawns in the center are strong.
            \item{Which play shall I choose?}
            I shall choose a static play

            The main asset of White is the passed pawns, which have been blocked by the Black rooks.

            The rooks on the e-file are not doing much because there is no open file. The knights are 
            important pieces. The Black knight keeps an eye on e4 so that the rooks cannot attack the 
            f4 pawn. 

            I have a holdable position because I have only on weakness and White must find a way
            to prove his advantage. Therefore I choose a static play without changing the position.
            \item{Whas is the move?}

            At some point, White may play \symrook e4 to attack the f4 pawn and exchange the Black knight.

            Alternatively, White may play \symknight e4 to exchange the Black knight. In either case, e4 is 
            an important square and must be protected in advance. \variation[invar]{25... Bg6} is a good move.
        \end{itemize}
	}
\fi 