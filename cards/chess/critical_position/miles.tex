\ifdefined\miles
    % The command is already defined, do nothing.
\else
    \newcommand{\miles}[1]{
	\dynamicchessproblem{#1}{Miles}{3Q4/q4ppk/p3pn1p/1p6/1P6/P1r1PB1P/5PP1/3R2K1 w - - 0 1}
    }
\fi

\ifdefined\milesAnswer
    % The command is already defined, do nothing.
\else
    \newcommand{\milesAnswer}[1]{
		\subsection*{#1. Miles}
        \newchessgame[
            setfen=3Q4/q4ppk/p3pn1p/1p6/1P6/P1r1PB1P/5PP1/3R2K1 w - - 0 1,
        ]
        \chessboard

        \begin{itemize}
            \item{Who has a static advantage?}

            White has a static advantage: His king is safer and his bishop is better than 
            Black's knight. 
            \item{Which plan should I choose?}

            White can choose both, either accumulate his advantage slowly or
            search for a tactical blow. 
            \item{What is the move?}

            Let's examine some ideas.
        \end{itemize}

        We first notice White can trap Black's queen:

        \variation[invar]{1. Be4 Nxe4 2. Rd7 Rc1+ 3. Kh2 }

        \chessboard[setfen=3Q4/q2R1ppk/p3p2p/1p6/1P2n3/P3P2P/5PPK/2r5 b - - 3 3]

        The queen is trapped. What can Black do? 

        \variation[invar]{3... Nd2!}

        \chessboard[setfen=3Q4/q2R1ppk/p3p2p/1p6/1P6/P3P2P/3n1PPK/2r5 w - - 4 4]

        \variation[invar]{4. Rxa7 Nf1+ 5. Kh1 Nd2+} Black has perpetual check.
        \variation[invar]{4. Rxd2 Rc7} The position looks equal.
        White can try \variation[invar]{4. h4!?}, threatening \symqueen xa7 now.

        \chessboard[setfen=3Q4/q2R1ppk/p3p2p/1p6/1P5P/P3P3/3n1PPK/2r5 b - - 0 4]

        In this position, Black must find \variation[invar]{4... Rd1! 5. Rxa7 Nf1+ 6. Kh1 Nd2+} with perpetual check.
	
        In this variation, Black must find two good moves \variation[invar]{3... Nd2} and \variation[invar]{4... Rd1} to 
        draw the game. Depending on the time situation, White has some practical chance to win the game.

        We go back to the first move again.

        \chessboard[setfen=3Q4/q4ppk/p3pn1p/1p6/1P6/P1r1PB1P/5PP1/3R2K1 w - - 0 1]

        In the first variation, the White king at the back rank allows Black some counterplay.
        Therefore we choose to play with prophylaxis.

        \variation[invar]{1. Kh2}

        \chessboard[setfen=3Q4/q4ppk/p3pn1p/1p6/1P6/P1r1PB1P/5PPK/3R4 b - - 1 1]

        Again, \symbishop e4 is threatened. Black must choose the endgame to defend:

        \variation[invar]{1... Qc7+ 2. Qxc7 Rxc7 3. Rd6 Ra7 4. Rb6}

        \chessboard[setfen=8/r4p1k/pR2pn1p/1p4p1/1P6/P3PB1P/5PPK/8 w - - 0 5]

        \variation[invar]{1...a5 2. bxa5 Rxa3 3. Be2 Nd5 4. Rc1 Qe7 5. Qxe7 Nxe7 6. Rc7 Nd5 7. Rxf7 Rxa5 8. Rb7 Ra2 9. Bd3+ Kg8 10. Kg3 b4 11. h4 }

        \chessboard[setfen=6k1/1R4p1/4p2p/3n4/1p6/3BP1KP/r4PP1/8 w - - 0 11]

        In both variations, White has some advantage in the endgame. It is however 
        far from easy to win.

        We go back to the first move again.

        \chessboard[setfen=3Q4/q4ppk/p3pn1p/1p6/1P6/P1r1PB1P/5PP1/3R2K1 w - - 0 1]

        In the previous variation, Black can always play \variation[invar]{1... Qc7}, 
        forcing a queen exchange. A further idea is playing \variation[invar]{1. h4}

        \variation[invar]{
            1. h4 Qc7 2. Qxc7 Rxc7 3. Rd6 Ra7 4. h5 g6 5. hxg6+ Kxg6 6. Rb6 
        }

        \chessboard[setfen=8/r4p2/pR2pnkp/1p6/1P6/P3PB2/5PP1/6K1 b - - 1 6] 

        White has some advantage.

        Conclusion: either we play \symbishop e4, hoping Black cannot find the moves to win the game quickly,
        or we choose between \symking h2 and h4 to play the endgame. The decision is far from easy to make.
    }
\fi