\ifdefined\nielsenDreev
    % The command is already defined, do nothing.
\else
    \newcommand{\nielsenDreev}[1]{
	    \dynamicchessproblem{#1}{Nielsen - Dreev}{r4rk1/pp1nb1p1/2p1p2p/5q1P/3PQBN1/8/PPP2P2/1K1R3R w - - 0 1}
    }
\fi

\ifdefined\nielsenDreevAnswer
    % The command is already defined, do nothing.
\else
    \newcommand{\nielsenDreevAnswer}[1]{
		\subsection*{#1. Nielsen - Dreev}
		\chessboard[
			setfen=r4rk1/pp1nb1p1/2p1p2p/5q1P/3PQBN1/8/PPP2P2/1K1R3R w - - 0 1,
            markstyle=circle,
            linewidth=0.05em,
            markfields={h5, f2},
            pgfstyle=straightmove,
            linewidth=0.05em,
            markmove={e6-e5},
        ]

        \begin{itemize}
            \item{Who has a static advantage?}

            Black has the static advantage. His pawn structure is better because
            White has weaknesses on h5 and f2. 

            Black also intends to play e5 at some point
            to remove his weak e6 pawn. 

            \item{Which play shall I choose?}

            I must therefore choose a dynamic play. 

            \item{Whas is the move?}
            Note Black can neutralize White's aggression on the king side 
            since he has enough defenders. (queen, rook, bishop) 
            
            White can choose to exchange the queens and change the dynamic 
            of the game by playing

            \variation[invar]{19. Qxf5 Rxf5 20. Bxh6 gxh6 21. Nxh6+ Kh7 22. Nxf5 exf5 23. d5 
            \xskakcomment{ White has a rook and two pawns for the minor pieces. He also
            has a passed pawn}}
            (\variation{23... exd5 24. Rxd5 \xskakcomment{ Black loses his f5 pawn.}}) 
        \end{itemize}

        In the actual game, Nielsen chooses \variation{19. Qe2}, has to therefore
        defend an inferior position with no counter play and loses the game.
    }
\fi