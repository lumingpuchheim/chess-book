In 2025 I travelled through Southeast Asia for half a year. After the journey,
I felt as if my mind were overflowing. Every day I dreamed of the places I had
seen, from Malaysian jungle to Vietnamese beaches, from the mountains of Nepal
to the islands of Thailand. At some point I realised that I needed to let my
mind rest if I wanted to live a more meaningful life.

Chess feels similar. There are countless ideas waiting to be discovered.
I have read many chess books --- from middlegame themes to endgame technique,
from opening theory to annotated grandmaster games. Often I feel overwhelmed
by the richness of the game, and it is a pity how quickly these ideas fade
if I do not write them down.

I am also not entirely satisfied with most books on the market. They often
contain long, computer-generated variations, many of which are irrelevant or
misleading for a human player. It is difficult for me to extract clear,
practical knowledge from such forests of lines.

I believe that ideas are far more valuable than variations. If I only remember
opening variations, I may play a ``perfect'' opening, but soon lose the thread
of the game, shuffling pieces until my opponent launches a decisive attack.
What I really need are guiding ideas: how to handle certain pawn structures,
how to play typical positions that arise from my openings.

People say that life is a game of chess. I agree. I often feel overwhelmed
both in life and at the board. That is why I am writing this notebook.
On one hand, I hope to preserve the ideas I have collected. On the other,
and more importantly, I hope that by organising my thoughts, I can find a
little more inner peace.

\vspace{2em}
\begin{flushright}
Ming Lu\\
\emph{Aegina, Greece}\\
\emph{\today}
\end{flushright}
