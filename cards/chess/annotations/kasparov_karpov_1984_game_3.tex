\section{Anatoly Karpov - Garry Kasparov, FIDE World Championship Match 1984/85, Round 3}
\epigraph{An impulsive, nervous reaction, the preceding game would appear to have put me in a not altogether correct
frame of mind, such that I could solve all my problems immediately with the help of a `sharp' pawn sacrifice}{\textit{Garry Kasparov}}

\keywords{Maroczy Bind, Hedgehog Position, Impulsive Reaction}
% My first annotation without books and computers. I chose this game
% because it is quite short. The strategic win from Karpov is also instructive according
% to chessgames.com

% I am happy with the result that I found out 12... Nc5 was bad. 
% I didnt figure out 16... d5 was explicitly because my analysis was not thourough enough and I 
% turned to books too early..

% The winning of Karpov was instructive. I should have taken more time.
\newchessgame[
 id=main,
 event={FIDE World Championship Match 1984/85},
 white={Karpov, Anatoly},
 black={Kasparov, Garry},
 round={3}]

\mainline{1.e4 c5 2.Nf3 e6 3.d4 cxd4 4.Nxd4 Nc6 5.Nb5 d6 6.c4 Nf6 7.N1c3
a6 8.Na3 Be7 9.Be2 O-O 10.O-O b6 } 

\chessboard[
  markstyle=circle,
  linewidth=0.05em,
  markfields={b6},
]

Black has weakness on b6. The next moves of White are logical, attacking 
the weakness and developing his pieces.


\mainline{11.Be3 Bb7} 
Black has a so-called hedgehog position. Typically Black wants to play 
b6-b5 and d6-d5 to free himself. With White's Maroczy bind (c4 and e4 pawns),
he can stop Blacking from playing d5. With White's next move, he stops both b5 and d5 break.

\chessboard[
    pgfstyle=straightmove,
    linewidth=0.05em,
    markmove={b6-b5, d6-d5},
    markstyle=circle,
    linewidth=0.05em,
    markfields={b6},
]

\mainline{12. Qb3}

\chessboard

b6 pawn is under attack. \variation[invar]{12... Nd7} is a natural move.
after \variation[invar]{13. Rad1 \xskakcomment{ Preventing Nc5 with \symbishop xc5. The bishop on
b7 is unprotected..} Qc7 \xskakcomment{intending \symknight c5}} Black is still in the game.

\mainline[level=1]{12... Na5?} 

Kasparov chose to force the matter. After this move White has 3 vs 1 
on the queenside and Black has a backward pawn on the d-file.

\chessboard

\variation[invar]{13. Bxb6 Nxb3 14. Bxd8 Rfxd8 15. axb3 Nxe4 16. Nxe4 Bxe4} is not good enough because
Black's bishop pair gives him enough chance to defend.

\mainline[level=1]{ 13.Qxb6
Nxe4 14.Nxe4 Bxe4 15.Qxd8 Bxd8} (\variation[invar]{15... Rxd8 16. Bb6} This is by the way another
drawback of the move \variation{12... Na5}) 


\mainline[level=1]{ 16.Rad1} 

Quite often the most natural move is the best.

\chessboard

% I wrote "This position must be estimated carefully before Black's 12th move."
% Actually Kasparov has expected only 16. Rfd1 in his preparation and was instantaneously nervous after
% seeing this move. If he could not foresee such a move, I shouldnt expect myself
% to foresee it on the board. 


\mainline{16... d5?}

\quote{An impulsive, nervous reaction, the
preceding game would appear to have put
me in a not altogether correct frame of
mind, such that I could solve all my
problems immediately with the help of a
`sharp' pawn sacrifice} \cite{Kasparov:2008}



\variation[invar]{16... Be7 17. Nb1 \xskakcomment{ (improving his worst piece) } Rac8} Black can still hold on.

\chessboard

\mainline{ 17.f3 Bf5 }

\chessboard

\mainline{18.cxd5
exd5 19.Rxd5 Be6 20.Rd6 Bxa2 21.Rxa6 Rb8 }

\chessboard

\mainline{22.Bc5! Re8 23.Bb5! }
Notice how Karpov used bishops to win tempi and protect
the b2 pawn from the b8 rook. Also note how the Black knight on a5 is doomed by the b5 bishop. 


\mainline{23... Re6
24.b4 Nb7 25.Bf2 Be7 26.Nc2 Bd5 27.Rd1 Bb3 28.Rd7! Rd8} (\variation[invar]{28... Bxc2 29. Rxe6}) \mainline[level=1]{29.Rxe6
Rxd7 30.Re1 Rc7 31.Bb6} 1-0

\subsection*{Lessons Learned}
\begin{itemize}
  \item{Clear your mind, no impulsive, nervous reaction!}
  \item{Evaluate carefully when forcing the exchanges}
  \item{One can make simple natural moves has the advantage}
\end{itemize}