\newpage
\section{Spassky - Polugaevsky, USSR Championship 1961}


\epigraph{Had White carried out his planned 34.Kf6!! Qxd4+ 35.Kf7! this would have been one of Spassky's best wins: such bold raids by the king under a hail of bullets can be counted on the fingers of one hand, and are part of the golden treasury of chess. \cite{kasparov:2004}}{\textit{Garry Kasparov}}

\begin{multicols}{2}
    \chessgameinfo{USSR Championship}{B.Spassky}{L.Polugaevsky}{10}{1961.01.26}{0-1}
    \newchessgame
    \mainline[level=1]{
    1. d4 Nf6 2. c4 e6 3. Nf3 b6 4. Nc3 Bb7 5. Bg5 Be7 6. e3 Ne4
7. Nxe4 Bxe4 8. Bf4 O-O 9. Bd3 Bb4+ 10. Kf1 Bxd3+ 11. Qxd3 Be7}

\begin{chessdiagram}
    \chessboard
\end{chessdiagram}

White has a lead in development, so playing for a kingside attack is logical.

\mainline[level=1]{12. h4 f5?!}

In defense, it is important not to create weaknesses for the attacker.
The text move closes the b1-h7 diagonal but makes a g4 thrust possible.

White has a simple plan: open the g-file and bring the queenside 
rook to g1 for the attack.

\mainline[level=1]{13. Ke2 d6 14. g4 Nd7 15. Rag1}

\begin{chessdiagram}
    \chessboard
\end{chessdiagram}

\mainline[level=1]{15... fxg4 16. Rxg4 Nf6
17. Rg5 Qd7}

\begin{chessdiagram}
    \chessboard
\end{chessdiagram}

White has two possible plans: double rooks on the g-file then play h5, or play h5 directly. 

If White doubles rooks on the g-file, Black can simply play ...\symrook f7. 
After h5 and ...h6, Black protects the g7 pawn and waits for hxg7. In that case, White's rook on g1 would be better placed on h1. 

Therefore White plays h5 directly.

\mainline[level=1]{18. h5 Ne8 19. Rg2 b5 }

\begin{chessdiagram}
    \chessboard
\end{chessdiagram}

White could have played h6 and then attacked the weak g6 pawn, gaining a material advantage:

\variation[invar]{
    20. h6 g6 21. Rxg6+! hxg6 22. h7+ Kh8 23. Be5+! 
    Ng7 \xskakcomment{ (20... dxe5 21. \symknight xe5 and then \symknight xg6)} 24. Qxg6 Bf6 25. Bxf6 Qf7 26. Bxg7+ Qxg7 27. Qxg7+ Kxg7 28. Ng5 
}

\begin{chessdiagram}
    \chessboard[setfen=r4r2/p1p3kP/3pp3/1p4N1/2PP4/4P3/PP2KP2/7R b - - 1 28]
\end{chessdiagram}

\mainline[level=1]{20. c5}

\begin{chessdiagram}
    \chessboard
\end{chessdiagram}

\mainline[level=1]{20... dxc5}

Black voluntarily weakens his e5 square. ...\symbishop f6 would have been better.

\mainline[level=1]{21. h6 Rf5}

The standard response g6 loses quickly.
\variation[invar]{21...g6 22. Rxg6+ hxg6 23. Qxg6+ Kh8 24. Ne5 }

\begin{chessdiagram}
    \chessboard
\end{chessdiagram}

\mainline[level=1]{22. Be5! c4 23. Qe4 }

\variation[invar]{23. Qxd5 exd5 24. hxg7} is more prosaic, exchanging pieces in a winning position. 

\mainline[level=1]{23...Qd5 24. Qg4 c3 25. b3 b4}

\begin{chessdiagram}
    \chessboard
\end{chessdiagram}

Black has created a protected passed pawn. If he can survive the attack, he will have an advantage.

White is playing for mate. 
Exchanging pieces with \variation[invar]{
    26. Bxg7 Qxf3+ 27. Qxf3 Rxf3 28. Kxf3
} would lead to a straightforward win. 

\mainline[level=1]{26. e4 Qb5+ 27. Ke3 Rf7 28. hxg7 Nf6}

``I had a lot of time. I thought: I'll have 7.5 out of 10, I'm the leader of the qualifying tournament. 
I was overjoyed when I played this forced variation'' (Boris Spassky).

\mainline[level=1]{29. Bxf6 Rxf6 30. Rxh7}

\begin{chessdiagram}
    \chessboard
\end{chessdiagram}

After Black takes the h7 rook, White will play \variation[invar]{
    31. Rh2 Kg8 32. Rh8+ Kf7 33. g8+
}. Black gives checks, hoping to save the game.

\mainline[level=1]{30... Rxf3+ 31. Kxf3 Qd3+ 32. Kf4 Bd6+ 33. Kg5 Kxh7 }

\begin{chessdiagram}
    \chessboard
\end{chessdiagram}

``Erratic thinking at moments of terrible tension, especially with the 
opponent's flag about to fall, occurs even with great players'' (Kasparov). 

``That's the position I wanted, and I saw that there's a very simple win: 
\variation[invar]{
    34. Kf6!! Qxd4 35. Kf7
} and, as Kazimirych would say, amen to the pies, and it would turn out very nicely, 
in Borisenko's words, but, alas, it didn't happen. After this game, I had a 
straight road to the Interzonal tournament, and then further... But something 
terrible happened.
I was haunted by that position for years. Now I'm not—such things happen once 
in a lifetime. But I wanted to do everything the best possible way, 
and I saw a one-move win. So I thought, if I have one move, 
why do I have to get the King to f6 and then f7? So, with 7.5/10 and 
leadership in mind, I played the next move'' (Boris Spassky).

\mainline[level=1]{34. Kh5} 

``And you know—I just forgot about the b5 square. Just forgot. 
Look, in chess, there are some situations when you lose a game, but still get 
something good out of it. You get useful experience, or you see that you 
shouldn't play so rashly and should calculate more precisely, or something'' (Boris Spassky).

\mainline[level=1]{ 34...Qb5+ 35. Kh4 Be7+
36. Kh3 Qg5 37. Qxg5 Bxg5 38. Rxg5 Rd8}

\begin{chessdiagram}
    \chessboard
\end{chessdiagram}

\mainline[level=1]{ 39. f4?}

\variation[invar]{
    39. Kg4 Kg8 40. Rc5 Rxd4 41. Rxc7 Rxe4+ 42. Kg5 Kh7 43. Rxa7 Re2 44. f3 Rg2+ 45. Kf4 Rxg7 46. Rxg7+ Kxg7 47. Ke3 
} would lead to a draw. Kasparov did not point this out in his commentary.

\begin{chessdiagram}
    \chessboard
\end{chessdiagram}

Black finds the right move by \vocab{prophylaxis}{Prophylaxis}{}. Black wins the endgame. \index{Prophylaxis}

\mainline[level=1]{39...Kg8! 40. Rc5 Rxd4
41. Rxc7 Rxe4 42. Kg4 e5 43. a3 Rxf4+ 44. Kg5 a5 45. axb4 axb4
46. Kg6 Rg4+ 47. Kf6 Kh7 48. g8=Q+ Kxg8 49. Kxe5 Rg1 50. Kf6
Rf1+ 51. Ke5 Rb1}
\end{multicols}

\subsection*{Lessons Learned}
It is far from the truth that Spassky was too fragile. He made 
one mistake when he could have mated his opponent. The endgame that followed 
was too difficult to handle. Even Kasparov did not find the 
right move decades later, analyzing with computers. 

Spassky's only mistake was playing too hastily when 
his opponent was in time trouble. By ignoring the clock and playing the right move,
he would have won the game with a brilliancy.