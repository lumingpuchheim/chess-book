\newpage
\section{Emanuel Lasker - Wilhelm Steinitz, World Championship 1894, Game 7}
\epigraph{The hardest game to win is a won game}{\textit{Emanuel Lasker}}

\keywords{Intermediate moves, Prophylaxis}

\begin{multicols}{2}
    \chessgameinfo{World Championship}{Lasker}{Steinitz}{7}{1894.08.26}{1-0}
\newchessgame[
    id=main,
    event={World Championship 1894},
    white={Lasker},
    black={Steinitz},
    round={7}
]

\mainline[level=1]{1. e4 e5 2. Nf3 Nc6 3. Bb5 d6 4. d4 Bd7 5. Nc3 Nge7 6. Be3 Ng6}

\begin{chessdiagram}
    \chessboard
\end{chessdiagram}

I usually don't comment on openings, especially from the old masters. 
Just observing the position we can conclude White is much better. Assuming White castles
on the queen side and Black castles on the king side. White is ready to launch with his h-pawn.
Black's knight on g6 is unfortunate. Not only does it waste a tempo (on f6 would be much better), 
but it is also a perfect target for White's h-pawn.

Black, on the other hand, needs much more time to prepare. His queen cannot be activated soon 
because his c-pawn is blocked by the knight on c6. If White plays \variation[invar]{7. d5}, closing the center, 
I can hardly see any counter play. Lasker has however a different plan.

\mainline[level=1]{7. Qd2 Be7 8. O-O-O a6 9. Be2 exd4 10. Nxd4 Nxd4 11. Qxd4 Bf6 12. Qd2 Bc6 13. Nd5 O-O }

\begin{chessdiagram}
\chessboard
\end{chessdiagram}

Optically, White has more more active pieces. \variation[invar]{14. h4} would be natural.
For example \variation[invar]{14. h4 Bxh4 15. g3 Bf6 16. f4 Re8 17. Bf3 Nf8 18. Qh2}

\begin{chessdiagram}
\chessboard[
    setfen=r2qrnk1/1pp2ppp/p1bp1b2/3N4/4PP2/4BBP1/PPP4Q/2KR3R b - - 4 18
]
\end{chessdiagram}

White has already doubled the heavy pieces on the h-file whilte Black has not launched the pawn!

We go back to the game.

\begin{chessdiagram}
    \chessboard
    \end{chessdiagram}

\mainline[level=1]{14. g4?}

The move loses a pawn without compensation.


\mainline{14... Re8 15. g5 Bxd5} 

\begin{chessdiagram}
\chessboard
\end{chessdiagram}

The candidate moves are exd5, gxf6, or \symqueen xd5.

After \variation[invar]{16. exd5 Rxe3 17. fxe3 Bxg5!}, Black solves all
his problems and the position is equal. However, such an exchange 
sacrifice becomes standard in the 20th century, decades after this game.
I am not sure if Steinitz could find this move.

After \variation[invar]{16. gxf6 Bxe4 17. Rhg1 Qxf6} Black is simply better while
White has no counter play.

Lasker chooses the third move.

\mainline[level=1]{ 16. Qxd5 Re5} 

\begin{chessdiagram}
\chessboard
\end{chessdiagram}

After \variation[invar]{17. Qxb7 Bxg5 18. Bxg5 Rxg5 19. Bc4 Rc5 20. Bd5 Rb8 21. Qxa6 Qf6},
Black has active pieces while White has one more pawn.

Most likely, Lasker is continuing the moves he has planned on his 14th move and
misses Steinitz' refutation.

\begin{chessdiagram}
\chessboard[
    setfen=1r4k1/2p2ppp/Q2p1qn1/2rB4/4P3/8/PPP2P1P/2KR3R w - - 1 22
]
\end{chessdiagram}

We go back to the current game. \mainline[level=1]{ 17. Qd2 Bxg5! 18. f4 Rxe4!}

\emph{
    This is the problem: doubling heavy pieces on the e-file
    allows Black to recover the piece
} (Neishtadt)

\mainline[level=1]{19. fxg5 Qe7}

% Dynamic play problem
\begin{chessdiagram}
\chessboard[
    setfen=r5k1/1pp1qppp/p2p2n1/6P1/4r3/4B3/PPPQB2P/2KR3R w - - 1 20
]
\end{chessdiagram}

Clearly Black has the static advantage: White is two pawns down and Black has more active heavy pieces on e-file.
Playing such a position needs nerve.

One possibility as recommended by Kasparov is take one pawn and then defend an inferior but defensible position.

\variation[invar]{20. Bf3 Rxe3 21. Bxb7 Rb8 22. Rhe1 Re5 23. Bxa6 Qxg5 24. Qxg5 Rxg5} 

\emph{Objectively, Kasparov is right. One pawn down gives greater saving chances than two pawns down.} (Dvoretsky)

Lasker chooses to continue the struggle:
\mainline[level=1]{20. Rdf1?! Rxe3 21. Bc4}

\begin{chessdiagram}
    \chessboard
    \end{chessdiagram}

The f7 pawn is under attack. \variation[invar]{21... Rf8} is an obvious
alternative. The knight can be activated by moving to e5 at some point.

\mainline[level=1]{21...Nh8} 

\emph{Typical Steinitz! The commentators admired this eccentric move, although it is apparently
not the strongest} (Kasparov)

I believe Lasker expects this move since it is typical Steinitz. 
 

\mainline[level=1]{ 22. h4 c6 23. g6}

\begin{chessdiagram}
\chessboard
\end{chessdiagram}

\mainline[level=1]{ 23... d5?} This move throws away all the advantage.

We can see from now on, the knight is stuck at the corner at the game. Black is therefore one piece down for
two pawns. Steinitz, however, still believes he has the advantage.

\variation[invar]{23... d5} is an automatic bad move. It closes the center,
it wins a tempo by chasing the White bishop. How can this be a wrong move,
it fails to see an intermediate move.

After \variation[invar]{23...hxg6 24. h5 gxh5 25. Rxh5 Re8 26. Rhh1} Black has a clear advantage because
he is up three pawns and White fails to start an attack on h-file. 

\begin{chessdiagram}
\chessboard
\end{chessdiagram}

\mainline[level=1]{24. gxh7+! Kxh7 25. Bd3+ Kg8 26. h5 Re8 27. h6 g6 28. h7+ Kg7}

\begin{chessdiagram}
\chessboard[
    setfen=4r2n/1p2qpkP/p1p3p1/3p4/8/3Br3/PPPQ4/2K2R1R w - - 1 29
]
\end{chessdiagram}

% TODO: add a prophylaxis problem
\mainline[level=1]{29. Kb1?! } \index{Prophylaxis!King Move} 

\emph{One of Lasker's characteristic ``changes of rhythm''. As long as his opponent
has not yet created any direct threats, White has a little time to make the useful
prophylaxy moves. In the ensuing complications, Black will no longer be able to exploit tactical resources involving
the enemy king's vulnerability. Such play requires both a healthy evaluating capacity andtremendous coolnesss.} (Dvoretsky)

This move has a good \vocab{prophylaxis}{prophylaxis}{} idea. However \variation[invar]{29. a3} first is more accurate \index{Prophylaxis}.

\begin{chessdiagram}
\chessboard[
    markstyle=border,
    linewidth=0.05em,
    markfields={h8},
]
\end{chessdiagram}

Black is playing practically one piece down, since his knight
is at the corner. Activating the knight urgent now.

\mainline[level=1]{29...Qe5} 
\variation[invar]{29... f6! 30. Qf2 \xskakcomment{ (\symqueen h2 is now impossible, due to back rank
weekness.)} Qe6 31. Ka2 Nf7} The knight is finally free again. We also see
why White's last move is an inaccuracy. If \variation[invar]{29. a3} has been played first,
White can play \variation[invar]{30. Qh2} after \variation[invar]{29... f6}.

This subtlety was also overlooked by Kasparov and Dvoretsky. 
White intends to play both \symking b1 and a3, but the order matters. 
Calculating the optimal sequence through brute force is practically impossible here. 
However, by comparing the consequences of each move, one can deduce that a3 must be played first, 
because \symking b1 creates a potential back-rank weakness that restricts White's options.

\mainline[level=1]{ 30. a3!}

Note that White's king castling is quite solid. The bishop on d3 protects the c2 pawn and e2 square, making Black's invasion on the second rank impossible. Black needs two tempi (c5 and then c4) to break this defense, which allows White to take his time to maneuver his queen.

\emph{In this game there is something of the `Tal' element: White's attack is rather abstract,
but it will not come to an end - all the time some threats arise!} (Kasparov)

\emph{Lasker's last two quiet moves were completely inexplicable to his comtemporaries: 
how can you play this way when two pawns are down?} (Kasparov)

\mainline{30...c5 31. Qf2 c4 32. Qh4 f6 33. Bf5}

\begin{chessdiagram}
    \chessboard[
    setfen=4r2n/1p4kP/p4pp1/3pqB2/2p4Q/P3r3/1PP5/1K3R1R b - - 1 33,
    markstyle=circle,
    linewidth=0.05em,
    markfields={g6},
    markstyle=border,
    linewidth=0.05em,
    markfields={h8},
    pgfstyle=straightmove,
    linewidth=0.05em,
    markmove={h1-g1},
]
\end{chessdiagram}


 The position is deeply complex and demands careful calculation.

\begin{itemize}
    \item{Where are the weaknesses?}

     The g6 square is the most vulnerable point in Black's camp.
    \item{Which is the worst-placed piece?}

     The h8-knight is completely sidelined, and there is no clear route to bring it back.
    \item{What is my opponent's idea?}

     White has nothing concrete yet, but \symrook hg1 followed by a sacrifice on g6 is in the air.
\end{itemize}

 White enjoys a static edge: his king is well protected and the knight on h8 remains sleeping.

 Black, in contrast, lacks counterplay. Occupying the second rank is either impossible or far too slow.

 With that in mind, Black must choose between several options:

\begin{itemize}
    \item {(Exchange piece) \symqueen g3}

    \variation[invar]{33... Qg3 34. Qh6+ Kf7 35. Ka2 \xskakcomment{ Prophylaxis! 35. \symrook hg1 is too early because 35... \symrook e1, exchanging all the rooks} Re1 \xskakcomment{Forced, to parry \symrook hg1} 36. Qd2 R1e5 37. Rh3 Qg5 38. Qxg5 fxg5 39. Bd7+ }
    
    \chessboard[setfen=4r2n/1p1B1k1P/p5p1/3pr1p1/2p5/P6R/KPP5/5R2 b - - 1 39]
    
    White has a clear advantage.
    \item {(Accept the challenge) gxf5}

    The threat is stronger than the execution. Now White has no more threat on g6.

    \variation[invar]{33... gxf5 34. Rhg1+ Kf7 35. Qh5+ Ke7 36. Rxf5 Qe6 37. Rxd5 Re1+ 38. Rd1 Rxg1 39. Rxg1 Kd8 }
    
    \chessboard[setfen=3kr2n/1p5P/p3qp2/7Q/2p5/P7/1PP5/1K4R1 w - - 1 40]
    
    The ending is equal.
    \item {(Create some counter play) c3}

    \variation[invar]{33... c3 34. Qh6+ Kf7 35. Rhg1 Rg3 36. Bd7 Re7 37. Rxg3 Qxg3 38. Rxf6+ Kxf6 39. Qf8+ Kg5 40. Qxe7+ Kh6 41. Qf8+ Kxh7 42. Qe7+ Kh6 43. Qf8+ Kh7 44. Qe7+ Kh6 45. Qf6 }
    
    \chessboard[setfen=7n/1p1B4/p4Qpk/3p4/8/P1p3q1/1PP5/1K6 b - - 7 45]
    
    The ending is equal.

    \variation[invar]{33... c3 34. b3 Kf7 35. Bd3 f5 \xskakcomment{ Intending \symrook xd3 then \symqueen e2} 36. Bxf5! gxf5 37. Qg5 Ke6 38. Rxf5 Qe4 39. Qf6+ Kd7 40. Rd1 Kc8 41. Rfxd5 Kb8 42. Rd7 Re1 43. Rxe1 Qxe1+ 44. Ka2 }

    \chessboard[setfen=1k2r2n/1p1R3P/p4Q2/8/8/PPp5/K1P5/4q3 b - - 1 44]

    The ending is equal.

     \item {(Bring the king to safety) \symking f7}

    Steinitz' choice. The middle game struggle continues.
\end{itemize}

 None of these lines is forced; over the board one must eliminate the wrong plan and select the move that best suits one’s style.

\mainline[level=1]{33...Kf7 34. Rhg1 gxf5 35. Qh5+ Ke7 36. Rg8}

\mainline[level=1]{36...Kd6}

\begin{chessdiagram}
\chessboard
\end{chessdiagram}
\emph{I believe it was exactly here that Steinitz made the decisive error:
unlike the other commentators, I fail to see where Black could havesaved himself after this.} (Dvoretsky)

This assessment is exaggerated; only after Black's next move does White obtain a tangible advantage.

\mainline{37. Rxf5}

% TODO: add an intermediate move problem
\begin{chessdiagram}
    \chessboard[
    setfen=4r1Rn/1p5P/p2k1p2/3pqR1Q/2p5/P3r3/1PP5/1K6 b - - 0 37
]
\end{chessdiagram}


\mainline[level=1]{37... Qe6?} 

The queen is under attack and it feels natural to move her immediately.
Yet both Kasparov and Dvoretsky overlooked
\variation[invar]{37... Re1+ 38. Ka2 Qe2 39. Rxd5+ Kc6 40. Rc5+ Kb6 41. Qxe2 R1xe2 42. Rc4}.
Once the queens come off, White's attack evaporates and the passed pawn is firmly blockaded.
The position should be equal. 

\mainline[level=1]{ 38. Rxe8 Qxe8 39. Rxf6+ Kc5 40. Qh6 Re7 41. Qh2 Qd7 42. Qg1+ d4 43. Qg5+ Qd5 44. Rf5 Qxf5 45. Qxf5+ Kd6 46. Qf6+ \xskakcomment{ 1-0}}

\end{multicols}

What an exhilarating fight this was. Lasker emerged two pawns down straight from the opening, yet he managed to keep complicating the position and piling up fresh questions for his opponent. For a long stretch the engine verdict is close to equal—though Steinitz was convinced he was the one pressing—because every time Black solved a problem, Lasker conjured another. Eventually, after neutralizing one threat after another, Steinitz finally slipped.

Game 7 marked the start of five consecutive losses to Lasker. This was an unprecedented humiliation for a man
who had been unbeaten in match play for over 25 years and had previously declared he would win without doubt.

Steinitz attributed his collapse to poor physical condition, particularly his disability which prevented him from walking and exercising properly, causing
``insomnia, rushing of blood to the head, and general depression.''

However, the true cause of Steinitz's defeat was not his physical condition, but rather that he was facing a form of chess he had never encountered before. Lasker was decades ahead of his time. He played this game in the style of Tal with an abstract attack, about 40 years before Tal was born. Even after 100 years, annotators failed to find Steinitz's last mistake even with the help of computers. Steinitz didn't need to be so harsh on himself---it was no wonder he failed to understand Lasker's revolutionary play and why he lost the game. 

\subsection*{Lessons Learned}

One must refrain from playing automatic moves and pay special attention to intermediate moves. In this game, Steinitz's \variation[invar]{23... d5} appeared natural---it closed the center and gained a tempo by attacking the bishop. However, it failed to account for the intermediate move \variation[invar]{24. gxh7+!}, which completely changed the evaluation of the position. Automatic moves often overlook tactical nuances that can turn a winning position into a losing one.

Before launching an attack, it is essential to play some quiet moves prophylactically, removing any potential counterattack in advance. Lasker's \variation[invar]{29. Kb1} and \variation[invar]{30. a3!} exemplify this principle. These seemingly passive moves eliminated tactical resources involving the enemy king's vulnerability, allowing White to proceed with his attack without fear of back-rank or back-rank-related tactics. Such prophylactic thinking requires both accurate evaluation and tremendous composure, especially when material is down.