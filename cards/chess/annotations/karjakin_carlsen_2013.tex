\section{S.Karjakin - Magnus Carlsen 2013}

\keywords{Ruy Lopez, Prophylaxis, Process of Elimination, Playability}

\begin{multicols}{2}
\newchessgame

\mainline[level=1]{1. e4 e5 2. Nf3 Nc6 3. Bb5 a6 4. Ba4 Nf6 5. O-O Be7 6. Re1 b5 7. Bb3 d6 8. c3 O-O
9. h3 Nb8 10. d4 Nbd7 11. Nbd2 Bb7} 

\chessboard

A typical idea in Ruy Lopez. Black intends to play \symrook e8, \symbishop f8, exd4 and 
then take the e4 pawn. White must protect his e4 pawn sooner or later.

\mainline[level=1]{ 12. Bc2 Re8 13. a4 Bf8} 

\chessboard

White usually maneuvers his d2 knight to f5 via f1, e3 or g3. At the moment
this maneuver is not possible because Black can play exd4 then take the e4 pawn.

\mainline{ 14. Bd3 c6 15. Qc2 Rc8
16. axb5 axb5 17. b4 Qc7 18. Bb2 Ra8 19. Rad1 Bb6 20. c4 bxc4 21. Nxc4 Nxc4 22. Bxc4 h6
23. dxe5 dxe5 24. Bc3 Ba6 25. Bb3 c5 26. Qb2 c4 27. Ba4 Re6 28. Nxe5 Bb7}

\chessboard[
    markstyle=circle,
    linewidth=0.05em,
    markfields={c4, e4, e5},
]

A critical position! The position is full of tactics so that
it is difficult to estimate statically. At the moment both kings
are safe. Piece balance and pawn structure will change dramatically.

White's bishop is being attacked. Black has a weakness
on c4. A natural idea is \variation[invar]{29. Bb5 Ba6 30. Ra1 Bb7 31. Rxa8 Bxa8 
32. Bxc4 Rxe5 33. Bxf7+ Qxf7 34. Bxe5 Nxe4 }.

After a more or less forced line we have the following position. 

\chessboard[
    setfen=b4bk1/5qp1/7p/4B3/1P2n3/7P/1Q3PP1/4R1K1 w - - 0 35
]

White has one rook and two pawns for two pieces. His king is safe. 
The position should be playable for him.

Alternatively, White moves his bishop to c2. He is forced to weaken his 
king safety to protect his weak e4 pawn. 

\chessboard[
    setfen=r4bk1/1bq2pp1/4rn1p/4N3/BPp1P3/2B4P/1Q3PP1/3RR1K1 w - - 1 29
]

\mainline[level=1]{29. Bc2 Rae8 
30. f4 Bd6} 

\chessboard

It was quite easy for Black to find the previous moves. 
White has weaknesses on e5 and e4. Black only needed to double his 
rooks on the e-file and use his pieces to target the e5 square.

However, it is difficult for White to find the right plan. He has 
already weakened his king safety with his kingside pawn movements. He will
also have to play g3 at some point to further weaken his king. 
From this development, we can conclude that White's 29th move was wrong. \variation[invar]{29. Bb5}
would have been better.

Looking forward, White still must find his next move.

\chessboard[
    setfen=4r1k1/1bq2pp1/3brn1p/4N3/1Pp1PP2/2B4P/1QB3P1/3RR1K1 w - - 1 31,
    markstyle=circle,
    linewidth=0.05em,
    markfields={e4,e5},
    pgfstyle=straightmove,
    linewidth=0.05em,
    markmove={f6-h5},
]

\begin{itemize}
    \item{Where are the weaknesses?}

    e4 and e5.
    \item{Which is the worst-placed piece?}
   
    Not clear.
    \item{What is my opponent's idea?}

    He wants to play \symknight h5 to attack the f4 pawn.
\end{itemize}

When Black plays \symknight h5, White must play g3. As prophylaxis,
it is a good idea to protect the g3 pawn first. There are two options
\variation[invar]{31. Kh2} and \variation[invar]{31. Re3}.

Karjakin should have eliminated \variation[invar]{31. Kh2} first. 
His king is on the same diagonal as the Black queen and bishop. 
\variation[invar]{31. Re3} is the right move.
\variation[invar]{31. Re3 Nh5 32. g3 g5} 
(\variation{32...f6 33. Nxc4 Bxf4 34. gxf4 Nxf4 35. Bb3 \xskakcomment{ Black king
is now exposed, White has advantage.}}) \variation {33. Ba4 R8e7 34. Qe2} White is fine.

\mainline[level=1]{ 31. Kh2? Nh5 32. g3 f6 33. Ng6 Nxf4}

\chessboard

\variation[invar]{34. gxf4 Bxf4+ 35. Kh1 Rxe4 36. Bxe4 Rxe4 37. Kg1 Be3+ 38. Rxe3 Rxe3 39. Qh2 Qxh2+ 40. Kxh2 Rxc3 41. Qh3}
Black is winning. Karjakin finds the only defense.

\mainline[level=1]{34. Rxd6 Nxg6 35. Rxe6 Rxe6 36. Bd4 f5
37. e5} 

\chessboard

White is still paying debts for weakening his own king safety voluntarily.
Black can attack along the main diagonal. Actually he only needs to make
natural moves and again White must come up with good defense.

\mainline[level=1]{37... Nxe5! 38. Bxe5 Qc6}

\chessboard

Here again, White must find the right defense. 

\variation[invar]{39. Be4 fxe4 40. Re3} is his last chance. After the text move, he
has no more chances.

\mainline[level=1]{39. Rg1? Qd5 40. Bxf5 Rxd5 41. Bg4 h5 42. Bd1 c3 43. Qf2 Rf5 44. Qe3 Qf7 45. g4 Re5 46. Qd4 Qc7 \xskakcomment{ 0-1}}

\end{multicols}

\subsection*{Lessons Learned}
Practically a player should choose to play playable positions.
We see again and again during the game, White struggles
to find the right move while Black only needs to make
logical moves. Such play is of course tiring for White and 
at some point, he makes decisive mistakes and then loses the game.

King safety is always the most important thing. It is unwise 
to weaken king safety.

Prophylaxis and Process of Eliminations are important
tools to find the right move without calculating too much. 
At move 31, White finds the right idea with prophylaxis.
However, he fails to choose the right move with Process of 
Elimination.