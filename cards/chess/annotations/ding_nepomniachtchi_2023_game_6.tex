\section{Ding, Liren - Ian Nepomniachtchi 2023 World Championship Game 6}

\keywords{London System, Knight Maneuvering}
\begin{multicols}{2}
    \newchessgame[      
        id=main,
        event={FIDE World Championship 2023},
        white={Ding, Liren},
        black={Nepomniachtchi, Ian},
        round={6}]
    \mainline[level=1]{1. d4 Nf6 2. Nf3 d5 3. Bf4 c5 4. e3 Nc6 5. Nbd2 cxd4
    6. exd4 Bf5 7. c3 e6} 
    
    \chessboard[
    ]
    
    Modern chess is full of subtleties. The position is actually a reversed
    Queen's Gambit Declined Exchange Variation. Typically after 
    \variation[invar]{1. d4 d5 2. c4 e6 3. Nc3 Nf6 4. cxd5 exd5 5. Bg5 Be7 6. e3 c6}
    
    \chessboard[
        setfen=r1bqk2r/pp1nbppp/2p2n2/3p2B1/3P4/2NBP3/PP3PPP/R2QK1NR w KQkq - 2 8
    ]
    
    We have a so-called Karlsbad structure. White has the following typical ideas:

    \begin{itemize}
        \item{Minority Attack}

        White advances his queenside pawns with the main purpose of creating weaknesses in black’s structure.

        \chessboard[
            setfen=8/pp3ppp/2p5/3p4/3P4/4P3/PP3PPP/8 w KQkq - 2 8,
            pgfstyle=straightmove,
            linewidth=0.05em,
            markmove={b2-b4, b4-b5, b5-c6, b7-c6},
        ]
        
        \item{Playing for the e3-e4 push}

        This can be done with or without the support of the f pawn (by pushing f3) then e4.
    
        \chessboard[
            setfen=8/pp3ppp/2p5/3p4/3P4/4P3/PP3PPP/8 w KQkq - 2 8,
            pgfstyle=straightmove,
            linewidth=0.05em,
            markmove={e3-e4},
        ]   
    \end{itemize}
    
    Black must defend accordingly. His ideas are
    \begin{itemize}
        \item{Moving the knight from g8 to e4}
        \item{Moving the knight from b8 to c4 via d7 and b6.}
    \end{itemize}
    
    We will see in this game that Ding adopts both of Black's ideas from the Queen's Gambit Declined Exchange Variation.  
    Personally I am not a fan of Black's opening. If in the Queen's Gambit
    Declined Exchange Variation, Black can defend well. With one more
    tempo and colors reversed, White should have some advantage. How can Black still
    defend? 

    Let's go back to the current game.

    \chessboard[
        pgfstyle=straightmove,
        linewidth=0.05em,
        markmove={e6-e5},
    ]

    The position is static and slow, meaning both players must
    seek plans episode after episode to improve their positions.

    \begin{itemize}
        \item{Where are the weaknesses?}
    
        It is still too hard to tell.
        \item{Which is the worst-placed piece?}
       
        The bishop on f1 must be developed, then castling.
        \item{What is my opponent's idea?}
    
        He wants to play e5 to free himself.
    \end{itemize}

    Understanding these, White should develop his bishop and then castle.
    He should also pay attention to Black's e5 thrust. So \symrook e1 is natural.

    \mainline[level=1]{ 8. Bb5}

    \chessboard[
        pgfstyle=straightmove,
        linewidth=0.05em,
        markmove={e6-e5},
    ]

    \begin{itemize}
        \item{Where are the weaknesses?}
    
        It is still too hard to tell.
        \item{Which is the worst-placed piece?}
       
        The bishop on f8 must be developed, then castling.
        \item{What is my opponent's idea?}
    
        Not clear yet.
    \end{itemize}

    White has a strong bishop outside its pawn chain on f4. Exchanging it
    with \symbishop d6 and developing is logical. The next few moves are natural.

    \mainline[level=1]{ 8...Bd6 9. Bxd6 Qxd6 10. O-O O-O
    11. Re1 h6} 
    
    \chessboard [
        markstyle=circle,
        linewidth=0.05em,
        markfields={c5, e5},
        markstyle=border,
        linewidth=0.05em,
        markfields={a1, d2},
        pgfstyle=straightmove,
        markmove={e6-e5},
    ]
    
    \begin{itemize}
        \item{Where are the weaknesses?}
    
        c5 and e5 are weak. White may have knights to occupy these
        squares.
        \item{Which is the worst-placed piece?}
       
        Not clear, neither the rook on a1 nor the knight on d2.
        \item{What is my opponent's idea?}
    
        e5 thrust
    \end{itemize}
    
    \mainline{ 12. Ne5} 
    
    \begin{itemize}
        \item{Where are the weaknesses?}
    
        c5 and e5 are weak. White may have knights to occupy these
        squares.
        \item{Which is the worst-placed piece?}
       
        Not clear, either the rook on a1 or the knight on d2.
        \item{What is my opponent's idea?}
    
        e5 thrust
    \end{itemize}
    
    \chessboard [
        markstyle=circle,
        linewidth=0.05em,
        markfields={c5, e5},
    ]

    \begin{itemize}
        \item{Where are the weaknesses?}
    
        c5 and e5 are weak squares.
        \item{Which is the worst-placed piece?}
       
        Not clear.
        \item{What is my opponent's idea?}
    
        \symbishop xc6, then occupy c5 with his knight.
    \end{itemize}

    \mainline{12... Ne7 } 
    
    % TODO:
    % r4rk1/pp2npp1/3qpn1p/1B1pNb2/3P4/2P5/PP1N1PPP/R2QR1K1 w - - 2 13
    % play out!


    \mainline{13. a4 a6}
    This move is okay in itself. However, Black starts a wrong plan that
    causes his position to deteriorate soon.

    \mainline{ 14. Bf1}
    
    \chessboard [
        markstyle=circle,
        linewidth=0.05em,
        markfields={c5},
        markstyle=border,
        linewidth=0.05em,
        markfields={a1, d2},
        pgfstyle=straightmove,
        markmove={a4-a5},
    ]

    \begin{itemize}
        \item{Where are the weaknesses?}
    
        c5 is weak.
        \item{Which is the worst-placed piece?}
       
        Not clear.
        \item{What is my opponent's idea?}
    
        a5 fixing the pawn structure.
    \end{itemize}

    Focus on the position only, \variation[invar]{14... a5} is the right move.
    Ironically, the opponent's plan \variation[invar]{15. a5} exists
    only since Black's last move. It is also psychologically difficult to play
    \variation[invar]{14... a5}, admitting \variation[invar]{13... a6} was a wrong
    plan. 

    \mainline[level=1]{14...  Nd7?} 
    
    \mainline{ 15. Nxd7 Qxd7} Natural moves!
    \mainline{16. a5!} We already know White's plan.
    
    
    \mainline{16... Qc7} 
    
    \chessboard

    White intends to install his knight on c5. A direct \variation[invar]{17. Nb3}
    would be answered with \variation[invar]{17... Nc6}. Black can
    also assault with \variation[invar]{18... Bc2}. White cannot improve his position.

    The queen on c7 is a defender of White's idea. So he tries to exchange it.
    Ding does this skillfully.

    \mainline[level=1]{ 17. Qf3 Rfc8 18. Ra3!} 
    
    \mainline[level=1]{18... Bg6 19. Nb3 Nc6} \mainline[level=1]{ 20. Qg3} 
    
    \chessboard

    Here Black could have defended patiently by playing \variation[invar]{20... Qxg3 21. hxg3 Bf5 22. Nc5 Rc7}.
    His position would be quite solid.
    
    However, Nepo chooses an active plan by countering with e5 in the center. His pieces
    are too passive for such a dynamic plan. White soon gets the upper hand.
    \mainline[level=1]{20... Qe7
    21. h4!}  A typical pawn move to remove back rank weakness in a better position
    before launching an attack.
    
    
    \mainline{ 21... Re8 22. Nc5 e5?} 
    Result of a wrong plan.
    
    \mainline[level=1]{23. Rb3 Nxa5 24. Rxe5 Qf6 25. Ra3 Nc4
    26. Bxc4 dxc4 } 
    
    \chessboard

    Black should have foreseen and estimated this position before he started 
    his plan of countering in the center on his 20th move. His moves 
    are forced while White may still have some improvements.

    Who has a better position? Obviously White. He has full control of 
    the center. Had Nepo estimated this position correctly, he would have 
    chosen a different plan. 
    
    \chessboard
    \mainline{27. h5?  } 
    
    \variation[invar]{27. Nxb7 Rxe5 28. dxe5 Qb6 29. Nd6 Qxb2 30. Nxc4 } would be winning.
    Note White should keep his strong knight to attack the Black king.

    Missed an opportunity!
    \variation{27... Rxe5 28. dxe5 Qd8 \xskakcomment{ This move is hard to see.} 29. Qf3 Qd2 30. hxg6 Qe1+ 31. Kh2 Qxe5+ 32. Kg1 Qe1+ 33. Kh2 Qe5+ 34. g3 \xskakcomment{ Otherwise perpetual check} Qxc5 35. Qxf7+ Kh8 }
    would have saved the game.
    \mainline[level=1]{ 27... Bc2? 28. Nxb7 Qb6 29. Nd6 Rxe5 30. Qxe5 Qxb2}

    \chessboard

    In this position, White can use process of elimination to find the best move.

    \variation[invar]{31. Nxc4?! Qc1+ 32. Kh2 Bd3 33. Qe3 Qd1 34. Ne5 Qxh5+ 35. Qh3 Qxh3+ 36. gxh3 }
    Black has a good chance to defend. 

    \chessboard [
        markstyle=circle,
        linewidth=0.05em,
        markfields={f7},
        markstyle=border,
        linewidth=0.05em,
        markfields={a3}
    ]

    \begin{itemize}
        \item{Where are the weaknesses?}
    
        f7 pawn.
        \item{Which is the worst-placed piece?}
       
        The rook on a3.
        \item{What is my opponent's idea?}
    
        Not clear
    \end{itemize}

    Black has a weakness on f7. \symrook a5-c5-c7 is natural to attack it.
    By the way, White's rook is attacked. There is no other way to parry the threat.

    \mainline[level=1]{ 31. Ra5! Kh7 32. Rc5 Qc1+ 33. Kh2 f6 34. Qg3 a5 35. Nxc4 a4
    36. Ne3 Bb1 37. Rc7 Rg8 38. Nd5 Kh8 39. Ra7 a3 40. Ne7 Rf8
    41. d5 a2 42. Qc7 Kh7 43. Ng6 Rg8 44. Qf7 \xskakcomment{ 1-0}}
    \chessboard
    \end{multicols}


\subsection*{Lessons Learned}
A very deep game played by Ding. He gains an advantage in the middlegame and 
waits patiently until Nepo takes immature measures.

The London system can become a Queen's Gambit Exchange Variation reversed.
This suggests the question of how to handle a Queen's Gambit Exchange Variation.

One example of an answer is that both sides have chances, both sides have their 
own prospects for attack and it is the player who has the greater knowledge and 
skill (and so who can, for example, employ resources that are less obvious than 
the resources which the opponent can employ) who will get the better of it.

When defending, one must choose between passive and active defense.
One can only use active defense when one has practical chances to defend.
Otherwise, an active defense is just suicide.