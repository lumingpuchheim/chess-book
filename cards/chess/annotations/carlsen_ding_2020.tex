\section{Magnus Carlsen - Ding Liren, Magnus Carlsen Chess Tour Finals}
% \keywords{London System, French Defence, Square Play, Diagonal Play}

\epigraph{ In difficult positions I make moves that do 
not lose by force.}{\textit{Anatoly Karpov}}

\keywords{London System, French Defence, Square Play, Diagonal Play}

\begin{multicols}{2}
\newchessgame[
 id=main,
 event={Magnus Carlsen Chess Tour Finals},
 white={Magnus Carlsen},
 black={Ding Liren},
 round={1}]

 \mainline[level=1]{
    1. d4 Nf6 2. Nf3 d5 3. Bf4 c5 4. e3 e6 5. c3 Bd6 6. Nbd2}
    
\chessboard[
    markstyle=border,
    linewidth=0.05em,
    markfields={c8},
]

A typical London System opening: with a hidden cunning idea. We 
reach the first critical moment of the game. White offers a pawn sacrifice. Black 
has more options to choose from. 

\begin{enumerate}
    \item{Decline the sacrifice}

    Black has a pawn stucture similar
    to French Defence. His light squared bishop is not active. He could
    choose to proceed with a normal French Defence setup:
    \variation[invar]{6... Bxf4 7. exf4 Qb6 8. Qc2 Qc7 9. g3 b6 10. Bg2 Bg7}

    \chessboard[
        setfen=rn2k2r/pbq2ppp/1p2pn2/2pp4/3P1P2/2P2NP1/PPQN1PBP/R3K2R w KQkq - 2 11
    ]
    \item{Accept the sacrifice, take on c3}

    \variation[invar]{6... cxd4 7. Bxd6 dxc3} White has compensation and can play \variation[invar]
    {8.Qa4+} or \variation[invar]{8.Ba3} as in the game.

    \chessboard[
        setfen=rnbqk2r/pp3ppp/3Bpn2/3p4/8/2p1PN2/PP1N1PPP/R2QKB1R w KQkq - 0 8
    ]

    \item{Accept the sacrifice, take on e3}

    As in the current game:    
\end{enumerate}
    
\mainline[level=1]{6... cxd4 7. Bxd6 dxe3 8. Ba3 exd2+ 9. Qxd2}

\chessboard[
    pgfstyle=straightmove,
    linewidth=0.05em,
    markmove={d2-g5},
    markstyle=border,
    linewidth=0.05em,
    markfields={c8},
    markstyle=circle,
    linewidth=0.05em,
    markfields={g7},
]


Here another critical moment.

Let us first list the static advantages for White:
\begin{itemize}
    \item{Bishop pair}
    \item{Better King safety}
    \item{Black has an inactive bishop on c8.}
    \item{Black has weakness on the dark squares.}
\end{itemize}
Dynamically White has lead in development for a sacrificed pawn.

Black has however a complete pawn structure. He can castle after some moves. So his
position is still defendable.

White intends to play \variation[invar]{10. Qg5} threatens the g7 pawn. One move is to play
\variation[invar]{10... Ne4 11. Qe3 Qb6 12. Nd4 f6}.
Black can move his king to f7:

\chessboard[
    setfen=r1b4r/pp3kpp/1qn1pp2/3p4/3Nn3/B1PBQ3/PP3PPP/R4RK1 w - - 4 14
]

In the game, Ding chose to play 9... \symknight c6. This move is per se not bad. 
He must however play accurately afterwards

\mainline[level=1]{9... Nc6 10. Qg5 Rg8 11. Bd3 h6
12. Qe3 Qb6 13. Qe2 Bd7 14. O-O O-O-O 15. b4} 

\chessboard[
    pgfstyle=straightmove,
    linewidth=0.05em,
    markmove={b4-b5},
    markstyle=circle,
    linewidth=0.05em,
    markfields={e5},
    markstyle=border,
    linewidth=0.05em,
    markfields={d7},
]

Another critical moment.
White threatens to play b5 to drive away the knight on c6 and then controls the e5 square with his knight:
Once White achieves his goal, Black has a desperate position. Therefore he must react now!

Understanding White's idea, Black can play \variation[invar]{15... e5 16. b5 e4 17. bxc6 Qxc6 18. Ne5 Qc7 
19. Nxf7 Bg4 20. Qe3 Qxf7 21. Bc2}. This move also activates the bishop on d7 as a bonus.

\chessboard[
    setfen=2kr2r1/pp3qp1/5n1p/3p4/4p1b1/B1P1Q3/P1B2PPP/R4RK1 b - - 1 21
]

Black equalizes.

In the game, Ding's move is a serious mistake. 
\mainline[level=1]{15... Kb8? 16. b5 \xskakcomment{ Of course!} Na5
17. Ne5 Be8 18. Bb4 Rc8 19. a4} 

\chessboard

Black is passive but has no weaknesses. He should have waited (e.g. \variation[invar]{19... Qc7}) and
let White to prove his advantage. Here it is important
to remember what Karpov said: in difficult positions I make moves that do 
not lose by force.

After the text move, White wins quickly.
\mainline[level=1]{19... Ne4 20. Bxe4 dxe4 21. Qxe4 f6
22. Qh7 Nb3 23. Qxg8 Nxa1 24. Qxe8 \xskakcomment{ 1-0}
 }

 \chessboard

\end{multicols}
 % 2025. 11.6: I am happy to find the critical positions on my own: 15... Kb8
 \subsection*{Lessons Learned}
 \begin{itemize}
    \item{Knowledge in other openings with similar pawn structure
 can help to find the right idea.}
    \item{Don't make moves that lose by force in difficult positions.}
 \end{itemize}