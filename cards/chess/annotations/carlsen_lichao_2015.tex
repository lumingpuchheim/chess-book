\newpage
\section{Magnus Carlsen - Li Chao, Qatar Masters 2015}

\begin{multicols}{2}
\chessgameinfo{Qatar Masters}{Magnus Carlsen}{Li Chao}{}{2015.12.24}{1-0}
\newchessgame

\mainline[level=1]{1. d4 Nf6 2. c4 g6 3. f3 d5 4. cxd5 Nxd5 5. e4 Nb6 6. Nc3 Bg7 7. Be3 O-O 8. Qd2 Nc6 9. O-O-O f5 10. e5 Nb4 11. Nh3 Qe8 12. Kb1}

\chessboard

This is a typical position in the Grünfeld Defense. White has a pawn center, and Black's
active pieces allow for counterplay.

Here we see the famous quiet king move again. Carlsen moved his king to 
a safer square before starting his attack on the kingside. \variation[invar]{12. Nf4} would also be possible. 

\mainline[level=1]{12... a5}

The kings castle on opposite sides. Mate or being mated is the most probable
outcome. Therefore, tempi are precious. 
Black was trying to postpone \symbishop e6 as much as possible, since White
could win a tempo with \symknight f4 and then start the attack with h4.

White was trying to postpone \symknight f4 as much as possible, since Black 
could react with g5 and f4, halting White's attack.

Both sides were making useful moves until reaching a ``zugzwang'' position.

\mainline[level=1]{13. Be2 c6}

In the post-game interview, Magnus said that c6 was a good move,
but he was happy to see it because now the queen on e8 cannot 
really go to a4, and there will be no mate threat.

\mainline[level=1]{14. Rc1 Kh8 15. Ka1}

\chessboard

\mainline[level=1]{15... Be6}

Now Black played the unpleasant move!

Black could still play \variation[invar]{15... N6d5 16. Nxd5 Nxd5}. This 
position would have been much more solid than in the game.

\mainline[level=1]{16. Nf4 Qf7 17. h4}

\chessboard

What does White want? Obviously, he wants to play h5, opening the h-file. \variation[invar]{17... g5} would 
be refuted with \variation[invar]{18. Ng6}. As \vocab{prophylaxis}{prophylaxis}, Black should play \variation[invar]{17... Rfd8}
so that \variation[invar]{18. Ng6} is not possible. \index{Prophylaxis}

\mainline[level=1]{17... Bxa2}

The text move is too slow.

\mainline{18. h5}


\mainline[invar]{18...Kg8 }

Black must parry \symknight g6. However, it feels wrong. The king had moved from g8 to h8 and then back to g8 again.

\mainline[level=1]{19. hxg6 hxg6 20. g4 }


\chessboard

Here prophylactic thinking is helpful. What does White want? He wants to move his 
queen to the h-file, check on h7, and then take the pawn with his knight. Black
could drive the knight away with \variation[invar]{20... g5 21. Nh3 f4 22. Bf2 Bd5 23. Nxg5 Qg6} and
could still hold the position.

\mainline[level=1]{20...Bb3?}

Now we see how White makes natural moves to attack.

\mainline{21. Bd1! a4 22. Qh2 Rfd8 23. Qh7+ Kf8 }

\chessboard

\mainline[level=1]{24. d5! }

White needed to find the decisive blow. \variation{24. Nxg6 Ke8} goes nowhere. Noticing
that Black's queen lacks space, the text move blocks the support from Black's bishop, and the 
Black queen will be trapped.

\mainline[level=1] {24...Nc4 25. Nxg6+ Ke8 26. e6}

\chessboard

Apparently, the Black king is trapped now.

\mainline{26... a3!}

Exclamation mark for the entertainment value. 

\chessboard

Black is also threatening mate: \variation[invar]{27... axb2 28. Kb1 Ra1#}.

\mainline[level=1]{27. exf7+ Kd7 28. Ne5+ Bxe5 29. Qxf5+ Kc7 30. Qxe5+ Nxe5}

\chessboard 

Now the game is effectively over. White has a material advantage, and Black has no more mating threats.

\mainline[level=1]{31. Bxb3 axb2+ 32. Kb1 Nxc1 33. Rxc1 Kc8 34. dxc6 bxc6 35. f4} Black resigned.

\end{multicols}