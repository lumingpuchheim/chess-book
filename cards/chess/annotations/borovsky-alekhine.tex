\newpage
\begin{multicols}{2}
    \chessgameinfo{Borovsky - Alekhine}{Borovsky}{Alekhine}{}{1933}{0-1}
    \newchessgame

    \mainline[level=1]{
        1. e4 e5 2. Nf3 Nc6 3. Bb5 a6 4. Ba4 Nf6 5. O-O d6 6. c3 Bd7 7. Re1 Be7 8. d4 O-O 9. Nbd2 Be8 10. Bxc6 Bxc6 11. dxe5 dxe5 12. Nxe5 Bxe4 13. Nxe4 Qxd1 14. Nxf6+ gxf6 15. Rxd1 fxe5 }
        
    \chessboard

    Alekhine wrote that Black had an advantage here and presented a plan to win the game.

    I do not believe he had a plan during the game. In the game, he realized 
    his advantage step by step, as if he had a plan already. We describe his 
    objectives step by step.

    \mainline[level=1]{16. Bh6 Rfd8 17. Kf1}
    
    \chessboard

    This is an endgame, and the king must be activated or centralized. e6 is a great square because 
    White cannot give check there. Also, the king on e6 prevents White's infiltration with the rook 
    at d7. Candidate moves are ...f6 and ...f5.
    ...f6 can be rejected because this makes the dark-squared bishop bad. 

    \mainline[level=1]{17...f5! 18. Rxd8+ Rxd8 19. g3 Kf7 20. Be3 }
    
    \chessboard

    If Black wanted to win the game, he had to play something on the kingside now.
    Otherwise, White could play \symking e2 and then \symrook d1 to exchange the rooks.

    Alekhine wrote that Black's ``plan'' was ``operating with the rook on the 
    open g-file and advancing the h-pawn, forcing the opening of the h-file''.
    
    White could defend this idea by playing h4, closing the h-file completely. 

    Alekhine still played ...h5 and hoped White was not aware of his plan.
    
    \mainline[level=1]{20...h5 21. Ke2 Ke6 22. Rd1 Rg8 23. f3}
    
    \chessboard

    Alekhine now opened the h-file. In his plan, ``the White king and also eventually the 
    bishop would be kept busy in order to prevent the intrusion of the Black rook at h1 or 
    h2''. Actually, this was his hope, and he succeeded again.

    \mainline[level=1]{23...h4 24. Bf2 hxg3 25. hxg3 Rh8 26. Bg1 Bd6}
    
    \chessboard

    Black had managed to tie down White's bishop on the kingside. He 
    hoped to open the queenside and infiltrate with his rook there. 
    White could have frustrated Black's plan by playing a4 himself.
    For example: 
    \variation[invar]{
    27. a4 b5 28. axb5 axb5 29. Ra1
    }

    The position would be equal since White controls the a-file.
    
    \mainline[level=1]{27. Kf1}
    
    \chessboard

    Black now hoped to open the a-file and infiltrate with his rook. 
    White was unaware of Black's idea. 

    \mainline[level=1]{27...Rg8 28. Bf2 b5 29. b3 a5 30. Kg2 a4 31. Rd2 axb3 32. axb3 Ra8 }
    
    \chessboard

    Black had achieved something now. White was completely passive while Black had moved 
    his rook to the kingside to attack White's weak pawns there. White's position was, however, 
    still holdable.

    \mainline[level=1]{33. c4 Ra3 }
    
    \chessboard 

    \mainline[level=1]{34. c5} 
    
    The decisive error! White helped Black by creating a weak c5 pawn. 
    The pawn would be attacked by Black's rook and bishop, and White could only 
    defend it with his bishop. The pawn would fall and the game would be lost. 
    
    \mainline[level=1]{34...Be7 35. Rb2 b4 36. g4 f4 37. Kf1 Ra1+ 38. Ke2 Rc1 39. Ra2 Rc3 40. Ra7 Kd7 41. Rb7 Rxb3 42. Rb8 Rb2+ 43. Kf1 b3 44. Kg1 Kc6 45. Kf1 Kd5 46. Rb7 e4 47. fxe4+ Kxe4 48. Rxc7 Kf3 49. Rxe7 Rxf2+ 50. Ke1 b2 51. Rb7 Rc2 52. c6 Kg3 53. c7 f3 54. Kd1 Rxc7 55. Rxb2 f2}
    
    
\end{multicols}

\subsection*{Lessons Learned}
    That is the game. It was not a grand plan that helped Black to win the game, but improvements step by step. Also, with the help of his opponent, Black finally won the game.