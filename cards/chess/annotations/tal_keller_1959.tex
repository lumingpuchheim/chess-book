\newpage
\section{Mikhail Tal - Dieter Keller, Zurich 1959} 

\begin{multicols}{2}
    Here is one of Tal's most impressive games—so complicated that 
    Tal gave it in his book \emph{The Life and Games of Mikhail Tal} without
    any notes. I call this game an absolute TAL-Fahrt.
    \chessgameinfo{Zurich 1959}{Mikhail Tal}{Dieter Keller}{}{1959.05.27}{1-0}
    \newchessgame
    \mainline[level=1]{
        1.Nf3 Nf6 2.c4 e6 3.Nc3 d5 4.d4 c6 5.Bg5 dxc4 6.e4 b5}
    
    \chessboard 

    Tal could have also played \variation[invar]{7.e5 h6 8. Bh4 g5 9. Nxg5 hxg5 10. Bxg5 Nbd7},
    the ultra-sharp Botvinnik variation that required much more preparation.
        
    \mainline[level=1]{7.a4 Qb6 8.Bxf6 gxf6 9.Be2 a6 10.O-O Bb7}
    

    \chessboard
    
    \mainline[level=1]{11.d5 }
    
    \chessboard

    Black also has some other options. Reckoning that White has the initiative only
    because the Black king is in the center, he could also play \variation[invar]{11... Nbd7 
    \xskakcomment{ intending to move his king to the queenside. For example } 12. dxe6 fxe6 13. Nd4 c5  \xskakcomment{ the e6 pawn is now protected.}  14. Bh5+ Kd8 15. Nc2 b4}. Black has some advantage.
    
    \mainline[level=1]{11...cxd5 12.exd5}
    
    \chessboard

    \mainline[level=1]{12...b4}  
    
    Taking the central pawn is out of the question because it is too dangerous for the Black king.
    It is natural to play the text move. As mentioned before, \variation[invar]{12... Nbd7} covering the king would have been better.

    \chessboard

    Here White can choose between \variation[invar]{13. a5} and \variation[invar]{13. dxe6}.

    After \variation[invar]{13. dxe6 fxe6 14. a5 Qc6} White still has some advantage. By
    the way \variation[invar]{13. dxe6 bxc3 14. exf7 Kxf7 15. Bc4+} is too dangerous for Black.

    As a comparison, after \variation[invar]{13. a5}, \queen c6 is impossible. Therefore,
    a5 is the better move.
    
    \mainline[level=1]{13. a5}

    \chessboard

    We have here three candidate moves: \symqueen d8, \symqueen d6, \symqueen c7.

    \symqueen d8 is bad because \variation[invar]{14... Qd8 15. dxe6 Qxd1 16. exf7+ Kxf7 17. Bc4+ Kg7 18. Nxd1}
    White has more material and a lead in development, or \variation[invar]{14... Qd8 15. dxe6 fxe6 16. Nd4 bxc3 17. Bh5+ Ke7 18. Re1} White 
    has an unstoppable attack.

    \symqueen d6 is a good move. White cannot gain an advantage after taking the e6 pawn,
    because \symknight d4 doesn't win a tempo. \variation[invar]{14... Qd6 15. Ne4 Qf4! 16. dxe6! Qxe4 17. exf7+ Kxf7 18. Rc1 } White has an attack against the Black king.

    The text move is also good.
    
    \mainline[level=1]{13...Qc7 14.dxe6 }

    \chessboard

    Now taking the pawn is bad because \variation[invar]{14... fxe6 15. Nd4}.

    \mainline[level=1]{14... bxc3 15.Nd4 Rg8 16.Qa4+ }
    
    The last few moves are quite straightforward. Black declines \variation[invar]{16... Nc6 17. exf7+}.
    
    \mainline[level=1]{16...Kd8 17.g3}
    
    Don't forget to defend! 

    \chessboard

    We are now in the middlegame. In the next few moves, both players have some resources like playing cards.
    It is important to recognize these cards and decide when to play which one.

    White has:
    \begin{itemize}
        \item \symbishop xc4
        \item bxc3
        \item \symrook fd1 or \symrook ad1
        \item exf7
    \end{itemize}

    Black has:
    \begin{itemize}
        \item \symknight c6
        \item cxb2
        \item \symbishop c5
        \item \symbishop d5
        \item \symrook g5
    \end{itemize}

    White has compensation because Black's knight and rook are on the corner while White 
    has some attack against the Black king.

    We can eliminate some moves first: \index{Process of Elimination}
    \variation[invar]{17... Bc5 18. exf7 Qxf7 19. Rad1 Kc8 20. Bxc4 Bd5 21. Bxd5 Qxd5 22. Nb3} White has a big advantage.

    \variation[invar]{17... Nc6 18. exf7 Qxf7 19. Nxc6} White gains back the material and keeps the advantage. 

    \variation[invar]{17... cxb2 18. exf7 Qxf7 19. Rad1} White is crushing.

    This means only \symbishop d5 or \symrook g5 are playable.
    
    \mainline[level=1]{17...Bd5}
    
    \chessboard

    \mainline[level=1]{18.Rfd1?}

    \variation[invar]{18. bxc3 Bd6 19. Rfd1 fxe6 20. Bf3 Kc8 21. Nxe6 Bxe6 22. Bxa8}
    The material is roughly equal. White has a safer king and therefore some advantage.
    There is however no attack anymore.

    Black blocks the d-file. White must hurry to place his rook on the d-file.

    The question is which one. Noticing that
    Black can play cxb2 attacking the rook on a1, Tal should have eliminated \symrook fd1.
    Therefore, \symrook ad1 is the best move.

    \chessboard
    
    \mainline[level=1]{18...Kc8?}

    Now Black has some candidate moves: cxb2, \symking c8. The move cxb2 wins a tempo for 
    him and White must place his a1 rook on b1. On the other hand, once White plays bxc3, 
    his rook on a1 will be active. 
    
    Playing \symking c8 also allows White to play \variation[invar]{19. Qe8+ Kb7 20. bxc3} activating 
    the rook on a1 as we have discussed. By the way, \variation[invar]{19... Qd8 20. exf7} gives White an advantage.
    
    It is therefore better to play \variation[invar]{18... cxb2}. 

    \chessboard

    \mainline[level=1]{19.bxc3}
    
    Black gives White the opportunity to play \symqueen e8+. The opportunity disappears
    when Black plays \symbishop in the next move. So \variation[invar]{19. Qe8+ Kb7 20. bxc3} is better.
    
    \mainline[level=1]{19... Bc5!}
    
    The last card played!

    \chessboard
    
    \mainline[level=1]{20.e7} 
    
    White still has the \symbishop xc4 card. After \variation[invar]{
        20. Bxc4 Bxc4? 21. Qxc4 Bxd4 22. Rxd4 Qxc4 23. Rxc4+ Kd8 24. Rd1+ Ke7 25. exf7 Kxf7 26. Rc7+ Kg6 27. c4 }

    White has total domination. Black cannot bring up his knight and rook on the queenside.

    It is better for Black to play \variation[invar]{
        20... fxe6 21. Bxd5 exd5 22. c4 Bxd4 23. Rxd4 Nc6 24. Rxd5 Rd8 25. Rad1 Rb8 26. Rxd8+ Nxd8 27. Qc2 
    } The position is equal.

    \chessboard
    
    \mainline[level=1]{20... Nc6}
    
    For human players, it is easier to play \symqueen xe7 or \symbishop xe7, eliminating White's dangerous pawn. \symbishop xe7 is probably better 
    because it also eliminates White's \symbishop xc4 card.

    After \variation[invar]{20... Bxe7 21. Rab1? Nd7!} Black finally brings his knight and has some advantage.
    \variation[invar]{20... Bxe7 21. Nf5 Be6 22. Nd4 Bd5} leads to a draw after repetition.

    \chessboard

    \mainline[level=1]{21.Bg4+}
    
    Another beautiful line is:
    \variation[invar]{21. Nf5 Qe5 22. Bxc4 Bxf2+! 23. Kf1 Qxf5}

    \chessboard[setfen=r1k3r1/4Pp1p/p1n2p2/P2b1q2/Q1B5/2P3P1/5b1P/R2R1K2 w - - 0 24]
    
    Black is threatening mate. Now the show begins:
    \variation[invar]{24. Rxd5 Qh3+ 25. Kxf2 Qxh2+ 26. Kf1 Kc7 27. Rb1 Qh1+ 28. Kf2! }
    The material is equal and both kings are exposed. 

    \variation[invar]{24. e8=Q+!}

    \chessboard[setfen=r1k1Q1r1/5p1p/p1n2p2/P2b1q2/Q1B5/2P3P1/5b1P/R2R1K2 b - - 0 24]

    \variation[invar]{24... Rxe8 25. Rxd5 Qh3+ 26. Kxf2 Kc7 27. Kg1 } White has advantage because 
    his king is safer.

    \variation[invar]{24... Kc7!! 25. Qaxc6+ Bxc6 26. Qxf7+ Bd7 27. Qxd7+ Qxd7 28. Rxd7+ Kxd7 29. Bxg8 Rxg8 30. Kxf2 } Finally, a draw.

    Let's go back to the mainline.
    \mainline[level=1]{21...Kb7}
    
    \chessboard

    The position has changed completely. Both sides have played all the 
    cards, and therefore it is time for a new evaluation. Intuitively, Black 
    will soon activate all his pieces. It would be good for White if 
    he can find some immediate drawing chances.

    \mainline[level=1]{22.Nb5?}
    
    \variation[invar]{
        22. Rab1+ Ka7 23. Nxc6+ Bxc6 24. Qxc4 Rg5}  

    \chessboard[setfen=r7/k1q1Pp1p/p1b2p2/P1b3r1/2Q3B1/2P3P1/5P1P/1R1R2K1 w - - 1 25]
    
    Noticing that Black's rook can be trapped, White can play naturally. Black is forced to sacrifice.

    \variation[invar]{25. Kf1 Re5 26. f4 Bb5 27. Rxb5 axb5 28. Qxb5 Rxe7 
    29. Be2 Qc8 30. Rd6 Qh3+ 31. Ke1 Rxe2+ 32. Kxe2 } 

    \chessboard[setfen=r7/k4p1p/3R1p2/PQb5/5P2/2P3Pq/4K2P/8 b - - 0 32]

    The position is equal.

    Let's go back to the mainline.

    \chessboard

    \mainline[level=1]{22...Qe5!}
    
    Black can use \vocab{PoE}{process of elimination} to find the correct move.

    \variation[invar]{22... axb5 23. Qxb5+ Ka7 24. Qxc5+ Ka6 25. Rxd5 Rxg4 26. Qb5+ Ka7 27. Rd7 } White wins.

    The bishop on d5 must be protected. So \symqueen e5 is also a natural move.

    \chessboard

    \mainline[level=1]{23.Re1}
    
    \symrook xd5 doesn't work unfortunately because \variation[invar]{23. Rxd5 Qxd5 24. Rd1 Bxf2+ 25. Kf1 Qxb5 }. 
    \symrook ab1 is too slow because \variation[invar]{23. Rab1 Rxg4 24. Nd6+ Kc7 25. e8=N+ Rxe8 26. Nxe8+ Kc8 27. Rxd5 Qxd5 28. Nxf6 Bxf2+ 29. Kxf2 Qf5+ 30. Kg2 Qxf6 }. 

    Tal played the only practical move.

    \chessboard

    \mainline[level=1]{23...Be4?}

    Here again, using process of elimination can help to find the correct move.

    We know that \symrook ab1 is the resource for White. As we will see, White can 
    win material with this move. Therefore Black should reject this move. The only move possible 
    is \variation[invar]{23... Qg5}.

\mainline[level=1]{24.Rab1 Rxg4 25.Rxe4 Qxe4 26.Nd6+ Kc7 27.Nxe4 Rxe4 28.Qd1}

\chessboard

\mainline[level=1]{28...Re5??}

Keller was unable to withstand the tension. The e7 pawn has haunted him for a long time. 
Therefore he should have taken it. \variation[invar]{28... Bxe7} would have been better.

\chessboard

\mainline[level=1]{29.Rb7+! Kxb7 30.Qd7+ Kb8 31.e8=Q+ Rxe8 32.Qxe8+ Kb7 33.Qd7+ Kb8 34.Qxc6}

This is indeed a very complex game with lots of tactical elements. Finding the right 
move through calculation leads to fatigue and mistakes. As human players,
it is much easier to find the right move by recognizing the resources and using the process of elimination. 
\end{multicols}