\chapter{Annotations}
I annotate chess games as a means of self-improvement and personal learning. The market offers a wealth of excellent chess literature, much of which is dense with variations. Additionally, there are countless chess games available, and uncovering hidden insights within these games provides a valuable experience for me.

My primary focus is on identifying critical moments and key moves using straightforward methods, rather than getting lost in complex variations. I prefer to analyze games with fewer moves, often avoiding endgames, as they tend to require extensive calculations and are not my main interest.

After pinpointing these critical moments, I verify my findings using computer analysis. There are times when I miss key points, but other times I succeed in my assessments. This process is an integral part of my learning journey.

I intentionally refrain from using computers to evaluate the players, as I believe such an approach will not contribute to my growth as a chess player.

Ultimately, I aspire to discover effective strategies for playing good chess through simple methods, drawing inspiration from the styles of great players like Karpov and Petrosian.

% Games with high value annotation
\newpage
\section{Mikhail Tal - Dieter Keller, Zurich 1959} 

\begin{multicols}{2}
    Here is one of Tal's most impressive games—so complicated that 
    Tal gave it in his book \emph{The Life and Games of Mikhail Tal} without
    any notes. I call this game an absolute TAL-Fahrt.
    \chessgameinfo{Zurich 1959}{Mikhail Tal}{Dieter Keller}{}{1959.05.27}{1-0}
    \newchessgame
    \mainline[level=1]{
        1.Nf3 Nf6 2.c4 e6 3.Nc3 d5 4.d4 c6 5.Bg5 dxc4 6.e4 b5}
    
    \chessboard 

    Tal could have also played \variation[invar]{7.e5 h6 8. Bh4 g5 9. Nxg5 hxg5 10. Bxg5 Nbd7},
    the ultra-sharp Botvinnik variation that required much more preparation.
        
    \mainline[level=1]{7.a4 Qb6 8.Bxf6 gxf6 9.Be2 a6 10.O-O Bb7}
    

    \chessboard
    
    \mainline[level=1]{11.d5 }
    
    \chessboard

    Black also has some other options. Reckoning that White has the initiative only
    because the Black king is in the center, he could also play \variation[invar]{11... Nbd7 
    \xskakcomment{ intending to move his king to the queenside. For example } 12. dxe6 fxe6 13. Nd4 c5  \xskakcomment{ the e6 pawn is now protected.}  14. Bh5+ Kd8 15. Nc2 b4}. Black has some advantage.
    
    \mainline[level=1]{11...cxd5 12.exd5}
    
    \chessboard

    \mainline[level=1]{12...b4}  
    
    Taking the central pawn is out of the question because it is too dangerous for the Black king.
    It is natural to play the text move. As mentioned before, \variation[invar]{12... Nbd7} covering the king would have been better.

    \chessboard

    Here White can choose between \variation[invar]{13. a5} and \variation[invar]{13. dxe6}.

    After \variation[invar]{13. dxe6 fxe6 14. a5 Qc6} White still has some advantage. By
    the way \variation[invar]{13. dxe6 bxc3 14. exf7 Kxf7 15. Bc4+} is too dangerous for Black.

    As a comparison, after \variation[invar]{13. a5}, \queen c6 is impossible. Therefore,
    a5 is the better move.
    
    \mainline[level=1]{13. a5}

    \chessboard

    We have here three candidate moves: \symqueen d8, \symqueen d6, \symqueen c7.

    \symqueen d8 is bad because \variation[invar]{14... Qd8 15. dxe6 Qxd1 16. exf7+ Kxf7 17. Bc4+ Kg7 18. Nxd1}
    White has more material and a lead in development, or \variation[invar]{14... Qd8 15. dxe6 fxe6 16. Nd4 bxc3 17. Bh5+ Ke7 18. Re1} White 
    has an unstoppable attack.

    \symqueen d6 is a good move. White cannot gain an advantage after taking the e6 pawn,
    because \symknight d4 doesn't win a tempo. \variation[invar]{14... Qd6 15. Ne4 Qf4! 16. dxe6! Qxe4 17. exf7+ Kxf7 18. Rc1 } White has an attack against the Black king.

    The text move is also good.
    
    \mainline[level=1]{13...Qc7 14.dxe6 }

    \chessboard

    Now taking the pawn is bad because \variation[invar]{14... fxe6 15. Nd4}.

    \mainline[level=1]{14... bxc3 15.Nd4 Rg8 16.Qa4+ }
    
    The last few moves are quite straightforward. Black declines \variation[invar]{16... Nc6 17. exf7+}.
    
    \mainline[level=1]{16...Kd8 17.g3}
    
    Don't forget to defend! 

    \chessboard

    We are now in the middlegame. In the next few moves, both players have some resources like playing cards.
    It is important to recognize these cards and decide when to play which one.

    White has:
    \begin{itemize}
        \item \symbishop xc4
        \item bxc3
        \item \symrook fd1 or \symrook ad1
        \item exf7
    \end{itemize}

    Black has:
    \begin{itemize}
        \item \symknight c6
        \item cxb2
        \item \symbishop c5
        \item \symbishop d5
        \item \symrook g5
    \end{itemize}

    White has compensation because Black's knight and rook are on the corner while White 
    has some attack against the Black king.

    We can eliminate some moves first: \index{Process of Elimination}
    \variation[invar]{17... Bc5 18. exf7 Qxf7 19. Rad1 Kc8 20. Bxc4 Bd5 21. Bxd5 Qxd5 22. Nb3} White has a big advantage.

    \variation[invar]{17... Nc6 18. exf7 Qxf7 19. Nxc6} White gains back the material and keeps the advantage. 

    \variation[invar]{17... cxb2 18. exf7 Qxf7 19. Rad1} White is crushing.

    This means only \symbishop d5 or \symrook g5 are playable.
    
    \mainline[level=1]{17...Bd5}
    
    \chessboard

    \mainline[level=1]{18.Rfd1?}

    \variation[invar]{18. bxc3 Bd6 19. Rfd1 fxe6 20. Bf3 Kc8 21. Nxe6 Bxe6 22. Bxa8}
    The material is roughly equal. White has a safer king and therefore some advantage.
    There is however no attack anymore.

    Black blocks the d-file. White must hurry to place his rook on the d-file.

    The question is which one. Noticing that
    Black can play cxb2 attacking the rook on a1, Tal should have eliminated \symrook fd1.
    Therefore, \symrook ad1 is the best move.

    \chessboard
    
    \mainline[level=1]{18...Kc8?}

    Now Black has some candidate moves: cxb2, \symking c8. The move cxb2 wins a tempo for 
    him and White must place his a1 rook on b1. On the other hand, once White plays bxc3, 
    his rook on a1 will be active. 
    
    Playing \symking c8 also allows White to play \variation[invar]{19. Qe8+ Kb7 20. bxc3} activating 
    the rook on a1 as we have discussed. By the way, \variation[invar]{19... Qd8 20. exf7} gives White an advantage.
    
    It is therefore better to play \variation[invar]{18... cxb2}. 

    \chessboard

    \mainline[level=1]{19.bxc3}
    
    Black gives White the opportunity to play \symqueen e8+. The opportunity disappears
    when Black plays \symbishop in the next move. So \variation[invar]{19. Qe8+ Kb7 20. bxc3} is better.
    
    \mainline[level=1]{19... Bc5!}
    
    The last card played!

    \chessboard
    
    \mainline[level=1]{20.e7} 
    
    White still has the \symbishop xc4 card. After \variation[invar]{
        20. Bxc4 Bxc4? 21. Qxc4 Bxd4 22. Rxd4 Qxc4 23. Rxc4+ Kd8 24. Rd1+ Ke7 25. exf7 Kxf7 26. Rc7+ Kg6 27. c4 }

    White has total domination. Black cannot bring up his knight and rook on the queenside.

    It is better for Black to play \variation[invar]{
        20... fxe6 21. Bxd5 exd5 22. c4 Bxd4 23. Rxd4 Nc6 24. Rxd5 Rd8 25. Rad1 Rb8 26. Rxd8+ Nxd8 27. Qc2 
    } The position is equal.

    \chessboard
    
    \mainline[level=1]{20... Nc6}
    
    For human players, it is easier to play \symqueen xe7 or \symbishop xe7, eliminating White's dangerous pawn. \symbishop xe7 is probably better 
    because it also eliminates White's \symbishop xc4 card.

    After \variation[invar]{20... Bxe7 21. Rab1? Nd7!} Black finally brings his knight and has some advantage.
    \variation[invar]{20... Bxe7 21. Nf5 Be6 22. Nd4 Bd5} leads to a draw after repetition.

    \chessboard

    \mainline[level=1]{21.Bg4+}
    
    Another beautiful line is:
    \variation[invar]{21. Nf5 Qe5 22. Bxc4 Bxf2+! 23. Kf1 Qxf5}

    \chessboard[setfen=r1k3r1/4Pp1p/p1n2p2/P2b1q2/Q1B5/2P3P1/5b1P/R2R1K2 w - - 0 24]
    
    Black is threatening mate. Now the show begins:
    \variation[invar]{24. Rxd5 Qh3+ 25. Kxf2 Qxh2+ 26. Kf1 Kc7 27. Rb1 Qh1+ 28. Kf2! }
    The material is equal and both kings are exposed. 

    \variation[invar]{24. e8=Q+!}

    \chessboard[setfen=r1k1Q1r1/5p1p/p1n2p2/P2b1q2/Q1B5/2P3P1/5b1P/R2R1K2 b - - 0 24]

    \variation[invar]{24... Rxe8 25. Rxd5 Qh3+ 26. Kxf2 Kc7 27. Kg1 } White has advantage because 
    his king is safer.

    \variation[invar]{24... Kc7!! 25. Qaxc6+ Bxc6 26. Qxf7+ Bd7 27. Qxd7+ Qxd7 28. Rxd7+ Kxd7 29. Bxg8 Rxg8 30. Kxf2 } Finally, a draw.

    Let's go back to the mainline.
    \mainline[level=1]{21...Kb7}
    
    \chessboard

    The position has changed completely. Both sides have played all the 
    cards, and therefore it is time for a new evaluation. Intuitively, Black 
    will soon activate all his pieces. It would be good for White if 
    he can find some immediate drawing chances.

    \mainline[level=1]{22.Nb5?}
    
    \variation[invar]{
        22. Rab1+ Ka7 23. Nxc6+ Bxc6 24. Qxc4 Rg5}  

    \chessboard[setfen=r7/k1q1Pp1p/p1b2p2/P1b3r1/2Q3B1/2P3P1/5P1P/1R1R2K1 w - - 1 25]
    
    Noticing that Black's rook can be trapped, White can play naturally. Black is forced to sacrifice.

    \variation[invar]{25. Kf1 Re5 26. f4 Bb5 27. Rxb5 axb5 28. Qxb5 Rxe7 
    29. Be2 Qc8 30. Rd6 Qh3+ 31. Ke1 Rxe2+ 32. Kxe2 } 

    \chessboard[setfen=r7/k4p1p/3R1p2/PQb5/5P2/2P3Pq/4K2P/8 b - - 0 32]

    The position is equal.

    Let's go back to the mainline.

    \chessboard

    \mainline[level=1]{22...Qe5!}
    
    Black can use \vocab{PoE}{process of elimination} to find the correct move.

    \variation[invar]{22... axb5 23. Qxb5+ Ka7 24. Qxc5+ Ka6 25. Rxd5 Rxg4 26. Qb5+ Ka7 27. Rd7 } White wins.

    The bishop on d5 must be protected. So \symqueen e5 is also a natural move.

    \chessboard

    \mainline[level=1]{23.Re1}
    
    \symrook xd5 doesn't work unfortunately because \variation[invar]{23. Rxd5 Qxd5 24. Rd1 Bxf2+ 25. Kf1 Qxb5 }. 
    \symrook ab1 is too slow because \variation[invar]{23. Rab1 Rxg4 24. Nd6+ Kc7 25. e8=N+ Rxe8 26. Nxe8+ Kc8 27. Rxd5 Qxd5 28. Nxf6 Bxf2+ 29. Kxf2 Qf5+ 30. Kg2 Qxf6 }. 

    Tal played the only practical move.

    \chessboard

    \mainline[level=1]{23...Be4?}

    Here again, using process of elimination can help to find the correct move.

    We know that \symrook ab1 is the resource for White. As we will see, White can 
    win material with this move. Therefore Black should reject this move. The only move possible 
    is \variation[invar]{23... Qg5}.

\mainline[level=1]{24.Rab1 Rxg4 25.Rxe4 Qxe4 26.Nd6+ Kc7 27.Nxe4 Rxe4 28.Qd1}

\chessboard

\mainline[level=1]{28...Re5??}

Keller was unable to withstand the tension. The e7 pawn has haunted him for a long time. 
Therefore he should have taken it. \variation[invar]{28... Bxe7} would have been better.

\chessboard

\mainline[level=1]{29.Rb7+! Kxb7 30.Qd7+ Kb8 31.e8=Q+ Rxe8 32.Qxe8+ Kb7 33.Qd7+ Kb8 34.Qxc6}

This is indeed a very complex game with lots of tactical elements. Finding the right 
move through calculation leads to fatigue and mistakes. As human players,
it is much easier to find the right move by recognizing the resources and using the process of elimination. 
\end{multicols}
\newpage
\section{Ding, Liren - Gukesh D, World Championship Match, Round 12}

\epigraph{I drank coffee. I changed my haircut a little}{\textit{Ding Liren}}

\keywords{Simple Play, Dynamic Play}

\begin{multicols}{2}
	\chessgameinfo{World Championship Match}{Ding, Liren}{Gukesh D}{12}{2025.02.03}{1-0}
It was the 12th game of a 14-game match, and Ding found himself trailing by one point after his recent loss. Playing with the white pieces, he knew he needed to make a strong move to regain momentum. In a strategic decision, he opted for a rather calm opening, setting the stage for the challenges that lay ahead. 

\newchessgame[
 id=main,
 storefen=example,
 event={Ding - Gukesh World Championship Match},
 white={Ding, Liren},
 black={Gukesh D},
 round={12}]
 \mainline[level=1]{1. c4 e6 2. g3 d5 3. Bg2 Nf6 4. Nf3 d4 5. O-O Nc6 6. e3 Be7 7. d3 dxe3 8. Bxe3 e5 9. Nc3 O-O 10. Re1 h6}

 \chessboard[
   	pgfstyle=straightmove,
	linewidth=0.05em,
 	markmove=d3-d4
 ]

 Ding Liren is known for his solid, strategic approach to chess. He excels in positional play and endgames, often outmaneuvering opponents with his patience and precision. Here the position is calm and it's all about piece maneuvering. 


He thought for a long time about his next move to provoke. 

 \mainline[level=1]{
11. a3 } 
In this position, White wants to push d3-d4. His knight on f3 is critical for the breakthrough and should not be exchanged.  \variation[invar]{11.h3} is a good alternative.

\mainline[level=1]{11...a5}

\chessboard[
   	pgfstyle=straightmove,
        linewidth=0.05em,
 	markmove=d3-d4,
 	markstyle=circle,
 	linewidth=0.05em,
 	markfields={b5},
 ]

It was understandable that Gukesh wanted to prevent b4 with this move. However, compared to the previous position,
he now has one serious weakness on b5 which White can easily exploit with his knight.

Black would love to move his knight on c6 to play c6, guarding the weakness on b5 and restricting the bishop on g2.
He cannot do so because the knight must guard e5. We can also understand Ding's next move, which keeps the 
pressure on e5 by preventing Black from playing \variation{11...Bg4}, exchanging the knight on f3 and releasing the pressure
on his e5 pawn.

\mainline[level=1]{12. h3} 

\resumechessgame
A direct \variation[invar]{12.d4 exd4 13. Nxd4 Nxd4 14. Bxd4 c6 15. c5} according to the engine. The position opens up and White has some advantage.

 \chessboard[
        setfen=r1bq1rk1/1p2bpp1/2p2n1p/p1P5/3B4/P1N3P1/1P3PBP/R2QR1K1 b - - 0 15
        ]

I don't know if the position favors White. After exchanging some pieces, Black should have more room to maneuver than 
several moves ago, where he had a cramped position. Chess law states it is better to avoid exchanging when one has a space advantage. This is exactly the case for White.

\chessboard[
   	pgfstyle=straightmove,
        linewidth=0.05em,
 	markmove=d3-d4,
 	markstyle=circle,
 	linewidth=0.05em,
 	markfields={b5},
 ]

It is instructive to observe the position here. White still has his trumps: d3-d4 and an outpost on b5, and Black can do nothing to parry. So there is no need for White to rush. With h3, White deprives Black of placing a bishop or knight on
g4. White slowly improves his position and Black has no counterattack. 

\resumechessgame[id=main]
\mainline[level=1]{12...Be6 13. Kh2} 

No need to rush! \vocab{prophylaxis}{Prophylaxis} against \symqueen d7 \index{Prophylaxis}.

\mainline{13...Rb8}

\chessboard[
   	pgfstyle=straightmove,
        linewidth=0.05em,
 	markmove={d3-d4,e5-e4},
 	markstyle=circle,
 	linewidth=0.05em,
 	markfields={b5},
 ]

Here White has more options. Again, \variation{14. d4 exd4 15. Nxd4 Nxd4 16. Bxd4 c6 17. c5} loses his advantage. White loses his trump b5 now because the c6 pawn guards it. 
\variation{14. Nb5 Nh7 15. Qd2 Ng5 16. Nxg5 hxg5 17. Rad1} 

\chessboard[
        setfen=1r1q1rk1/1pp1bpp1/2n1b3/pN2p1p1/2P5/P2PB1PP/1P1Q1PBK/3RR3 b - - 1 17,
	markstyle=circle,
 	linewidth=0.05em,
 	markfields={e4},
        ]
White keeps the advantage because Black now has a new weakness on e4.

Ding's next move is prophylaxis against any Black potential e4 thrust.
\mainline[level=1]{14. Qc2 Re8} 

\chessboard[
   	pgfstyle=straightmove,
        linewidth=0.05em,
 	markmove={d3-d4,e5-e4},
 	markstyle=circle,
 	linewidth=0.05em,
 	markfields={b5},
 ]

\mainline{ 15. Nb5}

\chessboard[
   	pgfstyle=straightmove,
        linewidth=0.05em,
 	markmove={d3-d4,e5-e4},
 ]

Finally the move! White achieves his goal on b5. Now White is preparing d4. 

\mainline{15... Bf5}

\chessboard[
   	pgfstyle=straightmove,
        linewidth=0.05em,
 	markmove={d3-d4,e5-e4},
 	markstyle=border,
 	linewidth=0.05em,
 	markfields={a1},
 ]

The rook on a1 is not so active. It can move to d1 to support the d3-d4 thrust. The next
move is logical. Even a World Chess Champion game can be so simple!

\mainline{16. Rad1 Nd7}

\chessboard[
   	pgfstyle=straightmove,
        linewidth=0.05em,
 	markmove={d3-d4,e7-g5},
 ]

White wants to play d4 and prevent Black from playing \variation[invar]{16...Bg5}. His next move is logical.

\mainline[level=1]{ 17. Qd2} 

\chessboard[
   	pgfstyle=straightmove,
        linewidth=0.05em,
 	markmove={d3-d4},
 ]

 Grandmaster Dorfman writes: ``If for one of the players the static balance is negative, he must without hesitation 
 employ dynamic means, and be ready to go in for extreme measures.''  \cite{Dorfman:2001}

 Here White has obviously a static advantage: he has a clear plan to improve his position while
 Black just has no clear plan. For this reason, Black must play dynamically!
\mainline{17... Bg6? } 

\variation[invar]{17... Nc5 18. d5 Nd3 19. d5 Nxe1 20. Qxe1 Nd4 21. Nfxd4 exd4 22. Nxd4 Bh7 23. Qxa5}

\chessboard[
   	setfen=1r1qr1k1/1pp2pp1/6bp/Q2P2b1/2PN4/P3B1PP/1P3PBK/3R4 w - - 1 24
 ]

White still has a static advantage, with the price of one exchange. Black has more
space to maneuver after exchanging some pieces. The position is much more playable
than in the actual game.

Go back to the actual game. White has prepared everything. It is now the time!

\chessboard[
   	pgfstyle=straightmove,
        linewidth=0.05em,
 	markmove={d3-d4},
 ]

\mainline[level=1]{18. d4 e4}

The only move. \variation[invar]{18...exd4 19. Bf4! Rc8 20. Nfxd4 Nxd4 21. Qxd4 Nc5 22. Nxc7!} White wins easily.

\chessboard[
 	markstyle=border,
 	linewidth=0.05em,
 	markfields={f3},
	pgfstyle=straightmove,
        linewidth=0.05em,
 	markmove={f3-g1, g1-e2, e2-f4},
 ]

\mainline[level=1]{ 19. Ng1 Nb6 20. Qc3! }

\chessboard[
	pgfstyle=straightmove,
        linewidth=0.05em,
 	markmove={d4-d5},
 ]

It is quite an instructive move. The queen deprives the knight on c6 of the a5 and e5 squares.
Meanwhile, d5 is coming.

\mainline{20...Bf6 21. Qc2 a4 22. Ne2 Bg5} 

\chessboard[
	pgfstyle=straightmove,
        linewidth=0.05em,
 	markmove={e2-f4},
 ]

\mainline{ 23. Nf4 Bxf4 24. Bxf4 Rc8 25. Qc3 Nb8}

\chessboard[
	pgfstyle=straightmove,
        linewidth=0.05em,
 	markmove={d4-d5},
 ]

\mainline{26. d5 Qd7 27. d6 c5 28. Nc7 Rf8 29. Bxe4 Nc6} 
(\variation[invar]{29...Bxe4 30. Rxe4 Nc6 31. Bxh6 gxh6 32. Rg4+ Kh7 33. Qg7#})

\mainline[level=1]{ 30. Bg2 Rcd8
31. Nd5 Nxd5 32. cxd5 Nb8 } 

\chessboard

Gukesh could have resigned here.

\mainline{33. Qxc5 Rc8 34. Qd4 Na6 35. Re7 Qb5
36. d7 Rc4 37. Qe3 Rc2 38. Bd6 f6} 

\chessboard

The live commentator said:``This is the only time Ding has calculated during the game!''

\mainline{39. Rxg7+} Black resigned.

 \chessboard

 Certainly this was not Gukesh's best game. Besides the game,
 it was more instructive to hear how Ding prepared for the game after the defeat the previous day:
 ``I drank coffee, I changed my haircut a little bit.''

\end{multicols}
 \subsection*{Lessons Learned}
 \begin{itemize}
    \item{Simplicity is often the best strategy. ``I drank coffee, I changed my haircut a little bit.''}
	\item{Play dynamically when one has a static disadvantage!}
\end{itemize}

\newpage
\section{Emanuel Lasker - Wilhelm Steinitz, World Championship 1894, Game 7}
\epigraph{The hardest game to win is a won game}{\textit{Emanuel Lasker}}

\keywords{Intermediate moves, Prophylaxis}

\begin{multicols}{2}
    \chessgameinfo{World Championship}{Lasker}{Steinitz}{7}{1894.08.26}{1-0}
\newchessgame[
    id=main,
    event={World Championship 1894},
    white={Lasker},
    black={Steinitz},
    round={7}
]

\mainline[level=1]{1. e4 e5 2. Nf3 Nc6 3. Bb5 d6 4. d4 Bd7 5. Nc3 Nge7 6. Be3 Ng6}

\begin{chessdiagram}
    \chessboard
\end{chessdiagram}

I usually don't comment on openings, especially from the old masters. 
Just observing the position we can conclude White is much better. Assuming White castles
on the queen side and Black castles on the king side. White is ready to launch with his h-pawn.
Black's knight on g6 is unfortunate. Not only does it waste a tempo (on f6 would be much better), 
but it is also a perfect target for White's h-pawn.

Black, on the other hand, needs much more time to prepare. His queen cannot be activated soon 
because his c-pawn is blocked by the knight on c6. If White plays \variation[invar]{7. d5}, closing the center, 
I can hardly see any counter play. Lasker has however a different plan.

\mainline[level=1]{7. Qd2 Be7 8. O-O-O a6 9. Be2 exd4 10. Nxd4 Nxd4 11. Qxd4 Bf6 12. Qd2 Bc6 13. Nd5 O-O }

\begin{chessdiagram}
\chessboard
\end{chessdiagram}

Optically, White has more more active pieces. \variation[invar]{14. h4} would be natural.
For example \variation[invar]{14. h4 Bxh4 15. g3 Bf6 16. f4 Re8 17. Bf3 Nf8 18. Qh2}

\begin{chessdiagram}
\chessboard[
    setfen=r2qrnk1/1pp2ppp/p1bp1b2/3N4/4PP2/4BBP1/PPP4Q/2KR3R b - - 4 18
]
\end{chessdiagram}

White has already doubled the heavy pieces on the h-file whilte Black has not launched the pawn!

We go back to the game.

\begin{chessdiagram}
    \chessboard
    \end{chessdiagram}

\mainline[level=1]{14. g4?}

The move loses a pawn without compensation.


\mainline{14... Re8 15. g5 Bxd5} 

\begin{chessdiagram}
\chessboard
\end{chessdiagram}

The candidate moves are exd5, gxf6, or \symqueen xd5.

After \variation[invar]{16. exd5 Rxe3 17. fxe3 Bxg5!}, Black solves all
his problems and the position is equal. However, such an exchange 
sacrifice becomes standard in the 20th century, decades after this game.
I am not sure if Steinitz could find this move.

After \variation[invar]{16. gxf6 Bxe4 17. Rhg1 Qxf6} Black is simply better while
White has no counter play.

Lasker chooses the third move.

\mainline[level=1]{ 16. Qxd5 Re5} 

\begin{chessdiagram}
\chessboard
\end{chessdiagram}

After \variation[invar]{17. Qxb7 Bxg5 18. Bxg5 Rxg5 19. Bc4 Rc5 20. Bd5 Rb8 21. Qxa6 Qf6},
Black has active pieces while White has one more pawn.

Most likely, Lasker is continuing the moves he has planned on his 14th move and
misses Steinitz' refutation.

\begin{chessdiagram}
\chessboard[
    setfen=1r4k1/2p2ppp/Q2p1qn1/2rB4/4P3/8/PPP2P1P/2KR3R w - - 1 22
]
\end{chessdiagram}

We go back to the current game. \mainline[level=1]{ 17. Qd2 Bxg5! 18. f4 Rxe4!}

\emph{
    This is the problem: doubling heavy pieces on the e-file
    allows Black to recover the piece
} (Neishtadt)

\mainline[level=1]{19. fxg5 Qe7}

% Dynamic play problem
\begin{chessdiagram}
\chessboard[
    setfen=r5k1/1pp1qppp/p2p2n1/6P1/4r3/4B3/PPPQB2P/2KR3R w - - 1 20
]
\end{chessdiagram}

Clearly Black has the static advantage: White is two pawns down and Black has more active heavy pieces on e-file.
Playing such a position needs nerve.

One possibility as recommended by Kasparov is take one pawn and then defend an inferior but defensible position.

\variation[invar]{20. Bf3 Rxe3 21. Bxb7 Rb8 22. Rhe1 Re5 23. Bxa6 Qxg5 24. Qxg5 Rxg5} 

\emph{Objectively, Kasparov is right. One pawn down gives greater saving chances than two pawns down.} (Dvoretsky)

Lasker chooses to continue the struggle:
\mainline[level=1]{20. Rdf1?! Rxe3 21. Bc4}

\begin{chessdiagram}
    \chessboard
    \end{chessdiagram}

The f7 pawn is under attack. \variation[invar]{21... Rf8} is an obvious
alternative. The knight can be activated by moving to e5 at some point.

\mainline[level=1]{21...Nh8} 

\emph{Typical Steinitz! The commentators admired this eccentric move, although it is apparently
not the strongest} (Kasparov)

I believe Lasker expects this move since it is typical Steinitz. 
 

\mainline[level=1]{ 22. h4 c6 23. g6}

\begin{chessdiagram}
\chessboard
\end{chessdiagram}

\mainline[level=1]{ 23... d5?} This move throws away all the advantage.

We can see from now on, the knight is stuck at the corner at the game. Black is therefore one piece down for
two pawns. Steinitz, however, still believes he has the advantage.

\variation[invar]{23... d5} is an automatic bad move. It closes the center,
it wins a tempo by chasing the White bishop. How can this be a wrong move,
it fails to see an intermediate move.

After \variation[invar]{23...hxg6 24. h5 gxh5 25. Rxh5 Re8 26. Rhh1} Black has a clear advantage because
he is up three pawns and White fails to start an attack on h-file. 

\begin{chessdiagram}
\chessboard
\end{chessdiagram}

\mainline[level=1]{24. gxh7+! Kxh7 25. Bd3+ Kg8 26. h5 Re8 27. h6 g6 28. h7+ Kg7}

\begin{chessdiagram}
\chessboard[
    setfen=4r2n/1p2qpkP/p1p3p1/3p4/8/3Br3/PPPQ4/2K2R1R w - - 1 29
]
\end{chessdiagram}

% TODO: add a prophylaxis problem
\mainline[level=1]{29. Kb1?! } \index{Prophylaxis!King Move} 

\emph{One of Lasker's characteristic ``changes of rhythm''. As long as his opponent
has not yet created any direct threats, White has a little time to make the useful
prophylaxy moves. In the ensuing complications, Black will no longer be able to exploit tactical resources involving
the enemy king's vulnerability. Such play requires both a healthy evaluating capacity andtremendous coolnesss.} (Dvoretsky)

This move has a good \vocab{prophylaxis}{prophylaxis}{} idea. However \variation[invar]{29. a3} first is more accurate \index{Prophylaxis}.

\begin{chessdiagram}
\chessboard[
    markstyle=border,
    linewidth=0.05em,
    markfields={h8},
]
\end{chessdiagram}

Black is playing practically one piece down, since his knight
is at the corner. Activating the knight urgent now.

\mainline[level=1]{29...Qe5} 
\variation[invar]{29... f6! 30. Qf2 \xskakcomment{ (\symqueen h2 is now impossible, due to back rank
weekness.)} Qe6 31. Ka2 Nf7} The knight is finally free again. We also see
why White's last move is an inaccuracy. If \variation[invar]{29. a3} has been played first,
White can play \variation[invar]{30. Qh2} after \variation[invar]{29... f6}.

This subtlety was also overlooked by Kasparov and Dvoretsky. 
White intends to play both \symking b1 and a3, but the order matters. 
Calculating the optimal sequence through brute force is practically impossible here. 
However, by comparing the consequences of each move, one can deduce that a3 must be played first, 
because \symking b1 creates a potential back-rank weakness that restricts White's options.

\mainline[level=1]{ 30. a3!}

Note that White's king castling is quite solid. The bishop on d3 protects the c2 pawn and e2 square, making Black's invasion on the second rank impossible. Black needs two tempi (c5 and then c4) to break this defense, which allows White to take his time to maneuver his queen.

\emph{In this game there is something of the `Tal' element: White's attack is rather abstract,
but it will not come to an end - all the time some threats arise!} (Kasparov)

\emph{Lasker's last two quiet moves were completely inexplicable to his comtemporaries: 
how can you play this way when two pawns are down?} (Kasparov)

\mainline{30...c5 31. Qf2 c4 32. Qh4 f6 33. Bf5}

\begin{chessdiagram}
    \chessboard[
    setfen=4r2n/1p4kP/p4pp1/3pqB2/2p4Q/P3r3/1PP5/1K3R1R b - - 1 33,
    markstyle=circle,
    linewidth=0.05em,
    markfields={g6},
    markstyle=border,
    linewidth=0.05em,
    markfields={h8},
    pgfstyle=straightmove,
    linewidth=0.05em,
    markmove={h1-g1},
]
\end{chessdiagram}


 The position is deeply complex and demands careful calculation.

\begin{itemize}
    \item{Where are the weaknesses?}

     The g6 square is the most vulnerable point in Black's camp.
    \item{Which is the worst-placed piece?}

     The h8-knight is completely sidelined, and there is no clear route to bring it back.
    \item{What is my opponent's idea?}

     White has nothing concrete yet, but \symrook hg1 followed by a sacrifice on g6 is in the air.
\end{itemize}

 White enjoys a static edge: his king is well protected and the knight on h8 remains sleeping.

 Black, in contrast, lacks counterplay. Occupying the second rank is either impossible or far too slow.

 With that in mind, Black must choose between several options:

\begin{itemize}
    \item {(Exchange piece) \symqueen g3}

    \variation[invar]{33... Qg3 34. Qh6+ Kf7 35. Ka2 \xskakcomment{ Prophylaxis! 35. \symrook hg1 is too early because 35... \symrook e1, exchanging all the rooks} Re1 \xskakcomment{Forced, to parry \symrook hg1} 36. Qd2 R1e5 37. Rh3 Qg5 38. Qxg5 fxg5 39. Bd7+ }
    
    \chessboard[setfen=4r2n/1p1B1k1P/p5p1/3pr1p1/2p5/P6R/KPP5/5R2 b - - 1 39]
    
    White has a clear advantage.
    \item {(Accept the challenge) gxf5}

    The threat is stronger than the execution. Now White has no more threat on g6.

    \variation[invar]{33... gxf5 34. Rhg1+ Kf7 35. Qh5+ Ke7 36. Rxf5 Qe6 37. Rxd5 Re1+ 38. Rd1 Rxg1 39. Rxg1 Kd8 }
    
    \chessboard[setfen=3kr2n/1p5P/p3qp2/7Q/2p5/P7/1PP5/1K4R1 w - - 1 40]
    
    The ending is equal.
    \item {(Create some counter play) c3}

    \variation[invar]{33... c3 34. Qh6+ Kf7 35. Rhg1 Rg3 36. Bd7 Re7 37. Rxg3 Qxg3 38. Rxf6+ Kxf6 39. Qf8+ Kg5 40. Qxe7+ Kh6 41. Qf8+ Kxh7 42. Qe7+ Kh6 43. Qf8+ Kh7 44. Qe7+ Kh6 45. Qf6 }
    
    \chessboard[setfen=7n/1p1B4/p4Qpk/3p4/8/P1p3q1/1PP5/1K6 b - - 7 45]
    
    The ending is equal.

    \variation[invar]{33... c3 34. b3 Kf7 35. Bd3 f5 \xskakcomment{ Intending \symrook xd3 then \symqueen e2} 36. Bxf5! gxf5 37. Qg5 Ke6 38. Rxf5 Qe4 39. Qf6+ Kd7 40. Rd1 Kc8 41. Rfxd5 Kb8 42. Rd7 Re1 43. Rxe1 Qxe1+ 44. Ka2 }

    \chessboard[setfen=1k2r2n/1p1R3P/p4Q2/8/8/PPp5/K1P5/4q3 b - - 1 44]

    The ending is equal.

     \item {(Bring the king to safety) \symking f7}

    Steinitz' choice. The middle game struggle continues.
\end{itemize}

 None of these lines is forced; over the board one must eliminate the wrong plan and select the move that best suits one’s style.

\mainline[level=1]{33...Kf7 34. Rhg1 gxf5 35. Qh5+ Ke7 36. Rg8}

\mainline[level=1]{36...Kd6}

\begin{chessdiagram}
\chessboard
\end{chessdiagram}
\emph{I believe it was exactly here that Steinitz made the decisive error:
unlike the other commentators, I fail to see where Black could havesaved himself after this.} (Dvoretsky)

This assessment is exaggerated; only after Black's next move does White obtain a tangible advantage.

\mainline{37. Rxf5}

% TODO: add an intermediate move problem
\begin{chessdiagram}
    \chessboard[
    setfen=4r1Rn/1p5P/p2k1p2/3pqR1Q/2p5/P3r3/1PP5/1K6 b - - 0 37
]
\end{chessdiagram}


\mainline[level=1]{37... Qe6?} 

The queen is under attack and it feels natural to move her immediately.
Yet both Kasparov and Dvoretsky overlooked
\variation[invar]{37... Re1+ 38. Ka2 Qe2 39. Rxd5+ Kc6 40. Rc5+ Kb6 41. Qxe2 R1xe2 42. Rc4}.
Once the queens come off, White's attack evaporates and the passed pawn is firmly blockaded.
The position should be equal. 

\mainline[level=1]{ 38. Rxe8 Qxe8 39. Rxf6+ Kc5 40. Qh6 Re7 41. Qh2 Qd7 42. Qg1+ d4 43. Qg5+ Qd5 44. Rf5 Qxf5 45. Qxf5+ Kd6 46. Qf6+ \xskakcomment{ 1-0}}

\end{multicols}

What an exhilarating fight this was. Lasker emerged two pawns down straight from the opening, yet he managed to keep complicating the position and piling up fresh questions for his opponent. For a long stretch the engine verdict is close to equal—though Steinitz was convinced he was the one pressing—because every time Black solved a problem, Lasker conjured another. Eventually, after neutralizing one threat after another, Steinitz finally slipped.

Game 7 marked the start of five consecutive losses to Lasker. This was an unprecedented humiliation for a man
who had been unbeaten in match play for over 25 years and had previously declared he would win without doubt.

Steinitz attributed his collapse to poor physical condition, particularly his disability which prevented him from walking and exercising properly, causing
``insomnia, rushing of blood to the head, and general depression.''

However, the true cause of Steinitz's defeat was not his physical condition, but rather that he was facing a form of chess he had never encountered before. Lasker was decades ahead of his time. He played this game in the style of Tal with an abstract attack, about 40 years before Tal was born. Even after 100 years, annotators failed to find Steinitz's last mistake even with the help of computers. Steinitz didn't need to be so harsh on himself---it was no wonder he failed to understand Lasker's revolutionary play and why he lost the game. 

\subsection*{Lessons Learned}

One must refrain from playing automatic moves and pay special attention to intermediate moves. In this game, Steinitz's \variation[invar]{23... d5} appeared natural---it closed the center and gained a tempo by attacking the bishop. However, it failed to account for the intermediate move \variation[invar]{24. gxh7+!}, which completely changed the evaluation of the position. Automatic moves often overlook tactical nuances that can turn a winning position into a losing one.

Before launching an attack, it is essential to play some quiet moves prophylactically, removing any potential counterattack in advance. Lasker's \variation[invar]{29. Kb1} and \variation[invar]{30. a3!} exemplify this principle. These seemingly passive moves eliminated tactical resources involving the enemy king's vulnerability, allowing White to proceed with his attack without fear of back-rank or back-rank-related tactics. Such prophylactic thinking requires both accurate evaluation and tremendous composure, especially when material is down.
\newpage
\section{Spassky - Polugaevsky, USSR Championship 1961}


\epigraph{Had White carried out his planned 34.Kf6!! Qxd4+ 35.Kf7! this would have been one of Spassky's best wins: such bold raids by the king under a hail of bullets can be counted on the fingers of one hand, and are part of the golden treasury of chess. \cite{kasparov:2004}}{\textit{Garry Kasparov}}

\begin{multicols}{2}
    \chessgameinfo{USSR Championship}{B.Spassky}{L.Polugaevsky}{10}{1961.01.26}{0-1}
    \newchessgame
    \mainline[level=1]{
    1. d4 Nf6 2. c4 e6 3. Nf3 b6 4. Nc3 Bb7 5. Bg5 Be7 6. e3 Ne4
7. Nxe4 Bxe4 8. Bf4 O-O 9. Bd3 Bb4+ 10. Kf1 Bxd3+ 11. Qxd3 Be7}

\begin{chessdiagram}
    \chessboard
\end{chessdiagram}

White has a lead in development, so playing for a kingside attack is logical.

\mainline[level=1]{12. h4 f5?!}

In defense, it is important not to create weaknesses for the attacker.
The text move closes the b1-h7 diagonal but makes a g4 thrust possible.

White has a simple plan: open the g-file and bring the queenside 
rook to g1 for the attack.

\mainline[level=1]{13. Ke2 d6 14. g4 Nd7 15. Rag1}

\begin{chessdiagram}
    \chessboard
\end{chessdiagram}

\mainline[level=1]{15... fxg4 16. Rxg4 Nf6
17. Rg5 Qd7}

\begin{chessdiagram}
    \chessboard
\end{chessdiagram}

White has two possible plans: double rooks on the g-file then play h5, or play h5 directly. 

If White doubles rooks on the g-file, Black can simply play ...\symrook f7. 
After h5 and ...h6, Black protects the g7 pawn and waits for hxg7. In that case, White's rook on g1 would be better placed on h1. 

Therefore White plays h5 directly.

\mainline[level=1]{18. h5 Ne8 19. Rg2 b5 }

\begin{chessdiagram}
    \chessboard
\end{chessdiagram}

White could have played h6 and then attacked the weak g6 pawn, gaining a material advantage:

\variation[invar]{
    20. h6 g6 21. Rxg6+! hxg6 22. h7+ Kh8 23. Be5+! 
    Ng7 \xskakcomment{ (20... dxe5 21. \symknight xe5 and then \symknight xg6)} 24. Qxg6 Bf6 25. Bxf6 Qf7 26. Bxg7+ Qxg7 27. Qxg7+ Kxg7 28. Ng5 
}

\begin{chessdiagram}
    \chessboard[setfen=r4r2/p1p3kP/3pp3/1p4N1/2PP4/4P3/PP2KP2/7R b - - 1 28]
\end{chessdiagram}

\mainline[level=1]{20. c5}

\begin{chessdiagram}
    \chessboard
\end{chessdiagram}

\mainline[level=1]{20... dxc5}

Black voluntarily weakens his e5 square. ...\symbishop f6 would have been better.

\mainline[level=1]{21. h6 Rf5}

The standard response g6 loses quickly.
\variation[invar]{21...g6 22. Rxg6+ hxg6 23. Qxg6+ Kh8 24. Ne5 }

\begin{chessdiagram}
    \chessboard
\end{chessdiagram}

\mainline[level=1]{22. Be5! c4 23. Qe4 }

\variation[invar]{23. Qxd5 exd5 24. hxg7} is more prosaic, exchanging pieces in a winning position. 

\mainline[level=1]{23...Qd5 24. Qg4 c3 25. b3 b4}

\begin{chessdiagram}
    \chessboard
\end{chessdiagram}

Black has created a protected passed pawn. If he can survive the attack, he will have an advantage.

White is playing for mate. 
Exchanging pieces with \variation[invar]{
    26. Bxg7 Qxf3+ 27. Qxf3 Rxf3 28. Kxf3
} would lead to a straightforward win. 

\mainline[level=1]{26. e4 Qb5+ 27. Ke3 Rf7 28. hxg7 Nf6}

``I had a lot of time. I thought: I'll have 7.5 out of 10, I'm the leader of the qualifying tournament. 
I was overjoyed when I played this forced variation'' (Boris Spassky).

\mainline[level=1]{29. Bxf6 Rxf6 30. Rxh7}

\begin{chessdiagram}
    \chessboard
\end{chessdiagram}

After Black takes the h7 rook, White will play \variation[invar]{
    31. Rh2 Kg8 32. Rh8+ Kf7 33. g8+
}. Black gives checks, hoping to save the game.

\mainline[level=1]{30... Rxf3+ 31. Kxf3 Qd3+ 32. Kf4 Bd6+ 33. Kg5 Kxh7 }

\begin{chessdiagram}
    \chessboard
\end{chessdiagram}

``Erratic thinking at moments of terrible tension, especially with the 
opponent's flag about to fall, occurs even with great players'' (Kasparov). 

``That's the position I wanted, and I saw that there's a very simple win: 
\variation[invar]{
    34. Kf6!! Qxd4 35. Kf7
} and, as Kazimirych would say, amen to the pies, and it would turn out very nicely, 
in Borisenko's words, but, alas, it didn't happen. After this game, I had a 
straight road to the Interzonal tournament, and then further... But something 
terrible happened.
I was haunted by that position for years. Now I'm not—such things happen once 
in a lifetime. But I wanted to do everything the best possible way, 
and I saw a one-move win. So I thought, if I have one move, 
why do I have to get the King to f6 and then f7? So, with 7.5/10 and 
leadership in mind, I played the next move'' (Boris Spassky).

\mainline[level=1]{34. Kh5} 

``And you know—I just forgot about the b5 square. Just forgot. 
Look, in chess, there are some situations when you lose a game, but still get 
something good out of it. You get useful experience, or you see that you 
shouldn't play so rashly and should calculate more precisely, or something'' (Boris Spassky).

\mainline[level=1]{ 34...Qb5+ 35. Kh4 Be7+
36. Kh3 Qg5 37. Qxg5 Bxg5 38. Rxg5 Rd8}

\begin{chessdiagram}
    \chessboard
\end{chessdiagram}

\mainline[level=1]{ 39. f4?}

\variation[invar]{
    39. Kg4 Kg8 40. Rc5 Rxd4 41. Rxc7 Rxe4+ 42. Kg5 Kh7 43. Rxa7 Re2 44. f3 Rg2+ 45. Kf4 Rxg7 46. Rxg7+ Kxg7 47. Ke3 
} would lead to a draw. Kasparov did not point this out in his commentary.

\begin{chessdiagram}
    \chessboard
\end{chessdiagram}

Black finds the right move by \vocab{prophylaxis}{Prophylaxis}{}. Black wins the endgame. \index{Prophylaxis}

\mainline[level=1]{39...Kg8! 40. Rc5 Rxd4
41. Rxc7 Rxe4 42. Kg4 e5 43. a3 Rxf4+ 44. Kg5 a5 45. axb4 axb4
46. Kg6 Rg4+ 47. Kf6 Kh7 48. g8=Q+ Kxg8 49. Kxe5 Rg1 50. Kf6
Rf1+ 51. Ke5 Rb1}
\end{multicols}

\subsection*{Lessons Learned}
It is far from the truth that Spassky was too fragile. He made 
one mistake when he could have mated his opponent. The endgame that followed 
was too difficult to handle. Even Kasparov did not find the 
right move decades later, analyzing with computers. 

Spassky's only mistake was playing too hastily when 
his opponent was in time trouble. By ignoring the clock and playing the right move,
he would have won the game with a brilliancy.

\section{Anatoly Karpov - Garry Kasparov, FIDE World Championship Match 1984/85, Round 3}
\epigraph{An impulsive, nervous reaction, the preceding game would appear to have put me in a not altogether correct
frame of mind, such that I could solve all my problems immediately with the help of a `sharp' pawn sacrifice}{\textit{Garry Kasparov}}

\keywords{Maroczy Bind, Hedgehog Position, Impulsive Reaction}
% My first annotation without books and computers. I chose this game
% because it is quite short. The strategic win from Karpov is also instructive according
% to chessgames.com

% I am happy with the result that I found out 12... Nc5 was bad. 
% I didnt figure out 16... d5 was explicitly because my analysis was not thourough enough and I 
% turned to books too early..

% The winning of Karpov was instructive. I should have taken more time.
\newchessgame[
 id=main,
 event={FIDE World Championship Match 1984/85},
 white={Karpov, Anatoly},
 black={Kasparov, Garry},
 round={3}]

\mainline{1.e4 c5 2.Nf3 e6 3.d4 cxd4 4.Nxd4 Nc6 5.Nb5 d6 6.c4 Nf6 7.N1c3
a6 8.Na3 Be7 9.Be2 O-O 10.O-O b6 } 

\chessboard[
  markstyle=circle,
  linewidth=0.05em,
  markfields={b6},
]

Black has weakness on b6. The next moves of White are logical, attacking 
the weakness and developing his pieces.


\mainline{11.Be3 Bb7} 
Black has a so-called hedgehog position. Typically Black wants to play 
b6-b5 and d6-d5 to free himself. With White's Maroczy bind (c4 and e4 pawns),
he can stop Blacking from playing d5. With White's next move, he stops both b5 and d5 break.

\chessboard[
    pgfstyle=straightmove,
    linewidth=0.05em,
    markmove={b6-b5, d6-d5},
    markstyle=circle,
    linewidth=0.05em,
    markfields={b6},
]

\mainline{12. Qb3}

\chessboard

b6 pawn is under attack. \variation[invar]{12... Nd7} is a natural move.
after \variation[invar]{13. Rad1 \xskakcomment{ Preventing Nc5 with \symbishop xc5. The bishop on
b7 is unprotected..} Qc7 \xskakcomment{intending \symknight c5}} Black is still in the game.

\mainline[level=1]{12... Na5?} 

Kasparov chose to force the matter. After this move White has 3 vs 1 
on the queenside and Black has a backward pawn on the d-file.

\chessboard

\variation[invar]{13. Bxb6 Nxb3 14. Bxd8 Rfxd8 15. axb3 Nxe4 16. Nxe4 Bxe4} is not good enough because
Black's bishop pair gives him enough chance to defend.

\mainline[level=1]{ 13.Qxb6
Nxe4 14.Nxe4 Bxe4 15.Qxd8 Bxd8} (\variation[invar]{15... Rxd8 16. Bb6} This is by the way another
drawback of the move \variation{12... Na5}) 


\mainline[level=1]{ 16.Rad1} 

Quite often the most natural move is the best.

\chessboard

% I wrote "This position must be estimated carefully before Black's 12th move."
% Actually Kasparov has expected only 16. Rfd1 in his preparation and was instantaneously nervous after
% seeing this move. If he could not foresee such a move, I shouldnt expect myself
% to foresee it on the board. 


\mainline{16... d5?}

\quote{An impulsive, nervous reaction, the
preceding game would appear to have put
me in a not altogether correct frame of
mind, such that I could solve all my
problems immediately with the help of a
`sharp' pawn sacrifice} \cite{Kasparov:2008}



\variation[invar]{16... Be7 17. Nb1 \xskakcomment{ (improving his worst piece) } Rac8} Black can still hold on.

\chessboard

\mainline{ 17.f3 Bf5 }

\chessboard

\mainline{18.cxd5
exd5 19.Rxd5 Be6 20.Rd6 Bxa2 21.Rxa6 Rb8 }

\chessboard

\mainline{22.Bc5! Re8 23.Bb5! }
Notice how Karpov used bishops to win tempi and protect
the b2 pawn from the b8 rook. Also note how the Black knight on a5 is doomed by the b5 bishop. 


\mainline{23... Re6
24.b4 Nb7 25.Bf2 Be7 26.Nc2 Bd5 27.Rd1 Bb3 28.Rd7! Rd8} (\variation[invar]{28... Bxc2 29. Rxe6}) \mainline[level=1]{29.Rxe6
Rxd7 30.Re1 Rc7 31.Bb6} 1-0

\subsection*{Lessons Learned}
\begin{itemize}
  \item{Clear your mind, no impulsive, nervous reaction!}
  \item{Evaluate carefully when forcing the exchanges}
  \item{One can make simple natural moves has the advantage}
\end{itemize}
\section{Boris Spassky - Tigran Vartanovich Petrosian, World Championship Match 1969 Game 5}


\newchessgame[
 id=main,
 event={Petrosian - Spassky World Championship Match},
 white={Boris Spassky},
 black={Tigran Petrosian},
 round={5}]

\mainline{
    1. c4 Nf6 2. Nc3 e6 3. Nf3 d5 4. d4 c5 5. cxd5 Nxd5 6. e4 Nxc3
    7. bxc3 cxd4 8. cxd4}
    
\chessboard [
    markstyle=border,
    linewidth=0.05em,
    markfields={c8},
]

I don't like Black's position. White already has a static advantage
because his pawn formation is better and Black has a passive
light square bishop.

On the other hand, White has some natural moves to develop:
\symbishop c4, O-O, \symrook ad1, \symrook fe1 etc.

It is however difficult to recommend any dynamic play for Black here.

\mainline{8... Bb4+ 9. Bd2 Bxd2+ 10. Qxd2 O-O 11. Bc4
    Nc6 12. O-O b6 13. Rad1 Bb7 14. Rfe1} 
    
\chessboard [
    pgfstyle=straightmove,
    linewidth=0.05em,
    markmove={d4-d5},
]

\mainline {14... Rc8 } 

White is planning the d5 thrust to create a central passed pawn.
Black could also tried
\variation[invar]{14... Na5 15. Bf1 Rc8 16. d5 Nc4 17. Bxc4 Rxc4}
    \variation{18. d6 Rxe4 19. Ne5 Rxe1 20. Rxe1 Bd5\xskakcomment{Black can defend the ending}}

% 2025. 11.5: I wrote the following during my own analysis. 
% the conclusion is premature because I miss error on Black's next move
% "Now White has a straight forward move, his light square bishop can
% exchange an important defender, Black's light square bishop 
% while the Black's knight can't help too much in the defense."

\chessboard [
	pgfstyle=straightmove,
    linewidth=0.05em,
 	markmove={d4-d5},
]


% 2025. 11.5: In my analysis, I saw the knight stuck at a5 for a long time.
% I failed to point out, that 16... Na5 is an error.
\mainline[level=1]{15. d5}

\chessboard
% 2025. 11.5: I missed the error completely. Alas, it is a critical position and I failed to 
% see it. 
Black should have played \varriation[invar]{15... Na5}
\variation{16. dxe6 Nxc4? 17. exf7+ Kh8 18. Qxd8 Rcxd8 19. Rxd8 Rxd8 20. e5} White wins.
\variation{16. dxe6 Qxd2 17. exf7+ Kh8 18. Nxd2 Nxc4 19. Nxc4 Rxc4 20. e5 Bc8 21. e6 Bxe6 22. Rxe6} equal.
Petrosian didn't like \variation{16. Bd3 exd5 17. e5! Nc4 18. Qf4} White has an attack. Indeed, Polugaevsky won a brilliant game
against Tal later. Black must find the only defense \symrook c6.

\mainline{15... exd5 16. Bxd5
    Na5 17. Qf4} 


\chessboard

Black has a desperate position. He cannot defend the d5 passed pawn.
White could also play \symqueen f5, \symknight g5 attack the h7 pawn.
    

\mainline{17... Qc7 18. Qf5 Bxd5 19. exd5 Qc2 20. Qf4 Qxa2 21. d6
    Rcd8 22. d7 Qc4 23. Qf5 h6 24. Rc1 Qa6 25. Rc7 b5}
   
\chessboard
    
\mainline{26. Nd4 Qb6
    27. Rc8 Nb7 28. Nc6 Nd6 29. Nxd8 Nxf5 30. Nc6 
}
1-0


\section{Magnus Carlsen - Ding Liren, Magnus Carlsen Chess Tour Finals}
% \keywords{London System, French Defence, Square Play, Diagonal Play}

\epigraph{ In difficult positions I make moves that do 
not lose by force.}{\textit{Anatoly Karpov}}

\keywords{London System, French Defence, Square Play, Diagonal Play}

\begin{multicols}{2}
\newchessgame[
 id=main,
 event={Magnus Carlsen Chess Tour Finals},
 white={Magnus Carlsen},
 black={Ding Liren},
 round={1}]

 \mainline[level=1]{
    1. d4 Nf6 2. Nf3 d5 3. Bf4 c5 4. e3 e6 5. c3 Bd6 6. Nbd2}
    
\chessboard[
    markstyle=border,
    linewidth=0.05em,
    markfields={c8},
]

A typical London System opening: with a hidden cunning idea. We 
reach the first critical moment of the game. White offers a pawn sacrifice. Black 
has more options to choose from. 

\begin{enumerate}
    \item{Decline the sacrifice}

    Black has a pawn stucture similar
    to French Defence. His light squared bishop is not active. He could
    choose to proceed with a normal French Defence setup:
    \variation[invar]{6... Bxf4 7. exf4 Qb6 8. Qc2 Qc7 9. g3 b6 10. Bg2 Bg7}

    \chessboard[
        setfen=rn2k2r/pbq2ppp/1p2pn2/2pp4/3P1P2/2P2NP1/PPQN1PBP/R3K2R w KQkq - 2 11
    ]
    \item{Accept the sacrifice, take on c3}

    \variation[invar]{6... cxd4 7. Bxd6 dxc3} White has compensation and can play \variation[invar]
    {8.Qa4+} or \variation[invar]{8.Ba3} as in the game.

    \chessboard[
        setfen=rnbqk2r/pp3ppp/3Bpn2/3p4/8/2p1PN2/PP1N1PPP/R2QKB1R w KQkq - 0 8
    ]

    \item{Accept the sacrifice, take on e3}

    As in the current game:    
\end{enumerate}
    
\mainline[level=1]{6... cxd4 7. Bxd6 dxe3 8. Ba3 exd2+ 9. Qxd2}

\chessboard[
    pgfstyle=straightmove,
    linewidth=0.05em,
    markmove={d2-g5},
    markstyle=border,
    linewidth=0.05em,
    markfields={c8},
    markstyle=circle,
    linewidth=0.05em,
    markfields={g7},
]


Here another critical moment.

Let us first list the static advantages for White:
\begin{itemize}
    \item{Bishop pair}
    \item{Better King safety}
    \item{Black has an inactive bishop on c8.}
    \item{Black has weakness on the dark squares.}
\end{itemize}
Dynamically White has lead in development for a sacrificed pawn.

Black has however a complete pawn structure. He can castle after some moves. So his
position is still defendable.

White intends to play \variation[invar]{10. Qg5} threatens the g7 pawn. One move is to play
\variation[invar]{10... Ne4 11. Qe3 Qb6 12. Nd4 f6}.
Black can move his king to f7:

\chessboard[
    setfen=r1b4r/pp3kpp/1qn1pp2/3p4/3Nn3/B1PBQ3/PP3PPP/R4RK1 w - - 4 14
]

In the game, Ding chose to play 9... \symknight c6. This move is per se not bad. 
He must however play accurately afterwards

\mainline[level=1]{9... Nc6 10. Qg5 Rg8 11. Bd3 h6
12. Qe3 Qb6 13. Qe2 Bd7 14. O-O O-O-O 15. b4} 

\chessboard[
    pgfstyle=straightmove,
    linewidth=0.05em,
    markmove={b4-b5},
    markstyle=circle,
    linewidth=0.05em,
    markfields={e5},
    markstyle=border,
    linewidth=0.05em,
    markfields={d7},
]

Another critical moment.
White threatens to play b5 to drive away the knight on c6 and then controls the e5 square with his knight:
Once White achieves his goal, Black has a desperate position. Therefore he must react now!

Understanding White's idea, Black can play \variation[invar]{15... e5 16. b5 e4 17. bxc6 Qxc6 18. Ne5 Qc7 
19. Nxf7 Bg4 20. Qe3 Qxf7 21. Bc2}. This move also activates the bishop on d7 as a bonus.

\chessboard[
    setfen=2kr2r1/pp3qp1/5n1p/3p4/4p1b1/B1P1Q3/P1B2PPP/R4RK1 b - - 1 21
]

Black equalizes.

In the game, Ding's move is a serious mistake. 
\mainline[level=1]{15... Kb8? 16. b5 \xskakcomment{ Of course!} Na5
17. Ne5 Be8 18. Bb4 Rc8 19. a4} 

\chessboard

Black is passive but has no weaknesses. He should have waited (e.g. \variation[invar]{19... Qc7}) and
let White to prove his advantage. Here it is important
to remember what Karpov said: in difficult positions I make moves that do 
not lose by force.

After the text move, White wins quickly.
\mainline[level=1]{19... Ne4 20. Bxe4 dxe4 21. Qxe4 f6
22. Qh7 Nb3 23. Qxg8 Nxa1 24. Qxe8 \xskakcomment{ 1-0}
 }

 \chessboard

\end{multicols}
 % 2025. 11.6: I am happy to find the critical positions on my own: 15... Kb8
 \subsection*{Lessons Learned}
 \begin{itemize}
    \item{Knowledge in other openings with similar pawn structure
 can help to find the right idea.}
    \item{Don't make moves that lose by force in difficult positions.}
 \end{itemize}
\section{Ding, Liren - Ian Nepomniachtchi 2023 World Championship Game 4}

\epigraph{
    Ian Nepomniachtchi is a great dynamic player. Such players often
    find it difficult to sit and defend passively. And it seems that
    this position requires exactly that.
}{Grandmaster David Navara}

\keywords{Pawn Sacrifice, Patience in Defense, Practical Play, Passed Pawns}

\begin{multicols}{2}
    \newchessgame[      
        id=main,
        event={FIDE World Championship 2023},
        white={Ding, Liren},
        black={Nepomniachtchi, Ian},
        round={4}]
    \mainline[level=1]{
        1. c4 Nf6 2. Nc3 e5 3. Nf3 Nc6 4. e3 Bb4 5. Qc2 Bxc3 6. bxc3 d6 7. e4 O-O 8. Be2 Nh5 9. d4}

    \chessboard

    There is a fierce fight in the center around d4. An alternative is \variation[invar]{9... Qf6 \xskakcomment{ a natural idea.}
    10. d5 Na5 11. g3 Bg4 } 

    \chessboard[
        setfen=r4rk1/ppp2ppp/3p1q2/n2Pp2n/2P1P1b1/2P2NP1/P1Q1BP1P/R1B1K2R w KQ - 1 12
    ]

    Black has a solid but passive position. It is probably not to Nepo's taste. 

    \mainline[level=1]{9...Nf4 10. Bxf4 exf4 11. O-O Qf6 12. Rfe1 Re8}
    
    \chessboard

    In the middlegame, it is often more about choosing moves and positions according to the 
    players' style than finding the objectively best move.

    White has a solid center and Black has a compact position. Continuing in this manner 
    would suit Ding better because Nepo, as a dynamic player, cannot sit and wait passively.

    \variation[invar]{13. c5 dxc5 14. e5 Qh6 15. Rad1 Bg4 16. Qb3} would be a good idea but not a good practical decision:

    \chessboard[
        setfen=r3r1k1/ppp2ppp/2n4q/2p1P3/3P1pb1/1QP2N2/P3BPPP/3RR1K1 b - - 4 16
    ]

    The position is sharp. White may have some advantage. However, the position 
    is open and Black has counterplay and open lines. What is the point of allowing
    a dynamic player tactical opportunities?

    Ding chooses a natural move, improving his position slowly and not allowing Nepo any counterplay.
    
    \mainline[level=1]{ 13. Bd3 Bg4 14. Nd2}
    
    \chessboard
    
    \mainline[level=1]{14...Na5?}
    
    A very strange move. The knight on a5 has no future. \variation[invar]{14...Rad8} would be a natural move.

    \chessboard
    
    \mainline[level=1]{ 15. c5! \xskakcomment{ With his sacrifice White activates his pieces.} dxc5 16. e5 Qh6 17. d5 Rad8 18. c4}
    
    \chessboard

    White has some advantage here. His center is strong and Black has 
    a passive knight on a5. 

    While the position may still be equal according to the computer, for human players White has a much easier position to play. He controls the center, while his opponent, as a dynamic player, can only sit and wait. At some point, his opponent would lose patience while defending and make mistakes, as happened in this game. 

    \mainline[level=1]{18... b6 19. h3 Bh5}

    \chessboard[
        pgfstyle=straightmove,
        linewidth=0.05em,
        markmove=f4-f3,
        markstyle=circle,
        linewidth=0.05em,
        markfields={e5},
    ]

    \begin{itemize}
        \item Where are the weaknesses?
    
        The e5 pawn is the pivot of the position and must be protected.
        \item Which is the worst-placed piece?
       
        The rooks must be activated.
        \item What is my opponent's idea?
    
        He wants to play \symknight f3, creating some counterplay.
    \end{itemize}
    
    By answering the questions above, White can find the next few moves:
    \begin{itemize}
        \item Move his bishop to e4 then f3 (if Black exchanges the bishop, White has a knight on f3, which further strengthens the e5 pawn).
        \item Move his queen to c3 to protect the e5 pawn.
        \item Double his rooks on the e-file to protect the e5 pawn.
    \end{itemize}

    Black, however, must play move by move.

    \mainline[level=1]{20. Be4 Re7 21. Qc3 Rde8 22. Bf3 Nb7 23. Re2} 
    
    As mentioned above, White has a clear plan and needs only to execute it, without thinking too much. 

    \chessboard

    Black makes a difficult decision here, allowing White to create a passed pawn but gaining 
    a good square on d6 for his knight.
    
    \mainline[level=1]{23... f6 24. e6 Nd6 25. Rae1} 
    
    \chessboard[
        setfen=4r1k1/p1p1r1pp/1p1nPp1q/2pP3b/2P2p2/2Q2B1P/P2NRPP1/4R1K1 b - - 2 25
    ]

    White has executed his plan and Black has defended well. How should Black defend next?

    I believe the question can be answered logically without calculating a lot.
    The main asset of White is the passed pawns, which have been blocked by the Black rooks.

    The rooks on the e-file are not doing much because there is no open file. The knights are 
    important pieces. The Black knight keeps an eye on e4 so that the rooks cannot attack the 
    f4 pawn. 
    
    At some point, White may play \symrook e4 to attack the f4 pawn and exchange the Black knight.

    Alternatively, White may play \symknight e4 to exchange the Black knight. In either case, e4 is 
    an important square and must be protected in advance. \variation[invar]{25... Bg6} is a good move.

    It is unclear how White can make progress here. In the actual game, Black chooses to defend
    actively and soon makes a severe mistake.
    
    \mainline[level=1]{25... Nf5?!} 
    
    \chessboard

    White's bishop cannot improve White's position, while Black's bishop can defend the important e4
    square. It is therefore logical to exchange the bishops and then occupy the e4 square with the rook.
    
    \mainline[level=1]{ 26. Bxh5 Qxh5 27. Re4 Qh6 }
    
    \chessboard
    
    \mainline{28. Qf3}

    Again, a very logical move, attacking the weak f4 pawn. Ding's moves 
    are natural, although not the perfect computer moves. By playing these moves,
    he sets problems for Nepo that cannot be solved using dynamic play---a 
    very practical choice!
    
    \chessboard

    After \variation[invar]{28... g5 29. g4 Nd6}, the position is still defensible for Black. 

    \mainline[level=1]{28...Nd4?? } 
    
    \chessboard

    I am not sure whether Nepo sees a trap that backfires because Ding has set a deeper one. After \variation[invar]{29. Qxf4 Qxf4 30. Rxf4 c6 31. Nf3 Nxf3 32. Rxf3 cxd5 33. cxd5 Rd8 34. Rd3 Rd6} 
    
    \chessboard[setfen=6k1/p3r1pp/1p1rPp2/2pP4/8/3R3P/P4PP1/4R1K1 w - - 3 35]

    White has no advantage.

    More plausibly, Nepo loses his patience in defense and wants to force a draw. 
    
    \chessboard

    \mainline[level=1]{29. Rxd4! } 
    
    Of course, the knight is much more valuable than the rook.

    \mainline[level=1]{29...cxd4 30. Nb3 g5 31. Nxd4 Qg6 32. g4 fxg3 33. fxg3 h5 34. Nf5 Rh7 35. Qe4 Kh8 36. e7 Qf7 37. d6 cxd6 38. Nxd6 Qg8 39. Nxe8 Qxe8 40. Qe6 Kg7 41. Rf1 Rh6 42. Rd1 f5 43. Qe5+ Kf7 44. Qxf5+ Rf6 45. Qh7+ Ke6 46. Qg7 Rg6 47. Qf8}

    \chessboard
\end{multicols}

\subsection*{Lessons Learned}

In the middlegame, it is often more important to choose positions that match your playing style than to find the objectively best move. Ding chose a solid, positional approach that suited him better than sharp tactical lines that would favor his dynamic opponent. He correctly avoided opening the position unnecessarily, which would have given Nepomniachtchi counterplay and tactical chances, even though it might have been objectively good. By choosing positions where he felt comfortable and his opponent would struggle, Ding created opportunities for mistakes.

A well-timed pawn sacrifice can activate your pieces and create a strong center. Sometimes material is less important than piece activity and positional control. When your pieces become active and you gain strategic advantages, a pawn sacrifice can be a powerful tool.

When defending a difficult position, patience is essential. Black's position was still defensible, but an impatient move led to immediate defeat. Dynamic players often struggle with passive defense, and maintaining patience can be the difference between holding the position and losing.

Ding's moves were natural and practical, even if not always the computer's top choice. By setting problems that his opponent couldn't solve with dynamic play, he achieved a practical advantage. Sometimes the best move is not the objectively strongest one, but the one that creates the most problems for your opponent in a practical game.

\section{Ding, Liren - Ian Nepomniachtchi 2023 World Championship Game 6}

\keywords{London System, Knight Maneuvering}
\begin{multicols}{2}
    \newchessgame[      
        id=main,
        event={FIDE World Championship 2023},
        white={Ding, Liren},
        black={Nepomniachtchi, Ian},
        round={6}]
    \mainline[level=1]{1. d4 Nf6 2. Nf3 d5 3. Bf4 c5 4. e3 Nc6 5. Nbd2 cxd4
    6. exd4 Bf5 7. c3 e6} 
    
    \chessboard[
    ]
    
    Modern chess is full of subtleties. The position is actually a reversed
    Queen's Gambit Declined Exchange Variation. Typically after 
    \variation[invar]{1. d4 d5 2. c4 e6 3. Nc3 Nf6 4. cxd5 exd5 5. Bg5 Be7 6. e3 c6}
    
    \chessboard[
        setfen=r1bqk2r/pp1nbppp/2p2n2/3p2B1/3P4/2NBP3/PP3PPP/R2QK1NR w KQkq - 2 8
    ]
    
    We have a so-called Karlsbad structure. White has the following typical ideas:

    \begin{itemize}
        \item{Minority Attack}

        White advances his queenside pawns with the main purpose of creating weaknesses in black’s structure.

        \chessboard[
            setfen=8/pp3ppp/2p5/3p4/3P4/4P3/PP3PPP/8 w KQkq - 2 8,
            pgfstyle=straightmove,
            linewidth=0.05em,
            markmove={b2-b4, b4-b5, b5-c6, b7-c6},
        ]
        
        \item{Playing for the e3-e4 push}

        This can be done with or without the support of the f pawn (by pushing f3) then e4.
    
        \chessboard[
            setfen=8/pp3ppp/2p5/3p4/3P4/4P3/PP3PPP/8 w KQkq - 2 8,
            pgfstyle=straightmove,
            linewidth=0.05em,
            markmove={e3-e4},
        ]   
    \end{itemize}
    
    Black must defend accordingly. His ideas are
    \begin{itemize}
        \item{Moving the knight from g8 to e4}
        \item{Moving the knight from b8 to c4 via d7 and b6.}
    \end{itemize}
    
    We will see in this game that Ding adopts both of Black's ideas from the Queen's Gambit Declined Exchange Variation.  
    Personally I am not a fan of Black's opening. If in the Queen's Gambit
    Declined Exchange Variation, Black can defend well. With one more
    tempo and colors reversed, White should have some advantage. How can Black still
    defend? 

    Let's go back to the current game.

    \chessboard[
        pgfstyle=straightmove,
        linewidth=0.05em,
        markmove={e6-e5},
    ]

    The position is static and slow, meaning both players must
    seek plans episode after episode to improve their positions.

    \begin{itemize}
        \item{Where are the weaknesses?}
    
        It is still too hard to tell.
        \item{Which is the worst-placed piece?}
       
        The bishop on f1 must be developed, then castling.
        \item{What is my opponent's idea?}
    
        He wants to play e5 to free himself.
    \end{itemize}

    Understanding these, White should develop his bishop and then castle.
    He should also pay attention to Black's e5 thrust. So \symrook e1 is natural.

    \mainline[level=1]{ 8. Bb5}

    \chessboard[
        pgfstyle=straightmove,
        linewidth=0.05em,
        markmove={e6-e5},
    ]

    \begin{itemize}
        \item{Where are the weaknesses?}
    
        It is still too hard to tell.
        \item{Which is the worst-placed piece?}
       
        The bishop on f8 must be developed, then castling.
        \item{What is my opponent's idea?}
    
        Not clear yet.
    \end{itemize}

    White has a strong bishop outside its pawn chain on f4. Exchanging it
    with \symbishop d6 and developing is logical. The next few moves are natural.

    \mainline[level=1]{ 8...Bd6 9. Bxd6 Qxd6 10. O-O O-O
    11. Re1 h6} 
    
    \chessboard [
        markstyle=circle,
        linewidth=0.05em,
        markfields={c5, e5},
        markstyle=border,
        linewidth=0.05em,
        markfields={a1, d2},
        pgfstyle=straightmove,
        markmove={e6-e5},
    ]
    
    \begin{itemize}
        \item{Where are the weaknesses?}
    
        c5 and e5 are weak. White may have knights to occupy these
        squares.
        \item{Which is the worst-placed piece?}
       
        Not clear, neither the rook on a1 nor the knight on d2.
        \item{What is my opponent's idea?}
    
        e5 thrust
    \end{itemize}
    
    \mainline{ 12. Ne5} 
    
    \begin{itemize}
        \item{Where are the weaknesses?}
    
        c5 and e5 are weak. White may have knights to occupy these
        squares.
        \item{Which is the worst-placed piece?}
       
        Not clear, either the rook on a1 or the knight on d2.
        \item{What is my opponent's idea?}
    
        e5 thrust
    \end{itemize}
    
    \chessboard [
        markstyle=circle,
        linewidth=0.05em,
        markfields={c5, e5},
    ]

    \begin{itemize}
        \item{Where are the weaknesses?}
    
        c5 and e5 are weak squares.
        \item{Which is the worst-placed piece?}
       
        Not clear.
        \item{What is my opponent's idea?}
    
        \symbishop xc6, then occupy c5 with his knight.
    \end{itemize}

    \mainline{12... Ne7 } 
    
    % TODO:
    % r4rk1/pp2npp1/3qpn1p/1B1pNb2/3P4/2P5/PP1N1PPP/R2QR1K1 w - - 2 13
    % play out!


    \mainline{13. a4 a6}
    This move is okay in itself. However, Black starts a wrong plan that
    causes his position to deteriorate soon.

    \mainline{ 14. Bf1}
    
    \chessboard [
        markstyle=circle,
        linewidth=0.05em,
        markfields={c5},
        markstyle=border,
        linewidth=0.05em,
        markfields={a1, d2},
        pgfstyle=straightmove,
        markmove={a4-a5},
    ]

    \begin{itemize}
        \item{Where are the weaknesses?}
    
        c5 is weak.
        \item{Which is the worst-placed piece?}
       
        Not clear.
        \item{What is my opponent's idea?}
    
        a5 fixing the pawn structure.
    \end{itemize}

    Focus on the position only, \variation[invar]{14... a5} is the right move.
    Ironically, the opponent's plan \variation[invar]{15. a5} exists
    only since Black's last move. It is also psychologically difficult to play
    \variation[invar]{14... a5}, admitting \variation[invar]{13... a6} was a wrong
    plan. 

    \mainline[level=1]{14...  Nd7?} 
    
    \mainline{ 15. Nxd7 Qxd7} Natural moves!
    \mainline{16. a5!} We already know White's plan.
    
    
    \mainline{16... Qc7} 
    
    \chessboard

    White intends to install his knight on c5. A direct \variation[invar]{17. Nb3}
    would be answered with \variation[invar]{17... Nc6}. Black can
    also assault with \variation[invar]{18... Bc2}. White cannot improve his position.

    The queen on c7 is a defender of White's idea. So he tries to exchange it.
    Ding does this skillfully.

    \mainline[level=1]{ 17. Qf3 Rfc8 18. Ra3!} 
    
    \mainline[level=1]{18... Bg6 19. Nb3 Nc6} \mainline[level=1]{ 20. Qg3} 
    
    \chessboard

    Here Black could have defended patiently by playing \variation[invar]{20... Qxg3 21. hxg3 Bf5 22. Nc5 Rc7}.
    His position would be quite solid.
    
    However, Nepo chooses an active plan by countering with e5 in the center. His pieces
    are too passive for such a dynamic plan. White soon gets the upper hand.
    \mainline[level=1]{20... Qe7
    21. h4!}  A typical pawn move to remove back rank weakness in a better position
    before launching an attack.
    
    
    \mainline{ 21... Re8 22. Nc5 e5?} 
    Result of a wrong plan.
    
    \mainline[level=1]{23. Rb3 Nxa5 24. Rxe5 Qf6 25. Ra3 Nc4
    26. Bxc4 dxc4 } 
    
    \chessboard

    Black should have foreseen and estimated this position before he started 
    his plan of countering in the center on his 20th move. His moves 
    are forced while White may still have some improvements.

    Who has a better position? Obviously White. He has full control of 
    the center. Had Nepo estimated this position correctly, he would have 
    chosen a different plan. 
    
    \chessboard
    \mainline{27. h5?  } 
    
    \variation[invar]{27. Nxb7 Rxe5 28. dxe5 Qb6 29. Nd6 Qxb2 30. Nxc4 } would be winning.
    Note White should keep his strong knight to attack the Black king.

    Missed an opportunity!
    \variation{27... Rxe5 28. dxe5 Qd8 \xskakcomment{ This move is hard to see.} 29. Qf3 Qd2 30. hxg6 Qe1+ 31. Kh2 Qxe5+ 32. Kg1 Qe1+ 33. Kh2 Qe5+ 34. g3 \xskakcomment{ Otherwise perpetual check} Qxc5 35. Qxf7+ Kh8 }
    would have saved the game.
    \mainline[level=1]{ 27... Bc2? 28. Nxb7 Qb6 29. Nd6 Rxe5 30. Qxe5 Qxb2}

    \chessboard

    In this position, White can use process of elimination to find the best move.

    \variation[invar]{31. Nxc4?! Qc1+ 32. Kh2 Bd3 33. Qe3 Qd1 34. Ne5 Qxh5+ 35. Qh3 Qxh3+ 36. gxh3 }
    Black has a good chance to defend. 

    \chessboard [
        markstyle=circle,
        linewidth=0.05em,
        markfields={f7},
        markstyle=border,
        linewidth=0.05em,
        markfields={a3}
    ]

    \begin{itemize}
        \item{Where are the weaknesses?}
    
        f7 pawn.
        \item{Which is the worst-placed piece?}
       
        The rook on a3.
        \item{What is my opponent's idea?}
    
        Not clear
    \end{itemize}

    Black has a weakness on f7. \symrook a5-c5-c7 is natural to attack it.
    By the way, White's rook is attacked. There is no other way to parry the threat.

    \mainline[level=1]{ 31. Ra5! Kh7 32. Rc5 Qc1+ 33. Kh2 f6 34. Qg3 a5 35. Nxc4 a4
    36. Ne3 Bb1 37. Rc7 Rg8 38. Nd5 Kh8 39. Ra7 a3 40. Ne7 Rf8
    41. d5 a2 42. Qc7 Kh7 43. Ng6 Rg8 44. Qf7 \xskakcomment{ 1-0}}
    \chessboard
    \end{multicols}


\subsection*{Lessons Learned}
A very deep game played by Ding. He gains an advantage in the middlegame and 
waits patiently until Nepo takes immature measures.

The London system can become a Queen's Gambit Exchange Variation reversed.
This suggests the question of how to handle a Queen's Gambit Exchange Variation.

One example of an answer is that both sides have chances, both sides have their 
own prospects for attack and it is the player who has the greater knowledge and 
skill (and so who can, for example, employ resources that are less obvious than 
the resources which the opponent can employ) who will get the better of it.

When defending, one must choose between passive and active defense.
One can only use active defense when one has practical chances to defend.
Otherwise, an active defense is just suicide.
\section{S.Karjakin - Magnus Carlsen 2013}

\keywords{Ruy Lopez, Prophylaxis, Process of Elimination, Playability}

\begin{multicols}{2}
\newchessgame

\mainline[level=1]{1. e4 e5 2. Nf3 Nc6 3. Bb5 a6 4. Ba4 Nf6 5. O-O Be7 6. Re1 b5 7. Bb3 d6 8. c3 O-O
9. h3 Nb8 10. d4 Nbd7 11. Nbd2 Bb7} 

\chessboard

A typical idea in Ruy Lopez. Black intends to play \symrook e8, \symbishop f8, exd4 and 
then take the e4 pawn. White must protect his e4 pawn sooner or later.

\mainline[level=1]{ 12. Bc2 Re8 13. a4 Bf8} 

\chessboard

White usually maneuvers his d2 knight to f5 via f1, e3 or g3. At the moment
this maneuver is not possible because Black can play exd4 then take the e4 pawn.

\mainline{ 14. Bd3 c6 15. Qc2 Rc8
16. axb5 axb5 17. b4 Qc7 18. Bb2 Ra8 19. Rad1 Bb6 20. c4 bxc4 21. Nxc4 Nxc4 22. Bxc4 h6
23. dxe5 dxe5 24. Bc3 Ba6 25. Bb3 c5 26. Qb2 c4 27. Ba4 Re6 28. Nxe5 Bb7}

\chessboard[
    markstyle=circle,
    linewidth=0.05em,
    markfields={c4, e4, e5},
]

A critical position! The position is full of tactics so that
it is difficult to estimate statically. At the moment both kings
are safe. Piece balance and pawn structure will change dramatically.

White's bishop is being attacked. Black has a weakness
on c4. A natural idea is \variation[invar]{29. Bb5 Ba6 30. Ra1 Bb7 31. Rxa8 Bxa8 
32. Bxc4 Rxe5 33. Bxf7+ Qxf7 34. Bxe5 Nxe4 }.

After a more or less forced line we have the following position. 

\chessboard[
    setfen=b4bk1/5qp1/7p/4B3/1P2n3/7P/1Q3PP1/4R1K1 w - - 0 35
]

White has one rook and two pawns for two pieces. His king is safe. 
The position should be playable for him.

Alternatively, White moves his bishop to c2. He is forced to weaken his 
king safety to protect his weak e4 pawn. 

\chessboard[
    setfen=r4bk1/1bq2pp1/4rn1p/4N3/BPp1P3/2B4P/1Q3PP1/3RR1K1 w - - 1 29
]

\mainline[level=1]{29. Bc2 Rae8 
30. f4 Bd6} 

\chessboard

It was quite easy for Black to find the previous moves. 
White has weaknesses on e5 and e4. Black only needed to double his 
rooks on the e-file and use his pieces to target the e5 square.

However, it is difficult for White to find the right plan. He has 
already weakened his king safety with his kingside pawn movements. He will
also have to play g3 at some point to further weaken his king. 
From this development, we can conclude that White's 29th move was wrong. \variation[invar]{29. Bb5}
would have been better.

Looking forward, White still must find his next move.

\chessboard[
    setfen=4r1k1/1bq2pp1/3brn1p/4N3/1Pp1PP2/2B4P/1QB3P1/3RR1K1 w - - 1 31,
    markstyle=circle,
    linewidth=0.05em,
    markfields={e4,e5},
    pgfstyle=straightmove,
    linewidth=0.05em,
    markmove={f6-h5},
]

\begin{itemize}
    \item{Where are the weaknesses?}

    e4 and e5.
    \item{Which is the worst-placed piece?}
   
    Not clear.
    \item{What is my opponent's idea?}

    He wants to play \symknight h5 to attack the f4 pawn.
\end{itemize}

When Black plays \symknight h5, White must play g3. As prophylaxis,
it is a good idea to protect the g3 pawn first. There are two options
\variation[invar]{31. Kh2} and \variation[invar]{31. Re3}.

Karjakin should have eliminated \variation[invar]{31. Kh2} first. 
His king is on the same diagonal as the Black queen and bishop. 
\variation[invar]{31. Re3} is the right move.
\variation[invar]{31. Re3 Nh5 32. g3 g5} 
(\variation{32...f6 33. Nxc4 Bxf4 34. gxf4 Nxf4 35. Bb3 \xskakcomment{ Black king
is now exposed, White has advantage.}}) \variation {33. Ba4 R8e7 34. Qe2} White is fine.

\mainline[level=1]{ 31. Kh2? Nh5 32. g3 f6 33. Ng6 Nxf4}

\chessboard

\variation[invar]{34. gxf4 Bxf4+ 35. Kh1 Rxe4 36. Bxe4 Rxe4 37. Kg1 Be3+ 38. Rxe3 Rxe3 39. Qh2 Qxh2+ 40. Kxh2 Rxc3 41. Qh3}
Black is winning. Karjakin finds the only defense.

\mainline[level=1]{34. Rxd6 Nxg6 35. Rxe6 Rxe6 36. Bd4 f5
37. e5} 

\chessboard

White is still paying debts for weakening his own king safety voluntarily.
Black can attack along the main diagonal. Actually he only needs to make
natural moves and again White must come up with good defense.

\mainline[level=1]{37... Nxe5! 38. Bxe5 Qc6}

\chessboard

Here again, White must find the right defense. 

\variation[invar]{39. Be4 fxe4 40. Re3} is his last chance. After the text move, he
has no more chances.

\mainline[level=1]{39. Rg1? Qd5 40. Bxf5 Rxd5 41. Bg4 h5 42. Bd1 c3 43. Qf2 Rf5 44. Qe3 Qf7 45. g4 Re5 46. Qd4 Qc7 \xskakcomment{ 0-1}}

\end{multicols}

\subsection*{Lessons Learned}
Practically a player should choose to play playable positions.
We see again and again during the game, White struggles
to find the right move while Black only needs to make
logical moves. Such play is of course tiring for White and 
at some point, he makes decisive mistakes and then loses the game.

King safety is always the most important thing. It is unwise 
to weaken king safety.

Prophylaxis and Process of Eliminations are important
tools to find the right move without calculating too much. 
At move 31, White finds the right idea with prophylaxis.
However, he fails to choose the right move with Process of 
Elimination.
\newpage
\section{Ding Liren - Levon Aronian, Alekhine Memorial 2013}

\begin{multicols}{2}
    \chessgameinfo{Alekhine Memorial}{Ding Liren}{Levon Aronian}{}{2013.09.18}{1-0}
    \newchessgame
\mainline[level=1]{1. d4 d5 2. c4 c6 3. Nf3 Nf6 4. Nc3 a6 5. e3 e6 6. c5 Nbd7 7. b4 b6 8. Bb2 a5 9. a3 Be7 10. Bd3 O-O 11. O-O Ba6 12. Ne1} 

\chessboard

\mainline{12...Bc4?}

Black has less space, therefore he should exchange the bishops. \variation[invar]{12...Bxd3} would have been better.
Aronian overestimates his passed pawn. The pawn has no future in the middle game because
it is blocked by the White pieces. On the other hand, he gives up the center.
White will soon have the initiative because he will occupy the center with e3-e4 thrust.

\mainline[level=1]{13. Bxc4 dxc4 14. Qe2 Rb8 15. Ra2 b5}

\chessboard[
    markstyle=circle,
    linewidth=0.05em,
    markfields={d6},
    pgfstyle=straightmove,
    markmove={e4-e5, e3-e4},
]

Black has created a protected passed pawn as he has planned. The price is however too high. 
White occupies the center and can maneuver freely while Black can only sit and wait in restricted space.

It is instructive to see how White exploits the weakness on d6:
\begin{enumerate}
    \item{Protect the d4 pawn}
    \item{Target d6 with bishop and knight}
    \item{Occupy d6 with the knight}
\end{enumerate}

It is also important to remember, don't rush!

\mainline[level=1]{16. e4 Rb7 17. Nc2 Nb8 18. Raa1 Qc8 19. Rad1 Rd8 20. Bc1 Na6 21. Bf4 Rbd7 22. h3 Ne8 23. Qe3 Bf6 24. e5! Be7 25. Ne4!
 Nac7 26. Nd6}
 
 \chessboard

 Look at the position. White dominates the center while Black is cramped in his own half.

\mainline[level=1]{26... Qa8 27. Qg3 Nd5 28. Ne3 Nc3 29. Rde1 Bxd6 30. exd6 Ne4 31. Qh4 Nd2}

\chessboard

\mainline[level=1]{32. Nd5! Nxf1 33. Nb6! Qa7 34. Rxf1 Nf6 35. Be5 Nd5 36. Nxd5 exd5}

% TODO: create a tactical problem
\chessboard[
    setfen=3r2k1/q2r1ppp/2pP4/ppPpB3/1PpP3Q/P6P/5PP1/5RK1 w - - 0 37
]

The final blow!
\mainline[level=1]{37. Bxg7 Kxg7 38. Qg5+ Kf8 39. Qf6 Kg8 40. Qg5+ Kf8 41. Qf6 Kg8 42. Re1 axb4 43. Re5 h6 44. Rh5 Qxa3 45. Qxh6 f6 46. Qxf6 \xskakcomment{ 1-0}
}

\end{multicols}

\subsection*{Lessons Learned}
It is crucial to carefully weigh the advantages and disadvantages of any strategic decision.
Giving up control of the center in exchange for a protected but immobile passed pawn is rarely a good trade.
The center provides flexibility and initiative, while a blocked passed pawn offers little practical value in the middlegame.

When liquidating an advantage while the opponent has no counterplay, thorough preparation is essential. 
One must protect all weaknesses and eliminate any potential counterplay before delivering the final blow. 
Rushing to convert an advantage can allow the opponent to generate unexpected complications.
\newpage
\section{Magnus Carlsen - Li Chao, Qatar Masters 2015}

\begin{multicols}{2}
\chessgameinfo{Qatar Masters}{Magnus Carlsen}{Li Chao}{}{2015.12.24}{1-0}
\newchessgame

\mainline[level=1]{1. d4 Nf6 2. c4 g6 3. f3 d5 4. cxd5 Nxd5 5. e4 Nb6 6. Nc3 Bg7 7. Be3 O-O 8. Qd2 Nc6 9. O-O-O f5 10. e5 Nb4 11. Nh3 Qe8 12. Kb1}

\chessboard

This is a typical position in the Grünfeld Defense. White has a pawn center, and Black's
active pieces allow for counterplay.

Here we see the famous quiet king move again. Carlsen moved his king to 
a safer square before starting his attack on the kingside. \variation[invar]{12. Nf4} would also be possible. 

\mainline[level=1]{12... a5}

The kings castle on opposite sides. Mate or being mated is the most probable
outcome. Therefore, tempi are precious. 
Black was trying to postpone \symbishop e6 as much as possible, since White
could win a tempo with \symknight f4 and then start the attack with h4.

White was trying to postpone \symknight f4 as much as possible, since Black 
could react with g5 and f4, halting White's attack.

Both sides were making useful moves until reaching a ``zugzwang'' position.

\mainline[level=1]{13. Be2 c6}

In the post-game interview, Magnus said that c6 was a good move,
but he was happy to see it because now the queen on e8 cannot 
really go to a4, and there will be no mate threat.

\mainline[level=1]{14. Rc1 Kh8 15. Ka1}

\chessboard

\mainline[level=1]{15... Be6}

Now Black played the unpleasant move!

Black could still play \variation[invar]{15... N6d5 16. Nxd5 Nxd5}. This 
position would have been much more solid than in the game.

\mainline[level=1]{16. Nf4 Qf7 17. h4}

\chessboard

What does White want? Obviously, he wants to play h5, opening the h-file. \variation[invar]{17... g5} would 
be refuted with \variation[invar]{18. Ng6}. As \vocab{prophylaxis}{prophylaxis}, Black should play \variation[invar]{17... Rfd8}
so that \variation[invar]{18. Ng6} is not possible. \index{Prophylaxis}

\mainline[level=1]{17... Bxa2}

The text move is too slow.

\mainline{18. h5}


\mainline[invar]{18...Kg8 }

Black must parry \symknight g6. However, it feels wrong. The king had moved from g8 to h8 and then back to g8 again.

\mainline[level=1]{19. hxg6 hxg6 20. g4 }


\chessboard

Here prophylactic thinking is helpful. What does White want? He wants to move his 
queen to the h-file, check on h7, and then take the pawn with his knight. Black
could drive the knight away with \variation[invar]{20... g5 21. Nh3 f4 22. Bf2 Bd5 23. Nxg5 Qg6} and
could still hold the position.

\mainline[level=1]{20...Bb3?}

Now we see how White makes natural moves to attack.

\mainline{21. Bd1! a4 22. Qh2 Rfd8 23. Qh7+ Kf8 }

\chessboard

\mainline[level=1]{24. d5! }

White needed to find the decisive blow. \variation{24. Nxg6 Ke8} goes nowhere. Noticing
that Black's queen lacks space, the text move blocks the support from Black's bishop, and the 
Black queen will be trapped.

\mainline[level=1] {24...Nc4 25. Nxg6+ Ke8 26. e6}

\chessboard

Apparently, the Black king is trapped now.

\mainline{26... a3!}

Exclamation mark for the entertainment value. 

\chessboard

Black is also threatening mate: \variation[invar]{27... axb2 28. Kb1 Ra1#}.

\mainline[level=1]{27. exf7+ Kd7 28. Ne5+ Bxe5 29. Qxf5+ Kc7 30. Qxe5+ Nxe5}

\chessboard 

Now the game is effectively over. White has a material advantage, and Black has no more mating threats.

\mainline[level=1]{31. Bxb3 axb2+ 32. Kb1 Nxc1 33. Rxc1 Kc8 34. dxc6 bxc6 35. f4} Black resigned.

\end{multicols}
\
