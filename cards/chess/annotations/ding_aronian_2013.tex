\newpage
\section{Ding Liren - Levon Aronian, Alekhine Memorial 2013}

\begin{multicols}{2}
    \chessgameinfo{Alekhine Memorial}{Ding Liren}{Levon Aronian}{}{2013.09.18}{1-0}
    \newchessgame
\mainline[level=1]{1. d4 d5 2. c4 c6 3. Nf3 Nf6 4. Nc3 a6 5. e3 e6 6. c5 Nbd7 7. b4 b6 8. Bb2 a5 9. a3 Be7 10. Bd3 O-O 11. O-O Ba6 12. Ne1} 

\chessboard

\mainline{12...Bc4?}

Black has less space, therefore he should exchange the bishops. \variation[invar]{12...Bxd3} would have been better.
Aronian overestimates his passed pawn. The pawn has no future in the middle game because
it is blocked by the White pieces. On the other hand, he gives up the center.
White will soon have the initiative because he will occupy the center with e3-e4 thrust.

\mainline[level=1]{13. Bxc4 dxc4 14. Qe2 Rb8 15. Ra2 b5}

\chessboard[
    markstyle=circle,
    linewidth=0.05em,
    markfields={d6},
    pgfstyle=straightmove,
    markmove={e4-e5, e3-e4},
]

Black has created a protected passed pawn as he has planned. The price is however too high. 
White occupies the center and can maneuver freely while Black can only sit and wait in restricted space.

It is instructive to see how White exploits the weakness on d6:
\begin{enumerate}
    \item{Protect the d4 pawn}
    \item{Target d6 with bishop and knight}
    \item{Occupy d6 with the knight}
\end{enumerate}

It is also important to remember, don't rush!

\mainline[level=1]{16. e4 Rb7 17. Nc2 Nb8 18. Raa1 Qc8 19. Rad1 Rd8 20. Bc1 Na6 21. Bf4 Rbd7 22. h3 Ne8 23. Qe3 Bf6 24. e5! Be7 25. Ne4!
 Nac7 26. Nd6}
 
 \chessboard

 Look at the position. White dominates the center while Black is cramped in his own half.

\mainline[level=1]{26... Qa8 27. Qg3 Nd5 28. Ne3 Nc3 29. Rde1 Bxd6 30. exd6 Ne4 31. Qh4 Nd2}

\chessboard

\mainline[level=1]{32. Nd5! Nxf1 33. Nb6! Qa7 34. Rxf1 Nf6 35. Be5 Nd5 36. Nxd5 exd5}

% TODO: create a tactical problem
\chessboard[
    setfen=3r2k1/q2r1ppp/2pP4/ppPpB3/1PpP3Q/P6P/5PP1/5RK1 w - - 0 37
]

The final blow!
\mainline[level=1]{37. Bxg7 Kxg7 38. Qg5+ Kf8 39. Qf6 Kg8 40. Qg5+ Kf8 41. Qf6 Kg8 42. Re1 axb4 43. Re5 h6 44. Rh5 Qxa3 45. Qxh6 f6 46. Qxf6 \xskakcomment{ 1-0}
}

\end{multicols}

\subsection*{Lessons Learned}
It is crucial to carefully weigh the advantages and disadvantages of any strategic decision.
Giving up control of the center in exchange for a protected but immobile passed pawn is rarely a good trade.
The center provides flexibility and initiative, while a blocked passed pawn offers little practical value in the middlegame.

When liquidating an advantage while the opponent has no counterplay, thorough preparation is essential. 
One must protect all weaknesses and eliminate any potential counterplay before delivering the final blow. 
Rushing to convert an advantage can allow the opponent to generate unexpected complications.