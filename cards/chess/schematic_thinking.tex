Thinking schematically means that you don’t look at specific moves or even plan, but instead just imagine which position you want to reach. When you have found your dream position, you can then try to find the moves to reach it. This thinking technique is so useful since it prevents you from getting lost in many different variations and gives you a clear position you want to reach.

Many great endgame players are known for their schematic thinking skills and Capablanca is a great example. He often made chess look easy and effortless because he was thinking in schemes and focused on the important parts of the position.