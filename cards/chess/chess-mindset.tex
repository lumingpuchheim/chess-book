\chapter{Chess Mindset}

Alexander Kotov's idea of candidate moves brings order into the 
thinking process. Before calculating, the player first evaluates the position 
and identifies its main demands. Only then are a few logical candidate moves 
selected, based on forcing ideas or clear positional improvements.

Each candidate move is calculated separately and thoroughly, without jumping 
between lines. Kotov stressed that mixing variations leads to confusion and 
repeated calculation. After comparing the results, inferior moves are eliminated, 
and the right move is simply the best remaining candidate. In this method, 
good chess is less about brilliance and more about disciplined, structured thinking.

This systematic approach works perfectly for a computer with a rigid algorithm and 
extensive memory. However, trying to emulate a computer misses the point completely.

\section{The Human Challenge}

Chess playing is energy-consuming. There are countless variations to calculate, 
and for human beings, visualizing a given position is already difficult enough. 
Navigating through different variations and remembering the differences 
costs a great deal of mental energy.

\section{A Mindset-Based Approach}

At the end of the day, we are only interested in finding the right moves. 
Quite often, adopting the correct mindset—using the Process of Elimination, 
the Principle of Choices, and Prophylaxis—makes it possible to find the right move 
without consuming excessive energy. These conceptual tools help us focus our 
calculation on what truly matters, allowing us to work with our human limitations 
rather than against them.

\section{Process of Elimination}
\vocab{PoE}{Process of Elimination}{Ruling out bad moves, not immediately finding the perfect one. Commonly used in calculation, defense, and endgames}
 is the method instead of searching for a brilliant move directly, 
you narrow the field\cite{Dvoretsky:2015}.

How it works in practice:

\begin{itemize}
\item List candidate moves that are natural or forcing.
\item Eliminate bad moves by concrete reasons.
\item Compare the remaining moves and choose the one that best fits the position.
\end{itemize}
\index{Process of Elimination} 

\section{Principle of Choices}

If you believe you are in a winning position, you must avoid 
making moves that give your opponent choices, as different 
outcomes may arise depending on an important tempo. The same 
principle applies when you are defending and your position is drawn.

On the other hand, if you are playing when the outcome is unclear, it is practical 
to give your opponent a choice, as they may make a mistake. \index{Principle of Choices}

Combining Principle of Choices and Process of Elimination makes it easier to find the correct 
move.

\begin{multicols}{2}
\newchessgame[
    setfen=6k1/8/5PP1/3p4/3Kp3/8/8/8  w - - 0 1
]

\chessboard

White believes he is winning (it is a chess problem, draw can be easily achieved. Obviously
White must find the winning move.)

The candidate moves are
\begin{itemize}
    \item{f7}
    \item{g7}
    \item{\symking e3}
    \item{Other king moves}
\end{itemize}

Other king moves are bad because Black can play ...e3 and then ...d4. He improves 
his position.

g7 is bad because White cannot make any further progress. He must move 
his king to e6 which can be easily parried by ...\symking f7. Not to mention when 
Black queens at e1, he gives a check.

To eliminate f7 we can use the Principle of Choices. After f7, Black 
can either play ...\symking f8 or ...\symking g7 and they lead to different result.
This means f7 leads to a draw when Black responds correctly.
Note we have eliminated the f7 without any calculation. Such a strategy saves 
energy and time in real games since we are only interested in finding the good move.

This leaves \symking e3. The game continues with 
\mainline[level=1]{1. Ke3 Kf8}

\chessboard

Here again we can use Process of Elimination.
Candidate moves are f7, g7, \symking d4 and other king moves.

Other king moves are bad because of the same reason as before.

\symking d4 is bad because Black can repeat the position. 

g7 is bad because after ...\symking Kf7, White has no further threats.

Therefore f7 is the move.

\mainline[level=1]{2. f7 Ke7}

\chessboard 

Of course \symking d4 is the only move.

\mainline[level=1]{3. Kd4 Kf8}

\chessboard

Even now we can use Process of Elimination! \symking e3 only repeats the position.
Certainly White doesnt want to sacrifice with g7 yet. Which leaves only \symking xd5.

\mainline[level=1]{4. Kxd5! e3}

\chessboard

The game finishes with a mate.

\mainline[level=1]{5. Ke6! e2 6. Kf6 e1=Q 7. g7#} 
\end{multicols}
Prophylaxis in chess refers to the strategic concept of anticipating and preventing an opponent's threats and plans before they materialize. It involves making moves that not only advance your own position but also disrupt or hinder the opponent's potential strategies. This proactive approach allows players to maintain control of the game and dictate the flow of play, rather than simply reacting to their opponent's moves.

The importance of prophylaxis in chess cannot be overstated. It helps players to:

\begin{itemize}
\item{Maintain Initiative}

By anticipating the opponent's plans, a player can make moves that keep them in control of the game, often leading to more favorable positions.
\item{Minimize Risks}

Prophylactic moves can help to diminish the threats posed by the opponent, thereby reducing the chances of falling into traps or losing material.
\item{Enhance Position}

By focusing on both one’s own plans and those of the opponent, players can improve their own position while simultaneously weakening the opponent’s setup.
\item{Create Strategic Plans}

Understanding the opponent’s potential moves allows a player to formulate a more cohesive and effective strategy, as they can plan based on what the opponent is likely to do.

\end{itemize}

In essence, prophylaxis is a key element of high-level chess play, as it embodies the principle of strategic foresight and the ever-important balance between attack and defense. By incorporating prophylactic thinking into their game, players can significantly enhance their overall performance and decision-making skills on the board.