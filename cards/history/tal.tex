Tal's play shows perfectly how uncertainty differs from risk. He didn't simply gamble with bad moves hoping for luck. His sacrifices and daring ideas were not blind risks; they were calculated invitations into positions where ordinary logic no longer worked.

When he sacrificed a piece, he often couldn't prove that the combination was sound-but he sensed that the position contained possibilities his opponent would not fully grasp. That is uncertainty: stepping into a realm where the outcome cannot be measured, but creative intuition guides the way. He wasn't risking in the sense of playing unsoundly; he was venturing into uncharted territory where imagination had more value than accuracy.

If he had only taken risks-making moves with a clear probability of failure-he would never have reached the top. But by embracing uncertainty, Tal turned the game into an adventure of discovery His opponents tried to calculate; he invited them into a fog where calculation lost its meaning. That is why his chess, though dangerous, wsa not reckless-it was the art of living comfortably in the unknown.

Most chess players rely on familarity-they recognize recurring patterns, or ``chunks'', built from years of experience. These mental shortcuts let them navigate normal positions efficiently, uch like walking through a well-lit city why know by hear. Tal disrupted that comfort. He led his opponents into positions that broke all established patterns, where intuition had to replace memory.

Paradoxically, this reuced his own risk: in irrational, chaotic positions where others felt lost, Tal was at home. He turned confusion into his advantage, transforming the unknown into his territory. What looked dangerous to others was, for him, the natural landscape of creativity.