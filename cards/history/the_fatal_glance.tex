Orpheus was the greatest musician who ever lived. When he played his lyre, trees bent to listen, rivers changed their cource, and wild beasts grew gentle. His wife Eurydice was the love of his life, but their happiness was cut short when she died from a serpent's bite while fleeing an unwanted pursuer.

Consumed by grief, Orpheus did what no living mortal had dared: he descended into the underworld itself to bring her back.

His music softend even the hardest hearts in Hades. The tortured souls ceased their weeping to listen. Cerberus, the three-headed guard dog, lay down peacefully. Even the Furies, ancient goddesses of vengeance, wept at his songs. Finally, Orpheus stood before Hades and Persephone, the king and queen of the dead, and sang of his love and his lose.

The rulers of the underworld was moved. Hades, who had never granted such a request, made Orpheus an offer:``You may take Eurydice back to the world of the living. But you must lead the way, and you must not look back at her until you both stand in the sunlight. If you turn around even once before then, she will return to me forever.'' 

It seemed simple enough. Orpheus had already accomplished the impossible-he had descended to the underworld and persuaded Death itself to release his beloved. All he had to do now was walk the path back to the surface without looking behind him. The hard part was over. Victory was within his grasp.

\subsection*{The Ascent}
Orpheus began the long climb up through the dark passages toward the world above. Behind him, he could hear nothing-no footsteps, no breathing, no sign that Eurydice was following. The underworld was utterly silent except for his own footfalls echoing in the gloom.

As he climbed, doubt began to creep into his mind. Was she really there? Had Hades truly kept his word? Perhaps this was a cruel trick, and he was climbing alone. He strained to hear some sound of her presence, but the silence was complete.

The path seemed endless. Every step brought him closer to the surface, closer to success-but also amplified his uncertainty. He had taken such an enormous risk descending into Hades. He had gambled everything on this journey. Shouldn't he make sure his gamble had paid off? What if he emerged intosunlight alone, having trusted blindly when a single glance could have given him certainty?

He tried to reason with himself: I must trust. I've come this far. Just keep walking. But another voice argued back: But what if she's not there? Better to know now. Better to turn back and try again than to discover at the end that you've been walking alone.

Light began to appear ahead-the entrance to the world of the living was near. He was almost there. Just a few more steps and they would both be free. The ordeal was nearly over.

\subsection*{The Fatal Choice}
And it was precisely then, withj victory so close he could see the sunlight, that Orpheus faced his terrible choice. He could take the risk of trusting-of walking forward into that light without confirmation, gambling that Eurydice was behind him. Or he could play it safe, turn around, and verify that she was truly there before taking those final steps. The safe choice seemed so reasonable. Just one quick look to confirm. Then he could turn back around and complete the journey with confidence rather than doubt. Surely the gods would understand-he was so close to the exit anyway. What difference could one glance make when the journey was nearly complete?

Ah the very threshold, with sunlight on his face and living world just step away, Orpheus made his decision. He swould not take the final risk. He would not walk blindly into the light. He turned around to look. Eurydice was there. She had been there all along, following faithfully behind him, just as Hades had promised. Their eyes met for one brief, beautiful moment.

An then she began to fade. Her form grew transparent, dissolving like moning mist. She reached out toward him, but her hand passed through his like smoke. ``Farewell,'' she whispered, and the word echoed as she was pulled back down into the darkness, back to the realm of the dead-this time forever.