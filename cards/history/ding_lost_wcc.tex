In 2024, the chess world was abuzz with excitement as Ding Liren, the reigning World Chess Champion, faced off against the young prodigy, Gukesh D. S. It was a match that had been building for months, a fierce clash of generations. Gukesh, the 18-year-old Indian grandmaster, had risen through the ranks with a remarkable blend of creativity and calculation, making him a formidable opponent for even the most seasoned players.

The final game of the World Chess Championship unfolded in a magnificent hall brimming with spectators, media representatives, and fans from around the globe. An electric tension permeated the atmosphere as both players took their seats, acutely aware of the stakes. With the score tied at 6.5 - 6.5, the pressure was palpable; a draw would lead to a rapid playoff, a scenario that favored Ding. As a result, Gukesh was determined to push aggressively for victory.

For Ding, the weight of the title rested heavily on his shoulders; he had fought hard to reach this point and was resolute in defending his title. Gukesh, on the other hand, was hungry for victory, poised to seize the moment and make history as the youngest World Chess Champion.

As the match commenced, Ding employed a conservative and familiar opening, relying on strategies that had historically served him well. In contrast, Gukesh opted for a more aggressive approach, choosing unconventional lines and applying early pressure on Ding’s position.

In the middle game, Ding found himself in a recognizable scenario—he was slightly better, with a solid position and minor advantages in pawn structure. However, as he pushed for a secure and strategic victory, he meticulously weighed every option, contemplating the risks associated with each move.

Gradually, Ding lost his advantage after a few moves. Now, he had to defend an endgame with one pawn down. The position was theoretically a draw, but concentration remained vital. Ultimately, Ding blundered, overlooking Gukesh's ability to simplify the position and win in a straightforward pawn endgame.

Perhaps Ding was too tired, or maybe he was overly cautious. Playing for a win when holding an advantage might have been a more viable path than aiming for a draw.

I still remember what Grandmaster Peter Leko commented during the match ``sometimes the biggest risk is not to take a risk.'' 