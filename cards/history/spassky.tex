In the 1958 USSR Championship, Boris Spassky was leading the tournament at the beginning and needed only a draw in the last round agianst Mikhail Tal. After a dramatic struggle, he lost - from a winning position - and with it, his chance to qualify for the Interzonal, the gateway to the World Championship.

Three years later in 1961, Spassky was again leading the championship, undefeated. In his critical game against Lev Polugaevsky, he launched a brilliant sacrifice that could have secured victory, but he missed the win and eventually lost. The defeat broke his momentum; he collapsed in the remaining rounds and once more failed to reach Interzonal.

Commenators said Spassky would never become World Champion - that he was too emotional, too fragile at decisive moments.

But something changed in Spassky after that second fall. The pain was too deep to escape- so he stopped escaping. Instead of blaming luck or nerves, he begain to look inward. He understood that talent alone was not enough, and that the real enemy was not Tal, Polugaevsky- it was his own attachment to victory.

He started to play chess not to prove himself, but to express harmony. Out of that calmness, strength emerged. He began to see the board as a reflection of life itself - full of uncertainty, but also of order for those who stayed still enough to see it. When he finally reached the World Championship in 1969, he reached his own Nirvana - not mystical state, but a release from the endless grasping for results. Failure had burned away his illusions, and what remanined was clarity. The losses that once shattered him had in truth prepared himFor only through suffering did he learn the art of detachment - and in detachment, he found freedom, both in chess and in life.