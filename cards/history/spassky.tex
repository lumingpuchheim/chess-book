In the 1958 USSR Championship, Boris Spassky was leading the tournament at the beginning and needed only a draw in the last round against Mikhail Tal. After a dramatic struggle, he lost - from a winning position - and with it, his chance to qualify for the Interzonal, the gateway to the World Championship.

Three years later in 1961, Spassky was again leading the championship, undefeated. In his critical game against Lev Polugaevsky, he launched a brilliant sacrifice that could have secured victory, but he missed the win and eventually lost. 


The defeat broke his momentum; he collapsed in the remaining rounds and once more failed to reach Interzonal.

Commentators said Spassky would never become World Champion—that he was too emotional, too fragile at decisive moments.

In 1966, Spassky became the final challenger and lost to Petrosian. In 1969, he became the challenger again and this time he beat Petrosian to become World Champion.