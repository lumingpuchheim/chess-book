Wilhelm Steinitz was more than just a great chess player-he wa a revolutionary. Born in Prague in 1836, Steinitz transformed chess from a romantic game of dashing attacks into a scientific discipline based on positional principles. In 1886, he became the first official World Chess Champion, a title he defended successfully for years. His theories about chess strategy formed the foundation of modern chess thinking, and for over a quarter-century, he was considered virtually invincible in match play. 

But the weight of maintaining supremacy in chess, a game that demands absolute mental precision and psychological fortitude, would eventually exact a terrible price.

In 1894, at age 58, Steinitz faced a young challenger named Emanuel Lasker in a match for the world championship. What followed was a psychological catastrophe that revealed the devastating toll competitive chess can take on human mind. 

Game 7 marked the start of five consecutive losses to Lasker. This was an unprecedented humiliation for a man who had been unbeaten in match play for over 25 years and had previously declared he would win without doubt.

After this crushing five-game streak, Steinitz asked for a week's rest-a telling admission of his shattered mental state. In his own words after one of these losses, he wrote: ``Mr. Lasker then broke into my game in the most woful manner and won a Pawn, blocking my pieces, and he had things almost all his own way.'' Steinitz attributed his collapse to poor physical condition, particularly his disability which prevented him from walking and exercising properly, cause ``insomnia, rushing of blood to the head, and general depression.''

He had lost his crown, but worse was yet to come.

\subsection*{The 1896-97 ematch: Complete Mewntal Disintegration}
Desperate to reclaim his title and prove his theories were still sound, Steinitz challenged Lasker to a rematch. The rematch was even more devastating. Steinitz's performance at the Nuremberg tournament before the match was subpar, finishing sixth place, which predetermined his bad result to an extent. In the rematch held in Moscow from November 1896 to January 1897, Steinitz won only 2 games, drawing 5 and losing 10-a crushing defeat.

The aftermath was tragic: Just four weeks after the match Steinitz lost his mind and had to seek psychiatric help. Shortly after the match, he had a mental breakdown and was confined for 40 days in Moscow sanatorium, where he played chess with the inmates. There, in his confinement, he would play chess against other patients, eventually even making claims that he was playiung chess games with God himself-believing he could give God pawn and move odds and still win.

His mental state never fully recovered. Two years later, he reportedly experienced delusions while returning by ship from the London 1899 tournament. Commitment to a series of mental hospitals followed beginning in Febrary 1900, and he died in the state mental hospital on Ward's Island on August 12, 1900. The death certificate listed ``chronic endocardia (mitral stenosis)'' and ``acute melancholia'' as causes of death. The first World Chess Champion died nearly penniless, his brilliant mind shattered by the very game he had mastered.