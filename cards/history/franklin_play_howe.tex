It began with a game of chess. In December 1774 Benjamin Franklin met Caroline Howe at the Royal Society in London. She challenged him to a game, which turned into a series of chess matches over several days.

On Christmas Day, she introduced the American to her brother, Lord Richard Howe, who told Franklin that some members of the British government ``were extremely well disposed to any reasonable accommodation'' between the British and the American colonists. The two men continued to use the chess matches as a front for a series of secret meetings to negotiate a piece. Nothing concrete resulted form the meetings, which ended in March 1775, but the Franklin and Howe concluded their talks with a mutual respect for one another.

The following year the admiral was appointed commander of the British navy in North America. Historian Walter Isaacson summarizes what happened just after the Continental Congress issued its Declaration of Independence. [Admiral Howe] carried a detailed proposal that offered a truce, pardons for the rebel leaders (with John Adams secretly exempted) and rewards for any American who helped restore peace. 

Because the British did not recognize the Continental Congress as a legitimate body, Lord Howe was unsure where to direct his proposals. So when he reached Sandy Hook, New Jersey, he sent a letter to Franklin, whom he addressed as ``my worthy friend.'' He had ``hopes of being serviceable'' Howe declared, ``in promoting the establishment of lasting peace and union with the colonies''.

Congress granted Franklin permission to reply, which he did on July 30. Franklin opened the following letter with a cordial statement, but the tone quickly sours as he rejected the offer with fury: ``it is impossible we should think of submission to a government'' whose ``atrocious injuries have extinguished avery remaining Spark of Affection for that Parent Coiuntry we once held so dear.''