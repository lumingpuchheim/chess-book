\section{Rook and Pawn vs. Rook}

\subsection*{Common Attack and Defense}

\begin{multicols}{2}
Most commonly, the attacker tries to promote his pawn while the 
defender tries to stop it.

We start from pawn on the 7th rank.

\newchessgame[
    setfen=4K3/4P1k1/8/8/8/8/r7/5R2 w - - 0 1,
]

\begin{chessdiagram}[Pawn on the 7th Rank]
    \chessboard[
        showmove=false
    ]
    \label{diagram:Lucena}
\end{chessdiagram}

\paragraph{Side Checks}
If it is Black's turn to move, he gets a draw with
\vocab{side-check}{Side Checks}{The weaker side gives checks using the rook from 
the side. The defense is successful when the stronger side cannot avoid 
the check because otherwise it loses the pawn}: \variation{1... Ra8+ 2. Rd7 Ra7+ 3. Kd6 Ra6+ 4. Kc7 Ra7+} etc.
\index{Rook Endgame!Side Checks}

Side checks are successful only when the rook and the pawn are separated 
by at least 3 files. That's why the weaker side should place their 
rook on the \vocab{long-side}{Long Side}{A central pawn or a bishop pawn divides
the chessboard into two unequal parts, one is long and the other is short}.

The king and the rook must defend together. Since the rook should be on the long side,
the king should be on the short side, in order not to interfere with the side checks.

\paragraph{Cutting Off}
In diagram \ref{digram: rook cut off} we have already seen a rook 
cutting off a king to stop it from joining the battle. 
In diagram \ref{diagram:Lucena}, if it is White to move, he can play

\mainline[level=1]{1. Rg1+ Kh7} 
 \variation{1... Kf6}{2. Kf8} Black cannot stop White from promoting.

The Black king is cut off and cannot stop the passed pawn anymore. 
Black's rook is on its own. \index{Rook Endgame!Cut off}

\paragraph{Bridging}
\mainline[level=1]{2. Rg4} Black cannot use side checks anymore because 
White's king can hide at f7. Black tries \vocab{rear-checks}{Rear Checks}{
    The weaker side tries to defend by giving checks from the rear
}. \index{Rook Endgame!Rear Checks}

\mainline[level=1]{2... Rd2 3. Kf7 Rf2+ 4. Ke6 Re2+ 5. Kf6 Rf2+ 6. Ke5 Re2+ 7. Re4} 

\begin{chessdiagram}[Vertical Bridge]
    \chessboard[
        showmove=false
    ]
    \label{diagram:bridge one}
\end{chessdiagram}

White's rook builds a \vocab{bridging}{Bridging}{The stronger side 
uses his rook to interfere with a side check or a rear check in rook endgame} \index{Rook Endgame!Bridging}.

\newchessgame[
    setfen=4K3/4P1k1/8/8/8/8/r7/5R2 w - - 0 1,
]

White can also win the game by a horizontal bridging:

\mainline[level=1]{
    1. Rg1 Kh7 2. Rd1 Ra8+ 3. Rd8 Kg7 4. Kd7 Ra7+ 5. Ke6 Ra6+ 6. Rd6
}

\begin{chessdiagram}[Horizontal Bridge]
    \chessboard[
        showmove=false
    ]
    \label{diagram:bridge two}
\end{chessdiagram}

Black loses because he cannot stop White from promoting his pawn.

\end{multicols}