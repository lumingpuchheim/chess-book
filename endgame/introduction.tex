\chapter*{Introduction}
\epigraph{
    ``Knowledge is power.''
    \begin{flushright}
        --- Petyr Baelish, \emph{Game of Thrones}
    \end{flushright}
}

One day I was studying how Ulf Andersson beat Ivanov in 2000 using 
the Catalan Opening. He steered the game into an endgame 
and then won it with a precise king maneuver. 

That same evening I came across the famous Rubinstein pawn 
endgame where he won with exactly the same king maneuver.
To my astonishment, in both games the defender had the same pawn structure and 
the attacker followed the same king invasion plan.

The Andersson game:
\begin{chessdiagram}
    \chessboard[
        setfen=3r2k1/5p1p/p3pp2/1pb5/8/2N3P1/PP2PPKP/5R2 w - - 0 20
    ]
\end{chessdiagram}

The Rubinstein game:
\begin{chessdiagram}
    \chessboard[
        setfen=8/pp2kppp/4p3/8/1P6/P3PP2/5P1P/2K5 b - - 0 25
    ]
\end{chessdiagram}

At that moment it was clear to me that Rubinstein's game must have 
inspired Andersson. He had prepared a Catalan Opening line 
specifically to reach a similar endgame and execute the same king maneuver.

This experience reminded me of the German sociologist Niklas Luhmann. 
He wrote his articles and books with the help of a vast Zettelkasten 
(slip-box) containing thousands of handwritten index cards, each 
recording a small idea and linked non‑linearly to related cards. 
This network of notes was central to his productivity and originality.

I have written this book using a small Zettelkasten of my own. 
You will see individual ideas I have collected, and you will also see 
how they are woven together in complete games. An opening idea to sacrifice 
a pawn may later reappear as an attacking idea in the middlegame or a winning 
plan in the endgame. I hope that, in the end, you will recognize these 
connections and win more games.

You may read the book in any order, since the ideas are largely self‑contained. 
In the section ``Index'' you will find an idea index, where you can look up 
all the ideas in the book. I hope you will find it useful.

\section*{Endgame Transitions}
Having knowledge about certain endgames will save us time and energy in calculation.
Chess endgames change from one to another. For example, a rook endgame may transpose into 
a queen and rook endgame/middlegame, a pawn endgame, or a queen vs. rook endgame.
It is therefore important not only to remember the tricks in a certain endgame,
but also to be aware of or proactively search for endgames that may arise from 
the current position. 

In this way, we are not only searching for the best move, but 
also for the best possible transposed position.

We have knowledge about certain endgames. This knowledge 
is useful when we can connect these endgames to our current position.
For example, pawn endgame 
positions in which two rook pawns are facing each other,
with one side having a distant passed pawn, are quite common. When we are defending 
a rook endgame where the pawns are described as above, the question becomes: can 
we force a rook exchange, and is the resulting pawn endgame a draw? If both answers are 
yes, we have a clear plan for how to defend.
\addcontentsline{toc}{chapter}{Introduction}
\section{Explanation of Symbols}

The chessboard with its coordinates:

\chessboard[
    setfen=rnbqkbnr/pppppppp/8/8/8/8/PPPPPPPP/RNBQKBNR w KQkq - 0 1,
    showmover=false,
]

\vspace{1em}

\begin{tabular}{@{}ll@{}}
    \symking & King \\
    \symqueen & Queen \\
    \symrook & Rook \\
    \symbishop & Bishop \\
    \symknight & Knight \\
    $\pm$ & White stands slightly better \\
    $\mp$ & Black stands slightly better \\
    $\pm$ & White stands better \\
    $\mp$ & Black stands better \\
    $+-$ & White has a decisive advantage \\
    $-+$ & Black has a decisive advantage \\
    $=$ & balanced position \\
    $!$ & good move \\
    $!!$ & excellent move \\
    $?$ & bad move \\
    $??$ & blunder \\
    $!?$ & interesting move \\
    $?!$ & dubious move \\
\end{tabular}



\newglossaryentry{prophylaxis}{name={Prophylaxis},description={The strategic concept of anticipating and preventing an opponent's threats and plans before they materialize. It involves making moves that not only advance your own position but also disrupt or hinder the opponent's potential strategies. This proactive approach allows players to maintain control of the game and dictate the flow of play, rather than simply reacting to their opponent's moves.}}%