\section{The Floating Square}
\begin{multicols}{2}
There are cases in which the king must do battle with two separated passed pawns.
In these cases, a useful rule is the \emph{floating square rule}, suggested by 
Studenecki in 1939.

If a square whose two corners are occupied by pawns (on the same rank)
reaches the edge of the board, then one of those pawns will queen.
\index{Floating Square}

If the square does not reach the edge of the board, 
the the king can hold the pawns. If there are two files between the pawns,
the king can capture both; if the distance is any greater, he can only prevent their further advance.

\newchessgame[
    setfen=8/6k1/6P1/p1K1p2P/8/8/8/8/8 b - - 0 1,
    moveid=1b
]

\chessboard[
    pgfstyle=lefttoprightborder,
    linewidth=0.1em,
    markregion={a1-e5},
    pgfstyle=bottomborder,
    markregion={a1-e5}
]

The square have reached the edge of the board, the pawns will queen, regardless 
of whose move it is.

\mainline[level=1]{
    1...a4 2. Kb4 e4 3. Kxa4 e3
} \textbf{$-+$}

\newchessgame[
    setfen=8/6k1/p3p1P1/2K4P/8/8/8/8/8 b - - 0 1,
    moveid=1b
]

\chessboard[
    pgfstyle=lefttoprightborder,
    linewidth=0.1em,
    markregion={a2-e6},
    pgfstyle=bottomborder,
    markregion={a2-e6}
]

The square now reaches only to the 2nd rank and the position becomes a draw.

\mainline[level=1]{1...Kf6!}

\variation[invar]{1... a5? 2. Kb5 e5 3. Kxa5 } \textbf{$+-$};

\variation[invar]{1... Kh6? 2. Kd6! a5 3. Kxe6 a4 4. Kf7 a3 5. g7 a2 6. g8=Q a1=Q 7. Qg6#}

\mainline[level=1]{2. Kc6}

White cannot capture both pawns. Trying to do so is a suicide:
\variation[invar]{
    2. Kb6? e5 3. Kc5 a5
} \textbf{$-+$}

\mainline[level=1]{2... Kg7 3. Kc5} \textbf{$=$};
\end{multicols}