\newpage
\section{Shouldering}
Quite often, one must choose a route for the king that gives
a ``shouldering'' to the enemy king to prevent the enemy 
from arriving in time at an important part of the board.

\begin{multicols}{2}
\newchessgame[
    setfen=8/p4K2/P7/8/8/8/1k6/8/8 w - - 0 1
]
\begin{chessdiagram}[Schlage-Ahues, Berlin 1921]
    \chessboard
\end{chessdiagram}

White will win the pawn at a7. Black can save himself only if 
he can succeed in locking the White king into the corner with ...\symking c7.

The game was drawn after \variation[invar]{
    1. Ke6 Kc3 2. Kd6? Kd4 3. Kc6 Ke5 4. Kb7 Kd6 5. Kxa7 Kc7
}

The right play would be:

\mainline[level=1]{
1. Ke6 Kc3 2. Kd5!
}

White's king approaches the a7-pawn while simultaneously ``shouldering''
the enemy king, keeping it from approahching the c7-square. \index{Shouldering}

\newchessgame[
    setfen=8/1p6/8/8/8/8/1P6/K6k w - - 0 1
]

\begin{chessdiagram}[J. Moravec, 1940]
    \chessboard
\end{chessdiagram}

White only gets a draw out of \variation[invar]{
    1. Ka2? Kg2 2. Kb3 Kf3 3. Kc4 Ke4 4. b4 Ke5 5. Kc5 Ke6
    6. Kb6 Kd5 7. Kxb7 Kc4
} \textbf{$=$}.

It is important to keep Black's king farther away from the pawns:

\mainline[level=1]{
    1. Kb1! Kg2 2. Kc2 Kf3 3. Kd3! Kf4 4. Kd4 Kf5 5. Kd5 Kf6
    6. Kd6 Kf7 7. b4 Ke8 8. Kc7 b5 9. Kc6
}
White wins.


\end{multicols}