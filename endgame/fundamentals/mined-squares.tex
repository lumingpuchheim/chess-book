\section{Mined Squares}

\vocab{mined-square}{Mined Square}{A square that is dangerous to step on, because it will cause the stepping player to be blown up, falling into zugzwang}
Sometimes, it is a single pair of squares that correspond.
Do not be the first to step on a mined square (\textbf{be the second!}), or you will be ``blown up'',
falling into zugzwang. You must either first allow your opponent to step 
on the mined square, or move forward, accurately avoiding it.
\index{Mined Squares}
\begin{multicols}{2}

    \begin{chessdiagram}
        \chessboard[
            setfen=8/2k5/8/1Pp3p1/6P1/1K6/8/8/8 w - - 0 1,
            pgfstyle=circle,
            markfields={b6, c4},
            showmover=false,
        ]
    \end{chessdiagram}
    White's king shuttles between b3, c3 and d3 while the Black king
    goes from c7 to b7 to a7. Neither of them is able to attack 
    the pawn, because c4 and b6 are mined. 

    However, White should never move his king to the other squares because 
    Black can play \variation[invar]{1... Kb6}. Since White's king cannot 
    move to c4, his pawn is lost.

    \newchessgame[
        setfen=8/8/1k1p4/3P1K2/8/8/8/8 w - - 0 1,      
    ]

    \begin{chessdiagram}
        \chessboard[
            pgfstyle=circle,
            markfields={c5, e6},
        ]
    \end{chessdiagram}

    Here e6 and c5 squares are mined. White wins by forcing his opponent
    to go to the mined square first.

    \mainline[level=1]{1. Kf6 Kb5 2. Ke7 Kc5 3. Ke6} \textbf{$+-$}

    \begin{chessdiagram}
        \chessboard[
            setfen=8/8/8/8/5k2/2pP4/2P5/4K3 b - - 0 1,
            pgfstyle=circle,
            markfields={a2, b4},
        ]
    \end{chessdiagram}

    a2 and b4 squares are mined. 

    Black must defend the key square e2 and f2. So the candidate moves are 
    ...\symking e3 or ...\symking f3. Let's analyze first.

    The following position is a mutual zugzwang. If it is 
    Black's turn, he must move his king and then his c3 pawn is lost.
    If it is White's turn, he has nothing better but moving his king back.

    \begin{chessdiagram}
        \chessboard[
            setfen=8/8/8/8/1k6/2pP4/K1P5/8 b - - 0 1,   
        ]
    \end{chessdiagram}

    With White's king at a2, Black's king must be at b4 (a4 is too far from the kingside.)

    We will have the whole packet of corresponding squares by moving the squares 
    to the kingside: a2-b4, b1-c5, c1-d4, d1-e3 and e1-f3. Therefore Black's correct move is ...\symking f3. \index{Mined Squares} \index{Corresponding Squares}
\end{multicols}