\newpage
\section{Drawn Positions: Bishop and Pawn} \label{section:bishop-and-pawn-drawn-positions}

\begin{multicols}{2}
    If the bishop does not control the rook pawn's 
    queening square, then the weaker side has 
    only to get his king in to corner (``safe corner''). 

    \begin{chessdiagram}[Safe Corner]
        \chessboard[
            setfen=6k1/8/6BP/8/8/8/2K5/8 b - - 0 1,
            showmover=false,
        ]
        \label{chess:bishop-endgame-1}
    \end{chessdiagram}

    The draw is obvious since White cannot drive the Black king 
    away from the corner.
    
    \begin{chessdiagram}[Pawns at h6 and h7]
        \chessboard[
            setfen=6k1/7p/4K2P/4B3/8/8/8/8 b - - 0 1,
            showmover=false,
        ]
        \label{chess:bishop-endgame-2}
    \end{chessdiagram}

    The position is a draw because Black's king only needs to move 
    between f8 and g8 and White cannot attack the h7 pawn. The evaluation 
    would not change if you added pawn pairs at g5/g6 and f4/f5.

    The position is also a draw when White has a light-squared bishop. 
    In this case, Black only needs to move his king between g8 and h8. 
    White can only stalemate the Black king when his bishop controls the g8 square.

    \begin{chessdiagram}[Pawns at h7 and Pawn at g6 (Ponziani)]
        \chessboard[
            setfen=8/6kB/6P1/5K2/8/8/8/8 b - - 0 1,
            showmover=false,
        ]
        \label{chess:bishop-endgame-3}
    \end{chessdiagram}
    
    This drawing position has been known since the 18th century. The 
    bishop cannot escape from its corner and giving it up leads to a 
    drawn pawn ending.

    \begin{chessdiagram}[Pawns at g6 and g7]
        \chessboard[
            setfen=5k2/6p1/3K2P1/3B4/8/8/8/8 b - - 0 1,
            showmover=false,
        ]
        \label{chess:bishop-endgame-4}
    \end{chessdiagram}

    The draw is obvious here and it would still be so if we added more 
    pawn pairs at h5/h6, f5/f6 and e4/e5.

    The position is also a draw if White had a dark-squared bishop. In this 
    case, Black only needs to move his king between g8 and h8. If White exchanges 
    his bishop for the g7 pawn, the resulting position is draw because 
    Black's king controls the key-square g7. \index{Key Square}
\end{multicols}

