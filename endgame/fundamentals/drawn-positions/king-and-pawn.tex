\newpage
\section{Drawn Positions: King Endgame}

\begin{multicols}{2}
    We know the idea of key square in the king endgame. \index{Drawn Positions!King Endgame}

    When the attacker's pawn arrives the 6th rank with his king, it depends 
    who controls the opposition. 
    
    If the defender controls the opposition, he can use stalemate as an important resource for the defender. \index{Stalemate}

    \newchessgame[
        setfen=4k3/8/4KP/8/8/8/8/8/8 w - - 0 1,
    ]
    \begin{chessdiagram}
        \chessboard
    \end{chessdiagram}

    White to move draw. \mainline[level=1]{1. f7 Kf8 2. Kf6 } stalemate.

    If the attacker controls the opposition, he can use zugzwang to win the game. \index{Zugzwang}
    
    \begin{chessdiagram}
        \chessboard[
            setfen=4k3/8/4KP/8/8/8/8/8/8 b - - 0 1,
        ]
    \end{chessdiagram}
    
    Black to move loses \variation{1... Kf7 2. f7} Black is in zugzwang and he 
    must move his king away.

    Rook-pawn is the exception. In the following position, no matter who has the move, 
    the position is drawn. 

    \begin{chessdiagram}
        \chessboard[
            setfen=k7/8/PK6/8/8/8/8/8/8 b - - 0 1,
            showmover=false,
        ]
        \label{chess:rook-pawn}
    \end{chessdiagram}

    The defender may also stalemate the attacker's king in the corner.

    \begin{chessdiagram}
        \chessboard[
            setfen=5k1K/7P/8/8/8/8/8/8 b - - 0 1,
            showmover=false,
        ]
    \end{chessdiagram}

    When Black's king is at f7, it is obviously also a stalemate.

    Against an h-pawn, the defender can combine defence, either lock the enemy's 
    king in the corner, or the king goes to corner himself.

    \begin{chessdiagram}
        \chessboard[
            setfen=K7/2k5/P7/8/8/8/8/8 w - - 0 1,
            showmover=false,
        ]
    \end{chessdiagram}

    If it is White to play, it is a draw. 

    \variation{
        1. Ka7 Kc7 2. Ka8 Kc8
    } etc.

    If it is Black to play, is is also a draw.

    \variation{
        1... Kc7 2. Ka7 Kc8 3. Kb6 Kb8
    } We have seen this position already in diagram \ref{chess:rook-pawn} and it is a draw.
\end{multicols}