\newpage
\section{Passed Pawn (Breakthrough)}

A breakthrough occurs when one or more pawns are sacrificed
in order to create a passed pawn and promote it.

We will examine a few of the standard structures how 
to achieve a breakthrough and how to prevent a breakthrough. \index{Passed Pawn} \index{Pawn Breakthrough}

\begin{multicols}{2}

\newchessgame[
    setfen=8/ppp5/8/PPP5/8/7k/8/7K w - - 0 1,
]

\chessboard

White wins by \mainline[level=1]{1. b6 cxb6 2. a6 bxa6 3. c6}.

Black to move has only one way to avoid the threatened breakthrough
\variation{1... b6}. Both \variation[invar]{1... a6 2. c6} and
\variation[invar]{1...c6 2. a6} are bad.

\chessboard[
    setfen=8/8/2p3k1/ppp5/6PK/2P4P/P1P5/8 w - - 0 1,
    showmover=false,
]

This is the structure we find in the Ruy Lopez Exchange Variation.
Black to move can create a passed pawn by 1...c4, followed by 
...c5, ...b4 ...a4 and ...b3.

White to move can play \vocab{prophylaxis}{prophylaxis}{} with 
1. c4, which guarantees him a decisive advantage.

When the opponent has a pawn majority with a doubled pawn, 
it is often a good move to play a pawn move on the file of 
the enemy's doubled pawn to prevent the opponent from creating a passed pawn.

\textbf{Maslov - Glebov, USSR 1936}

\newchessgame[
    setfen=6k1/6pp/5p2/4p3/p1p1P2P/2P2PP1/1K6/8 b - - 0 1,
    moveid=1b
]

\chessboard 

The possibility of a pawn breakthrough changes the evaluation of the position completely.

\mainline[level=1]{
    1... h5 2. Ka3
}

\variation[invar]{
    2. g4 g5
}

\chessboard[
    setfen=6k1/8/5k2/4p1pp/p1p1P1PP/2P2P2/1K6/8 b - - 0 1,
    showmover=false,
]

This structure is called a ``tetris'': four pawns meet in a tetris. Black 
will create a passed pawn on h4. \index{Pawn Breakthrough!Tetris}

Returning to the mainline. 

\mainline[level=1]{
    2...g5 3. Kxa4 f5! 
}

\chessboard 

There is no defence: \variation[invar]{
    4. hxg5 f4
} or \variation[invar]{
    4. exf5 g4 5. fxg4 e4
}

\mainline[level=1]{4. Kb5 f4 5. gxf5 gxh4} The h-pawn promotes.
\end{multicols}
