\section{Key Square}
\vocab{key-square}{Key Square}{What we call those squares whose occupation by the king assures victory, regardless of whose turn it is to move.
In other types of endgame, we may also speak of key squares for other pieces besides the king}
are what we call those squares whose occupation by the king assures victory, regardless of whose turn it is to move.
In other types of endgame, we may also speak of key squares for other pieces besides the king.
\index{Key Square}
\begin{multicols}{2}
    \chessboard[
        setfen=8/3k4/8/3K4/3P4/8/8/8/8 w - - 0 1,
        showmover=false,
        pgfstyle=cross,
        markfields={c6,d6,e6},
    ]
    
    The d5-square on which the king now stands is not a key square. White
    to move does not win. The key squares are c6, d6 and e6. Black 
    to move must retreat his king , allowing the enemy king onto one of the key squares.
    With White to move, the position is drawn, since he cannot move to any key square.

    \chessboard[
        setfen=1k6/8/1K6/1P6/8/8/8/8 w - - 0 1,
        showmover=false,
        pgfstyle=cross,
        markfields={a7,b7,c7,a6,c6},
    ]

    \newchessgame[
        setfen=1k6/8/1K6/1P6/8/8/8/8 w - - 0 1,
    ]

    With the pawn on the 5th rank, the key squares are not only 
    a7, b7 and c7, but also the similar 6-th rank squares a6, b6 and c6.

    White wins even if he is on the move: \mainline[level=1]{1. Ka6! Ka8 2. b6 Kb8 3. b7} \textbf{$+-$}

\end{multicols}