\newpage
\section{Triangulation}

\begin{multicols}{2}
    \vocab{triangulation}{Triangulation}{A king maneuver which aims to 
    lose a tempo, and leave the opponent with the move} refers to a king maneuver which aims to 
    lose a tempo, and leave the opponent with the move.

    The following position is very important, both for itself
    and as an illustration of the characteristic logic 
    of analyzing corresponding squares. \index{Triangulation}


    \newchessgame[
        setfen=8/1p1k4/1P6/2PK4/8/8/8/8 w - - 0 1
    ]

    \begin{chessdiagram}
        \chessboard
    \end{chessdiagram}

    The d5 and d7 square are in correspondence. The mobility 
    of Black's king is restricted: he must watch 
    for the c6 break, and also avoid being 
    pressed to the edge of the board. White can ``lose'' a tempo 
    and place his opponent in zugzwang.

    \mainline[level=1]{1. Ke5! Kc6 2. Kd4 Kd7 3. Kd5} 

    We achieve the same position with Black to move. Black is 
    in zugzwang.

    \mainline[level=1]{3... Kc8 4. Ke6 Kd8 5. Kd6 Kc8 6. Ke7 Kb8
    7. Kd7 Ka8 8. c6} \textbf{$+-$}

    
\end{multicols}