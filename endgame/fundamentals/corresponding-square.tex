\newpage
\section{Corresponding Square}

In chess, two squares are corresponding squares if the occupation of one of these squares by a king requires the enemy king to move to the other square in order to hold the position. Corresponding squares exist in some chess endgames, usually ones that are mostly blocked. Usually, there are several groups of corresponding squares. In some cases, they indicate which square the defending king must move to in order to keep the opposing king away. In other cases, a maneuver by one king puts the other player in a situation where he cannot move to the corresponding square, so the first king is able to penetrate the position.[2] The theory of corresponding squares is more general than opposition and is more useful in cluttered positions \cite{wiki:corresponding-square}.



The most common application of corresponding squares are \emph{opposition}, \emph{triangulation} and \emph{mined squares}.