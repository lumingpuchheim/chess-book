\section{Rook vs. Pawn}

\begin{multicols}{2}
    \newchessgame[
        setfen=8/8/1KP5/3r4/8/8/8/k7 w - - 0 1,
    ]
    \begin{chessdiagram}[Moving Downstairs]
        \chessboard
    \end{chessdiagram}

    \mainline[level=1]{1. c7 Rd6+}

    We can use \vocab{PoE}{Process of Elimination}{}
    to find the right move. 
    2. \symking b5 is bad because 2...\symrook d1, followed by
    3...\symrook c1, exchanging the rook for a pawn.

    2. \symking b7 is bad because 2... \symrook d7,
    which exchanges the rook for a pawn.

    \mainline[level=1]{
        2. Kb5 Rd5+ 3. Kb4 Rd4+ 4. Kb3 Rd3+ 5. Kc2 Rd4!
    }

    \begin{chessdiagram}
        \chessboard
    \end{chessdiagram}

    \variation[invar]{
        6. c8=Q Rc4+ 7. Qxc4
    } results in stalemate.

    \mainline[level=1]{
        6. c8=R! Ra4 7. Kb3!
    } \textbf{$+-$}.


    \newchessgame[
        setfen=7K/6R1/1k6/p7/8/8/8/8 w - - 0 1,
    ]
    \begin{chessdiagram}[Cutting the King off]
        \chessboard
        \label{digram: rook cut off}
    \end{chessdiagram}

    \mainline[level=1]{1. Rg5} \index{Rook Endgame!Cut off}

    When Black's pawn reaches a3, it will be captured by \symrook g3. 


    \newchessgame[
        setfen=R7/8/8/8/2K3p1/8/5k2/8 w - - 1 1
    ]

    \begin{chessdiagram}[An Intermediate Check]
        \chessboard
    \end{chessdiagram}

    \mainline[level=1]{
        1. Rf8+! Ke2 2. Rg8 Kf3}
        
    Because of the check, White succeeded in driving Black's 
    king back one square from f2 to f3.

    \mainline[level=1]{3. Kd3 g3 4. Kf8+ Kg2 5. Ke2} \textbf{$+-$}.

    \newchessgame[
        setfen=8/8/8/8/1K6/7R/1pk5/8 w - - 0 1,
    ]
    \begin{chessdiagram}[Pawn Promotion to a Knight]

        \chessboard
    \end{chessdiagram}
    
    \mainline[level=1]{1. Rh2+ Kc1 2. Kc3 b1=N! 3. Kd3 Na3 4. Ra2 Nb1!} $=$
    
    If Black has a rook-pawn, this method doesn't work.

    \newchessgame[
        setfen=8/8/8/3K4/8/p7/7R/1k6 w - - 0 1,
    ]
    \begin{chessdiagram}
        \chessboard
        \label{fig: promotion-to-knight-2}
    \end{chessdiagram}

    \mainline[level=1]{
        1. Kc4 a2 2. Kb3 a1=N 3. Kc3
    } \textbf{$+-$}.


    \newchessgame[
        setfen=7R/8/8/8/8/pk1K4/8/8 b - - 0 1,
        moveid=1b
    ]
    \begin{chessdiagram}[Stalemate]
        \chessboard
    \end{chessdiagram}
    
    \mainline[level=1]{1... Kb2 2. Rb8+ Kc1 3. Ra8 Kb2 4. Kd2 a2 5. Rb8+ Ka1!} $=$


    \newchessgame[
        setfen=8/8/8/8/2K5/7R/pk6/8 w - - 0 1,
    ]
    \begin{chessdiagram}[Shouldering]
        \chessboard
    \end{chessdiagram}
    
    \mainline[level=1]{1. Rh2+}

    \variation[invar]{
        1... Kb1 
    } is bad since we have already seen this in diagram \ref{fig: promotion-to-knight-2}.

    \mainline[level=1]{
        1... Ka3
    } $=$

    \newchessgame[
        setfen=8/4K3/8/3pk3/3R4/8/8/8 w - - 0 1,
    ]
    \begin{chessdiagram}[Outflanking]   
        \chessboard
    \end{chessdiagram}
    
    \mainline[level=1]{1. Rd2! d4 2. Rd1 Kd5 3. Kd7}

    \begin{chessdiagram}
        \chessboard
    \end{chessdiagram}

    In pawn endgames, we know the idea of \vocab{opposition}{Opposition}{}
    and outflanking. Here we have the same principle: White gains opposition 
    in order to outflank. \index{Opposition}

    Black is in zugzwang. \variation[invar]{3... Kc4 4. Ke6}
    or \variation[invar]{3... Ke4 4. Kc6}
\end{multicols}
