\newpage
\section{Rook vs. Connected Pawns}

\begin{multicols}{2}
    \newchessgame[
        setfen=8/8/5K2/7k/8/p7/1p6/1R6 w - - 0 1,
    ]
    \begin{chessdiagram}[Checkmate Threat]
        \chessboard
    \end{chessdiagram}

    \mainline[level=1]{1. Kf5 Kh4 2. Kf4 Kh3 3. Kf3 Kh2}

    With mating threats, White's king manages to help in time.
    As long as Black's king is on the 2nd rank, Black cannot 
    play ...a2 because \symrook xb2+.

    \mainline[level=1]{
        4. Ke3 Kg2 5. Kd3 Kf3 6. Kc2 a2 7. Kxb2 axb1=Q+ 8. Kxb1
    }

    \newchessgame[
        setfen=8/8/P7/1PK5/8/8/8/r3k3 w  w - - 0 1
    ]
    \begin{chessdiagram}[Auto Pilot]
        \chessboard
    \end{chessdiagram}

    \mainline[level=1]{
        1. Kb6 Kd2 2. Ka7!}
    
    \variation[invar]{
        2. a7
    } only draws. 

    \begin{chessdiagram}[Tail Hook]
        \chessboard[setfen=8/P7/1K6/1P6/8/8/3k4/r7 b - - 0 2]
    \end{chessdiagram}
    
    \variation[]{
        2... Kc3 3. Kb7 Kb4 4. b6 Kb5
    } \index{Tail Hook}

    Now we go back to the mainline.We see \vocab{auto-pilot}{Auto Pilot}{} again.
    White pushes the less advanced b-pawn. 
    Black's rook cannot prevent the march because it is placed on another
    file. \index{Auto Pilot}


    \mainline[level=1]{2...Kc3 3. b6 Kc4 4. b7 Rb1 5. b8=Q Rxb8 6. Kxb8
    } $1-0$

\end{multicols}