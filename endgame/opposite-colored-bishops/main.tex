\chapter{Opposite Colored Bishops} \label{chapter:opposite-colored-bishops}

From chapter \ref{chapter:bishop-and-pawns-endgame} and section \ref{section:bishop-and-pawn-drawn-positions} we already learned some basic ideas:
\begin{itemize}
    \item{
        \vocab{one-diagonal-principle}{One Diagonal Principle}{}
    }

    The bishop fulfills its functions from a single diagonal
    \item{
        \vocab{pawns-in-the-crosshairs}{Pawns in the Crosshairs}{}
    }

    attacking the enemy pawns with the bishop either to force the pawns to advance,
    which aids in the task of their subsequent blockade, or else to tie the king to their defense
    \item{
        \vocab{tail-hook}{Tail Hook}{}
    }

    Tying the king to the rearmost pawn from behind
\end{itemize}

\paragraph{Fortress}
Here it is frequently possible to save oneself even two or three pawns down.

Often in opposite-colored bishop endgames, the fortress is constructed rather than played.
First, it is necessary to determine the configuration of pawns and pieces which 
will render the position impenetrable; only then can we proceed with the calculation of variations which will 
prove whether or not we can attain the desired configuration and whether 
it is impenetrable in fact. 

\paragraph{Pawn Placement}
In opposite-colored bishop endings, the stronger side 
should place his pawns on squares of the opposite color to his bishop, while
the weaker side must, contrary to the general rule, keep his pawns on the same color squares 
as his own bishop. ``A bad bishop protects the good pawns.''

\paragraph{Positional Nuance}
When playing an opposite-colored bishop ending, the 
number of pawns is frequently less important than a small alteration 
in the placement of pieces or pawns. Therefore, we often encounter 
pawn sacrifices.

\begin{multicols}{2}
    \newchessgame[
        setfen=1k6/8/1b2pp2/6p1/2P4p/5K1P/2B5/8 w - - 0 1,
    ]
    \begin{chessdiagram}
        \chessboard
    \end{chessdiagram}

    \mainline[level=1]{
        1. c5 Bxc5 2. Bb3 e5 3. Be6 Kc7 4. Ke4
    }

    White achieves a draw by moving his bishop along the h3-c8 diagonal when 
    he is three pawns down. The final position is a \textbf{fortress}.
    White places his pawns on the same color as his own bishop.

    The bishop defends its pawn at h3 and holds the enemy pawns 
    \textbf{on one diagonal}.
\end{multicols}

\newpage
\section{Bishop and Two Connected Pawns vs. Bishop}
\begin{multicols}{2}
    \newchessgame[
        setfen=8/4k3/8/4PP2/b3K3/8/3B4/8 b - - 0 1,
        moveid=1b
    ]

    \begin{chessdiagram}
        \chessboard
    \end{chessdiagram}

    White threatens e6, \symking e5, and f6. Black must defend the 
    e6 square. The candidate moves are ...\symbishop b3 or ...\symbishop d7.


    We check first \variation[invar]{
        1... Bb3
    }

    \begin{chessdiagram}
        \chessboard[
            setfen=8/4k3/8/4PP2/4K3/1b6/3B4/8 w - - 1 2
        ]
    \end{chessdiagram}

    To win the game, White must infiltrate his king to support the e6 advance.
    He has two routes: either from the queenside or from the kingside.

    White can give a probe check with \symbishop g5. Depending on where the Black king 
    moves, White infiltrates from the opposite side.

    For example 
    \variation[invar]{
        2. Bg5+ Kf7 3. Kd4 Ka2 4. Kc5 Bb3 5. Kd6
    } or \variation{
        2. Bg5+ Kd7 3. Kf4 Ba2 4. Bh4 Bf7 5. Kg5 Ke7 6. Kh6+ Kd7 7. Kg7 Bc4 
        8. Kf6
    } Black cannot stop e6.

    However, Black can defend after \variation[invar]{
        2. Bb4 Kf7 3. Kd4 
    } 

    \begin{chessdiagram}
        \chessboard[
            setfen=8/5k2/8/4PP2/1B1K4/1b6/8/8 b - - 4 3
        ]
    \end{chessdiagram}

    \variation{
        3... Bc2
    }

    ``Pawns in the crosshairs!''

    \variation{
        4. e6+ Kf6 5. e7 Ba4
    } draws. 
    
    We know the stronger side must place his pawns on squares of the opposite color to his own bishop, and 
    \textbf{
        When the weaker side manages to force the stronger 
        side to place his pawns on the same color as his own bishop and 
        block them, he can defend.
    }

    Returning to the mainline. Black must move his bishop to d7.
    \mainline[level=1]{
        1...Bd7 2. Bg5+ Kf7
    }

    Black moves his bishop between d7 and c8, preventing 
    the e6 advance. White cannot make any progress. He cannot move his king to d5 to 
    infiltrate because he is tied to defending his f5 pawn. 

    \textbf{
        The defender must place his bishop where it prevents the advance 
        of one pawn and attacking the other.
    }

\end{multicols}
\newpage
\section{Bishop and Two Disconnected Pawns vs. Bishop}
\begin{multicols}{2}
    With disconnected passed pawns, the stronger side's 
    strategy is always: the king goes toward the pawn that the bishop is holding back.

    \begin{chessdiagram}
        \chessboard[
            setfen=8/2k1b3/2P5/3K1P1B/8/8/8/8/8 w - - 0 1,
        ]
    \end{chessdiagram}

    \variation{1. Bf3 Kd8 2. Ke6 Bb4 3. f6 Ba5 4. f7 Bb4 5. Kf6 Bc3+ 6. Kg6 Bb4 7. Kg7} \textbf{$+-$}.

    Sometimes the weaker side's king can help the bishop defend against both pawns 
    at once by shuttling to whichever fland it's needed to prevent the enemy king from 
    invading his territory.

    \begin{chessdiagram}
        \chessboard[
            setfen=8/2bB4/2P5/6k1/4K3/5P2/8/8 w - - 0 1,
        ]
    \end{chessdiagram}
    \variation{
        1. Kd5 Kf6
    } If the White king goes to b7, Black's goes to d8.
    Note Black's bishop defends with \vocab{one-diagonal-principle}{One Diagonal Principle}{} \index{Bishop Endgame!One Diagonal Principle}.

    The next position has great practical significance.

    The stronger side has a knight pawn and a center pawn separated by 
    two files. The weaker side has a bishop that controls the queening 
    square of the stronger side's knight pawn. \index{Opposite Colored Bishops!Berger-Kotlerman Position}

    \newchessgame[
        setfen=8/8/8/5B2/1p3b2/2k1p3/8/5K2 w - - 0 1,
    ]
    \begin{chessdiagram}[Berger-Kotlerman]
        \chessboard[
            
        ]
    \end{chessdiagram}
    \label{fig:berger-kotlerman}

    \mainline[level=1]{
        1. Ke2 b3 2. Kd1 Kb4 3. Bh7 Ka3 4. Bg6 Kb2
    }
    
    \variation[invar]{
        4...b2 \xskakcomment{ threatening 5... \symking a2} 5. Bb1 Kb3 6. Ke2 
    }

    \mainline[level=1]{
        5. Bf7
    }

    White's defence plan:
    
    If Black's king goes to a3, White places his bishop on b1-h7 diagonal. 
    In case Black plays ...b2, White can defend with ...\symbishop b1. Black can not 
    drive the bishop away because of the edge of the board.

    If Black's king goes to a2, White places his bishop on a2-g8 diagonal. Black 
    cannot advance his pawn because of the pin.

    Otherwise White moves his Bishop along b1-h7 or a2-g8 diagonal.

\end{multicols}
\newpage
\section{The King Blockades the Passed Pawn}
The stronger side often has a passed pawn, which 
must be blockaded either by the king or by the bishop.

\vocab{first-defensive-system}{
    The first Defensive System
} {
    The king blockades the enemy passed pawn while 
    the bishop defends its own pawns. 
} blockades the enemy passed pawn with the king while the bishop 
defends its own pawns. This is the basic 
and usually the most secure defensive arrangement.

\begin{multicols}{2}
    \newchessgame[
        setfen=8/1k3p2/1P3Kp1/7p/2b2P2/4B1P1/7P/8 w - - 0 1,
    ]
    \begin{chessdiagram}
        \chessboard
    \end{chessdiagram}

    Black's bishop defends the pawns and the king holds the 
    passed pawn. White cannot make any progress

    \mainline[level=1]{
        1. f5 Bd3
    } \textbf{$=$}.

    Breaking down the first defensive system always involve the creation of a second passed pawn, 
    frequently by a pawn breakthrough. \index{Pawn Breakthrough}

    \newchessgame[
        setfen=8/8/4b1p1/2B13p/5P1P/1pK1Pk2/8/8 b - - 0 1,
        moveid=1b
    ]
    \begin{chessdiagram}[Kotov-Botvinnik 1955]
        \chessboard
    \end{chessdiagram}

    A classic example of using pawn breakthrough to break down the first defensive system.

    \mainline[level=1]{
        1...g5! 2. fxg5 d4! 3. exd4 Kg3 4. Ba3}
    
    \variation[invar]{
        4. Be7 Kxh4 5. g6+ Kg4
    }

    Black's bishop has an excellent position. It protects the b3-pawn 
    and restrains both enemy pawns along the \textbf{one diagonal} a2-g8 (\vocab{one-diagonal-principle}{One Diagonal Principle}{}) \index{Bishop Endgame|One Diagonal Principle}

    White has no counterplay. Black just advances his h-pawn and wins the bishop for it.

    \mainline[level=1]{4... Kxh4 5. Kd3 Kxg5}
    

    \chessboard

    Now Black has a second passed pawn and can win the game.

    \mainline[level=1]{
        6. Ke4 h4 7. Kf3 Bd5+
    }

    White resigned because after \symking f2, Black's king goes after the 
    b3-pawn. The bishop defends the h-pawn while restraining the d-pawn 
    along the c8-h3 diagonal.
\end{multicols}