\newpage
\section{Bishop and Two Disconnected Pawns vs. Bishop}
\begin{multicols}{2}
    With disconnected passed pawns, the stronger side's 
    strategy is always: the king goes toward the pawn that the bishop is holding back.

    \begin{chessdiagram}
        \chessboard[
            setfen=8/2k1b3/2P5/3K1P1B/8/8/8/8/8 w - - 0 1,
        ]
    \end{chessdiagram}

    \variation{1. Bf3 Kd8 2. Ke6 Bb4 3. f6 Ba5 4. f7 Bb4 5. Kf6 Bc3+ 6. Kg6 Bb4 7. Kg7} \textbf{$+-$}.

    Sometimes the weaker side's king can help the bishop defend against both pawns 
    at once by shuttling to whichever fland it's needed to prevent the enemy king from 
    invading his territory.

    \begin{chessdiagram}
        \chessboard[
            setfen=8/2bB4/2P5/6k1/4K3/5P2/8/8 w - - 0 1,
        ]
    \end{chessdiagram}
    \variation{
        1. Kd5 Kf6
    } If the White king goes to b7, Black's goes to d8.
    Note Black's bishop defends with \vocab{one-diagonal-principle}{One Diagonal Principle}{} \index{Bishop Endgame!One Diagonal Principle}.

    The next position has great practical significance.

    The stronger side has a knight pawn and a center pawn separated by 
    two files. The weaker side has a bishop that controls the queening 
    square of the stronger side's knight pawn. \index{Opposite Colored Bishops!Berger-Kotlerman Position}

    \newchessgame[
        setfen=8/8/8/5B2/1p3b2/2k1p3/8/5K2 w - - 0 1,
    ]
    \begin{chessdiagram}[Berger-Kotlerman]
        \chessboard[
            
        ]
    \end{chessdiagram}
    \label{fig:berger-kotlerman}

    \mainline[level=1]{
        1. Ke2 b3 2. Kd1 Kb4 3. Bh7 Ka3 4. Bg6 Kb2
    }
    
    \variation[invar]{
        4...b2 \xskakcomment{ threatening 5... \symking a2} 5. Bb1 Kb3 6. Ke2 
    }

    \mainline[level=1]{
        5. Bf7
    }

    White's defence plan:
    
    If Black's king goes to a3, White places his bishop on b1-h7 diagonal. 
    In case Black plays ...b2, White can defend with ...\symbishop b1. Black can not 
    drive the bishop away because of the edge of the board.

    If Black's king goes to a2, White places his bishop on a2-g8 diagonal. Black 
    cannot advance his pawn because of the pin.

    Otherwise White moves his Bishop along b1-h7 or a2-g8 diagonal.

\end{multicols}