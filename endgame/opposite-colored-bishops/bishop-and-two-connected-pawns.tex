\newpage
\section{Bishop and Two Connected Pawns vs. Bishop}
\begin{multicols}{2}
    \newchessgame[
        setfen=8/4k3/8/4PP2/b3K3/8/3B4/8 b - - 0 1,
        moveid=1b
    ]

    \begin{chessdiagram}
        \chessboard
    \end{chessdiagram}

    White threatens e6, \symking e5, and f6. Black must defend the 
    e6 square. The candidate moves are ...\symbishop b3 or ...\symbishop d7.


    We check first \variation[invar]{
        1... Bb3
    }

    \begin{chessdiagram}
        \chessboard[
            setfen=8/4k3/8/4PP2/4K3/1b6/3B4/8 w - - 1 2
        ]
    \end{chessdiagram}

    To win the game, White must infiltrate his king to support the e6 advance.
    He has two routes: either from the queenside or from the kingside.

    White can give a probe check with \symbishop g5. Depending on where the Black king 
    moves, White infiltrates from the opposite side.

    For example 
    \variation[invar]{
        2. Bg5+ Kf7 3. Kd4 Ka2 4. Kc5 Bb3 5. Kd6
    } or \variation{
        2. Bg5+ Kd7 3. Kf4 Ba2 4. Bh4 Bf7 5. Kg5 Ke7 6. Kh6+ Kd7 7. Kg7 Bc4 
        8. Kf6
    } Black cannot stop e6.

    However, Black can defend after \variation[invar]{
        2. Bb4 Kf7 3. Kd4 
    } 

    \begin{chessdiagram}
        \chessboard[
            setfen=8/5k2/8/4PP2/1B1K4/1b6/8/8 b - - 4 3
        ]
    \end{chessdiagram}

    \variation{
        3... Bc2
    }

    ``Pawns in the crosshairs!''

    \variation{
        4. e6+ Kf6 5. e7 Ba4
    } draws. 
    
    We know the stronger side must place his pawns on squares of the opposite color to his own bishop, and 
    \textbf{
        When the weaker side manages to force the stronger 
        side to place his pawns on the same color as his own bishop and 
        block them, he can defend.
    }

    Returning to the mainline. Black must move his bishop to d7.
    \mainline[level=1]{
        1...Bd7 2. Bg5+ Kf7
    }

    Black moves his bishop between d7 and c8, preventing 
    the e6 advance. White cannot make any progress. He cannot move his king to d5 to 
    infiltrate because he is tied to defending his f5 pawn. 

    \textbf{
        The defender must place his bishop where it prevents the advance 
        of one pawn and attacking the other.
    }

\end{multicols}