\newpage
\section{The Bishop Restrains the Passed Pawn}
Situations in which the bishop stops a passed pawn (or two pawns on the same diagonal)
are called the \vocab{second-defensive-system}{Second Defensive System}{In 
opposite colored bishop endgame, in which the bishop stops a passed pawn (or two pawns on the same diagonal)}.
The weaker side's king must defend its pawns and limit the activity of the 
enemy king. \index{Opposite Colored Bishop Endgame!Second Defensive System}

\begin{multicols}{2}
    \newchessgame[
        setfen=8/P5p1/4k3/2B4p/4b3/4K1P1/1P5P/8 b - - 0 1,
        moveid=1b
    ]
    \begin{chessdiagram}[Euwe-Yanovsky 1946]
        \chessboard
    \end{chessdiagram}

    In the actual game, Black played \variation[invar]{1... Bg2} and 
    lost. We will see this later.
    Black cannot allow White's king to approach his pawns. 
    A draw can be achieved via
    \mainline[level=1]{
        1... Kf5 2. Bf8 g6 3. Kd4 Bg2 4. Kc5 Ke6 5. Kb6 Kd7 6. b4 Ba8
        7. b5 Kc8
    }

    Breaking the second defensive system often involves the king breaking through 
    to the passed pawn, frequently by using \vocab{widen-the-beachhead}{Widen the Beachhead}{}.
    
    Returning to the previous game, we see how White broke the second 
    defensive system.

    \newchessgame[
        setfen=8/P5p1/4k3/2B4p/4b3/4K1P1/1P5P/8 b - - 0 1,
        moveid=1b
    ]
    \begin{chessdiagram}
        \chessboard
    \end{chessdiagram}

    \mainline[level=1]{
        1... Bg2? 2. Bf4 g6 3. g4 
    } 

    Widening the beachhead!

    \mainline[level=1]{
        3... hxg4 4. Kxg4 Bh1 5. Kg5 Kf7 6. Bd4 Bg2 
        7. h4 Bh1 8. b4 Bg2 9. b5 Bh1
    }

    \begin{chessdiagram}
        \chessboard
    \end{chessdiagram}

    Black's king was tied to defending the g6 pawn. White 
    has strengthened his queenside to the maximum by advancing his pawns.
    Now White begins exchanging the pawns on the kingside to 
    start the final march to the queenside.

    \mainline[level=1]{
        10. Bf6 Bg2 11. h5! gxh5 12. Kf5
    } Black resigned because he cannot stop White's king's 
    invasion via \symking e6-d6-c7.

    \newchessgame[
        setfen=8/P4k2/4p1pp/3bBp2/3Pp2P/4P3/4KPP1/8 w - - 0 1
    ]

    \begin{chessdiagram}[Kaidanov-Antoshin 1984]
        \chessboard
    \end{chessdiagram}

    White has two possible plans, either his king marches to the 
    queenside to support his passed pawn, or he infiltrates 
    to the kingside to attack the weak g6-pawn.
    White should try to threaten both at the same time so that Black cannot 
    defend both: if his king arrives at e5 and his bishop controls e7-square,
    Black's king cannot defend d6 and f6 at the same time.

    White must be careful. \variation[invar]{1. Bf4 g5 2. hxg5 hxg5 3. Bxg5 Ke8}.
    Black switches to the \vocab{first-defensive-system}{First Defensive System}{}
    and White cannot create a second passed pawn.

    ...g5 alone is not a threat. \variation[1... g5 2. h5] Black cannot protect his kingside 
    pawns.

    White marches his king along dark squares.
    \mainline[level=1]{
        1. Kf1 Ba8 2. Kg1 Bd5 3. Kh2 Ba8 4. Kg3 Bd5 5. Bc7!}

    \begin{chessdiagram}
        \chessboard
    \end{chessdiagram}

    \mainline[level=1]{
        5...Ke7 }
    
    \variation[invar]{
        5... Ba8 6. Kf4 g5+ 7. Ke5 Ke7 8. h5 Bc6 9. Bc5+ Ke6
    } White's king infiltrates.    
        
    \mainline[level=1]{6. Bf4 g5 }
    
    \begin{chessdiagram}
        \chessboard
    \end{chessdiagram}
    
    Now White sacrifices his bishop for the pawn. The remaining endgame is winning 
    because Black cannot defend both passed pawns.
    
    \mainline[level=1]{7. Bxg5+ hxg5 8. hxg5 Kf7 9. f4 Kg6 10. Kh4 Ba8 11. g4 fxg4 12. Kxg4 Bd5 13. Kg3 Kf7 14. Kf2 Ke7 15. Ke1 Kd6 16. Kd2 Bc6 17. Kc3 Ba8 18. Kb4 Bd5 19. g6 Ke7 20. Kc5 Kf6 21. f5 Ba8 22. fxe6 Kxe6 23. d5+
    }
\end{multicols}