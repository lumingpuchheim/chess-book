\section{Bishop vs Disconnected Pawns}

\vocab{pants}{Pants}{
    The bishop stops pawns on two different diagonals.
    Advancing one of the pawns means the bishop must give up its guard over the other.
} refers to a situation where the bishop stops pawns on two different diagonals.
Advancing one of the pawns forces the bishop to give up its guard over the other.

Note that there are other ways to exploit 
a torn bishop. Sometimes, it may be driven away from the intersection 
of two diagonals by the king, or forced to move away by means of zugzwang.

\vocab{one-diagonal-principle}{One Diagonal Principle}{
    The bishop fulfills its functions from a single diagonal
} states that a bishop can defend from a single diagonal. \index{Bishop Endgame|One Diagonal Principle}

\begin{multicols}{2}
    \begin{chessdiagram}[Dvoretsky, 2000]
        \newchessgame[
            setfen=3Bk3/2K2p2/8/5pP1/2p5/8/8/8 w - - 0 1,
        ]
        \chessboard
    \end{chessdiagram}

    \mainline[level=1]{1. g6! }

    The bishop now fulfills both functions from the single diagonal c1-h6.

    \mainline[level=1]{
        1... f6
    }

    \variation[invar]{1... fxg6 2. Bg5} the draw is easily achieved.

    \begin{chessdiagram}
        \chessboard
    \end{chessdiagram}

    Here is another example of \vocab{PoE}{Process of Elimination}{}. 
    \variation[invar]{2. Bxf6 f4} fails.

    \variation[invar]{2. g7} gives up the pawn for nothing, which is bad. 

    This leaves only \mainline[level=1]{2. Kd6 Kf8 3. Kd5} with a draw.
\end{multicols}