\newpage
\section{Bishop vs Connected Pawns}

\begin{multicols}{2}
    The following ending demonstrates the most important 
    ideas for positions involving connected pawns.

    \newchessgame[
        setfen=8/8/8/6kb/PK6/8/1P6/8 w - - 0 1,
    ]
    \begin{chessdiagram}
        \chessboard
    \end{chessdiagram}

    We examine first \mainline[level=1]{
        1. Kc5 Kf6 2. Kd6 Bd1!
    }

    \begin{chessdiagram}
        \chessboard
    \end{chessdiagram}

    \vocab{pawns-in-the-crosshairs}{Pawns in the Crosshairs}{
        attacking the enemy pawns with the bishop either to force the 
        pawns to advance, which aids in the task of their subsequent blockade, 
        or else to tie the king to their defense
    } refers to attacking the enemy pawns with the bishop either to force the 
    pawns to advance, which aids in the task of their subsequent blockade, 
    or else to tie the king to their defense.

    \mainline[level=1]{
        3. a5 Be2 4. b4 Bf1 5. Kc6 Ke7 6. b5 Bxb5
    } \textbf{$=$}.

    White can also play \vocab{auto-pilot}{Auto Pilot}{
        moving the king ahead of the connected passed pawns 
        to secure their passage to the queening square
    }, moving the king ahead of the connected passed pawns to 
    secure their passage to the queening square.

    \newchessgame[
        setfen=8/8/8/6kb/PK6/8/1P6/8 w - - 0 1,
    ]

    \mainline[level=1]{
        1. Ka5 Kf6 2. b4
    }

    \begin{chessdiagram}
        \chessboard
    \end{chessdiagram}

    Black is saved by another important technique 
    \vocab{tail-hook}{Tail Hook}{
        Tying the king to the rearmost pawn from behind
    }, tying the king to the rearmost pawn from behind.

    \mainline[level=1]{
        2... Ke5 3. b5 Kd4 4. Kb6 Bf3 5. a5 Kc4
    }

    \begin{chessdiagram}
        \chessboard
    \end{chessdiagram}

    Black ties the king to the rearmost pawn. 

    \mainline[level=1]{
        6.a6 Kb4
    }

    Black ties the king to the rearmost pawn again

    \mainline[level=1]{
        7. a7 Ba8! 8. Ka6 Kc5 9. b6 Kc6
    } \textbf{$=$}
\end{multicols}