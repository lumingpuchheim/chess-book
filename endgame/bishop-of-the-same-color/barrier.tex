\newpage
\section{Barrier}

\begin{multicols}{2}
We remember \vocab{the-first-defensive-system}{First Defensive System}{
} in opposite colored bishop endgame, using the king to block the enemy passed pawn.

Such a system can also be used in bishops of the same color endgame.
The king blockades the passed pawn while the bishop keeps the 
attacker's king from invading.


\newchessgame[
    setfen=8/6p1/8/p2kbp1p/Pp6/1P2P2P/4K1P1/2B5 w - - 0 1
]

\begin{chessdiagram}
    \chessboard
\end{chessdiagram}

White can sacrifice his bishop to setup an impassable barrier before 
Black's king. 

\mainline[level=1]{
    1. Kd3 Bc3 2. e4+ fxe4 3. Ke2
} \textbf{$=$}.

\chessboard

Black's king cannot invade because all the squares on the 
fourth rank are covered. \index{Barrier}
White only needs to move his bishop between e3 and f4. 
With \vocab{one-diagonal-principle}{One Diagonal Principle}{

} White's bishop also prevents Black from creating a passed pawn on the king side.


\end{multicols}