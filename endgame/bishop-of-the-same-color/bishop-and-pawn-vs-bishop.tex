\section{Bishop and Pawn vs. Bishop}

\begin{multicols}{2}
    \newchessgame[
        setfen=8/2K5/3P4/1B2k3/6b1/8/8/8 w - - 0 1,
    ]

    \begin{chessdiagram}
        \chessboard
    \end{chessdiagram}

    White to move wins by driving off the enemy bishop from one diagonal
    and then interfering along the other.

    \mainline[level=1]{
        1. Bd7 Bd1 2. Bh3 Ba4 3. Bg2
    } White wins after 4. \symbishop c6.

    Black to move draws

    \variation{
        1... Kd4 2. Bd7 Bd1 3. Bh3 Ba4 4. Bg2 Kc5
    } \textbf{$=$}. 
    Thus, if the weaker side's king cannot get in front of the pawn, then the 
    basic defensive principle becomes:
    \vocab{king-behind-the-king}{King Behind the King}{
        In bishops of the same color endgame a 
        defendence resource if the weaker side's 
        king cannot get in front of the pawn
    }

    \vocab{the-short-diagonal}{The Short Diagonal}{
        In bishops of the same color endgame, 
        even with the ``right'' king position, the draw 
        is impossible if one of the diagonals along 
        which the beshop will restrain the pawn proves too short
    } In bishops of the same color endgame, 
    even with the ``right'' king position, the draw 
    is impossible if one of the diagonals along 
    which the beshop will restrain the pawn proves too short.

    \newchessgame[
        setfen=2B5/K7/1P6/k7/4B3/8/8/8 w - - 0 1
    ]
    \begin{chessdiagram}
        \chessboard
    \end{chessdiagram}

    \mainline[level=1]{
        1.Bb7 Bf5 2. Bf3 Bc8 3. Be2
    } Black is in zugzwang therefore he cannot prevent White from playing
    \symbishop a6 and wins the game.
\end{multicols}