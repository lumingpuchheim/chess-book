\newpage
\section{Dvoretsky's Rule}

\begin{multicols}{2}

Positions in which two rook's pawn are facing each other,
with one side having a distant passed pawn, are fairly common in 
practice. So it is useful to have a quick and accurate way of 
evaluating them. The plan to play for a win is obvious:
the king will go after the rook's pawn. His opponent musteliminate the pawn on the other wing 
and then rush the king over to the corner where it 
can stop the rook's pawn. Under what circumstances 
can he succeed?

\newchessgame[
    setfen=8/5k2/p7/5PK1/P7/8/8/8 w - - 0 1
]

\chessboard

White to move wins:

\mainline[level=1]{1. a5! Kg7 2. Kf4 Kf6 3. Ke4 Kf7 4. Kd5 Kf6 5. Kc6 Kxf5
6. Kb6 Ke6 7. Kxa6 Kd7 8. Kb7}

In practice, such situation often occur at the end of 
long calculations, and extending such calculations a few moves further 
could be most difficult. It would be good to have 
a definite evaluation of this position immediately as soon as 
we lay eye on it. \index{Pawn Endgame!Dvoretsky's Rule}


\subsection{Dvoretsky's Rule}
\begin{enumerate}
    \item{If the rook's pawn of the stronger side has crossed 
    the middle of the board, it's always a win.}
    \item{We define a ``normal'' position, which is drawn:}
    \begin{enumerate}
        \item{the rook's pawns, which block one another, are separated by 
        the middle of the board on a-file.}
        \item{
            Black's (the defender) king aiming for the c8 square, can reach 
            it without loss of time. This is because the passed pawn 
            has either traversed the key h3-c8 diagonal or stands on it.
        }
    \end{enumerate}
    \item{For the kingside passed pawn, every square behind the h3-c8 square is 
    a reserve tempo for White. 
    For example, the pawn at f4 means one reserve tempo for White.
    If the king is not beside the passed pawn, but in front of it, that's another reserve tempo
    for White.}
    \item{Every square the queenside pawns are behind the ``normal'' position is 
    a reserve tempo for Black. With pawn at a3/a4, Black has one reserve tempo in his favor.}
    \item{White wins only if the relative number of tempi calculated shown above is in his favor.}
\end{enumerate}

\newchessgame[
    setfen=8/8/5k2/8/5K2/p4P2/P7/8 w - - 0 1
]
\chessboard

White wins because the count is 3:2. His f-pawn is two squares 
behind the h3-c8 diagonal and the king is in front of the passed pawn.
Black has two tempi because the queenside pawns are two squares 
behind the ``normal'' position.

\mainline[level]{
    1. Ke4! Ke6 2. Kd4
} \textbf{ $+-$}
\end{multicols}