\newpage
\section{f- and h-Pawn Against h-Pawn}

\begin{multicols}{2}

    \newchessgame[
        setfen=8/7p/8/8/8/5p1k/7P/6K1 w - - 0 1
    ]

    \begin{chessdiagram}[N. Grigoriev, 1920]
        \chessboard
    \end{chessdiagram}

    \variation[invar]{1. Kh1? Kg4!} Black has an elementary 
    plan: his king goes to e3 and then he advances f2, 
    forcing the advance of White's h-pawn. Black then chooses 
    whether to move his h-pawn one or two squares to place his 
    opponent in zugzwang.

    \mainline[level=1]{1. Kf2 Kg4 2. Ke3!}

    Thanks to zugzwang, the pawn must leave the h7-square. White 
    must only make sure he chooses the right back rank square for his 
    king.

    \mainline[level=1]{2... h6 3. Kf2 Kf4} 

    \begin{chessdiagram}
        \chessboard
    \end{chessdiagram}

    Here White must be careful not to play \variation[invar]{4. Kf1? Ke3 5. Ke1 h5!}.
    White is losing: moving the h-pawn leas either to Fahrni-Alapin Position or lose
    the h-pawn. Move the king will make Black's f-pawn promote. 

    The h pawns need odd number of tempi to block each other so that White can play stalemate defense.
    Therefore White must search for anti-opposition.     \index{Pawn Endgame!Fahrni-Alapin Position}

    \mainline[level=1]{4. Ke1 Ke3 5. Kf1 h5 6. Ke1 f2+ }
    
    \variation[invar]{6... h4 7. Kf1 f2 8. h3} \equal
    But not \variation[invar]{7. h3?} Black wins as Fahrni-Alapin Position. 

    \mainline[level=1]{7. Kf1 Kf3 8. h3 Kg3 9. h4} \textbf{\equal}
\end{multicols}

\newpage
\section{Maizelis' Position} \label{sec:maizelis-position}
\begin{multicols}{2}
    The stronger side has an easy win if the rook's pawn is on the starting 
    square. If it is not the case, Maizelis' position serves as a most 
    important guidepost. The outcome of the battle depends on whether the 
    stronger side cn reach Maizelis' position and whose move it is. \index{Pawn Endgame!Maizelis' Position}

    \newchessgame[
        setfen=8/8/8/5p1p/5k2/8/5K1P/8 w - - 0 1
    ]

    \begin{chessdiagram}[Maizelis, 1955]
        \chessboard
        \label{chess:maizelis}
    \end{chessdiagram}

    With White to move, there is no win:
    \mainline[level=1]{
        1. Ke2 Ke4 2. Kf2 h4 3. Ke2 h3 4. Kf2 Kd3 5. Kf3 Kd2 6. Kf2
    } \textbf{\equal}

    Black to move wins:

    \newchessgame[
        setfen=8/8/8/5p1p/5k2/8/5K1P/8 b - - 0 1,
        moveid=1b
    ]

    \begin{chessdiagram}
        \chessboard
    \end{chessdiagram}

    \mainline[level=1]{
        1... Ke4 2. Ke2 h4
    } 

    White is in zugzwang: h3 leads to Fahrni-Alapin Position. \index{Pawn Endgame!Fahrni-Alapin Position}
    As we will see, moving the king allows Black's king to invade.

    \mainline[level=1]{
        3. Kf2 Kd3! 4. Kf3 h3 5. Kf2 Kd2 6. Kf3 }
        
    (\variation[invar]{6. Kf1 Ke3 7. Ke1 Kf3 8. Kf1 f4 9. Kg1 Ke2} $-+$)
    \mainline[level=1]{6... Ke1 7. Ke3 Kf1 8. Kf3 Kg1 
        9. Kg3 f4+ 10. Kf3 Kxf2 11. Kf2 f3
    } \textbf{$-+$}

    The central problem here is one reciprocal zugzwang. If Black's
    king moves to the 4th rank, with the pawn at h4 or h6, both
    must respond by taking the opposition.
    If the pawn is at h5, both must take the anti-opposition.
\end{multicols}