\newpage
\section{f- and h-Pawn Against h-Pawn}

\begin{multicols}{2}

    \newchessgame[
        setfen=8/7p/8/8/8/5p1k/7P/6K1 w - - 0 1
    ]

    \textbf{N. Grigoriev, 1920}

    \chessboard

    \variation[invar]{1. Kh1? Kg4!} Black has an elementary 
    plan: his king goes to e3 and then he advances f2, 
    forcing the advance of White's h-pawn. Black then chooses 
    whether to move his h-pawn one or two squares to place his 
    opponent in zugzwang.

    \mainline[level=1]{1. Kf2 Kg4 2. Ke3!}

    Thanks to zugzwang, the pawn must leave the h7-square. White 
    must only make sure he chooses the right back rank square for his 
    king.

    \mainline[level=1]{2... h6 3. Kf2 Kf4} 

    \chessboard

    Here White must be careful not to play \variation[invar]{4. Kf1? Ke3 5. Ke1 h5!}.
    White is losing: moving the h-pawn leas either to Fahrni-Alapin Position or lose
    the h-pawn. Move the king will make Black's f-pawn promote. 

    The h pawns need odd number of tempi to block each other so that White can play stalemate defense.
    Therefore White must search for anti-opposition.     \index{Pawn Endgame!Fahrni-Alapin Position}

    \mainline[level=1]{4. Ke1 Ke3 5. Kf1 h5 6. Ke1 f2+ }
    
    \variation[invar]{6... h4 7. Kf1 f2 8. h3} \equal
    But not \variation[invar]{7. h3?} Black wins as Fahrni-Alapin Position. 

    \mainline[level=1]{7. Kf1 Kf3 8. h3 Kg3 9. h4} \textbf{\equal}

\end{multicols}