\section{Reti's Idea}

It sometimes happens that a king outside the 
square of a passd pawn can still catch it.
The win of the missing tempo (or even several tempi) 
is accomplished by the creation of accompanying threats, most often
(althought not exclusively) involved with supporting one's 
own passed pawn.

\begin{multicols}{2}
    \newchessgame[
        setfen=7K/8/k1P5/7p/8/8/8/8 w - - 0 1,
        moveid=1w
    ]

    % Write in bold "R. Reti, 1921"
    \textbf{R. Reti, 1921}

    \begin{chessdiagram}
        \chessboard
    \end{chessdiagram}

    Black's king lies within the square of the c6-pawn,
    while White is short two tempi to catch the h5-pawn.
    Nevertheless, he can save himself. The trick is to ``to chase
    two birds at once''. The king's advance is dual-purpose: he chases
    after the h-pawn, while simultaneously approaching the queen's wing.
    
    \index{Pawn Endgame!Reti's Idea}
    \mainline[level=1]{1. Kg7 h4 2. Kf6 Kb6 3. Ke5 Kxc6}
    
    \variation[invar]{3... h3 4. Kd6 h2 5. c7} \equal
    
    \mainline[level=1]{4. Kf4} \textbf{\equal}



    And now a slightly different version of the same idea

    \newchessgame[
        setfen=3K4/7p/3k4/P7/8/8/8/8 w - - 0 1
    ]

    \textbf{L. Prokes, 1947}

    \begin{chessdiagram}
        \chessboard
    \end{chessdiagram}

    \mainline[level=1]{1. Kc8 Kc6 2. Kb8! Kb5 3. Kb7!}

    Thanks to the threat of a6, White wins a tempo and gets into the square of the 
    h-pawn.

    \mainline[level=1]{3...Kxa5 4. Kc6} \textbf{\equal}
\end{multicols}