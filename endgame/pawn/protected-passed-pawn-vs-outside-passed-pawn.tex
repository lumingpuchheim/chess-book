\newpage
\section{Protected Passed Pawn vs Outside Passed Pawn}

\begin{multicols}{2}
The following exceptionally complex example was first given in Fine's 
book, but unfortunately with a completely erroneous analysis. 
Maizelis (1956) did a much better job on the position and 
later his conclusions were refined and extended by other authors.


\newchessgame[
    setfen=8/p6p/2k3p1/4Pp2/2K2P1P/6P1/8/8 w - - 0 1
]

\begin{chessdiagram}
    \chessboard
\end{chessdiagram}

Here we have a protected passed pawn vs. an outside passed pawn.
Since White cannot win the a-pawn, he will have to trade it 
for his e-pawn.

\mainline[level=1]{1.  Kb4 Kb6}

Black cannot close up the kingside with 1...h5, since then the trade 
of pawns will give White an easy win:
\variation[invar]{
    1... h5 2. Ka5 Kb7 3. Kb5 Kc7 4. Ka6 Kb8 5. e6 Kc7 6. Kxa7 Kd6 
    7. Kb6 Kxe6 8. Kc6 Ke7 9. Kc7 Ke6 10. Kd8 \xskakcomment{ outflanking!} Kd5 11. Ke7 Ke4 12. Kf6 Kf3
    13. Kxg6 Kxg3 14. Kg5
}

\mainline[level=1]{2. Ka4 a5 3. h5!}

\begin{chessdiagram}
    \chessboard
\end{chessdiagram}

Before exchanging the pawns, it is necessary first to weaken the enemy kingside
pawn chain.

\mainline[level=1]{3... gxh5 4. e6! Kc6 5. Kxa5 Kd6 6. Kb6 Kxe6}

\begin{chessdiagram}
    \chessboard
\end{chessdiagram}

f5 pawn will fall and Black will play at some point ...h4. 
After the exchange we will have Maizelis' Position. \index{Pawn Endgame!Maizelis' Position}
White's king's position decides the result, as we know from Maizelis' Position.

\mainline[level=1]{7. Kc6!}

\variation[invar]{7. Kc5? Kd7 8. Kd5 Ke7 9. Ke5 Kf7 10. Kxf5 h4 11. gxh4 Ke7 
12. Ke5 Kf7 13. h5 Ke7} \textbf{\equal} Black achieves opposition when 
White's pawn is on h5.

Notice we didn't calculate until the end but only use the knowledge of Maizelis' Position
to decide the move.

We can also use \vocab{principle-of-choices}{Principle of Choices}{} to decide the move. \variation{7. Kc5} gives Black a choice 
between anti-opposition or opposition (one of them leads to draw according 
to the theory of Maizelis' Position). If White thinks he is winning, why 
giving Black such an opportunity? \index{Principle of Choices}

However if White believes the position is draw, then he should play \variation{7. Kc5}
to allow the opponent to make a mistake.

\mainline[level=1]{7... Kf7}

\variation[invar]{7... Ke7 8. Kd5 Kf6 9. Kd6 Kf7 10. Ke5 Kg6 11. Ke6 Kg7 
12. Kxf5 h4 13. gxh4 Kf7 14. Ke5 Ke7 15. h5} \textbf{$+-$} We know this winning already in diagram \ref{chess:maizelis}.

\begin{chessdiagram}
    \chessboard
\end{chessdiagram}

\variation[invar]{8. Kd5 Ke7 9. Ke5 Kf7 10. Kxf5 h4 11. gxh4 Ke7} Draw because 
Black achieves anti-opposition when White's pawn is on h4.

\mainline[level=1]{8. Kd7! Kf8 9. Kd6!}

\variation[invar]{9. Ke6} is because it gives Black a choice.

\mainline[level=1]{9... Kg7 10. Ke7 Kg8 11. Ke6 Kf8 12. Kf6}

\begin{chessdiagram}
    \chessboard
\end{chessdiagram}

Black is in zugzwang.

\mainline[level=1]{12...Kg8 13. Kxf5 h4 14. gxh4 f7 15. Ke5 Ke7 16. h5}

\begin{chessdiagram}
    \chessboard
\end{chessdiagram} 

White is winning and we have seen it already in \ref{sec:maizelis-position}.


\end{multicols}