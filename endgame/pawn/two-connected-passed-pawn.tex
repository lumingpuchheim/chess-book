\newpage
\section{Two Connected Passed Pawn}
\begin{multicols}{2}
In practice we have often a typical situation with
two connected passed pawns against one protected 
passed pawn. Draw is often the result since the stronger 
side's king is tied to the square of the protected passed pawn.

Sometimes, the stronger side's king can leave the square to help his 
pawns to queen or checkmate his opponent.

The stronger side's plan usually consists of the following
\begin{enumerate}
    \item The furthest possible advance of the pawns
    \item The optimum placement of the pawns - ready to roll
    \item Choose the best time for the king's decisive advance
\end{enumerate}

\newchessgame[
    setfen=8/8/8/1p6/1Pp5/2K3Pk/7P/8 w - - 0 1
]

\textbf{B. Horwitz, J. Kling, 1851}

\begin{chessdiagram}
    \chessboard
\end{chessdiagram}

Let's watch this plan in action. 

\mainline[level=1]{1. Kd4 Kg4 2. h4 Kh5 3. Ke3 Kg4 4. Ke4 Kh5 5. Kf4 Kh6
6. g4 Kg6 7. h5+ Kh6}

\begin{chessdiagram}
    \chessboard
\end{chessdiagram}

Stage 1 done. White's king has never left the square. In order to 
advance his pawns further, he uses a typical technique: \vocab{triangulation}{triangulation}! \index{Triangulation}

\mainline[level=1]{
    8. Kf3 Kg5 9. Ke4 Kh6 10. Kf4
}

\begin{chessdiagram}
    \chessboard
\end{chessdiagram}

Now Black's king must concede ground. As a result White's pawns can advance further.

\mainline[level=1]{
    10... Kh7 11. g5 Kg7 12. g6! Kf6
}

\begin{chessdiagram}
    \chessboard
\end{chessdiagram}

Ready to roll! \variation[invar]{12. h6?} is a mistake because the pawns 
cannot advance further.

White executes another triangulation for the decisive king advance.

\mainline[level=1]{
13. Ke4 Kg7 14. Kf3 Kf6 15. Kf4 
}

\begin{chessdiagram}
    \chessboard
\end{chessdiagram}

After another triangulation, Black must concede ground again.

\mainline[level=1]{
15... Kg7}

\begin{chessdiagram}
    \chessboard
\end{chessdiagram}

Stage 2 done. Now White has strengthened his position to its utmost, 
it is time for the decisive king advance. The die will be cast. White's king will leave 
the square. It is important to calculate the consequence of this move.

\mainline{16. Kg5! c3 17. h6+ Kg8 18. Kf6 c2 19. h7+ Kh8 20. g7+ Kxh7 21. Kf7 c1=Q 22. g8=Q+ Kh6 23. Qg6#
}
\end{multicols}