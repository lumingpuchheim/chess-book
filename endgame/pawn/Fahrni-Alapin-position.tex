\newpage
\section{Fahrni-Alapin Position}

\begin{multicols}{2}
    \newchessgame[
        setfen=2k5/8/p1P5/P2K4/8/8/8/8 w - - 0 1
    ]

    \textbf{Fahrni-Alapin, 1912}

    \begin{chessdiagram}
        \chessboard
    \end{chessdiagram}

    Two pairs of squares of reciprocal zugzwang are 
    obvious: d6-d8 and c5-c7. The squares d6 and c5 border on d5.

    For Black, the squares d8 and c7 border on c8. Thus a standard
    means of identifying a new correspondece: that of the d5 and 
    c8-square.

    Along with d5 and c5, White has two equally important squares
    c4 and d4; while Black has b8 and d8 adjoining the squares c7 and 
    c8.

    With Black's king on d8, White makes a waiting move with his king,
    from c4 to d4, Black's king will be forced to c7 or c8, 
    when White occupies the corresponding square and wins. 
    \index{Triangulation}
    \index{Pawn Endgame!Corresponding Squares}
    \index{Pawn Endgame!Fahrni-Alapin Position}

    \mainline[level=1]{
        1. Kc4 Kd8 2. Kd4 Kc8 3. Kd5 Kd8 4. Kd6 Kc8 5. c7
    } \textbf{1-0}
\end{multicols}