%%%%%%%%%%%%%%%%%%%%%%%%%%%%%%%%%%%%%%%%%
% The Legrand Orange Book
% LaTeX Template
% Version 3.1 (February 18, 2022)
%
% This template originates from:
% https://www.LaTeXTemplates.com
%
% Authors:
% Vel (vel@latextemplates.com)
% Mathias Legrand (legrand.mathias@gmail.com)
%
% License:
% CC BY-NC-SA 4.0 (https://creativecommons.org/licenses/by-nc-sa/4.0/)
%
%%%%%%%%%%%%%%%%%%%%%%%%%%%%%%%%%%%%%%%%%

%----------------------------------------------------------------------------------------
%	PACKAGES AND OTHER DOCUMENT CONFIGURATIONS
%----------------------------------------------------------------------------------------

\documentclass[
	11pt, % Default font size, select one of 10pt, 11pt or 12pt
	fleqn, % Left align equations
	a4paper, % Paper size, use either 'a4paper' for A4 size or 'letterpaper' for US letter size
	oneside, % Oneside mode, more suitable if the book is to be read on a screen or print-on-demand
]{memoir}

%----------------------------------------------------------------------------------------
%	CORE PACKAGES (replacing what the old class loaded)
%----------------------------------------------------------------------------------------

\usepackage[utf8]{inputenc}
\usepackage[T1]{fontenc}

\usepackage{graphicx}
\usepackage[usenames, svgnames, table]{xcolor}

\usepackage{lmodern} % Latin Modern = scalable Computer Modern
\usepackage{microtype}

\usepackage{geometry}
\geometry{
	top=3cm,
	bottom=2.5cm,
	inner=1.5cm,
	outer=1.5cm,
	headsep=10pt,
	headheight=14pt,
	footskip=1.4cm,
	columnsep=1cm
}

\usepackage{hyperref}
\usepackage[nonumberlist]{glossaries}
\makeglossaries

% Header/footer styling similar to the old Legrand template
\usepackage{fancyhdr}
\pagestyle{fancy}
\renewcommand{\headrulewidth}{0.5pt}
\fancyhf{}
\fancyhead[LE,RO]{\thepage}
\fancyhead[LO]{\rightmark}
\fancyhead[RE]{\leftmark}
\fancypagestyle{plain}{%
  \fancyhead{}%
  \renewcommand{\headrulewidth}{0pt}%
}

% Section and TOC formatting (you were already using these)
\usepackage[explicit]{titlesec}
\usepackage{tocloft}

% Bibliography and index setup (mirroring the old class)
\usepackage[
	backend=biber,
	bibstyle=numeric,
	citestyle=numeric,
	sorting=nyt,
	sortcites=true,
	abbreviate=false,
	backref=true
]{biblatex}
\defbibheading{bibempty}{}

\usepackage{makeidx}
\makeindex
\usepackage{etoolbox}

%----------------------------------------------------------------------------------------
%	BUILD FLAGS
%----------------------------------------------------------------------------------------
% Control whether to build all heavy chapters (ideas, games, annotations).
% Default: build everything. A wrapper file can define \nobuildall before
% loading this file to disable them.
\newif\ifbuildall
\buildalltrue
\ifdefined\nobuildall
  \buildallfalse
\fi

% PDF metadata and link appearance
\hypersetup{
	pdftitle={My Chess Notebook}, % Title field
	pdfauthor={Ming Lu}, % Author field
	pdfsubject={Chess}, % Subject field
	pdfkeywords={chess, ideas, games, annotations}, % Keywords
	pdfcreator={LaTeX}, % Content creator field
	% Make links colored instead of framed rectangles
	colorlinks=true,
	linkcolor=ocre,
	citecolor=ocre,
	urlcolor=ocre
}

\addbibresource{../sample.bib} % Bibliography file

\definecolor{ocre}{RGB}{243, 102, 25} % Define the color used for highlighting throughout the book

% These commands came from the old Legrand class; provide no-op versions so
% existing content still compiles under memoir. You can later redefine them
% if you want fancy chapter images again.
\newcommand{\chapterimage}[1]{} % Chapter heading image (ignored for now)
\newcommand{\chapterspaceabove}[1]{} % Whitespace above chapter title box
\newcommand{\chapterspacebelow}[1]{} % Whitespace below chapter title box

\chapterimage{../Images/zen-stone.jpg}
\chapterspaceabove{6.5cm}
\chapterspacebelow{6.75cm}

\usepackage{chessboard}
\usepackage{xskak}
\usepackage{sectsty}
\usepackage{tocloft}
\usepackage{epigraph}
\usepackage{multicol}
\usepackage{tikz}

% Define \titlepage command to match Legrand template style
% Usage: \titlepage{background image code}{title text}{author text}
\newcommand{\titlepage}[3]{%
	\thispagestyle{empty}%
	\newgeometry{left=0cm,right=0cm,top=0cm,bottom=0cm}%
	\begin{tikzpicture}[remember picture,overlay]%
		\node[anchor=center,inner sep=0] at (current page.center) {#1};%
		\node[anchor=center,align=center] at ([yshift=3cm]current page.center) {\rmfamily\fontsize{50}{60}\selectfont\bfseries #2};%
		\node[anchor=center,align=center] at ([yshift=-2cm]current page.center) {\rmfamily\fontsize{36}{44}\selectfont\bfseries #3};%
	\end{tikzpicture}%
	\restoregeometry%
	\clearpage%
}

%----------------------------------------------------------------------------------------

\begin{document}

%----------------------------------------------------------------------------------------
%	TITLE PAGE (memoir style)
%----------------------------------------------------------------------------------------

\titlepage % Output the title page
	{\includegraphics[width=\paperwidth,height=\paperheight,keepaspectratio=false]{../Images/Delicate chessboard with a soft accent.png}} % Code to output the background image, which should be the same dimensions as the paper to fill the page entirely; leave empty for no background image
	{My Chess Notebook} % Book title
	{Ming Lu} % Author name

%----------------------------------------------------------------------------------------
%	COPYRIGHT PAGE
%----------------------------------------------------------------------------------------

\thispagestyle{empty} % Suppress headers and footers on this page

~\vfill % Push the text down to the bottom of the page

\noindent Copyright \copyright\ 2025 Ming Lu\\ % Copyright notice

%----------------------------------------------------------------------------------------
%	TABLE OF CONTENTS
%----------------------------------------------------------------------------------------
\newpage
\pagestyle{empty} % Disable headers and footers for the following pages
\newcommand{\Color}[1]{\hypersetup{linkcolor=#1}\color{#1}}

\titlespacing*{\part}{10em}{0em}{0em}
% Optionally change font style
\renewcommand{\cftpartfont}{\Large\bfseries\Color{ocre}} 
\renewcommand{\cftchapfont}{\Large\bfseries\Color{ocre}} 
\renewcommand{\cftsecfont}{\large}  % Bold section font
\renewcommand{\cftsubsecfont}{\large\itshape} % Italic subsection font
\renewcommand{\cftsecpagefont}{\large}   % Page numbers for sections
\renewcommand{\cftsubsecpagefont}{\large} % Page numbers for subsections

\begingroup % start a TeX group
\color{ocre}% or whatever color you wish to use
\tableofcontents* % Output the table of contents (no self-entry)
\endgroup


\pagestyle{fancy} % Enable default headers and footers again

\cleardoublepage % Start the following content on a new page

%----------------------------------------------------------------------------------------
%	HEADING COLORS (chapters/sections/subsections in text)
%----------------------------------------------------------------------------------------

% Chapter heading: ocre, same font size as default
\titleformat{\chapter}
  {\normalfont\huge\bfseries\color{ocre}} % format
  {\thechapter}                            % label
  {1em}                                    % sep
  {#1}                                     % before-code (title text)

% Unnumbered chapters (\chapter*)
\titleformat{name=\chapter,numberless}
  {\normalfont\huge\bfseries\color{ocre}}
  {}
  {0pt}
  {#1}

% Section heading
\titleformat{\section}
  {\normalfont\Large\bfseries\color{ocre}}
  {\thesection}
  {1em}
  {#1}

% Unnumbered sections (\section*)
\titleformat{name=\section,numberless}
  {\normalfont\Large\bfseries\color{ocre}}
  {}
  {0pt}
  {#1}

% Subsection heading
\titleformat{\subsection}
  {\normalfont\large\bfseries\color{ocre}}
  {\thesubsection}
  {1em}
  {#1}

% Unnumbered subsections (\subsection*)
\titleformat{name=\subsection,numberless}
  {\normalfont\large\bfseries\color{ocre}}
  {}
  {0pt}
  {#1}

% Define a macro for keywords, the keywords are displayed in italic
\newcommand{\keywords}[1]{Keywords: \textit{#1}}

% Vocabulary/glossary helper:
%   #1 = key (no spaces, e.g. hanging-pawns)
%   #2 = word as printed in the text (e.g. hanging pawns)
%   #3 = meaning (definition)
\newcommand{\vocab}[3]{%
  \ifglsentryexists{#1}{%
    \gls{#1}% use existing entry
  }{%
    \newglossaryentry{#1}{name={#2},description={#3}}%
    \gls{#1}% first use also prints as a glossary link
  }%
}

% Include chess tools
% Macro to display chess game information in a rectangular box
% Usage: \chessgameinfo{event}{white}{black}{round}{date}{result}
% Optional parameters can be left empty: \chessgameinfo{Event}{White}{Black}{}{}{}
% Example: \chessgameinfo{FIDE World Championship 2023}{Ding, Liren}{Nepomniachtchi, Ian}{6}{2023.04.15}{1-0}
\newcommand{\chessgameinfo}[6]{%
    \vspace{0.5em}
    \begin{center}
    \begin{tikzpicture}
        \node[rectangle, 
              draw=ocre, 
              fill=ocre!5, 
              line width=1pt,
              rounded corners=5pt,
              inner xsep=12pt,
              inner ysep=10pt,
              text width=0.45\textwidth,
              align=left] (box) {
            \textbf{\textcolor{ocre}{Game Information}}\\[0.8em]
            \begin{tabular}{@{}l@{\hspace{0.5cm}}l@{}}
                \textbf{Event:} & #1 \\
                \ifstrempty{#4}{}{\textbf{Round:} & #4 \\}
                \ifstrempty{#5}{}{\textbf{Date:} & #5 \\}
                \textbf{White:} & #2 \\
                \textbf{Black:} & #3 \\
                \ifstrempty{#6}{}{\textbf{Result:} & #6 \\}
            \end{tabular}
        };
    \end{tikzpicture}
    \end{center}
    \vspace{0.5em}%
}

%----------------------------------------------------------------------------------------
%	Main Content
%----------------------------------------------------------------------------------------
\chapter*{Preface}
\addcontentsline{toc}{chapter}{Preface}
Chess is full of emotion. The veteran grandmaster Vassyl Ivanchuk, after defending brilliantly for 40 moves and emerging with a winning position, made one automatic move and watched victory slip through his fingers. He slumped over his board, tears streaming down his face—the kind of grief that comes from decades of passion, frustration, and desperate love for this impossible game. This is chess: not just calculation and strategy, but human experience compressed into 64 squares.

Yet when we look at the chess literature available today, we find a curious gap. There are few books that focus on the deeper lessons we can draw from chess—lessons about thinking, decision-making, and navigating complexity in real life. Most chess books fall into two categories: they are either full of computer-generated variations that show you what an engine would play, or they focus obsessively on opening theory, memorizing lines that may never appear in your games.

Neither of these approaches is particularly useful for real life. Computer analysis tells you what is optimal, but not how to think when you do not have a computer. Opening theory teaches you to memorize, but not to understand. What we need are the principles, patterns, and mental frameworks that chess masters use—the same tools that help them make better decisions not just on the board, but in life itself.

That is the reason I am writing this book. Here you will find ideas from chess that translate directly to life: how to use Process of Elimination to make better decisions, how pattern recognition shapes expertise, how simplicity can triumph over complexity under pressure, and how the same thinking methods that help grandmasters find good moves can help you navigate your own challenges. These are not abstract theories, but practical tools drawn from real games and real situations.

\vspace{2em}
\begin{flushright}
Ming Lu\\
\emph{Aegina, Greece}\\
\emph{December 19, 2025}
\end{flushright}

\chapter*{Introduction}
\addcontentsline{toc}{chapter}{Introduction}
\section{Explanation of Symbols}

The chessboard with its coordinates:

\chessboard[
    setfen=rnbqkbnr/pppppppp/8/8/8/8/PPPPPPPP/RNBQKBNR w KQkq - 0 1,
    showmover=false,
]

\vspace{1em}

\begin{tabular}{@{}ll@{}}
    \symking & King \\
    \symqueen & Queen \\
    \symrook & Rook \\
    \symbishop & Bishop \\
    \symknight & Knight \\
    $\pm$ & White stands slightly better \\
    $\mp$ & Black stands slightly better \\
    $\pm$ & White stands better \\
    $\mp$ & Black stands better \\
    $+-$ & White has a decisive advantage \\
    $-+$ & Black has a decisive advantage \\
    $=$ & balanced position \\
    $!$ & good move \\
    $!!$ & excellent move \\
    $?$ & bad move \\
    $??$ & blunder \\
    $!?$ & interesting move \\
    $?!$ & dubious move \\
\end{tabular}


\section*{Visualization the Three Questions}
We use a simple graphical system to illustrate the three questions as used by Aagaard in his book ``Grandmaster Preparations Positional Play''. 

We will use circle to identify weakness. We will use square to identify the worst-placed pieces. We will use arrows to illustrate the opponent's ideas.


Let us look at an example 

\subsection*{Hikaru Nakamura - Vladimir Kramnik}

 \chessboard[
        setfen=3r2k1/p4pbp/b3p3/npqpP1pP/8/2P3P1/P3QPBN/R3R1K1 w - - 0 1,
 	markstyle=circle,
 	linewidth=0.05em,
 	markfields={g5, f6},
	markstyle=border,
	markfields={g2},
	pgfstyle=straightmove,
 	markmove=h7-h6
        ]

Black has weakness on g5 and f6. All his pieces are bad. White's worst piece is the bishop on g2.

Black intends to play h6 to protect his biggest weakness: g5 pawn. To prevent this move White can play h6 himself and then \symqueen h5 attack the weak pawn on g5.




\section*{Process of Elimination}
\section{Process of Elimination}
\vocab{PoE}{Process of Elimination}{Ruling out bad moves, not immediately finding the perfect one. Commonly used in calculation, defense, and endgames}
 is the method instead of searching for a brilliant move directly, 
you narrow the field\cite{Dvoretsky:2015}.

How it works in practice:

\begin{itemize}
\item List candidate moves that are natural or forcing.
\item Eliminate bad moves by concrete reasons.
\item Compare the remaining moves and choose the one that best fits the position.
\end{itemize}
\index{Process of Elimination} 


%----------------------------------------------------------------------------------------
%	GLOSSARY
%----------------------------------------------------------------------------------------
\cleardoublepage
\phantomsection
\addcontentsline{toc}{chapter}{\textcolor{ocre}{Glossary}}
\printglossary[title={Glossary}]

\chapter{Chess Ideas and Patterns}

\ifbuildall
	\epigraph{
    Magnus Carlsen's exceptional memory and pattern recognition skills are central to his chess mastery. In a remarkable demonstration, he accurately recreated a 26-piece position after just a two-second glance at the board. This position was from Bobby Fischer's 1956 game against Donald Byrne, and Carlsen not only reconstructed it but also recalled the subsequent moves with near-perfect accuracy.
}{}

When Magnus Carlsen plays chess, he searches for similar positions in his brain so that he doesn't need to calculate too much. This ability is rooted in the cognitive process known as ``chunking,'' where individuals group information into meaningful clusters to enhance memory retention and recall. In chess, chunking allows players to perceive complex positions as manageable patterns, facilitating quick decision-making. 

Understanding and utilizing chunking is crucial in chess. By recognizing and recalling familiar patterns, players can reduce the need for extensive calculation, leading to more efficient and effective play.

Chess is about finding chunks that are familiar to you. It is therefore practical to collect these chunks for future use. In this part, we are going to list some ideas and hopefully you can use them in your future games.

% Opening ideas
% Benoni stuff
%\newpage
\section{The Modern Benoni Pawn Structure}

The Benoni Defense is a rare choice in high-level classical chess because it gives Black a long-term disadvantage. With accurate play, White can gradually find the best moves to secure a lasting advantage.

The Benoni structure typically arises from the Benoni Defense but can also emerge from the English Opening (reversed) or from the Ruy Lopez and Italian Game when Black concedes the center with ...exd4 and ...c5, and White responds with d5.

This structure is far from rare—even in World Chess Championship games, reversed Benoni structures have appeared. Understanding its key tactical and strategic ideas can be a valuable asset, even at the highest levels.

\newchessgame[
id=A,
moveid=1w,
setwhite={pa2, pb2, pd5, pe4, pf2, pg2, ph2},
addblack={pa7, pb7, pc5, pd6, pf7, pg6, ph7}]
\chessboard

The exchange of White's c-pawn for Black's e-pawn leaves White with a pawn majority in the centre and Black with one on the queenside. This asymmetry suggests that White will try to play on the kingside and in the centre, while Black will seek counterplay on the queenside. However, this simplistic generalization does not hold in many cases—depending on how the pieces are arranged, either side may be able to fight back on the flank where they are theoretically weaker.


\subsection*{White Plans}
\begin{itemize}
    \item{Kingside play}

    The central pawn majority is White's main positional trump in the Modern Benoni. By staking out an advantage in space on the kingside, it allows White to develop an initiative on that side of the board. The most important tool in White's arsenal is the e4-e5 pawn advance, which can open up lines and squares for the white pieces, and result in the creation of a passed d-pawn if Black answers with ...dxe5.
    \item{Queenside play}

    When Black prepares the ...b7-b5 pawn break with ...a6, White usually tries to hinder it by playing a2-a4, even though this advance weakens the b4-square. As a further deterrent to Black's queenside expansion, White often moves the knight on f3 to c4 via d2. With the knight on c4, Black's ...b7-b5 break may be met by axb5 followed by \symknight{}a5, when the arrival of a white knight on c6 could severely disrupt Black's position. The knight on c4 also attacks Black's backward pawn on d6, and White can often increase the pressure on this pawn by playing Bf4 or Nb5. The strength of White's knight on c4 often induces Black to exchange it off: typical ways of doing so are ...\symknight{}b6, ...\symknight{}e5, or ...b7-b6 followed by ...\symbishop{}a6.

    \item{Positional Play}
    Black has a weakness on d6. Bringing the knight from d2 to c4 is a common way to accumulate advantage.

\end{itemize}    

\subsection*{Black's Key Ideas}
\begin{itemize}
    \item{Queenside play}

    Queenside pawn advance supported by the g7 bishop to create a passed pawn. 

    \item{c5-c4}

    Push the pawn to make room for the d7 knight to create tactical chances. The d7 knight 
    moves to d3 via c5.
\end{itemize} 
\newpage
\section{e5 for White in the Modern Benoni Pawn Structure}
\begin{multicols}{2}
\chessgameinfo{Leipzig Olympiad 1960}{J.Penrose}{M.Tal}{11}{1960.11.18}{1-0}
At the time, Tal was unstoppable. He was about to win the World Championship in 1960,
while Penrose was a strong but less renowned player. However, Penrose crushed Tal in this game.

\newchessgame

\mainline[level=1]{1. d4 Nf6 2. c4 e6 3. Nc3 c5 4. d5 exd5 5. cxd5 d6 6. e4 g6 7. Bd3 Bg7 8. Nge2 O-O 9. O-O a6}

\chessboard

We have arrived at a typical position in the Modern Benoni. White 
must play a prophylaxis a4 to stop Black from playing ...b5. Then he 
can setup an e4-f4 center to control the e5 square.  \index{Modern Benoni!e5 Thrust}

It is hard to find a good plan for Black because the opening is dubious.
In the game, Black tries ...c4, ...\symknight c5, attempting ...\symknight d3. However, White's kingside 
attack comes much faster.

\mainline[level=1]{10. a4 Qc7 11. h3 Nbd7 12. f4 Re8 13. Ng3 c4 14. Bc2 Nc5 15. Qf3 Nfd7 16. Be3 b5 17. axb5 Rb8 18. Qf2 axb5 }

\chessboard[
    markstyle=circle,
    linewidth=0.05em,
    markfields={f7},
]

f7 is a clear weakness for Black. White plans to open the f-file and then break through.
The move 19. e5 is necessary because otherwise Black can play ...\symknight e5, gaining some counterplay.

\mainline[level=1]{19. e5 dxe5 20. f5 Bb7 21. Rad1 Ba8 22. Nce4 Na4 23. Bxa4 bxa4 24. fxg6 fxg6 }

\chessboard

White has a crushing position with a powerful attack on the kingside.

\mainline[level=1]{25. Qf7+ Kh8 26. Nc5 Qa7 27. Qxd7 Qxd7 28. Nxd7 Rxb2 29. Nb6 Rb3 30. Nxc4 Rd8 31. d6 Rc3 32. Rc1 Rxc1 33. Rxc1 Bd5 34. Nb6 Bb3 35. Ne4 h6 36. d7 Bf8 37. Rc8 Be7 38. Bc5 Bh4 39. g3} 1-0

\end{multicols}

\newpage
\section{Kingside Piece Attack for White in the Modern Benoni Pawn Structure}

\begin{multicols}{2}
In the next game, we show a Modern Benoni pawn structure coming 
from the Zaitsev variation of the Ruy Lopez. \index{Modern Benoni!Kingside Piece Attack}

\chessgameinfo{FIDE World Championship}{V.Anand}{M.Adams}{5}{2005.9.30}{1-0}

\newchessgame
\mainline[level=1]{
1. e4 e5 2. Nf3 Nc6 3. Bb5 a6 4. Ba4 Nf6 5. O-O Be7 6. Re1 b5 7. Bb3 d6 8. c3 O-O 9. h3
Bb7 10. d4 Re8 11. Nbd2 Bf8 12. a4 h6 13. Bc2 exd4 14. cxd4 Nb4 15. Bb1 c5 16. d5}

\chessboard

Here we have a Modern Benoni pawn structure coming from Ruy Lopez. The position 
looks intimidating for Black because all the White pieces are targeting the Black king 
(White can move the a1 rook to g3 via a3). Black tries to stop the attack with 
...c4, \symknight d7-c5-d3, blocking the important b1-h7 diagonal.

\mainline[level=1]{16...Nd7 17. Ra3 c4}

\chessboard

\mainline[level=1]{18. axb5?!}

This move is not good because it gives Black an additional resource: exchanging the rook on a3.
The Black rook can also attack on a1 when the White rook moves to g3.
\variation[invar]{18. Nd4} is better.

\mainline[level=1]{18...axb5 19. Nd4 Qb6 20. Nf5 Ne5 21. Rg3 g6 22. Nf3 Ned3}

\chessboard

White is starting a decisive attack while Black has deserted White's queenside and blocked White's light-squared bishop.

\mainline[level=1]{23. Qd2}

\chessboard

\mainline[level=1]{23...Bxd5?}

The most dangerous piece for Black is the knight on f3, which can move to h4 and sacrifice on g6.
Black must play \symknight xe1,
for example: \variation[invar]{23...Nxe1 24. Nxe1 Ra1 25. Nxh6+ Bxh6 26. Qxh6 Nxd5 27. Be3 Nxe3 28. Rxg6+ fxg6 29. Qxg6+ \xskakcomment{ perpetual check}};
\variation[invar]{23...Nxe1 24. Nxh6+ Bxh6 25. Qxh6 Nxf3+ 26. gxf3 Nd3 27. Be3 Qxe3 28. fxe3 Ra1 29. Kg2 Rxb1 \xskakcomment{ the position is equal}}

\mainline[level=1]{24. Nxh6+ Bxh6 25. Qxh6 Qxf2+ 26. Kh2 Nxe1}

\chessboard

\mainline[level=1]{27. Nh4! Ned3 28. Nxg6 Qxg3+ 29. Kxg3 fxg6 30. Qxg6+ Kf8 31. Qf6+ Kg8 32. Bh6} 1-0


\end{multicols}

% Open Sicilian stuff
\newpage
\section{Open Sicilian}

The Open Sicilian is a dynamic and aggressive chess opening that arises after the:

\newgame
\newchessgame[
id=A,
moveid=1w,
]

\mainline{1. e4 c5 2. Nf3 d6 3. d4 cxd4 4. Nxd4}

\chessboard

This opening is one of the most popular and complex responses to 1. e4, leading to sharp, tactical battles where both sides have numerous opportunities for creative play. By choosing the Open Sicilian, White aims to seize central space and accelerate piece development, while Black counterattacks using active piece play and pawn breaks such as ...d5 or ...b5.

White has a lead in development and extra kingside space, which White can use to begin a kingside attack. This is counterbalanced by Black's central pawn majority, created by the trade of White's d-pawn for Black's c-pawn, and the open c-file, which Black uses to generate queenside counterplay and even a queenside attack if White decides to castle there.
\newpage
\section{Najdorf Sicilian: White Sacrifice on d5}

\nocite{franco:2015}
When the Black king remains in the center, White often has the tactical option of 
 sacrificing on d5. If Black accepts the sacrifice, White opens the e-file with 
 exd5, gaining time and lines against the uncastled king and 
 obtaining a dangerous attack. Black must be very careful not to allow such a sacrifice 
 under favorable circumstances for White.

\begin{multicols}{2}
\chessgameinfo{USSR Chess Championship}{B.Spassky}{L.Polugaevsky}{}{1958}{}
\newchessgame
\mainline[level=1]{
    1. e4 c5 2. Nf3 d6 3. d4 cxd4 4. Nxd4 Nf6 5. Nc3 a6 6. Bg5 Nbd7 
    7. Bc4 Qa5 8. Qd2 e6 9. O-O-O b5 
}

\chessboard 

White can try the sacrifice and obtain an advantage if Black accepts: \variation[invar]{10. Bd5?! exd5?} 

\chessboard[
    setfen=r1b1kb1r/3n1ppp/p2p1n2/qp1p2B1/3NP3/2N5/PPPQ1PPP/2KR3R w kq - 0 11
]

After \variation[invar]{11. Nc6 Qb6 12. exd5} followed by \variation[invar]{13.Rhe1}, White 
 has a clear advantage, with the e-file opened and Black's king exposed in the center.

However, after the more precise \variation[invar]{10... b4 11. Bxa8 bxc3 12. bxc3 Qa2}, 
 Black has strong counterplay in the following position. \index{Najdorf Sicilian!White Sacrifice on d5}

\chessboard[
    setfen=B1b1kb1r/3n1ppp/p2ppn2/6B1/3NP3/2P5/q1PQ1PPP/2KR3R w k - 0 13
]

We now return to the main line.
\mainline[level=1]{10. Bb3}


 First, let's look at the natural but mistaken pawn thrust \variation[invar]{10... b4?}.

\chessboard[
    setfen=r1b1kb1r/3n1ppp/p2ppn2/q5B1/1p1NP3/1BN5/PPPQ1PPP/2KR3R w kq - 0 11
]

 Now White can play \variation[]{11. Nd5 exd5 12. exd5 Qb5 13. Rhe1}, obtaining 
  a strong, long-term attack along the e-file and against the black king.

 We now return to the actual game: \mainline[level=1]{10... Bb7 11. Rhe1 Be7 12. f4 Nc5}.

 Once again, if Black plays too ambitiously with \variation[invar]{12... b4 13. Nd5! exd5 14. exd5 Kf8} follows, and the king is driven into danger.

\chessboard[
    setfen=r4k1r/1b1nbppp/p2p1n2/q2P2B1/1p1N1P2/1B6/PPPQ2PP/2KRR3 w - - 1 15
]

 After the brilliant sequence 
 \variation[]{15. Rxe7! Kxe7 16. Nc6+ Bxc6 17. dxc6 Qc5 18. Qe2+ Kd8 19. cxd7 Kxd7 20. Bxf7}, 
  White obtains a powerful and enduring attack.

 The full game also features a later sacrifice on e6, which we will examine 
  in detail in a later section.
\end{multicols}
\newpage
\section{Najdorf Sicilian: White Sacrifice on e6}

Quite often Black protects his e6 pawn with f7 pawn only, 
white White targets the pawn with a light-squared bishop and a 
knight on d4. White can sacrifice on e6 when he has enough compensation,
for example, the Black king must stay in the center for a long time.

\begin{multicols}{2}

\chessgameinfo{Bled-Zagreb-Belgrade Candidates}{Mikhail Tal}{Tigran Vartanovich Petrosian}{}{1959.10.25}{1/2-1/2}
\newchessgame
\mainline[level=1]{
    1. e4 c5 2. Nf3 d6 3. d4 cxd4 4. Nxd4 Nf6 5. Nc3 Nbd7 6. Bc4 a6 7. Bg5 Qa5?! 8. Qd2 e6 9. O-O h6 10. Bh4 g5 11. Bg3 Nh5 }
    
\chessboard

Here White has two attackers towards the e6 pawn while Black has only one. So 
a sacrifice is possible. Which piece should White sacrifice? Sacrificing a 
knight would be incorrect since White has not enough open lines for the bishop.
On the other hand, sacrificing a bishop allows him to lock the Black king in 
the center.

    \mainline[level=1]{12. Bxe6 fxe6 13. Nxe6 }
    
    \chessboard
    
    \mainline[level=1]{13...Nxg3?}
    
    It is understandable that Black want to exchange pieces to 
    defend. After all, White's bishop is targeting the d6 pawn.

    It is however difficult for a human being to find the right defence:
    \variation[invar]{13...Ndf6 14. Nxf8 Kxf8 15. Bxd6+ Kg7 16. e5 Nh7 17. b4 Qd8 18. Nd5 Be6 19. c4 Re8 20. Rac1 }

    After the text move, Black's king is still in the center and he cannot develop his queenside pieces.
    
    \mainline[level=1]{14. fxg3 Ne5 15. Rxf8+ Rxf8 16. Qxd6 Rf6}
    
    \chessboard

    \mainline[level=1]{17. Nc7+?}
    
    Losing all the advantages! \variation[invar]{17. Qc7 Qxc7 18. Nxc7+ Kd8 19. Nxa8} is better.
    After the queens are exchanged, White's material advantage is palpable. \index{Najdorf Sicilian!White Sacrifice on e6}
    
    \mainline[level=1]{17...Kf7 18. Rf1 Rxf1+ 19. Kxf1 Nc4 20. Qxh6 Qc5? }

    After \variation[invar]{20... Qe5}, the position is equal.
    White cannot check on h8 to get advantage as we will see because it is covered by the Black queen.
    
    \chessboard 

    \mainline[level=1]{21. Nxa8? }

    Allows Black to draw with perpetual check.
    \variation[invar]{21. Qh7+ Kf8 22. Qh8+ Kf7 23. N3d5 Ne3+ 24. Nxe3 Qxe3 25. Qe8+ Kg7 26. Qe5+ Kg6 27. Nxa8} gives
    White chances to win.
    
    \mainline[level=1]{21...Nd2+ 22. Ke2 Bg4+ 23. Kd3 Qc4+ 24. Ke3 Qc5+ 25. Kd3} draw with perpetual check.
    
\end{multicols}
\newpage
\section{Najdorf Sicilian: White Attack with e4 and f4 Center}

Often in Najdorf Sicilian positions, White sets up a center with e4 and f4:

\chessboard[
    setfen=8/pp3ppp/3pp3/8/4PP2/8/PPP3PP/8 w HAha - 0 1,
    pgfstyle=straightmove,
    markmove={e4-e5, f4-f5},
]

White can either play e5 to open the f-file for the rook or sometimes f5
to open the a2--g8 diagonal for the bishop.

\begin{multicols}{2}
    \chessgameinfo{USSR Chess Championship}{B.Spassky}{L.Polugaevsky}{}{1958}{}
    \newchessgame
    \mainline[level=1]{
        1. e4 c5 2. Nf3 d6 3. d4 cxd4 4. Nxd4 Nf6 5. Nc3 a6 6. Bg5 Nbd7 
        7. Bc4 Qa5 8. Qd2 e6 9. O-O-O b5 10. Bb3 Bb7 11. Rhe1 Be7 12. f4 Nc5}.
    
    \chessboard

    White sets up a center with e4 and f4 and then thrusts with e5 to open the f-file.
 
\mainline[level=1]{13. e5}

Good idea, but the move order is wrong. The correct sequence is 
\variation[invar]{13. Bxf6 Bxf6 14. e5}.

\mainline[level=1]{13... dxe5 14. Bxf6 Bxf6?}

\variation[invar]{14... gxf6} is more accurate.

\mainline[level=1]{15. fxe5 Bh4 16. g3 Be7}

\chessboard

\mainline[level=1]{17. Bxe6!}

Here we see this thematic attack on e6 again. It is not a true sacrifice, 
since Black cannot safely accept it.

\chessboard

Accepting the sacrifice is a mistake: \variation[invar]{17... fxe6 18. Nxe6 Rd8 19. Nxg7+ \xskakcomment{This is a typical idea after the sacrifice: the g7-pawn is unprotected and White captures it with check.} Kf7 20. Qh6 \xskakcomment{White has a crushing attack.} }

\mainline[level=1]{17... O-O 18. Bb3 Rad8 19. Qf4 b4 20. Na4 h6 21. Nxc5 Qxc5 22. h4 }

\chessboard

White emerges a pawn up and with the initiative, and he eventually wins the game.

\mainline[level=1]{22... Bd5 23. Nf5 Bxb3 24. axb3 Rxd1+ 25. Rxd1 Rc8 26. Qe4 Bf8 27. e6 fxe6 28. Qxe6+ Kh8 29. Qe4 Qc6 30. Qd3 Re8 31. h5 Be7 32. Nxe7 Rxe7 33. Qg6 Qe8 34. g4 Re1 35. Qxe8+ Rxe8 36. Rd4 a5 37. Kd2 Re5 38. c3 bxc3+ 39. bxc3 Rg5 40. c4 Kg8 41. Rf4 g6 1-0}



\end{multicols}
\newpage
\section{Najdorf Sicilian: White Sacrifice/Exchange on b5}

A sacrifice on b5 is another common theme in the Najdorf Sicilian.
White exchanges a piece (typically a bishop) for pawns, aiming 
for rolling passed pawns on the queenside or targeting Black's
king in the center.

\begin{multicols}{2}    
    \chessgameinfo{Tilburg rapid}{Vassily Ivanchuk}{Erik Gustaf Ferdinand Hellers}{}{1993.11.18}{1-0}
    \newchessgame
    \mainline[level=1]{1. e4 c5 2. Nf3 d6 3. d4 cxd4 4. Nxd4 Nf6 5. Nc3 Nc6 6. Bg5 e6 7. Qd2 a6 8.
    O-O-O h6 9. Be3 Bd7 10. f3 b5 11. Bd3 Qc7}
    
    \chessboard


    \mainline[level=1]{12. Bxb5 axb5 13. Ndxb5 Qb8 14. Nxd6+ Bxd6 15. Qxd6 Qxd6 16. Rxd6}
    
    \chessboard

    Here we see the typical exchange. After giving up a bishop, White's
    knight targets both the queen (which must retreat) and the pawn on d6.

    This is not really a sacrifice. White exchanges a bishop for three pawns. In return 
    he obtains more coordinated pieces, and his pawns on the queenside will roll at some point.

    White eventually wins.
    \mainline[level=1]{16...Na5 17. b3 Rc8 18. Kb2 Bc6 19. Rhd1 O-O 20. Bb6 Nb7 21. R6d2 Ra8 22. a4 Rfc8 23. Nb5 Ne8 24. Na7 Rxa7 25. Bxa7 Nc7 26. c4 Kf8 27. Kc3 Ke8 28. Bb6 Na6 29. a5 f6 30. g4 e5 31. h4 Nb8 32. g5 hxg5 33. hxg5 Nd7 34. gxf6 gxf6 35. Rh1 Kf7 36. Rh7+ Ke6 37. Rg2 Ra8 38. Rgg7 Kd6 39. b4 Nxb6 40. axb6 Ra3+ 41. Kb2 Rxf3 42. Rxb7 Bxb7 43. Rxb7 Kc6 44. Rb8 Rf4 45. b5+ Kc5 46. b7 Kb6 47. c5+ Kc7 48. Rf8 Kxb7 49. Rf7+ Kb8 50. b6 Rxe4 51. Kb3 Re1 52. Kc4}

When discussing this sacrifice, we must also mention Mikhail Tal. His sacrifice 
idea has withstood the test of time, as it was even used in the year 2000.

\chessgameinfo{Hastings 1973/74}{Mikhail Tal}{Michael Francis Stean}{}{1974.01.09}{1-0}
\newchessgame
\mainline[level=1]{
    1. e4 c5 2. Nf3 d6 3. d4 cxd4 4. Nxd4 Nf6 5. Nc3 a6 6. Bg5 e6 7. f4 Nbd7 8. Qf3 Qc7 9. O-O-O b5 }
    
    \chessboard

    Again we see the same theme: White gives up his bishop for two pawns, and his knight
    hits Black's queen with a tempo. In this game, Tal aims his attack against Black's king in 
    the center.

    \mainline[level=1]{10. Bxb5 axb5 11. Ndxb5 Qb8}
    
    \chessboard

    \mainline[level=1]{12. e5 }
    
    Again we have an e4-f4 center, and White thrusts with e5. Unlike 
    exchanging pawns to open the f-file, this time White 
    sacrifices his pawn to open the d-file.
    \mainline[level=1]{12... Bb7 13. Qe2 dxe5 14. Qc4 Bc5 15. Bxf6 gxf6 }

    \chessboard
    
    White must open the d-file to continue his attack; otherwise his 
    material deficit will tell after Black consolidates.
    \mainline[level=1]{16. Rxd7 Be3+ 17. Kb1 Kxd7 18. Rd1+ }
    
    \chessboard

    \mainline[level=1]{18... Bd4?}

    Until now Stean has defended well, but here he falters. He must play \variation[invar]{18... Ke7}.
    After the text move, White continues his attack by exchanging Black's important 
    dark-squared bishop. Tal eventually wins the game.

    \mainline[level=1]{19. fxe5 fxe5 20. Nxd4 exd4 21. Qxd4+ Ke7 22. Qc5+ Kf6 23. Rf1+ Kg6 24. Qe7 f5 25. Qxe6+ Kg7 26. Qe7+ Kg6 27. h4 Ra5 28. h5+ Kxh5 29. Qf7+ Kh4 30. Qf6+ Kg3 31. Qg5+ Kh2 32. Qh4+ Kxg2 33. Rf2+ Kg1 34. Ne2# }
    
    
\end{multicols}

% Mobile Center stuff
\newpage
\section{Mobile Center}
A mobile pawn center is a central pawn formation where one side has at least two connected central pawns (usually on d4 and e4) 
that are free to advance, and the opponent has at most one pawn in the center. 
This gives the side with the pawn chain potential to push those pawns forward 
to gain space or create passed pawns, and often forces the opponent to defend 
against advancing pawns rather than contesting the center directly.

\chessboard[
    setfen=8/pp3ppp/4p3/8/3PP3/8/P4PPP/8 w KQkq - 2 1,
    pgfstyle=straightmove,
    linewidth=0.05em,
    markmove={d4-d5, e4-e5},
]

The typical strategic plan for the side with the mobile center is to advance pawns and try to create a passed pawn 
or use pawn pressure to push pieces away. We will see both strategies in this section.

\input{../cards/chess/ideas/mobile-center/white-create-passed-pawn}
\newpage
\section{Mobile Center: White Kingside Attack}
\begin{multicols}{2}

\chessgameinfo{USSR Championship}{L.Polugaevsky}{M.Tal}{2}{1969.9.7}{1-0}
\newchessgame
\mainline[level=1]{
    1. c4 Nf6 2. Nc3 e6 3. Nf3 d5 4. d4 c5 5. cxd5 Nxd5 6. e4 Nxc3 7. bxc3 cxd4 8. cxd4 Bb4+ 9. Bd2 Bxd2+ 10. Qxd2 O-O 11. Bc4 Nc6 12. O-O b6 13. Rad1 Bb7 14. Rfe1}

    \chessboard

    The same position also appeared in the World Chess Championship game between
    Spassky and Petrosian one month ago. Petrosian chose \variation{14... Rc8}. Tal 
    chose another variation in this game.
    \index{Mobile Center!White Kingside Attack}

    \mainline[level=1]{14... Na5 15. Bd3 Rc8}
    
    \chessboard

    White will thrust his pawn to e5. Before doing so, he sacrifices his d-pawn first.
    Otherwise, Black can exchange the knight with his bishop, removing an important attacker.

    \mainline[level=1]{16. d5 exd5 17. e5 Nc4 18. Qf4 Nb2? }
    
    \chessboard

    Here we will see a Greek Gift! \index{Attack!Greek Gift}
    \mainline[level=1]{19. Bxh7+ Kxh7 20. Ng5+ Kg6 21. h4!}
    
    Bring a further attacker!
    
    \mainline[level=1]{21...Rc4}
    
    \chessboard

    \mainline[level=1]{22. h5+?!}
    
    Too slow! \variation[invar]{22. Qg3 Kh6 23. e6 Qf6 24. exf7 Nxd1 25. Re6 } White wins the 
    game. White wins after an endgame after the text move.
    
    \mainline[level=1]{22... Kh6 23. Nxf7+ Kh7 24. Qf5+ Kg8 25. e6 Qf6 26. Qxf6 gxf6 27. Rd2 Rc6 28. Rxb2 Re8 29. Nh6+ Kh7 30. Nf5 Rexe6 31. Rxe6 Rxe6 32. Rc2 Rc6 33. Re2 Bc8 34. Re7+ Kh8 35. Nh4 f5 36. Ng6+ Kg8 37. Rxa7 
} 1-0
\end{multicols}

% Benko Gambit stuff
\newpage
\section{Benko Gambit}

The Benko Gambit (or Volga Gambit) is a chess opening characterised by the move 3...b5 in the Benoni Defence arising after:

\begin{multicols}{2}
\newchessgame
\mainline[level=1]{
1. d4 Nf6
2. c4 c5
3. d5 b5}

\chessboard
Black sacrifices a pawn for enduring queenside pressure. White can accept or decline the gambit.

When the sacrifice is accepted, White, who is already behind in development, must solve the problem of developing the f1-bishop.
An early e4 will lead to an exchange on f1, losing the right to castle.

Black also obtains fast development, good control of the a1–h8 diagonal, and can exert pressure down the half-open a- and b-files, where White's a- and b-pawns can become vulnerable. These benefits can last well into the endgame, and so, unusually for a gambit, Black does not generally mind if queens are exchanged.



\chessgameinfo{9th Canadian Open}{Gerald Aspler}{Pal Benko}{6}{1971.08.29}{0-1}
\newchessgame
\mainline[level=1]{
    1. d4 Nf6 2. c4 c5 3. d5 b5 4. cxb5 a6 5. bxa6 Bxa6}
    
\chessboard

This is a typical position in the Benko Gambit Accepted. Black has a clear plan:
attack along the a- and b-files. Note that the light-squared bishop on a6 prevents
White's normal development e4 (after exchanging on f1, White loses the right to castle).

\mainline[level=1]{6. Nc3 d6
7. Nf3 g6 8. g3 Bg7 9. Bg2 O-O 10. O-O Nbd7 11. Re1 Qb6 12. e4
Ng4 13. Qc2 Rfb8}

\chessboard

It is clear that Black is attacking along the a- and b-files. 
Black has a pawn structure similar to the Benoni structure with pawns on c5 and d6.
This means Black will also have a typical Benoni plan: 

\begin{itemize}
    \item{Advance the pawn to c4}
    \item{Move the knight to d3}
\end{itemize}

Note that the knight on d3 will be well protected by the c4 pawn and the a6 bishop.

\mainline[level=1]{14. h3 Nge5 15. Nxe5 Nxe5 16. b3 Nd3 17. Rd1}

\chessboard

\mainline[level=1]{17...c4! 18. Be3 Qb4 19. Bd2 Qc5 20. Rf1 cxb3 21. axb3 Nb4 22. Qb2
Rc8 23. Qa3 Bxc3 24. Bxc3 Qxc3 25. Rfc1 Qd4 26. Bf1 Rc2
} 0-1

We will also see, the great Capablanca played the idea of Benko Gambit 
with Ruy Lopez before Benko Gambitwas even invented.

\chessgameinfo{St. Petersburg}{Aron Nimzowitsch}{Jose Raul Capablanca}{1}{1914.04.21}{0-1}
\newchessgame

\mainline[level=1]{
    1. e4 e5 2. Nf3 Nc6 3. Nc3 Nf6 4. Bb5 d6 5. d4 Bd7 6. Bxc6
Bxc6 7. Qd3 exd4 8. Nxd4 g6}

\chessboard

The last move loses a pawn, yet the position is still playable. 
I believe Capablanca didn't play the gambit intentionally. 

\mainline[level=1]{9. Nxc6 bxc6 10. Qa6 Qd7 11. Qb7
Rc8 12. Qxa7 Bg7 13. O-O O-O}

\chessboard[
    markstyle=circle,
    linewidth=0.05em,
    markfields={e4},
]

\mainline[level=1]{14. Qa6?!}

This move simply loses one tempo. White's e4 pawn needs protection, 
one obvious idea is playing \variation[invar]{14. f3}. After 
the text move, Black can play some natural moves to develop while 
White can only react passively.

\mainline[level=1]{14...Rfe8 15. Qd3 Qe6 16. f3 Nd7 17. Bd2}

\chessboard[
    markstyle=circle,
    linewidth=0.05em,
    markfields={c3, b2},
]

Black's bishop targets all the White's weaknesses along the long diagonal.
White's d2 pawn is an important defender. A logic decision is to trade it.
Afterwards, Black can exert pressure on a- and b-file with the rooks.

\mainline[level=1]{17...Ne5! 18. Qe2 Nc4! 19. Rab1 Ra8 20. a4 Nxd2! 21. Qxd2}

\chessboard[
    pgfstyle=straightmove,
    linewidth=0.05em,
    markmove={b2-b3},
]

White intends to play b3 and when move his knight away. 
If he succeeds, he will rid of his weak a- and b-pawns and these pawns will
even become strength as passed pawns. Black must stop it.

\mainline[level=1]{21...Qc4! 22. Rfd1 Reb8}

\chessboard

Now Black has finally realized the Benko Gambit idea without playing 
the opening. White's knight is tied down. Black will double his rooks on the b-file in 
the next few moves.

\mainline[level=1]{23. Qe3 Rb4}

\chessboard

\mainline[level=1]{24. Qg5? Bd4+ 25. Kh1 Rab8
26. Rxd4 Qxd4 27. Rd1 Qc4 28. h4 Rxb2 29. Qd2 Qc5 30. Re1 Qh5
31. Ra1 Qxh4+ 32. Kg1 Qh5 33. a5 Ra8 34. a6 Qc5+ 35. Kh1 Qc4
36. a7 Qc5 37. e5 Qxe5 38. Ra4 Qh5+ 39. Kg1 Qc5+ 40. Kh2 d5
41. Rh4 Rxa7 42. Nd1
} 0-1
\end{multicols}

\newpage

\section{Blumenfeld Gambit}
\newchessgame

The Blumenfeld Countergambit is a chess opening characterised by the moves 3...e6 4.Nf3 b5 in the Benoni Defense arising after:

\mainline[level=1]{
1. d4 Nf6
2. c4 c5
3. d5 e6
4. Nf3 b5}

\chessboard

Black sacrifices a wing pawn to establish an imposing centre with pawns on c5, d5 and e6. The natural development of the bishops to b7 and d6, combined with the half-open f-file for a rook, tend to facilitate Black's play on the kingside.


\begin{multicols}{2}
\newchessgame
\chessgameinfo{Bad Pistyan}{Siegbert Tarrasch}{Alexander Alekhine}{1}{1922.04.07}{0-1}
\mainline[level=1]{
    1. d4 Nf6 2. Nf3 e6 3. c4 c5 4. d5 b5 5. dxe6 fxe6 6. cxb5 d5}

\chessboard

Black has a very intimidating center.

\mainline[level=1]{
7. e3?!}

This move is too passive. White must try to destroy Black's center before Black
starts an attack.  Tarrasch wanted to prevent
Black from playing ...d4, but this was not necessary. He should have played \symknight c3 first.
Black cannot play \variation[invar]{7... d4 8. Na4 Bd6 9. e3} because his 
center is destroyed completely. 
While after \variation[invar]{7... Nbd7 8. e4 d4 9. Na4 Bb7 10. Bd3 Bxe4 11. Bxe4 Nxe4 12. O-O Ndf6 13. b4 cxb4 14. Nxd4 }
White has nothing to fear. 

\mainline[level=1]{7...Bd6 8. Nc3 O-O 9. Be2 Bb7 10. b3 Nbd7 11. Bb2 Qe7
12. O-O Rad8 13. Qc2 e5 }

Black starts an attack. ...e4 is in the air.

\mainline[level=1]{14. Rfe1? e4 15. Nd2 Ne5 16. Nd1 Nfg4
17. Bxg4 Nxg4 18. Nf1 Qg5 19. h3 Nh6 20. Kh1 Nf5 21. Nh2 d4
22. Bc1 d3 23. Qc4+ Kh8 24. Bb2 Ng3+ 25. Kg1 Bd5 26. Qa4 Ne2+
27. Kh1 Rf7 28. Qa6 h5 29. b6 Ng3+ 30. Kg1 axb6 31. Qxb6 Ne2+
32. Kh1 Ng3+ 33. Kg1 d2 34. Rf1 Nxf1 35. Nxf1 Be6 36. Kh1 Bxh3
37. gxh3 Rf3 38. Ng3 h4 39. Bf6 Qxf6 40. Nxe4 Rxh3+
} 0-1

\end{multicols}
% Poisoned Pawn stuff
\newpage
\section{Poisoned Pawn}

The Poisoned Pawn Variation is any of several series of opening moves in chess in which a pawn is said to be "poisoned" because its capture can result in a positional loss of time or a loss of material. This commonly refers to a capture of the b-pawn by the queen, especially by Black.

In chess, a \vocab{poisoned-pawn}{Poisoned pawn}{
    A pawn that seems to be hanging or undefended, enticing an opponent to capture it. However, doing so often results in a significant positional or material loss due to hidden tactical threats or traps set by the opponent.
} is a pawn that seems to be hanging or undefended, enticing an opponent to capture it. However, doing so often results in a significant positional or material loss due to hidden tactical threats or traps set by the opponent.

Being aware of the concept of a poisoned pawn can protect you from a lot of trouble. If you spot a pawn hanging, it is always a good idea to double-check to see if it is poisoned. Not only that, but you can also use a poisoned pawn to set a trap for your opponents.
%\newpage

\section{Poisoned Pawn: London System}

\begin{multicols}{2}
    \newchessgame

    \mainline[level=1]{
        1. d4 d5 2. Nf3 Nf6 3. Bf4 c5 4. e3 Qb6 5. Nc3 Qxb2?}
    
    \chessboard

    Here Black is too greedy. White exploits Black's weakness on c7 and his c-pawn
    is too strong for Black to defend.
    
    \mainline[level=1]{    6. Nb5 Na6 7. a3 Bf5 8. dxc5 Bxc2 9. Qc1 Qxc1+ 10. Rxc1 Bf5 11. Nbd4 
    }

    \chessboard

    White is winning.
\end{multicols}
\newpage
\section{Poisoned Pawn: Torre Attack}
\begin{multicols}{2}
 Spassky seized the initiative by sacrificing a pawn in the next game. His
 attack developed and the Black king never escaped. 
 \newchessgame
 \chessgameinfo{USSR Championship}{B.Spassky}{V.Osnos}{}{1963}{1-0}   
 \mainline[level=1]{
 1. d4 Nf6 2. Nf3 e6 3. Bg5 c5 4. e3 Qb6 5. Nbd2 Qxb2 }
 
 \chessboard

 Here we have another poisoned pawn position. Taking the pawn is playable for Black
 since there are no clear weaknesses and the position is closed. White, on the other hand, 
 has some lead in development. The struggle continues.

 \mainline[level=1]{6. Bd3 cxd4?!}
 
 This move opens the position. White may have the open c-file in the future.
 \variation[invar]{6... d5} is better.

 \mainline[level=1]{7. exd4 Qc3 8. O-O d5 9. Re1 Be7 10. Re3}
 
 \chessboard

 Black cannot castle now because of \variation[invar]{10...O-O 11. Bxh7}, which loses the queen.
 The natural move is to remove the queen from danger.
 \mainline[level=1]{10... Qc7 11. Ne5}
 
 \chessboard

 Black still cannot castle because of the ``Greek Gift'':
 \variation[invar]{11...O-O 12. Bxf6 Bxf6 13. Bxh7+ Kxh7 14. Qh5+ Kg8 15. Rh3} Black will be mated soon.

 \mainline[level=1]{11... Nc6 12. c3}
 
 \chessboard

 \mainline[level=1]{12... Nxe5?!}
 
 A strange move, as the knight must retreat immediately.
 Since Black cannot castle on the kingside due to the sacrifice on h7 
 as mentioned, he must try to castle on the queenside. \variation[invar]{12...Bd7} is better.

 \mainline[level=1]{13. dxe5 Ng8 }
 
 Forced. \variation[invar]{13... Nd7 14. Bxe7 Kxe7 15. Qh5} Black's king is stuck in the center. 
 
 \mainline[level=1]{14. Nf3 h6 15. Bf4 Bd7 16. Nd4}
 
 \chessboard

 It is logical for Black to exchange pieces. The question is which one.
 White's knight is much more dangerous than the bishop. Therefore, Black should play 
 \variation[invar]{16...Bc5}.

 Note that \variation[invar]{16...Qxc3 17. Nxe6} destroys Black's center.
 
 \mainline[level=1]{16... Bg5 17. Bxg5 hxg5 18. Qg4 Qxc3 19. Nb3 Nh6?}
 
 \variation[invar]{19... Ne7} is better. After the text move, White is completely winning.

 \mainline[level=1]{20. Qxg5 Qb4 21. Rg3 Qf8 }
 
 \chessboard

 \mainline[level=1]{22. Rc1!}
 
 A very logical move, bringing all pieces into the attack!

 \mainline[level=1]{22... f6 23. Qe3 f5 24. Nc5 f4 }
 
 \chessboard
 
 \mainline[level=1]{25. Bg6+! Ke7 26. Qa3} 1-0
 
\end{multicols}	

% Catalan stuff
\newpage
\section{Catalan Opening}
The Catalan Opening is a chess opening that begins with 1.d4 and 2.c4, followed by g3 and Bg2. It often arises via the move order 1.d4 Nf6 2.c4 e6 3.g3, combining ideas from the Queen’s Gambit and the Réti.

\chessboard[
    setfen=rnbqkb1r/pppp1ppp/4pn2/8/2PP4/6P1/PP2PP1P/RNBQKBNR b KQkq - 0 3
]

The main idea of the Catalan is long-term positional pressure rather than immediate tactics. White fianchettoes the king’s bishop to g2, placing it on the long diagonal where it exerts strong pressure on the central squares d5 and e4, and later on the queenside. White usually allows tension in the center and does not rush to occupy it with pawns. \index{Endgame!Knight vs. Bishop}

A key theme is that Black may capture the pawn on c4. White often accepts this temporarily, using faster development and better piece activity to pressure Black until the pawn is recovered or Black is forced into concessions. This is known as the Open Catalan. If Black keeps the center closed and does not take on c4, the game becomes a Closed Catalan, where White slowly builds pressure against Black’s solid but somewhat passive structure.

Overall, the Catalan aims for stable central control, strong piece coordination, and a small but persistent advantage that often becomes clearer in the middlegame and endgame.

\section{Catalan Opening: Endgame}

White can steer the game toward a favorable endgame, as Ulf Andersson demonstrated in this instructive example.

\begin{multicols}{2}
\chessgameinfo{Swedish Team Championship}{Ulf Andersson}{Sergey Vladimirovich Ivanov}{}{2000}{1-0}
\newchessgame
\mainline[level=1]{1. Nf3 d5 2. d4 Nf6 3. c4 e6 4. g3 dxc4 5. Qa4+ Nbd7 6. Bg2 a6
7. Nc3 Rb8 8. Qxc4 b5 9. Qd3 Bb7 10. O-O c5 11. dxc5 Bxc5
12. Bf4 Rc8 13. Rad1 O-O 14. Ne5}

\chessboard

Many pieces will be exchanged, and Andersson deliberately heads for an endgame
where he can press with a persistent but risk-free advantage.

\mainline[level=1]{14... Bxg2 15. Kxg2 Nxe5 16. Bxe5
Qxd3 17. Rxd3 Rfd8 18. Rxd8+ Rxd8 19. Bxf6 gxf6}

\chessboard

This is the position Andersson had in mind when he played his 14th move.
How should we evaluate this endgame? A good starting point is to examine
the pawn structure.

\chessboard[
    setfen=8/5p1p/p3pp2/1p6/8/P5P1/1P2PP1P/8 w HAha - 0 1,
    showmover=false,
    markstyle=circle,
    linewidth=0.05em,
    markfields={a6, h7},
]

White has no real weaknesses in his camp. He will place his pawns on light
squares so that Black's dark-squared bishop cannot attack them.

On the other hand, Black has two clear weaknesses: the pawns on a6 and h7.
As the game continues, White will use his king to attack these pawns directly.
Also notice Black's doubled pawns on the f-file; they give White enough time
to bring his king into Black's position.

After the rooks are exchanged, White's pieces are actually too slow to attack
both weaknesses simultaneously, so Black has decent practical chances to hold.
However, White can simply improve his position and wait for a mistake, so his
winning chances remain very good.

\mainline[level=1]{20. Rc1 Be7}

\chessboard

One of the most neglected principles in chess is \emph{don't rush}. 
Before pushing forward, White must first prevent Black's rook from invading
on d2. His next move is highly instructive.

\mainline[level=1]{
21. Nb1! f5 22. e3 Bf6 23. b3 Kf8 24. Kf3 Ke7 25. h3 Rd5
26. Ke2 Kd7 27. Nd2 Be7 28. Nf3 Bf6 29. Ne1 Rd6 30. g4 fxg4
31. hxg4 Rc6 32. Rxc6 Kxc6 33. Nd3!}

\chessboard

White must now stop Black's king from invading. The d5 square is also
temporarily defended because after \variation[invar]{33...Kd5? 34. Nb4+}
Black would simply lose a pawn.

\mainline[level=1]{33...a5 34. e4!}

\chessboard 

Now d5 is also defended permanently. White is ready to march his king toward
Black's kingside while Black is reduced to waiting.
\mainline[level=1]{34...a4 35. Ke3 axb3
36. axb3 Kd6 37. Kf4 Bd8 }

\chessboard

\mainline[level=1]{38. g5!}

White now sacrifices a pawn temporarily to open lines for a decisive king invasion.

\mainline[level=1]{38...Ke7 39. Kg4 Kf8 40. f3 f6
41. Kh5 fxg5 42. Kh6 Kg8 43. Nc5 Kf7 44. Kxh7}

\chessboard

The first mission is accomplished: White's king has infiltrated Black's camp.
Next he will march toward the queenside to attack the remaining weakness.

\mainline[level=1]{44...Bb6 45. Nd3 Kf6
46. Kg8 Bg1 47. Kf8 e5 48. Ke8 Ke6 49. Kd8 Kd6 50. Kc8 Be3
51. Kb7}

\chessboard

\mainline[level=1]{51...Bd4?}

Until this point Black has defended well. After the move in the game, however,
White places him in zugzwang, wins a pawn, and with it the game. Black should
have played \variation[invar]{51... Bg1}
\mainline[level=1]{52. b4 Kd7 53. Nc5+ Kd6 54. Kb6 Bc3 55. Kxb5 Bd2
56. Nb7+ Kc7 57. Na5 Be3 58. Ka6} 1-0

\end{multicols}
\newpage
\section{Catalan Opening: Pressure Along the Long Diagonal}

In the Catalan, White fianchettoes the king's bishop to g2, placing it on the
long diagonal where it exerts strong pressure on the central squares d5 and e4,
and later on the queenside.

In the following game Black struggles to develop on the queenside. White
wins by following a clear strategic plan:
\begin{itemize}
    \item Pressing against the weakness on b7, forcing all of Black's pieces
          into increasingly passive positions.
    \item Creating a second weakness on g6.
    \item Infiltrating with his king to exploit both weaknesses and decide the game.
\end{itemize}

\begin{multicols}{2}

\newchessgame
\chessgameinfo{Training Match}{Garry Kasparov}{Ulf Andersson}{}{1985.06}{1-0}
\mainline[level=1]{1.d4 Nf6 2.c4 e6 3.g3 d5 4.Bg2 Be7 5.Nf3 O-O 6.Qc2 dxc4 7.Qxc4 a6 8.Bf4 Bd6 9.Ne5}

\chessboard

Because of White's bishop on the long diagonal, Black cannot develop his
queenside pieces naturally. The bishop cannot move, as it must defend the b7
pawn, and the b-pawn itself cannot advance because the rook on a8 would be
left hanging. Even \variation[invar]{9...Nbd7} does not really help Black's
development, since White can simply ignore it.

\mainline[level=1]{9...Nd5 10.Nc3 Nxf4 11.gxf4 Nd7 12.e3 Qe7
13.O-O Rb8 14.Ne4 Nf6 15.Nc5 c6}

\chessboard

Black finally completes queenside development. White now has a simple mission:
play a4 and b4, then b5, exchange on b5, and keep targeting b7 to drive Black's
pieces into passive defence.

\mainline[level=1]{16.b4 Kh8 17.a4 Nd5 18.b5 f6
19.Ned3 axb5 20.axb5 cxb5 21.Qxb5 Rd8 22.Ra7 Bxc5 23.Nxc5 Bd7
24.Nxd7 Rxd7 25.f5 g6 26.fxe6 Qxe6 27.Rfa1 Kg7 28.Qb3}

\chessboard

White is threatening winning Black's knight. Black must exchange the queens.

\mainline[level=1]{28...Nf4
29.Qxe6 Nxe6 30.Rb1 Nd8 31.Bf3 Rc8 32.Ra5 Rcc7 33.Rab5 f5}

\chessboard

Now Black is completely passive. Here the \emph{Principle of Two Weaknesses}
applies: to win a better position, especially in the endgame, it is often not
enough to attack a single weakness—you need to create a second target so your
opponent's pieces are overstretched. \index{Principle of Two Weaknesses}

White therefore creates a second weakness on g6 on the kingside.

\mainline[level=1]{
34.h4 Kf7 35.h5 Kg7 36.Kg2 Re7 37.Rb6 Rf7 38.Bd5 Rfd7 39.R1b5
Re7 40.Kg3 Red7 41.hxg6 hxg6}

\chessboard

White's king will now march over to join the attack and convert his advantage.
\mainline[level=1]{42.Kf4 Rc2 43.Kg5 Rxf2 44.Rxg6+
Kf8 45.Bb3 Nf7+ 46.Kf6 f4 47.e4 Rb2 48.e5 f3 49.e6 f2 50.Bc4
} 1-0
\end{multicols}
\newpage
\section{Catalan Opening: Setup Pawn Center}
If the circumstance allows, White can also setup a pawn center.

\begin{multicols}{2}
    \chessgameinfo{Amsterdam IBM Tournament}{B.Spassky}{D.Ciric}{}{1970}{1-0}

    \newchessgame
    \mainline[level=1]{
        1. d4 d5 2. c4 e6 3. Nf3 Nf6 4. g3 Be7 5. Bg2 O-O 6. O-O c6 7. b3 Nbd7 8. Bb2 b6 9. Nbd2 Bb7 10. Rc1 Rc8 11. e3}
    
        ``White is planning to place the queen on e2 and not on the usual c2 square, where
    it would be `exposed' to the c8 rook'' (Najdorf).

    \mainline[level=1]{11... c5 12. Qe2}
    
    \chessboard

    ``White is preparing \symrook fd1 and an eventual \symknight e5'' (Spassky).

    \mainline[level=1]{12... Rc7 13. cxd5}
    
    \chessboard
    
    Critical moment! The next move decides the character of the game.

    \variation[invar]{13... exd5 14. dxc5 bxc5 15. Ne5}

    \chessboard[
        setfen=3qr1k1/pbrnbppp/5n2/2ppN3/8/1P2P1P1/PB1NQPBP/2RR2K1 b - - 3 16
    ]

    Black has \vocab{hanging-pawns}{Hanging pawns}{two side-by-side pawns on adjacent files (here the c- and d-pawns) with no friendly pawns on neighboring files behind them to support them. They grant space and dynamic piece activity but can become long-term weaknesses if blockaded or forced to advance} on the c- and d-files. The position is playable for both sides.
    
    The text move allows White to setup a pawn center.
    \mainline[level=1]{13... Bxd5 14. e4 Bb7 15. e5 Nd5 16. Nc4}

    White has made some natural moves. 

    \mainline[level=1]{16...Qa8 }
    
    \chessboard 

    \mainline[level=1]{17. Nd6!}
    
    White sacrifices a pawn to activate his pieces (his pawn will fall in 
    the next a few moves). \index{Pawn Sacrifice!Piece Activity}

    \mainline[level=1]{17...Bxd6 18. exd6 Rc6 19. dxc5 bxc5 20. Ng5 Rxd6 21. Rfd1!}
    
    ``The strongest move, pinning the knight and bringing his last inactive piece into play'' (Bernard Cafferty).
    It is also good practical play, if one cannot find a decisive blow, just bring new 
    pieces into play.
    
    The double sacrifice on h7 and g7 is too hasty: \variation[invar]{21. Nxh7 Kxh7 22. Qh5+ Kg8 23. Bxg7 Kxg7 24. Qg5+ Kh8 25.Rc4
    Nf4! 26. Rxf4 Rd4!}

    \mainline[level=1]{21... Ra6}
    
    \chessboard

    \mainline[level=1]{22. Qe4}
    
    Amusingly, now the double sacrifice mentioned in the last move works because the Black rook has moved 
    to a6! Sometimes the opponent will help us. \index{Attack!Double Sacrifice}
    \variation[invar]{22. Nxh7 Kxh7 23. Qh5+ Kg8 24. Bxg7 Kxg7 25. Qg5+ Kh8 26.Rc4} White mate soon
    
    \mainline[level=1]{22...f5 23. Qc4 Qe8 24. Re1 Rxa2 25. Rxe6 Qa8 26. Bxd5 Bxd5 27. Qh4 h6}
    
    \chessboard

    \mainline[level=1]{28. Qxh6! Nf6 29. Rxf6}
    
    1-0. There is no defense after \variation[invar]{29... Rxf6 30. Qh7+ Kf8 31. Qh8+ Bg8 32. Bxf6}.
\end{multicols}

% Carlsbad stuff
\newpage
\section{Carlsbad Structure}
\newchessgame


The Carlsbad pawn structure can be reached via many openings but is best 
known from the Queen's Gambit Declined Exchange Variation, which occurs after 

\mainline[level=1]{1.d4 d5 2.c4 e6 3.Nc3 Nf6 4.cxd5 exd5 5.Bg5 Be7 6.e3 c6}

\chessboard[
    setfen=r1bqk2r/pp1nbppp/2p2n2/3p2B1/3P4/2NBP3/PP3PPP/R2QK1NR w KQkq - 2 8
]

The Carlsbad pawn structure is important because it is full of rich possibilities 
for both sides and because the structure that can be reached via many openings, 
including the Nimzo-Indian, Caro-Kann, Scandinavian, and multiple other queen's 
pawn openings.

\begin{itemize}
    \item{Minority Attack}

    White advances his queenside pawns with the main purpose of creating weaknesses in black’s structure.

    \chessboard[
        setfen=8/pp3ppp/2p5/3p4/3P4/4P3/PP3PPP/8 w KQkq - 2 8,
        pgfstyle=straightmove,
        linewidth=0.05em,
        markmove={b2-b4, b4-b5, b5-c6, b7-c6},
    ]
    
    \item{Playing for the e3-e4 push}

    This can be done with or without the support of the f pawn (by pushing f3) then e4.

    \chessboard[
        setfen=8/pp3ppp/2p5/3p4/3P4/4P3/PP3PPP/8 w KQkq - 2 8,
        pgfstyle=straightmove,
        linewidth=0.05em,
        markmove={e3-e4},
    ]   
\end{itemize}

Black must defend accordingly. His ideas are
\begin{itemize}
    \item{Moving the knight from g8 to e4}
    \item{Moving the knight from b8 to c4 via d7 and b6.}
    \item{Idea from Petrosian, moving the knight from b8 to d6 via d7, b6 and c4, in anticipation of
    the standard minority attack b2-b4-b5. \footnote{See the game Bobotsov-Petrosian, Lugano Olympiad 1968 \cite{kasparov:2004}}}
\end{itemize}

\newpage

\section{Carlsbad Structure: White Center Break}

\begin{multicols}{2}
    In this section, we focus on the center-break plan for White.
    Typically, White needs to complete these steps to break in the center:

    \begin{itemize}
        \item{f2-f3}
        \item{e3-e4}
        \item{e4-e5}
        \item{An attack on the king side}
    \end{itemize}

    We will illustrate the idea by a famous game. \index{Carlsbad Structure!White Center Break}

    \chessgameinfo{USSR Championship}{M.Botvinnik}{P.Keres}{}{1952.12.09}{1-0}
    \newchessgame
    \mainline[level=1]{
        1.d4 Nf6 2.c4 e6 3.Nc3 d5 4.cxd5 exd5 5.Bg5 Be7 6.e3 O-O 7.Bd3
        Nbd7 8.Qc2 Re8 9.Nge2 Nf8 10.O-O c6 }
    
    \chessboard

    We reach a typical Queen's Gambit Declined Exchange Variation position.
    One typical idea for White is the minority attack.

    \mainline[level=1]{11.Rab1}
    
    \chessboard

    \mainline[level=1]{11...Bd6}

    This move doesn't have a clear plan. The text move threats \symbishop xh2.
    However the threat is easy to defend and we will see soon that the bishop 
    will go back.  

    \mainline[level=1]{12.Kh1 Ng6}

    \chessboard

    \mainline[level=1]{
        13.f3}
    
    First mission accomplished. Now White aims for e4 and directs all pieces toward that square.
        
    \mainline{13...Be7 14.Rbe1! Nd7?}
    
    Black loses two tempi because the knight will return to f6. \variation[invar]{14... h6} is the best move.

    \mainline[level=1]{15.Bxe7 Rxe7 16.Ng3! Nf6 17.Qf2}
    
    Note how White prepared for the e4 break. In the last move, he protected the d4 
    pawn with his queen, since it will be loose after e4.
    
    \mainline[level=1]{17...Be6
        18.Nf5 Bxf5 19.Bxf5 Qb6 }
    
    \chessboard

    Now the second mission is achieved. It is instructive to notice the piece maneuvers:
    \symknight g1 - e2 - g3, 
    \symrook a1 - b1 - e1,
    \symqueen d1 - f2 
    
    All the pieces are helping for the e4 break.
    \mainline[level=1]{20.e4! dxe4 21.fxe4 Rd8}
    
    \chessboard

    Now for the third mission: White launches a kingside attack and wins.
    \mainline[level=1]{22.e5 Nd5 23.Ne4 Nf8 24.Nd6 Qc7 25.Be4 Ne6 26.Qh4 g6 27.Bxd5 cxd5 28.Rc1 Qd7 29.Rc3 Rf8 30.Nf5 Rfe8 31.Nh6+ Kf8 32.Qf6 Ng7 33.Rcf3 Rc8 34.Nxf7 Re6 35.Qg5 Nf5 36.Nh6 Qg7 37.g4} Black resigned.

    Actually such a plan doesn't need to be restricted in Carlsbad structure, it 
    can also be used in Nimzo-Indian defense. The next game is a classic.

    \chessgameinfo{AVRO}{M.Botvinnik}{J.Capablanca}{11}{1938.11.22}{1-0}
    \newchessgame

    \mainline[level=1]{1.d4 Nf6 2.c4 e6 3.Nc3 Bb4 4.e3 d5 5.a3 Bxc3+ 6.bxc3 c5 7.cxd5
    exd5 8.Bd3 O-O 9.Ne2 b6 10.O-O Ba6 11.Bxa6 Nxa6 12.Bb2 Qd7
    13.a4 Rfe8 14.Qd3 c4}
    
    \chessboard

    White is preparing for f3 and e4. Notice the piece maneuvers, we know them already. 
    Certainly Botvinnik did remember his plan and just played them during the game.
    
    \mainline[level=1]{15.Qc2 Nb8 16.Rae1! Nc6 17.Ng3! Na5 18.f3! Nb3 19.e4! Qxa4}
    
    Mission accomplished: f3 and e4. The next is e5 winning a tempo.

    \mainline[level=1]{20.e5 Nd7 21.Qf2 g6 22.f4 f5 23.exf6 Nxf6 24.f5
    Rxe1 25.Rxe1 Re8 26.Re6 Rxe6 27.fxe6 Kg7 28.Qf4 Qe8 29.Qe5 Qe7}

    \chessboard

    This is the classic finish:
    \mainline[level=1]{30.Ba3 Qxa3 31.Nh5+ gxh5 32.Qg5+ Kf8 33.Qxf6+ Kg8 34.e7 Qc1+
    35.Kf2 Qc2+ 36.Kg3 Qd3+ 37.Kh4 Qe4+ 38.Kxh5 Qe2+ 39.Kh4 Qe4+
    40.g4 Qe1+ 41.Kh5} Black resigned.

\end{multicols}
\newpage
\section{Minority Attack}
In chess, a minority attack \index{Minority Attack} is the advancement of one's pawns on the side of the board where one has fewer pawns than their opponent, intending to use their minority to strategically provoke a weakness (i.e, an isolated or backward pawn) in the opponent's pawn structure. The minority attack is a common middlegame plan that can be played in many pawn structures. The name might be misleading, as the "attack" does not involve tactics planned to produce checkmate or significant material gain, but rather a strategical and structural advantage for the attacking player.

The minority attack can be strengthened by the moving of one or both rooks to the files where the attacking player intends to advance their pawns, planning prophylactically for the opening of the files. Common openings that result in pawn structures where a minority attack is effective include the Queen's Gambit Declined and the Caro-Kann Defense. The minority attack occurs most commonly on the queenside, as players commonly castle kingside in openings where a minority attack is effective, and the advancement of the pawns on the side of the castled king is widely considered to severely weaken the king's safety.

\begin{multicols}{2}
\subsection*{Basic Form}
White thrusts a- and b-pawns to create a weakness for black on c6.


\newchessgame[
id=main,
moveid=23w,print,
showmover,
mover=b,% has no effect
castling=Q,enpassant=a3,
setwhite={pa2,pb2,pd4,pe3,pf2,pg2,ph2},
addblack={pa7,pb7,pc6,pd5,pf7,pg7,ph7}]


White has the following moves to start a minority attack:
\begin{itemize}
    \item{1. b4}
    \item{1. a4}
    \item{1. b5}
\end{itemize}

\chessgameinfo{}{Capablanca}{Golombek}{}{1939}{1-0}
%\centering
\newchessgame[
id=main,
moveid=23w,print,
showmover,
mover=b,% has no effect
castling=Q,enpassant=a3,
setwhite={pa2,pb4,pd4,pe3,pf2,pg2,ph3,qd3,kg1,rb1,rc1,na4},
addblack={pa7,pb7,pc6,pd5,pf7,pg6,ph6,kg8,qd6,ra8,re8,ng7}]

\mainline[level=1]{23.b5 cxb5}

White continues the minority attack and favorably changes the pawn structure. If black allows white to capture on c6, then he will have a backward c6-pawn. If black captures on b5, white will recapture with the queen and can target the isolated d5 and b7 pawn.

\mainline{24.Qxb5 Ne6}

\chessboard

\mainline{25. Nc3!}

Much better than 
\variation{25. Qxb7} which gives black chances after \variation{25... Reb8 26. Qc6 Qxc6 27. Rxb8+ Rxb8 28. Rxc6 Rb1+}.

\mainline{25... Red8 26. Qxb7 Qa3 27.Nxd5 Qxa2}

\chessboard

\mainline{28.Nb4 Qa4 29. Nc6}
1-0

For white not only threatens the rook but also the queen with \variation{30. Ra1}.

\end{multicols}

% Pawn Sacrifice stuff
\newpage
\section{Pawn Sacrifice}
A pawn sacrifice occurs when a player deliberately places one of their pawns in a position where it can be captured by the opponent. This tactic is often employed to achieve greater control over the board, disrupt the opponent's strategy, or create opportunities for more significant gains later in the game. 


%\newpage
\section{Pawn Sacrifice: Bishop Pair}

The bishop pair is a powerful strategic advantage in chess, allowing control over both color complexes and enhancing board coverage.

Sometimes it is a good idea to sacrifice a pawn for a bishop pair. \index{Pawn Sacrifice!Bishop Pair}

\begin{multicols}{2}
    \chessgameinfo{Linares 26th}{V.Anand}{L.Aronian}{4}{2009.02.20}{0-1}
    \newchessgame
    \mainline[level=1]{
        1. d4 d5 2. c4 c6 3. Nf3 Nf6 4. Nc3 e6 5. e3 Nbd7 6. Bd3 dxc4 7. Bxc4 b5 8. Bd3 Bd6 9. O-O O-O 10. Qc2 Bb7 11. a3 a6 12. Ng5}
    
    \chessboard
    
    White sacrificed a pawn for a bishop pair and a pawn center, expecting the ``Greek Gift'' from Black.

    \mainline[level=1]{12...Bxh2+ 13. Kxh2 Ng4+ 14. Kg1 Qxg5 15. f3 Ngf6 16. e4}
    
    \chessboard

    Black must stop White to double rook on the h-file
    \mainline[level=1]{16...Qh4!}
    
    The battle continued. White got an advantage but loses the game due to a blunder.

    \mainline[level=1]{17. Be3 e5 18. Ne2 Nh5 19. Qd2 h6 20. Rfd1 Rae8 21. Bc2 Re6 22. Bf2 Qe7 23. g4 Rg6 24. Kf1 Nhf6 25. Ng3 Nxg4 26. fxg4 Qh4 27. Nf5 Qxg4 28. Qc3 Re8 29. Qg3 Qh5 30. Qh4 Qf3 31. Rd3 Qg2+ 32. Ke2 exd4 33. Rg3 Rxg3 34. Qxg3 Rxe4+ 35. Kd2 Rg4 36. Qxg2 Rxg2 37. Ke2 c5 38. Rg1 Ne5 39. Rxg2 Bxg2 40. Kd2 h5 41. b4 Nc4+ 42. Kc1 Nxa3 43. Bd1 cxb4 44. Bxh5 g6 45. Ne7+ Kf8 46. Nxg6+ fxg6 47. Bxg6 Ke7 48. Bxd4 Kd6 49. Bd3 Nc4 50. Bg7 a5 51. Be2 Be4 52. Bf6 a4 53. Bg7 Kd5
    }

\end{multicols}
\newpage
\section{Pawn Sacrifice: Infiltration}
In many endgames the guiding principle is clear: piece activity is more important than material.
A rook is especially powerful on the seventh rank, where it attacks pawns and restricts the enemy king.
It is therefore quite natural to consider sacrificing a pawn in order to infiltrate with the rook.
\index{Pawn Sacrifice!Infiltration}

\begin{multicols}{2}
\newchessgame[
    setfen=2rr2k1/pb1nbpp1/1qp1p2p/1p6/3PN3/4PN2/PPQ1BPPP/2RR2K1 w - - 8 17,
    moveid=17w
]
\chessboard

In this position, Carlsen sacrificed a pawn to infiltrate with his rook. \index{Pawn Sacrifice!Infiltration}
\index{Endgame!Infiltration}

\mainline[level=1]{17. Nc5 Nxc5 18. dxc5 Rxd1+ 19. Rxd1 Qxc5 20. Qxc5 Bxc5 21. Rd7 }

\chessboard

\mainline{21... Ba8 22. Ne5 Bb6 23. Nxf7 Rc7}

\chessboard

Carlsen could have played \variation[invar]{24. Nxh6}, winning a pawn.
Instead, he preferred to keep the initiative and the activity of his rook.


\chessgameinfo{Candidates qf2}{Tigran Petrosian}{Lajos Portisch}{13}{1974}{1-0}
\newchessgame[
    setfen=3r2k1/p4pp1/1p2n2p/3p4/3P1q1P/1P1Q1NP1/P4PK1/2R5 w - - 1 22,
    moveid=22w
]

\chessboard

Black has just set up a ``trap'' for White.

\mainline[level=1]{
22. gxf4!}

White went for it, sacrificing a pawn. As we will see, his rook soon
infiltrated the enemy camp.

\mainline[level=1]{22... Nxf4+ 23. Kg3 Nxd3 24. Rc3 Nb4 25. a3 Na6 
26. b4 Nb8 27. Rc7}

\chessboard

This was the position Petrosian was aiming for. White had infiltrated his rook to 
the seventh rank while Black was totally passive.

\mainline[level=1]{27... a5 28. b5 Nd7 29. Kf4 h5 30. Ne5!}

\chessboard

White pushed Black's knight even further back. Exchanging the knights would be 
disastrous for Black: \variation[invar]{
    30...  Nxe5 31. Kxe5 f6+ 32. Kf5 Kh7 33. Ke6 Kg6 34. Rc6 
}

White eventually converted his advantage and won the game.

\mainline[level=1]{30... Nf8 31. Rb7 f6 32. Nc6 Ng6+ 33. Kg3 Rd6 34. Rxb6 Re6 35. Rb8+ Nf8 36. Ra8 Re1 37. Nd8 Kh7 38. b6 Rb1 39. b7 Nd7 40. Rxa5
}
\end{multicols}
\newpage
\section{Fish Bone in the Throat}
In King's Indian Defense, White has a thematic idea: placing 
his knight on e6 and sacrificing a pawn to increase his piece activity.
\begin{multicols}{2}
    \chessgameinfo{FIDE World Cup}{Van Wely}{Radjabov}{}{2005.12.03}{1-0}
    \newchessgame
    
    \mainline[level=1]{       
        1. d4 Nf6 2. c4 g6 3. Nc3 Bg7 4. e4 d6 5. Nf3 O-O 6. Be2 e5 7. O-O Nc6 8. d5 Ne7 9. b4 Nh5 10. Re1 f5 11. Ng5 Nf6}
    
    \chessboard

    The knight is aiming for e6. White is ready to sacrifice a pawn to increase his piece activity.
        
    \mainline[level=1]{
        12. f3 Kh8 13. Ne6 Bxe6 14. dxe6}

    \chessboard

    The e6 pawn looks weak. However, Black needs lots of tempi to catch the pawn.

    The White knight is aiming for d5. If Black tries to drive it away, he will weaken his 
    d6 pawn. Black cannot take the pawn with his queen because his c7 pawn hangs.
    
    \mainline[level=1]{14... fxe4 15. fxe4 Nc6 16. Nd5 Nxe4 17. Bf3 Nf6 18. b5 Nxd5 19. bxc6 Nb6 20. cxb7 Rb8 21. c5 e4 22. Rxe4 dxc5 23. Qxd8 Rfxd8 24. Bg5 Re8 25. Rd1 Bd4+ 26. Rexd4 cxd4 27. e7 h6 28. Bf6+ Kg8 29. Rxd4 Kf7 30. Rd8 Nd7 31. Bh4 g5 32. Rxd7 Ke6 33. Bg4+ Kf6 34. Be1 Rxb7 35. Bc3+ Kg6 36. Bf3 Rb1+ 37. Kf2 Rc1 38. Rxc7 Kf5 39. Bh5 Rc2+ 40. Kf3 1-0      
    }
\end{multicols}
\newpage

\section{Pawn Sacrifice: Center}

In the opening or middlegame, a good center usually means 
piece activity. It is therefore a good idea to sacrifice a pawn or two for a center.
\index{Pawn Sacrifice!Center}

\begin{multicols}{2}
    \chessgameinfo{World Championship}{Ding, Liren}{Nepomniachtchi, Ian}{4}{2023.04.13}{1-0}
    \newchessgame[      
        id=main,
        event={FIDE World Championship 2023},
        white={Ding, Liren},
        black={Nepomniachtchi, Ian},
        round={4}]
    \mainline[level=1]{
        1. c4 Nf6 2. Nc3 e5 3. Nf3 Nc6 4. e3 Bb4 5. Qc2 Bxc3 6. bxc3 d6 7. e4 O-O 8. Be2 Nh5 9. d4}

    \chessboard

    There is a fierce fight in the center around d4. An alternative is \variation[invar]{9... Qf6 \xskakcomment{ a natural idea.}
    10. d5 Na5 11. g3 Bg4 } 

    \chessboard[
        setfen=r4rk1/ppp2ppp/3p1q2/n2Pp2n/2P1P1b1/2P2NP1/P1Q1BP1P/R1B1K2R w KQ - 1 12
    ]

    Black has a solid but passive position. It is probably not to Nepo's taste. 

    \mainline[level=1]{9...Nf4 10. Bxf4 exf4 11. O-O Qf6 12. Rfe1 Re8}
    
    \chessboard

    In the middlegame, it is often more about choosing moves and positions according to the 
    players' style than finding the objectively best move.

    White has a solid center and Black has a compact position. Continuing in this manner 
    would suit Ding better because Nepo, as a dynamic player, cannot sit and wait passively.

    \variation[invar]{13. c5 dxc5 14. e5 Qh6 15. Rad1 Bg4 16. Qb3} would be a good idea but not a good practical decision:

    \chessboard[
        setfen=r3r1k1/ppp2ppp/2n4q/2p1P3/3P1pb1/1QP2N2/P3BPPP/3RR1K1 b - - 4 16
    ]

    The position is sharp. White may have some advantage. However, the position 
    is open and Black has counterplay and open lines. What is the point of allowing
    a dynamic player tactical opportunities?

    Ding chooses a natural move, improving his position slowly and not allowing Nepo any counterplay.
    
    \mainline[level=1]{ 13. Bd3 Bg4 14. Nd2}
    
    \chessboard
    
    \mainline[level=1]{14...Na5?}
    
    A very strange move. The knight on a5 has no future. \variation[invar]{14...Rad8} would be a natural move.

    \chessboard
    
    \mainline[level=1]{ 15. c5!} With his sacrifice White activates his pieces.
    \mainline[level=1]{ 15... dxc5 16. e5 Qh6 17. d5 Rad8 18. c4}
    
    \chessboard

    White has some advantage here. His center is strong and Black has 
    a passive knight on a5. 

    While the position may still be equal according to the computer, for human players White has a much easier position to play. He controls the center, while his opponent, as a dynamic player, can only sit and wait. At some point, his opponent would lose patience while defending and make mistakes, as happened in this game. 

    \mainline[level=1]{18... b6 19. h3 Bh5}

    \chessboard[
        pgfstyle=straightmove,
        linewidth=0.05em,
        markmove=f4-f3,
        markstyle=circle,
        linewidth=0.05em,
        markfields={e5},
    ]

    \begin{itemize}
        \item Where are the weaknesses?
    
        The e5 pawn is the pivot of the position and must be protected.
        \item Which is the worst-placed piece?
       
        The rooks must be activated.
        \item What is my opponent's idea?
    
        He wants to play \symknight f3, creating some counterplay.
    \end{itemize}
    
    By answering the questions above, White can find the next few moves:
    \begin{itemize}
        \item Move his bishop to e4 then f3 (if Black exchanges the bishop, White has a knight on f3, which further strengthens the e5 pawn).
        \item Move his queen to c3 to protect the e5 pawn.
        \item Double his rooks on the e-file to protect the e5 pawn.
    \end{itemize}

    Black, however, must play move by move.

    \mainline[level=1]{20. Be4 Re7 21. Qc3 Rde8 22. Bf3 Nb7 23. Re2} 
    
    As mentioned above, White has a clear plan and needs only to execute it, without thinking too much. 

    \chessboard

    Black makes a difficult decision here, allowing White to create a passed pawn but gaining 
    a good square on d6 for his knight.
    
    \mainline[level=1]{23... f6 24. e6 Nd6 25. Rae1} 
    
    \chessboard[
        setfen=4r1k1/p1p1r1pp/1p1nPp1q/2pP3b/2P2p2/2Q2B1P/P2NRPP1/4R1K1 b - - 2 25
    ]

    White has executed his plan and Black has defended well. How should Black defend next?

    I believe the question can be answered logically without calculating a lot.
    The main asset of White is the passed pawns, which have been blocked by the Black rooks.

    The rooks on the e-file are not doing much because there is no open file. The knights are 
    important pieces. The Black knight keeps an eye on e4 so that the rooks cannot attack the 
    f4 pawn. 
    
    At some point, White may play \symrook e4 to attack the f4 pawn and exchange the Black knight.

    Alternatively, White may play \symknight e4 to exchange the Black knight. In either case, e4 is 
    an important square and must be protected in advance. \variation[invar]{25... Bg6} is a good move.

    It is unclear how White can make progress here. In the actual game, Black chooses to defend
    actively and soon makes a severe mistake.
    
    \mainline[level=1]{25... Nf5?!} 
    
    \chessboard

    White's bishop cannot improve White's position, while Black's bishop can defend the important e4
    square. It is therefore logical to exchange the bishops and then occupy the e4 square with the rook.
    
    \mainline[level=1]{ 26. Bxh5 Qxh5 27. Re4 Qh6 }
    
    \chessboard
    
    \mainline{28. Qf3}

    Again, a very logical move, attacking the weak f4 pawn. Ding's moves 
    are natural, although not the perfect computer moves. By playing these moves,
    he sets problems for Nepo that cannot be solved using dynamic play---a 
    very practical choice!
    
    \chessboard

    After \variation[invar]{28... g5 29. g4 Nd6}, the position is still defensible for Black. 

    \mainline[level=1]{28...Nd4?? } 
    
    \chessboard

    I am not sure whether Nepo sees a trap that backfires because Ding has set a deeper one. After \variation[invar]{29. Qxf4 Qxf4 30. Rxf4 c6 31. Nf3 Nxf3 32. Rxf3 cxd5 33. cxd5 Rd8 34. Rd3 Rd6} 
    
    \chessboard[setfen=6k1/p3r1pp/1p1rPp2/2pP4/8/3R3P/P4PP1/4R1K1 w - - 3 35]

    White has no advantage.

    More plausibly, Nepo loses his patience in defense and wants to force a draw. 
    
    \chessboard

    \mainline[level=1]{29. Rxd4! } 
    
    Of course, the knight is much more valuable than the rook.

    \mainline[level=1]{29...cxd4 30. Nb3 g5 31. Nxd4 Qg6 32. g4 fxg3 33. fxg3 h5 34. Nf5 Rh7 35. Qe4 Kh8 36. e7 Qf7 37. d6 cxd6 38. Nxd6 Qg8 39. Nxe8 Qxe8 40. Qe6 Kg7 41. Rf1 Rh6 42. Rd1 f5 43. Qe5+ Kf7 44. Qxf5+ Rf6 45. Qh7+ Ke6 46. Qg7 Rg6 47. Qf8}

    \chessboard
\end{multicols}

% middlegame
% Middle Game Attack stuff
\newpage
%\section{Middlegame Attack}
%\newpage
\section{Middlegame Attack: Greek Gift}

The Greek Gift is a classic attacking motif where White sacrifices a bishop on h7 
(or h2 for Black) to rip open the enemy king's shelter. \index{Attack!Greek Gift}
After Bxh7+, the attacker usually follows with Ng5+ and Qh5, 
aiming for a direct mating attack. It works when the defending king lacks key 
defenders—especially the knight on f6 or sufficient control of g5. When the 
conditions are right, the sacrifice is not speculative; it's a forced assault 
that often leads to a decisive attack or mate.

\begin{multicols}{2}

In the first round of my first tournament (25 minutes, no increment), I played White as the underdog against 
a strong opponent. I was fourteen and he was eleven. Feeling I had little chance 
in a slow strategic game, I decided to attack.

Before the tournament I had learned how Botvinnik beat Capablanca by setting up 
a center with e4 and starting an attack on the kingside. 

The game started. He played his moves instantly.

\newchessgame 

\mainline[level=1]{
    1. d4 Nf6 2. c4 e6 3. Nc3 Bb4 4. e3 O-O 5. Bd3 c5 6. Nf3 Nc6 7. a3 Bxc3+ 8. bxc3 b6
     9. O-O Ba6}

     \chessboard

     During the game I had no idea what he was doing. (Only years later did I understand that his moves b6, \symbishop a6, \symknight c6, \symknight a5
     are typical against White's c4 pawn. The idea was not bad in itself, if only he had protected his pawn.)
     
     
     I just developed my center.

     \mainline[level=1]{ 10. e4 }
     
     \chessboard

     My heart started to beat faster. My entire attacking plan depended on his next move.

     \mainline[level=1]{10...Na5?}
     
     I was relieved. He should have played \variation[invar]{10... d5 11. e5 Ne4} blocking my 
     bishop to stop the attack.

     \mainline[level=1]{11. e5 Ne8}
     
     \chessboard

     Now we reached a typical Greek Gift position.

     \mainline[level=1]{12. Bxh7+ }
     
     \chessboard

     No matter what Black plays now, White has a standard solution.

     \variation[invar]{12... Kxh7 13. Ng5+ Kg8 14. Qh5};
     
     \variation[invar]{12... Kxh7 13. Ng5+ Kg6 14. h4! Rh8 15. h5+! Rxh5 16. Qd3+ f5 17. exf6+ Kxf6 18. Qf3+ Ke7 19. Qf7+ Kd6 20. dxc5+ bxc5 21. Qxh5 }
     
     \mainline[level=1]{12...Kh8 13. Ng5 g6 14. Qf3 Ng7 15. Qh3 Nh5 16. Nxf7+ Rxf7 17. Bxg6 Qe7 18. Qxh5+ 
     } Black resigned.

     When I visited my middle-school math teacher after being admitted to university, I saw 
     his name in the class register and told my teacher how I had beaten him in the tournament.


\end{multicols}
\newpage
\section{MiddlegameAttack: Double Sacrifice}

In the double sacrifice attack, the attacker sacrifices two pieces (typically two bishops) on h7 and g7 (or h2 and g2 for Black) to break down the opponent's king's pawn shield, ultimately delivering checkmate with the queen and rook. \index{Attack!Double Sacrifice}

\begin{multicols}{2}

\chessgameinfo{St. Petersburg}{Aron Nimzowitsch}{Siegbert Tarrasch}{}{1914.04.28}{0-1}
\newchessgame[
    setfen= 3r1rk1/p3qp1p/2bb2p1/2p5/3P4/1P6/PBQN1PPP/2R2RK1 b - - 0 19,
    moveid=19b
]

\chessboard

Black begins by destroying White's defensive pawns on h2 and g2. The move order is crucial: first, a check on h2 forces the king to capture, then the queen delivers a check on h4, and only after the king retreats does the second bishop sacrifice occur on g2.

\mainline[level=1]{19... Bxh2+ 20. Kxh2 Qh4+ 21. Kg1 Bxg2 }

\chessboard

If White accepts the second sacrifice by capturing the bishop, the position remains lost. \variation[invar]{
22. Kxg2 Qg4+ 23. Kh2 Rd5 \xskakcomment{ Black threatens mate. White must give up material.} 24. Qxc5 Rh5+ 25. Qxh5 Qxh5+ 26. Kg2 Qg5+ 27. Kh2 Qxd2 28. Bc3 Qxa2 
}

In the actual game, Black maintains a crushing attack even after White declines the second sacrifice, and the game concludes decisively.
\mainline[level=1]{22. f3 Rfe8 23. Ne4 Qh1+ 24. Kf2 Bxf1 25. d5 f5 26. Qc3 Qg2+ 27. Ke3 Rxe4+ 28. fxe4 f4+ 29. Kxf4 Rf8+ 30. Ke5 Qh2+ 31. Ke6 Re8+ 32. Kd7 Bb5# 0-1
}




\end{multicols}
\input{../cards/chess/ideas/attack/quiet-king-moves}
\newpage
\section{Doubling Rook on G-File from Hedgehog Position}
The Hedgehog is a pawn formation usually adopted by Black that 
can arise from several openings. Black trades the pawn on c5 for White's pawn on d4, then places pawns on a6, b6, d6, and e6. 
These pawns form a row of ``spines'' behind which Black develops pieces. 
Typically, the bishops sit on b7 and e7, knights on d7 and f6, the queen on c7, 
and rooks on c8 and e8 (or c8 and d8).

\chessboard[
    setfen=r4rk1/1bqnbppp/pp1ppn2/8/8/8/8/8 w HAhq - 0 1
]

\begin{multicols}{2}
\chessgameinfo{Exhibition Game}{B. Fischer}{U.Andersson}{}{1970}{1-0}

\newchessgame[
    setfen=r2qrbk1/1pp3pp/2n1bp2/p2np3/8/PP1PPN2/1BQNBPPP/R4RK1 w - - 2 13,
    moveid=13w
]

\chessboard

White reached a color-reversed Hedgehog.
Here Fischer began his original idea: move the king to h1, push g-pawn to g4,
and double rooks on the g-file.

The Black pawn on f6 allows White to play g5 and open the g-file. 

\mainline[level=1]{
    13. Kh1! Qd7 14. Rg1! Rad8 15. Ne4 Qf7 16. g4 g6 17. Rg3! Bg7 18. Rag1!}
    
\chessboard

White plans \symknight h4 and \symknight f5/g5. Black cannot stop the g5 thrust 
with ...g4 because it would weaken the f5 square.

White won the game in 25 moves.


\chessgameinfo{Kislovodsk}{M.Taimanov}{A.Yusupov}{}{1982}{0-1}

\newchessgame[
    setfen=1qr1r1k1/1bbn1ppp/pp1ppn2/8/2P1P3/1NN1BP2/PP4PP/2RR1BQK b - - 0 18,
    moveid=18b
]

\chessboard

Again we have a Hedgehog (color-reversed from the last game), again with a pawn on f3.
It was hard to find a fresh plan for Black. Yusupov knew Fischer's idea and played confidently.

\mainline[level=1]{18... Kh8 19. Rc2 Rg8! 20. Rcd2 g5! 21. Bd4 Rg6! 22. Nc1 Rcg8! 23. Nd3 Qf8}

\chessboard

Black is clearly planning the g4 thrust. 
\mainline[level=1]{24. Re1?}

White has an important resource because
the Black knight cannot jump to h4 as in the previous game due to the pin from the d4 bishop.

\variation[invar]{24. g4} stops Black's pawn thrust. The position stays equal. 


\mainline[level=1]{24...g4 25. fxg4 e5 26. Be3 Nxg4 27. Nd5 Bd8} Black won in the end.

\end{multicols}

\newpage
\section{Attack Along Two Diagonals}

When White castles kingside, he must be careful not 
to face an attack with. 
\begin{itemize}
    \item{Black's knight on g4}
    \item{Black's bishop on c5}
    \item{Black's bishop on b7}
    \item{Black's queen targeting h4}
\end{itemize}
There will be too many mate threats to handle.

\begin{multicols}{2}
\chessgameinfo{Tata Steel-A 75th}{L.Aronian}{V.Anand}{4}{2013.01.15}{0-1}

\newchessgame
    \mainline[level=1]{
    1. d4 d5 2. c4 c6 3. Nf3 Nf6 4. Nc3 e6 5. e3 Nbd7 6. Bd3 dxc4 7. Bxc4 b5 8. Bd3 Bd6 9. O-O O-O 10. Qc2 Bb7 11. a3 Rc8 12. Ng5 c5 13. Nxh7}

    \chessboard

    White has no defender on the king side. Black starts a dangerous attack.
    
    \mainline[level=1]{13...Ng4! 14. f4 cxd4 15. exd4 Bc5 }
    
    \chessboard

    \mainline[level=1]{16. Be2?}
    
    Black has a dangerous bishop on c5. White should have eliminated it with
    \variation[invar]{16. dxc5 Nxc5 17. h3 Nxd3 18. Nxf8 Qd4+ 19. Kh1 Ndf2+ 20. Rxf2 Nxf2+ 21. Kh2 Kxf8}

    \chessboard[
        setfen=2r2k2/pb3pp1/4p3/1p6/3q1P2/P1N4P/1PQ2nPK/R1B5 w - - 0 22
    ]
    White can defend. We go back to the mainline.
    
    \chessboard

    \mainline[level=1]{16...Nde5!}
    
    Black threats with \variation[invar]{17... Qd4+ 18. Kh1 Nf2+ 19. Kg1 Nh3+ 20. Kh1 Qg1+! 21. Rxg1 Nf2#}
    
    \mainline[level=1]{17. Bxg4 Bxd4+ 18. Kh1 Nxg4}
    
    We arrive the position mentioned at the beginning of this section.
    \mainline[level=1]{19. Nxf8}
    
    \chessboard

    \variation[invar]{19... Qh4?? 20. Qh7+} Black loses all the attack. 
    
    Black cuts off White's queen with his next move. 
    
    \mainline[level=1]{19...f5! 20. Ng6 Qf6 21. h3 Qxg6 22. Qe2 Qh5 23. Qd3 Be3! 
}  0-1

\chessboard

Let's remind ourselves of the finish to ``Rubinstein's Immortal''.

\chessgameinfo{}{G.Rotlewi}{A.Rubinstein}{}{1907.12.26}{0-1}

\newchessgame[
    setfen=2rr2k1/1b2qppp/pb2pn2/1p2P3/1P3P2/P1NB4/1B2Q1PP/R4R1K b - - 2 20,
    moveid=20b
]

\chessboard

\mainline[level=1]{20...Ng4 21. Be4 Qh4 22. g3 Rxc3!! 23. gxh4 Rd2! 24. Qxd2 Bxe4+ 25. Qg2 Rh3 26. Bd4 Bxd4} 0-1

\end{multicols}
\newpage
\section{Dynamic G-Pawn Play}
In some positions one side has static advantage and 
the other side has dynamic advantage. The side with 
dynamic advantage must seize the initiative without delay before the 
other side can neutralize it. A dynamic g-pawn thrust 
is often an excellent idea.

\begin{multicols}{2}
    \chessgameinfo{Candidate Quarter Final}{J.Timman}{A.Jusupov}{}{1986}{1/2-1/2}

    \newchessgame[
        setfen=4k2r/p2n1ppp/Npr1p3/3pP3/3p1P2/8/P2N2PP/R3K2R b KQk - 0 19,
        moveid=19b
    ]
    \chessboard

    The material is almost equal. If White can play \symknight f3 and O-O, he 
    will have some advantage. Black must hurry.

    \mainline[level=1]{19... g5!}

    After \variation{20. g3 gxf4 21. gxf4 Rg8} both Black rooks are very active.

    \mainline[level=1]{20. O-O gxf4 21. Rxf4 Nxe5}

    White's center is destroyed. Timman decided to return the piece because 
    in endgame the knight is inferior to the pawns. The game ended in a draw.

    \mainline[level=1]{22. Rxd4 Rg8 23. Nb4 Rc3 24. Nxd5 exd5 25. Rxd5 Rc5 26. Rxc5 bxc5 27. Ne4 Ke7 28. Nxc5 Rc8 29. Nb3 Rc2 30. Nd4 Rc4 31. Re1 f6 32. Nf3 Ke6 33. Nxe5 fxe5 34. Re3 Ra4 35. Rh3 Rxa2 36. Rxh7 e4 37. h4 e3 38. Kf1 a5 39. h5 Rf2+ 40. Ke1 Rxg2 }


    \chessgameinfo{Vinkovci}{M.Taimanov}{B.Larsen}{13}{1970}{0-1}

    \newchessgame[
        setfen=r1b2rk1/pp2bppp/2n5/q2p4/5B2/PQN1PN2/1P3PPP/2R1K2R b Kq - 0 14,
        moveid=14b
    ]

    \chessboard

    The last move \variation[invar]{14. Qb3} attacks the weak pawn. 

    If White can play O-O, he will consilidate his position and Black will have 
    difficulty to develop his c8 bishop. 

    ``Indeed, the d5 pawn can be easily be defended by ...\symrook d8,
    but can it be said that this would have ensured Black an easy life? After all, 
    the queen move also nailed down the c8 bishop, as in the event of the invasion of 
    the White queen at b7 the standard reply ...\symrook b8 is not possible.'' (T. Petrosian)
    
    Larsen decided to play dynamically:
    \mainline[level=1]{
        14...g5! 15.Bg3 g4 16.Nd4 Nxd4 17.exd4 Bg5!}

    \chessboard

    Black won the game after a tactical struggle.
    \mainline[level=1]{
        18.O-O Bxc1 19.Rxc1 Be6 20.h3 gxh3 21.Be5 f6 22.Ne4 fxe5
        23.Qg3+ Bg4 24.Qxg4+ Kh8 25.Ng5 Qd2 26.Rc7 Qxf2+ 27.Kh2 Qxg2+
        28.Qxg2 hxg2 29.dxe5 Rac8 30.Rxb7 Rc2 31.Nf7+ Kg7 32.e6 Kf6
        33.e7 g1=Q+ 34.Kxg1 Rg8+
    } 0-1
\end{multicols}

% Prophylactic
\newpage
\section{Prophylactic King Move}

Sometimes the opponnent needs a crucial check to achieve his 
goal. It is helpful to move the king to a safe square to 
eliminate the opponent's check.

\nocite{Dvoretsky:1996}

\begin{multicols}{2}
\chessgameinfo{Linares}{J.Nunn}{A.Jusupov}{}{1988}{0-1}
\newchessgame[
    setfen=r4rk1/p3npp1/4p2p/n2pP3/R7/2qBR3/2PN1PPP/3Q2K1 b - - 0 18,
    moveid=18b
]
\chessboard

White threatens to play \symbishop h7+, winning the queen. 

If Black moves his queen away, for example \variation[invar]{18... Qc7} White can play 19. \symqueen c5 
and then \symrook g3, maintaining the attack.

\mainline[level=1]{18... Kh8!}

``Despite looking rather awkward on c3 the queen at least attacks the d2-knight and so
limits the movement of the White queen.'' (Dvoretsky)

\mainline[level=1]{19. g4 Nac6 20. Nf3 Rab8 21. Bc4 Qb2 22. Bb3 Ng6 23. Ra2 Rxb3 24. Rxb2 Rxb2 }

\chessboard

Black has compensation for his queen. He wins the game eventually.


\chessgameinfo{Mar del Plata}{J.Timman}{B.Larsen}{}{1982}{0-1}
\newchessgame[
    setfen=1r2rbk1/2q3p1/p1p1bn1p/1pP1p3/8/1PN3P1/PBQ1PPB1/2RR2K1 b - - 0 23,
    moveid=23b
]

\chessboard

\mainline[level=1]{23... Kh8!}

\variation[invar]{23... Bxc5 24. Ne4 Nxe4 25. Qxe4} White will also gain 
the c6 pawn and have advantage.

\mainline[level=1]{24. e3}

Now \variation[24. Ne4 Bf5]. White cannot take the f6 knight since there is 
no check and his queen is hanging.

The game continues and Black wins eventually.
\end{multicols}
	\nocite{Dvoretsky:2016}
\nocite{Aagaard:2002}


\ifdefined\chessproblem
    % The command is already defined, do nothing.
\else
    \newcommand{\chessproblem}[3]{
        \subsection*{#1. #2}
        \chessboard[
            setfen=#3,
        ]
    }
\fi
\ifdefined\conquestNunn
    % The command is already defined, do nothing.
\else
    \newcommand{\conquestNunn}[1]{
	\chessproblem{#1}{Conquest - Nunn}{5rk1/pp3ppp/5b2/2p1pb2/3q4/2NP2P1/PPP3KP/R2QR3 b - - 0 1}

    }
\fi

\ifdefined\conquestNunnAnswer
    % The command is already defined, do nothing.
\else
    \newcommand{\conquestNunnAnswer}[1]{
		\subsection*{#1. Conquest - Nunn}
 \chessboard[
        setfen=5rk1/pp3ppp/5b2/2p1pb2/3q4/2NP2P1/PPP3KP/R2QR3 b - - 0 1,
 	markstyle=circle,
 	linewidth=0.05em,
 	markfields={g2},
	markstyle=border,
 	markfields={f5}
        ]
	\begin{itemize}
		\item{Where are the weaknesses?}

		White king as well as his main diagonal is weak.
		\item{Which is the worst-placed piece?}

		The light-squared bishop does nothing.
		\item{What is my opponent's idea?}

		He can only wait passively.
	\end{itemize}

	By answering the template question above, the correct move is \symbishop d7, improving the piece as well as exploiting White's weakness on the main diagonal.
    }
\fi





\ifdefined\fischerKeres
    % The command is already defined, do nothing.
\else
    \newcommand{\fischerKeres}[1]{
	\chessproblem{#1}{Fischer - Keres}{r5k1/1pq2ppp/2rb1n2/4n2P/p2pPB2/P2P2PB/R1P1Q3/1N3RK1 b - - 0 1}
    }
\fi

\ifdefined\fischerKeresAnswer
    % The command is already defined, do nothing.
\else
    \newcommand{\fischerKeresAnswer}[1]{
	\subsection*{#1. Fischer - Keres}
 \chessboard[
        setfen=r5k1/1pq2ppp/2rb1n2/4n2P/p2pPB2/P2P2PB/R1P1Q3/1N3RK1 b - - 0 1,
 	markstyle=circle,
 	linewidth=0.1em,
 	markfields={c2},
	markstyle=border,
 	markfields={a8}
        ]

	\begin{itemize}
		\item{Where are the weaknesses?}

		White has weakness on c2.
		\item{Which is the worst-placed piece?}

		The rook on a8 does nothing
		\item{Whas is my opponent's idea?}

		White has no active plan so black can choose his own maneuver.
	\end{itemize}

	\symrook c8 is not possible since white's bishop controls c8. \symrook a5 is natural because the 
	rook targets c5 aiming the weakness c2 in the next move.

	Serendipitously, the move also aims h5. White must figure out how to defend. At 
	the moment we don't need to concern how white will defend. Let's pass the ball to 
	white and decide then what to do.
    }
\fi





\ifdefined\petrosianBannik
    % The command is already defined, do nothing.
\else
    \newcommand{\petrosianBannik}[1]{
	\chessproblem{#1}{Petrosian - Bannik}{3r3r/ppk1b2p/1np2p2/4p1pP/2P1N3/1P2B1P1/P3PP2/2KR3R w - - 0 1}
    }
\fi

\ifdefined\petrosianBannikAnswer
    % The command is already defined, do nothing.
\else
    \newcommand{\petrosianBannikAnswer}[1]{
	\subsection*{#1. Petrosian - Bannik}
 \chessboard[
        setfen=3r3r/ppk1b2p/1np2p2/4p1pP/2P1N3/1P2B1P1/P3PP2/2KR3R w - - 0 1,
 	markstyle=circle,
 	linewidth=0.05em,
 	markfields={f6,e6},
	markstyle=border,
 	linewidth=0.05em,
 	markfields={e3},
        ]
	\begin{itemize}
		\item{Where are the weaknesses?}

		Black has weaknesses on e6 and f6.
		\item{Which is the worst-placed piece?}

		The bishop on e3 is inactive.
		\item{What is my opponent's idea?}

		Black has no active plan, so White can choose his own maneuver.
	\end{itemize}
	The best move is \symbishop c5 to exchange the inactive bishop and exploit Black's weakness on f6 by moving the king to e6. Let's see what Petrosian has to say about this position. More than the move, it's his comments that made a deep impression.
	\begin{quote}
		``In deciding on this move, it was imperative to weigh all the pros and cons thoroughly. The move looks illogical as White is voluntarily exchanging his good bishop for his opponent's bad one, instead of swapping the bishop for knight and securing his preponderance. However, if you probe into the position a little more deeply, it becomes obvious that after a possible exchange of rooks on the d-file and the transfer of king to e6, Black would cover his vulnerable points and create an impregnable formation. The role played in this by the bad bishop would be of no small importance. After 1.\symbishop c5 \symrook xd1 2.\symrook xd1 \symbishop xc5 3.\symknight xc5 White was threatening infiltration on e6 and after 3...\symrook e8 4.\symknight e4 \symrook e6 5.g4 He was clearly better as the f6 pawn is very weak.'' \cite{Petrosian:2015}
	\end{quote}
    }
\fi
\ifdefined\bebchukBakulin
    % The command is already defined, do nothing.
\else
    \newcommand{\bebchukBakulin}[1]{
	\chessproblem{#1}{Bebchuk - Bakulin}{r1b1k2r/bpp1nppp/p2p1q2/P2P4/7P/1N6/1PP1QPP1/R1B1KB1R w - - 0 1}
    }
\fi

\ifdefined\bebchukBakulinAnswer
    % The command is already defined, do nothing.
\else
    \newcommand{\bebchukBakulinAnswer}[1]{
	\subsection*{#1. Bebchuk - Bakulin}
 	\chessboard[
        	setfen=r1b1k2r/bpp1nppp/p2p1q2/P2P4/7P/1N6/1PP1QPP1/R1B1KB1R w - - 0 1,
 		markstyle=circle,
 		linewidth=0.05em,
 		markfields={e8,e7,f6},
 		markstyle=border,
 		linewidth=0.05em,
 		markfields={a1},
        ]

	\begin{itemize}
		\item{Where are the weaknesses?}

		Black needs to castle. His e-file is weak and his queen is in danger.
		\item{Which is the worst-placed piece?}

		The rook on a1 is inactive. 
		\item{What is my opponent's idea?}

		He wants to castle.
	\end{itemize}

	Trying to catch the queen doesn't work \variation[invar]{1.Bg5 Qe5} since White must exchange the queen. 
	\symrook a4 is a natural idea, intending \symrook e4 to catch the queen.
	
	\variation[level=1]{1. Ra4 O-O 2. Rf4 Bf5 3. g4 Rae8 4. Kd1 Qe5 5. Qxe5 dxe5 6. Rxf5 Nxf5 7. gxf5 Rd8 8. Bg2 Bxf2 9. Ke2} 
	
	White has some advantage. Black could have tried \symbishop f5. Let's play \symrook a4 and let Black find the right defense.
	
    }
\fi
\ifdefined\gelfandAnand
    % The command is already defined, do nothing.
\else
    \newcommand{\gelfandAnand}[1]{
	\chessproblem{#1}{Gelfand - Anand}{1r2k2r/pb1n1ppp/4p3/2q5/Q3B3/4P3/1P1N1PPP/R4RK1 b - - 0 18}
    }
\fi

\ifdefined\gelfandAnandAnswer
    % The command is already defined, do nothing.
\else
    \newcommand{\gelfandAnandAnswer}[1]{
        \subsection*{#1. Gelfand - Anand}
        \chessboard[
            setfen=1r2k2r/pb1n1ppp/4p3/2q5/Q3B3/4P3/1P1N1PPP/R4RK1 b - - 0 1,
            markstyle=circle,
            linewidth=0.05em,
            markfields={d7,d2},
            markstyle={border},
            markfields={h8}
                ]

        \begin{itemize}
            \item{Where are the weaknesses?}
            
            Both knight on d2 and d7 are weak.
            \item{Which is the worst-placed piece?}
            
            The rook on h8 does nothing.
            \item{What is my opponent's idea?}
            
            White wants to attack the king in the middle. Maybe with \symrook ac1, \symbishop xb7, \symrook c8. 
        \end{itemize}

        The position is sharp. Black must find a way to solve his problem of his weak knight. 

        The best way to solve the problem is to get rid of it.
        \variation{19... O-O 20. Qxd7 Rfd8 21. Bxh7 Kf8! 22. Qa4 Rxd2} Black has active pieces, which is enough
        for the pawn. All of White's pieces could find better squares.
        
        Aagaard wrote that if one could see both \variation{21. Bxh7} and \variation{21...Kf8}, he is ready for a tournament. Are you ready?
    }
\fi

\ifdefined\morovicKarpov
    % The command is already defined, do nothing.
\else
    \newcommand{\morovicKarpov}[1]{
	\chessproblem{#1}{Morovic - Karpov}{r4rk1/pp1qnpbp/2pb1p1/4p3/1PP1P3/P1N1P1P1/1B2QPBP/R4RK1 w - - 0 1}
    }
\fi

\ifdefined\morovicKarpovAnswer
    % The command is already defined, do nothing.
\else
    \newcommand{\morovicKarpovAnswer}[1]{
		\subsection*{#1. Morovic - Karpov}
		\chessboard[
			setfen=r4rk1/pp1qnpbp/2pb1p1/4p3/1PP1P3/P1N1P1P1/1B2QPBP/R4RK1 w - - 0 1,
			markstyle=circle,
			linewidth=0.05em,
			markfields={c4},
			markstyle=border,
			linewidth=0.05em,
			markfields={e7},
			pgfstyle=straightmove,
			markmove={e7-c8,c8-b6,b6-c4},
		]

		\begin{itemize}
			\item{Where are the weaknesses?}

			White has weakness on c4.
			\item{Which is the worst-placed piece?}

			The knight on e7 is inactive. 
			\item{Whas is my opponent's idea?}

			Not clear. He has difficulty to evolve his position.
		\end{itemize}

		White has weakness on c4. e7 knight is the worst piece. 
		So moving it aiming for c4 is a natural idea. \symknight c8 is the answer.
	
    }
\fi
\ifdefined\sanakoevLungdal
    % The command is already defined, do nothing.
\else
    \newcommand{\sanakoevLungdal}[1]{
	\chessproblem{#1}{Sanakoev - Lungdal}{2r1k2r/1b3ppp/p3p3/2qpP3/1p1Q1P2/2P5/PP2B1PP/R2R2K1 w - - 0 1}
    }
\fi

\ifdefined\sanakoevLungdalAnswer
    % The command is already defined, do nothing.
\else
    \newcommand{\sanakoevLungdalAnswer}[1]{
		\subsection*{#1. Sanakoev - Lungdal}
        \newchessgame[
            setfen=2r1k2r/1b3ppp/p3p3/2qpP3/1p1Q1P2/2P5/PP2B1PP/R2R2K1 w - - 0 1,
            moveid=1w
        ]
		\chessboard[		
			markstyle=circle,
			linewidth=0.05em,
			markfields={b7},
			markstyle=border,
			linewidth=0.05em,
			markfields={a1},
		]

		\begin{itemize}
			\item{Where are the weaknesses?}

			The bishop on b7 is unprotected.
			\item{Which is the worst-placed piece?}

			The rook on a1 is inactive. 
			\item{Whas is my opponent's idea?}

			Black may want to castle kingside and exchange the queens.
		\end{itemize}

		\symrook ab1 is the right move.

        According to Dvoretsky, the move deserves two exclamation marks. We can
        however find the answer with logic and simple calculation.

        White is leading in the development so he should have the advantage.
        The queen is pinned so there is no way to start an attack with a queen as usual when one leads in development.

        \variation{1. cxb4 Qxd4 2. Rxd4 Rc2 3. Bd3 Rxb2 4. Rc1 Kd7 5. Rc2 Kd7 6. Rc2 Rxc2 7. Rxc2} Black
        simplifies the position and equalizes. 

        White fails to protect the pawn on b2 after taking the pawn on b4. So the answer is simple:
        \mainline[level=1]{1. Rab1} threatening cxb4. 
        
        \mainline{1... Qxd4 2. Rxd4 bxc3 3. bxc3 } White wins a tempo.
        
        \mainline{3... Rc7 4.Rdb4} White has the only open file and has the advantage. Black still needs to castle. His bishop is hopeless.        
        
        \chessboard
    }
\fi
\ifdefined\zlotnikLopes
    % The command is already defined, do nothing.
\else
    \newcommand{\zlotnikLopes}[1]{
	\chessproblem{#1}{Zlotnik - Lopes}{r1b1kb1r/ppp3qp/3p1pp1/2nN4/2P2B2/6P1/PP2PPBP/R2QK2R w - - 0 1}
    }
\fi

\ifdefined\zlotnikLopesAnswer
    % The command is already defined, do nothing.
\else
    \newcommand{\zlotnikLopesAnswer}[1]{
		\subsection*{#1. Zlotnik - Lopes}
        \newchessgame[
            setfen=r1b1kb1r/ppp3qp/3p1pp1/2nN4/2P2B2/6P1/PP2PPBP/R2QK2R w - - 0 1,
            moveid=1w
        ]
		\chessboard[		
			markstyle=circle,
			linewidth=0.05em,
			markfields={f6, g7, h8, c7},
			markstyle=border,
			linewidth=0.05em,
			markfields={f4},
		]

		\begin{itemize}
			\item{Where are the weaknesses?}

			The pawns on f6, queen on g7 and rook on h8 are vulnerable. Pawn on c7 is also under attack. The black queen is busy with protecting two weaknesses c7 and f6.
			\item{Which is the worst-placed piece?}

			The bishop on f4 is inactive. 
			\item{Whas is my opponent's idea?}

			A plan is hard to plan. He may want to castle.
		\end{itemize}

		\symbishop d2 is the right move.

        The bishop on f4 does nothing and should be moved to c3, controlling the main diagonal and attacking
        the weaknesses.
    }
\fi
\ifdefined\carlsenDing
    % The command is already defined, do nothing.
\else
    \newcommand{\carlsenDing}[1]{
	\chessproblem{#1}{Carlsen - Ding}{2kr2r1/pp1b1pp1/1qn1pn1p/3p4/1P6/B1PB1N2/P3QPPP/R4RK1 b - - 0 15}
    }
\fi

\ifdefined\carlsenDingAnswer
    % The command is already defined, do nothing.
\else
    \newcommand{\carlsenDingAnswer}[1]{
	\subsection*{#1. Carlsen - Ding}
    \newchessgame[
        setfen=2kr2r1/pp1b1pp1/1qn1pn1p/3p4/1P6/B1PB1N2/P3QPPP/R4RK1 b - - 0 15,
        moveid=15b
    ]
 	\chessboard[     
        pgfstyle=straightmove,
        linewidth=0.05em,
        markmove={b4-b5},
        markstyle=circle,
        linewidth=0.05em,
        markfields={e5},
        markstyle=border,
        linewidth=0.05em,
        markfields={d7},
        ]

	\begin{itemize}
		\item{Where are the weaknesses?}

		e5 square will be weak after White plays b5 driving away the knight on c6.
		\item{Which is the worst-placed piece?}

		The bishop on d7 is inactive. 
		\item{What is my opponent's idea?}

		b5.
	\end{itemize}

	White threatens to play b5 to drive away the knight on c6 and then control the e5 square with his knight.
Once White achieves his goal, Black has a desperate position. Therefore he must react now!

Understanding White's idea, Black can play \variation[invar]{15... e5 16. b5 e4 17. bxc6 Qxc6 18. Ne5 Qc7 
19. Nxf7 Bg4 20. Qe3 Qxf7 21. Bc2} 

    In the actual game, Ding chose to play \mainline[level=1]{15... Kb8? 16. b5 \xskakcomment{ Of course!}} White has an
    advantage and wins the game in a few moves.
	
    }
\fi



\ifdefined\lelchukVoronova
    % The command is already defined, do nothing.
\else
    \newcommand{\lelchukVoronova}[1]{
	\chessproblem{#1}{Lelchuk - Voronova Analysis}{1b3q1r/3Q3p/pp1n1pp1/2pR4/P3pPPk/2P1N3/2P4P/7K w - - 0 10}
    }
\fi


\ifdefined\lelchukVoronovaAnswer
    % The command is already defined, do nothing.
\else
    \newcommand{\lelchukVoronovaAnswer}[1]{
		\subsection*{#1. Lelchuk - Voronova Analysis}
		\chessboard[
			setfen=1b3q1r/3Q3p/pp1n1pp1/2pR4/P3pPPk/2P1N3/2P4P/7K w - - 0 10,
			markstyle=circle,
			linewidth=0.05em,
			markfields={h4},
			markstyle=border,
			linewidth=0.05em,
			markfields={d7},
		]

		\begin{itemize}
			\item{Where are the weaknesses?}

			Black king on h4 is completely cut off.
			\item{Which is the worst-placed piece?}

			The White queen on d7.
			\item{What is my opponent's idea?}

			Not clear. 
		\end{itemize}

		The basic idea is to activate the White queen to attack the Black king.

        \variation{ 10. Qe6!!}

        This move has some very deep idea.    
        
        \begin{enumerate}
            \item{\variation[invar]{10... c4}}
        
            \chessboard[
                setfen=1b3q1r/7p/pp1nQpp1/3R4/P1p1pPPk/2P1N3/2P4P/7K w - - 0 11
            ]

            White intends to cut off the king completely and mate with \symqueen d5-d1 maneuver.

            \variation[invar]{11. Rh5+! gxh5 12. Ng2+ Kh3 13. g5+ f5 14. Qd5 h4 15. Qd1 Rg8 16. Ne3 Rxg5 17. Qf1+ Rg2 18. Qxg2# }
            
            \chessboard[setfen=1b3q2/7p/pp1n4/5p2/P1p1pP1p/2P1N2k/2P3QP/7K b - - 0 18]
        
            \item{\variation[invar]{10... f5}}

            \variation[invar]{
                10...f5 11. Rd3!}
                
                \chessboard[setfen=1b3q1r/7p/pp1nQ1p1/2p2p2/P3pPPk/2PRN3/2P4P/7K b - - 1 11]

                Black is doomed.

                \begin{enumerate}
                    \item {\variation[invar]{11...Nc4}}
                    
                    \variation{11...Nc4 12. Ng2+ Kxg4 13. Rg3+ Kh5 14. Rh3+ Kg4 15. Qxc4 Bxf4 16. Qf1 Be5 17. Nf4 Kg5 18. Ne6+ Kf6 }
                    \item {\variation[invar]{11...exd3}}

                    \variation{11... exd3 12. Ng2+ Kxg4 13. Qe1 Qe7 14. Qg3+ Kh5 15. Qh3+ Qh4 16. Qxh4# }
                \end{enumerate}
        \end{enumerate}
    }
\fi
\ifdefined\laskerSteinitz
    % The command is already defined, do nothing.
\else
    \newcommand{\laskerSteinitz}[1]{
	\chessproblem{#1}{Lasker - Steinitz}{4r2n/1p2qpkP/p1p3p1/3p4/8/3Br3/PPPQ4/2K2R1R w - - 1 29}
    }
\fi

\ifdefined\laskerSteinitzAnswer
    % The command is already defined, do nothing.
\else
    \newcommand{\laskerSteinitzAnswer}[1]{
	\subsection*{#1. Lasker - Steinitz}
 	\chessboard[
        setfen=4r2n/1p2qpkP/p1p3p1/3p4/8/3Br3/PPPQ4/2K2R1R w - - 1 29,
    ]

	\begin{itemize}
		\item{Where are the weaknesses?}

		White has potential back-rank weakness before starting an attack.

		\item{What is my opponent's idea?}

		He wants to activate his knight at the corner.
	\end{itemize}

    \variation{29. a3} is the right move. 

    White intends to play both \symking b1 and a3, but the order matters critically. 
    Calculating the optimal sequence through brute force is practically impossible here. 
    However, by comparing the consequences of each move, one can deduce that a3 must be played first, 
    because \symking b1 creates a potential back-rank weakness that restricts White's options.

    }

\ifdefined\dingAronian
    % The command is already defined, do nothing.
\else
    \newcommand{\dingAronian}[1]{
        \chessproblem{#1}{Ding - Aronian}{1r1q1rk1/3nbppp/2p1pn2/ppP5/1PpP4/P1N1P3/RB2QPPP/4NRK1 w - - 0 16}
    }
\fi

\ifdefined\dingAronianAnswer
    % The command is already defined, do nothing.
\else
    \newcommand{\dingAronianAnswer}[1]{
        \subsection*{#1. Ding - Aronian}
        \newchessgame[
            setfen=1r1q1rk1/3nbppp/2p1pn2/ppP5/1PpP4/P1N1P3/RB2QPPP/4NRK1 w - - 0 16,
            moveid=16w
        ]
        \chessboard[
            markstyle=circle,
            linewidth=0.05em,
            markfields={d6},
            pgfstyle=straightmove,
            markmove={e4-e5, e3-e4},
        ]

        \begin{itemize}
            \item{Where are the weaknesses?}
    
            d6 square is weak.
            \item{Which is the worst-placed piece?}
    
            Here we are talking about how to regroup the pieces. 
            White must find a way to exploit the weakness on d6.
            This can be done by placing the bishop on f4, the knight on e4. 
            The pawn on e3 must be placed on e5 at some point.

            \item{What is my opponent's idea?}
            
            He may look for counterplay along the a-file.
            He may also attack the d4 pawn. Therefore both of them must be protected.
        \end{itemize}

        It is always important to remember, don't rush!

        \mainline[level=1]{16. e4 Rb7 17. Nc2 Nb8 18. Raa1 Qc8 19. Rad1 Rd8 20. Bc1 Na6 21. Bf4 Rbd7 22. h3 Ne8 23. Qe3 Bf6 24. e5! Be7 25. Ne4!
 Nac7 26. Nd6}
 
        \chessboard

    }
\fi

\newpage
\chapter{Maneuvering}
\section{Excercises}
\begin{multicols}{2}
	\conquestNunn{1}
	\fischerKeres{2}
	\petrosianBannik{3}
	\bebchukBakulin{4}
	\gelfandAnand{5}
	\morovicKarpov{6}
	\sanakoevLungdal{7}
	\zlotnikLopes{8}
	\carlsenDing{9}
	\lelchukVoronova{10}
	\dingAronian{11}

	\subsection*{Prophylaxis}
	\laskerSteinitz{1}
\end{multicols}

\newpage
\section{Answers}
\begin{multicols}{2}
	\conquestNunnAnswer{1}
	\fischerKeresAnswer{2}
	\petrosianBannikAnswer{3}
	\bebchukBakulinAnswer{4}
	\gelfandAnandAnswer{5}
	\morovicKarpovAnswer{6}
	\sanakoevLungdalAnswer{7}
	\zlotnikLopesAnswer{8}
	\carlsenDingAnswer{9}
	\lelchukVoronovaAnswer{10}
	\dingAronianAnswer{11}

	\subsection*{Prophylaxis}
	\laskerSteinitzAnswer{1}
\end{multicols}
	\chapter{My Games}
\newpage
\section{Exciting Fight}

\multicols{2}
    \chessgameinfo{Correspondence Chess 960}{Ming}{Marius}{}{2025.01.17-2025.02.26}{1-0}

    \newchessgame[
        setfen=rnnkqbbr/pppppppp/8/8/8/8/PPPPPPPP/RNNKQBBR w HAha - 0 1,
    ]
    \chessboard

    The position is quite similar to the standard opening position:
    queen and king are in the middle while the rooks are on the edges.


    \mainline[level=1]{1. e4 e5 2. Bc4?!} 
    
    I dislike this move now that I am annotating. During the game, I was 
    thinking of a setup similar to the London System:

    \chessboard[setfen=8/8/8/8/2B1P3/2NPBP2/PPP1N1PP/R2KQ2R ]

    Schematic thinking in itself is not bad. However, such a setup that requires many tempi can be
    easily disturbed. 

    Observing the opening position: it is necessary to move the e- and f-pawns to develop the 
    bishops. \variation[invar]{2. f4} in the spirit of the Queen's Gambit is called for. 
    The engine recommends \variation[invar]{2. f4 f6 3. Nc3 Nc6 4. Nd3 Nd6 5. O-O-O O-O-O 6. g3 h5 7. Kb1 Kb8 8. Bf2 }

    \mainline[level=1]{2... Nd6! 3. Bb3 f6!}
    
    \chessboard

    Marius played natural moves that refuted my idea. My bishop will eventually 
    be exchanged. During the game, I was not happy with this development. The game must go on, 
    so we continued with development moves.

    \mainline{4. Nc3 Nc6 5. f3 Be6 6. N1e2 }

    \chessboard[
        pgfstyle=straightmove,
        linewidth=0.05em,
        markmove={d2-d4},
        markstyle=circle,
        linewidth=0.05em,
        markfields={g2},
        markstyle=border,
        linewidth=0.05em,
        markfields={f8},   
    ]

    \begin{itemize}
		\item{Where are the weaknesses?}

		g2 pawn.
		\item{Which is the worst-placed piece?}

		f8 bishop is inactive. 
		\item{What is my opponent's idea?}

		He wants to play d4.
	\end{itemize}
    \mainline[level=1]{6...O-O-O?!} 
    
    I was happy to see this move during the game because I saw that I might start 
    an attack along the a7-g1 diagonal with a battery. At some point, my knight can move
    to b5. 
    \variation[invar]{6... Qg6} would have been more energetic.  
    \mainline[level=1]{ 7. d3 }
    
    I cannot explain why I did not play d4 directly. The reason may be that I was so obsessed 
    with the attack on the a7-g1 diagonal that I wanted to place my queen in front of my bishop.
    
    \mainline[level=1]{7... Qg6 8. Qf2 Kb8} 
    
    \chessboard

    It is difficult to formulate a concrete plan for White. The problem is that the queen and 
    bishop battery has no future for the attack. \variation[invar]{9.Nb5} is and will remain impossible 
    as long as there is a Black knight on d6. The h1 rook cannot be activated because 
    of the battery. 
    
    \mainline[level=1]{ 9. d4?!}
    
    This move is too committal. White is not ready for an attack because Black's defenders 
    are well placed. A move like d4 will open the center and therefore change the game completely from 
    slow maneuvering to rapid piece contact. White is not ready for this yet. 
    
    A waiting move like \variation[invar] {9. h4} would have been better.
    
    \mainline[invar]{9...exd4 10. Nxd4 Nxd4 11. Qxd4 b6 12. Qf2 Nc4 13. O-O-O}
    
    
    \mainline[level=1]{13...Bc5 }
    
    Finally, I can forget about my battery and the attack. Ironically, this move forces 
    me to make a good move. Then Black must decide what to do with his attacked knight 
    on c4. In a few moves, we will see that Black makes a mistake.
    
    \mainline{14. Qe2 Qg5+ 15. Kb1}
    
    \chessboard

    \mainline[level=1]{15...Ne3?}
    
    Marius overlooked the fact that the knight on e3 was in danger.
    I also overlooked it during the game. \variation[invar]{15... Bxg1} would have been better.
    
    \chessboard[
        setfen=1k1r3r/p1pp2pp/1p2bp2/2b3q1/4P3/1BN1nP2/PPP1Q1PP/1K1R2BR w - - 8 16
    ]
    
    \mainline[level=1]{16. Qa6?!}
    
    During the game, I was happy to find this route/rout. Black cannot take the rook because of
    \variation[invar]{17. Nb5}, leading to mate in several moves. 

    The problem with this move was that it helped Black solve his e3 knight problem. 
    
    \variation[invar]{16. Re1} would have been better. Black cannot defend two weaknesses 
    simultaneously: the knight on e3 and the weakness around his king. For example:

    \variation[invar]{16. Re1 Bxb3 17. axb3 Nxg2 18. h4 Qg6 19. Qa6 }

    \chessboard[
        setfen=1k1r3r/p1pp2pp/Qp3pq1/2b5/4P2P/1PN2P2/1PP3n1/1K2R1BR b - - 2 19
    ]

    White threatens \variation[invar]{20. Nb5}. Therefore \variation[invar]{19... c6 20. Bxc5 bxc5 21. Reg1 Qg3 22. Qe2 Nf4 23. Qf1 }. 
    
    \chessboard[
        setfen=1k1r3r/p2p2pp/2p2p2/2p5/4Pn1P/1PN2Pq1/1PP5/1K3QRR b - - 5 23
    ]
    
    The Black queen is trapped.

    We go back to the mainline.
    \mainline[level=1]{16...c6 17. Bxe3 Bxe3 18. Rd3 Bxb3 19. axb3 }
    
    \chessboard

    During the game, I was looking for something like \symknight d5 at some point. 
    The idea was ambitious. However, I could play \symknight a4, threatening to sacrifice on b6.
    Here we can also see that \variation[invar]{15... Ne3} helped White solve his h1 rook
    problem. \symrook hd1 is now in play. 

    \mainline[level=1]{19...d6}

    Another possibility would be \variation[invar]{
        19...Qxg2 20. Rhd1 Bf4 21. Na4 Qg5 22. Nxb6! axb6 23. Qxb6+ Kc8 24. Qa6+ Kc7 25. Rxd7+ Rxd7 26. Qa7+ Kc8 27. Rxd7 Qb5 28. Rxg7 c5 29. Qa8+ Qb8 
    }

    \chessboard[
        setfen=2k4r/Q5Rp/5p2/1qp5/4Pb2/1P3P2/1PP4P/1K6 w - - 0 29
    ]
    White makes no sacrifices here because the material is equal. His pieces are more active 
    and his king is safer. It is unclear whether White can convert this advantage.

    \mainline[level=1]{20. Rhd1 Bc5}
    
    \chessboard[
        setfen=1k1r3r/p5pp/Qppp1p2/2b3q1/4P3/1PNR1P2/1PP3PP/1K1R4 w - - 2 21
    ]
    
    Here is a critical position. White has a significant advantage here. Black's bishop is stuck on c5 and must protect the d6 pawn.
    White would love to play b4. \variation[invar]{21. Na2} is hard to see. White threatens 
    to play b4. For example: \variation[invar]{21. Na2 Qxg2 22. b4 Bg1 23. Nc3 Rd7 24. b5 c5 25. R3d2 Qg6 26. Rxd6 Rxd6 27. Rxd6 Qe8 }

    \chessboard[
        setfen=1k2q2r/p5pp/Qp1R1p2/1Pp5/4P3/2N2P2/1PP4P/1K4b1 w - - 1 28
    ]

    Compared to the actual game, White has achieved more: his queen is active, and his rook and 
    knight will be dominant. We return to the actual game.
    \mainline[level=1]{21. Qa4 } The correct idea, though the execution could have been better.
    
    \mainline[level=1]{21...Kb7 22. b4 b5!}
    
    I overlooked this move and soon afterward made a mistake. 
    \mainline[level=1]{23. Qa5?!} This wastes a tempo! \variation[invar]{23. Qb3} would have been better.
    
    \mainline[level=1]{23... Bb6 24. Qa2 Rd7}
    
    \chessboard[
        7r/pk1r2pp/1bpp1p2/1p4q1/1P2P3/2NR1P2/QPP3PP/1K1R4 w - - 4 25
    ]

    \mainline[level=1]{25. Qe6?}

    I missed an opportunity! Perhaps I was afraid of losing too many pawns on the kingside. \variation[invar]{25. Rxd6 Rxd6 26. Qf7+} would have been better. 

    \mainline[level=1]{25... Rhd8 26. g3 Qe5!}
    
    \chessboard

    That is it. Now the position is equal again.

    \mainline[level=1]{27. Qxe5 fxe5 28. Ne2 Kc7}
    
    \chessboard

    
    \mainline[level=1]{29. f4?}
    Feeling that I had some advantage earlier, I felt I must try for a win. A waiting move 
    would have been better.
    
    \mainline[level=1]{29...d5! 30. exd5}
    
    \chessboard

    \mainline[level=1]{30...Rxd5?}
    
    Black actually has some advantage here. \variation[invar]{30... e4} would 
    have been better. His passed pawn would be too strong.

    \chessboard

    \mainline[level=1]{31. Rxd5?}
    Another missed opportunity. \variation[invar]{31. fxe5} would have been better. In the endgame,
    both of us made too many automatic bad moves.

    \mainline[level=1]{31...Rxd5 32. Rxd5 cxd5 33. fxe5}
    
    \chessboard
    
    The position is still equal, although White has one more pawn. His e5 pawn is too weak.

    \mainline{33...a5??}
    It is understandable that Black wants to exchange some pawns to secure a draw, but this move is a blunder. 
    
    \mainline{34. bxa5 Bxa5 35. Nd4 b4 36. Ne6+ Kd7 37. Nxg7 Bc7 38. e6+ Ke7 39. Nf5+ Kxe6 40. Nd4+ Kf6 41. Ka2 Bd6 42. Kb3 h5 43. Nc6 Kg5 44. Nxb4 Bxb4 45. Kxb4 Kg4 46. Kc5 Kh3 47. b4 } Black resigned.

    It was an exciting game with many mistakes from both sides. Both of us 
    tried hard to play good chess. I had some good ideas, but I was too eager to play them without 
    trying to find better moves.

    I tried hard to create a non-existent attack in the opening. Marius parried 
    it easily. Later in the middlegame, I had some attacking chances without trying hard, but I missed several opportunities.

    The endgame was fortunate for me. 
\end{multicols}
\newpage
\section{Boring but Perfect}

\begin{multicols}{2}
    \chessgameinfo{chess.com}{Ming}{Naroditsky-Bot}{}{2024.11.28}{1/2-1/2}
    \newchessgame

    \mainline[level=1]{1. d4 d5 2. c4 e6 3. Nc3 Nf6 4. cxd5 exd5 5. Bg5 Be7 }
    
    \chessboard

    Here we have the Queen's Gambit Declined Exchange Variation. I was familiar with the 
    so-called Carlsbad structure.  

    \mainline{6. Bxf6 Bxf6 7. Nf3 O-O 8.
    e3 c6}
    
    \chessboard

    \mainline{9. Bd3 Bg4 10. O-O Nd7 11. a3 a5 12. h3 Bxf3 13. Qxf3 Re8 14. Bf5 Be7 15.
    e4 dxe4 16. Nxe4}
    
    \chessboard
    
    \mainline{16... Nf6?}

    Here I missed the only chance to grab a pawn. \variation[invar]{16... Nf8} would
    have been better. Black threatens both the d4 pawn and \variation[invar]{17... g6 18. Bg4 f5} to win material.
    
    \mainline[level=1]{17. Rad1 Bf8 18. Rfe1 Nxe4 19. Bxe4 g6 20. d5 cxd5 21. Bxd5
    Rxe1+ 22. Rxe1 Qc7 23. g3 Bg7 24. b3 Rf8 25. Qe3 Qd7 26. Bg2 Bd4 27. Qd2 Rd8 28.
    b4 axb4 29. axb4 Qc7 30. Re4 Bb6 31. Qf4 Qxf4 32. Rxf4}
    

    \chessboard

    The game is a dead draw here. We have opposite-colored bishops plus a symmetric pawn structure.
    \mainline{32...Bc7 33. Rc4 Rd1+ 34. Kh2
    Bb6 35. Rc2 Rd7 36. f4 f5 37. Rc8+ Kg7 38. Rb8 Be3 39. Rxb7 Rxb7 40. Bxb7}
    
    \chessboard

    The rooks are exchanged. White cannot make any progress here, although he is one
    pawn up.
    \mainline{40...Kf6
    41. Kg2 h6 42. Kf3 Bb6 43. Bd5 g5 44. b5 Kg6 45. fxg5 hxg5 46. Bc6 Kf6 47. h4
    gxh4 48. gxh4}
    
    \chessboard

    We have a theoretical draw here. The Black king can defend the h-pawn alone. His 
    bishop defends the b-pawn. He does not mind losing his own f-pawn. It
    was unfortunate that I could not claim a draw. The game continued for 
    many more moves.

    \mainline{48...Kg6 49. Be8+ Kh6 50. Kf4 Bd8 51. h5 Ba5 52. Kxf5 Bd8 53. Bf7 Ba5
    54. Kf6 Bd8+ 55. Ke6 Ba5 56. Kd6 Bd8 57. Kd7 Ba5 58. Kc6 Bd8 59. Bg6 Ba5 60. Be8
    Bd8 61. Bg6 Ba5 62. Be8 Bd8 63. Bf7 Ba5 64. Bg6 Bd8 65. Be8 Ba5 66. Bg6 Bd8 67.
    Be8 Ba5 68. Bf7 Bd8 69. Bg8 Ba5 70. Be6 Bd8 71. Bf7 Ba5 72. Bc4 Bd8 73. Be2 Ba5
    74. Kb7 Bd8 75. Kc6 Ba5 76. Bg4 Bd8 77. Bd1 Ba5 78. Be2 Bd8 79. Bf3 Ba5 80. Bd1
    Bd8 81. Bf3 Ba5 82. Bd1 Bd8 83. Kb7 Ba5 84. Ka6 Bd8 85. Bf3 Bc7 86. Ka7 Bd8 87.
    Kb7 Bh4 88. b6 Bf2 89. Be2 Bxb6 90. Kxb6 Kh7 91. Ka5 Kh6 92. Ka6 Kh7 93. Ka5 Kh6
    94. Kb5 Kh7 95. Kb4 Kh6 96. Kc5 Kh7 97. Bf3 Kh6 98. Kd4 Kh7 99. Ke3 Kh6 100. Kd4
    Kh7 101. Bd5 Kh6 102. Bf7 Kh7 103. Kc4 Kh6 104. Bg6 Kg7 105. Kc3 Kh6 106. Kc4
    Kg7 107. Be4 Kh6 108. Bf3 Kh7 109. Kb4 Kh6 110. Be2 Kh7 111. Bd3+ Kh6 112. Bg6
    Kg7 113. Bf5 Kh6 114. Bg6 Kg7 115. Kb5 Kh6 116. Kc5 Kg7 117. Kc6 Kh6 118. Kd6
    Kg7 119. Kd5 Kh6 120. Kd6 Kg7 121. Bb1 Kh6 122. Ke7 Kxh5}

    While annotating, I am surprised to observe that I played almost perfectly.

\end{multicols}
\newpage
\section{Opening Ambush}

\begin{multicols}{2}
    \chessgameinfo{chess.com}{Ming}{Naroditsky-Bot}{}{2024.09.02}{1-0}
    \newchessgame
    \mainline[level=1]{1. d4 Nf6 2. c4 g6 3. Nc3 Bg7 4. e4 d6 5. h4 O-O 6. h5 Nxh5 7. Rxh5?!}
    
    \chessboard

    This exchange sacrifice was premature. White had not fully developed. His attack idea 
    was too simple: queen on h5 and knight on g5. Black should have been able to defend easily.

    \mainline[level=1]{7...gxh5 8. Qxh5}
    
    \chessboard

    \mainline[level=1]{8...c5}
    
    Black had a very simple idea to defend: \variation[invar]{8... f5} then \variation[invar]{9... Qe8}.
    After the queens were exchanged, Black could enjoy his material advantage.

    \mainline[level=1]{9. Nf3 Nc6 10. Ng5 h6 11. Nf3}
    
    \chessboard

    \mainline{11... Nb4??}
    
    This was a blunder. With material down, White would not have cared if he lost more. His 
    only goal was to mate. 

    Black could still have exchanged the queens.
    \variation{11... Qd7 12. d5 Nb4 13. Bxh6 Qg4 14. Qxg4 Bxg4 15. Bxg7 Kxg7 } would have 
    been better.
    \mainline[level=1]{12. Bxh6}
    
    \chessboard

    \mainline[level=1]{12... Nc2+??}
    
    Black had to play \variation[invar]{12... e5 13. dxe5 dxe5 14. Rc1 Bxh6 15. Qxh6 Qb6}. 
    His position was still defensible.

    \chessboard
    
    \mainline[level=1]{13. Kd2 Nxd4 14. Bxg7 Nxf3+ 15. gxf3 Kxg7 16. Qg5+ Kh8}
    
    \chessboard

    \mainline[level=1]{17. Bd3 Bh3 18. Qh6+ Kg8 19. Rg1+ Bg4 20. Rxg4#}

    I admit the exchange sacrifice was dubious. Black had good chances to defend since White was not fully
    developed. In the end, White was lucky to win the game because he simply played natural moves.
\end{multicols}
\newpage
\section{Lucky Defending}
\begin{multicols}{2}
    \chessgameinfo{chess.com}{Ming}{Francis-Bot}{}{2024.09.02}{1-0}
    \newchessgame

    \mainline[level=1]{1. d4 f5 2. Bf4 e6 3. Nf3 Nf6 4. e3 Bd6}
    
    \chessboard

    Black chose the Dutch Defense, which is not very solid, and White already had 
    an advantage. Here I faced a decision: how to exchange the bishops? 
    \variation{5. Bxd6 cxd6 6. c4} was a good alternative. White would still keep a lead 
    in development and his bishop would be much better than Black's.

    In the game I forgot this alternative entirely and chose to move the bishop.
    I wanted to open the h-file and create an attack along it.
    
    \mainline[level=1]{5. Bg3 O-O }
    
    \chessboard

    \mainline[level=1]{6. Bc4?}
    
    This move threw away the advantage because it would sooner or later
    lose a tempo. The bishop on c4 was not active. It had no 
    target and Black could later play the natural d5 to drive it away.
    
    \mainline[level=1]{6...Nc6 7. c3 Bxg3 8. hxg3 d5 9. Bd3 Qe7 }
    
    \chessboard
    
    \mainline[level=1]{10. Ne5?}
    
    One of my sins was impatience. What was the point of this move? After \variation[invar]{10...Nxe5 11. f4 Ng4} White 
    lost a pawn and Black had a clear advantage. A solid \variation[invar]{11. Nbd2} finishing the 
    London System would have been better.

    \mainline[level=1]{10...Bd7?} As we already know, \variation[invar]{10... Nxe5} would have been better.
    
    \chessboard
    
    \mainline[level=1]{11. f4?}
    
    \variation[invar]{11. Nxc6} would have been better; White would keep the better bishop. 
    In the game I was obsessed with the so-called Pillsbury setup (White placing a knight on e5 supported by pawns on d4 and f4).

    \mainline[level=1]{11...Nxe5 12. dxe5 Ng4}

    \chessboard 

    Oops. I had to defend the e3 pawn now. 
    
    \mainline[level=1]{13. Qe2 Qc5 }
    
    \chessboard
    
    \mainline[level=1]{14. Kd2?!}
    
    I needed to defend my e3 pawn. What else could I do? It turned out White 
    didn't need to defend that pawn. Losing it would not make the position inferior
    because the bishop was still much better. White needed to keep the pieces active.

    \variation[invar]{14. b4 Qxe3 15. Qxe3 Nxe3 16. Nbd2} The position is still equal. 
    Besides his better bishop, he can also hope for the active play along the h-file.
    \mainline[level=1]{14...Qb6 15. b3 a5 16. Na3}
    
    \chessboard

    Black had a small edge now because his pieces were more active. He needed to 
    act quickly because once White consolidated, White would have counterplay on 
    the h-file. 
    
    \mainline[level=1]{16...Qc5?!}
    
    Now White just needed to make a natural move. \variation{16... c5} targeting d4 would have been better.
    
    \mainline[level=1]{17. Nc2}
    
    \chessboard
    
    \mainline[level=1]{17...h6?}
    
    Now White had the advantage again. Black should have played prophylactic g6, removing any potential
    sacrifice for White with \symrook xh7.

    \mainline[level=1]{18. Rh4 Rfd8 19. Rah1 Re8 }
    
    \chessboard 

    White now only needed to play the natural sacrifice.
    
    \mainline[level=1]{20. Rxg4! fxg4 21. Qxg4 Kf8}
    
    \chessboard
    
    \mainline[level=1]{22. Qg6}
    
    \variation[invar]{22. Bh7} intending \variation{23. Rxh6 gxh6? 24. Qg6} mate in a few moves would have been stronger.
    
    \chessboard

    \mainline[level=1]{22...a4?} Now mate is inevitable.

    \variation[invar]{22...Qe7 23. Qh7 Qf7 24. Bg6 Qg8 25. Qxg8+ Kxg8 26. Bxe8 Bxe8 27. Nd4 } White has advantage while 
    Black can still play.
    
    \chessboard
    
    The sacrifice is again quite natural.

    \mainline[level=1]{23. Rxh6 gxh6 24. Qxh6+ Kf7 25. Qh7+ Kf8 26. Bg6 Qxc3+ 27. Kxc3 d4+ 28. exd4 axb3 29. Qf7# }

    A lucky win for me indeed. From an inferior position, I only needed to play natural moves.
    After my opponent missed a difficult move, I just followed through with the natural sacrifices. 
    To be honest, there was not much calculation involved. I am, however, proud that I took the opportunity.
\end{multicols}
	\chapter{Annotations}
I annotate chess games as a means of self-improvement and personal learning. The market offers a wealth of excellent chess literature, much of which is dense with variations. Additionally, there are countless chess games available, and uncovering hidden insights within these games provides a valuable experience for me.

My primary focus is on identifying critical moments and key moves using straightforward methods, rather than getting lost in complex variations. I prefer to analyze games with fewer moves, often avoiding endgames, as they tend to require extensive calculations and are not my main interest.

After pinpointing these critical moments, I verify my findings using computer analysis. There are times when I miss key points, but other times I succeed in my assessments. This process is an integral part of my learning journey.

I intentionally refrain from using computers to evaluate the players, as I believe such an approach will not contribute to my growth as a chess player.

Ultimately, I aspire to discover effective strategies for playing good chess through simple methods, drawing inspiration from the styles of great players like Karpov and Petrosian.

% Games with high value annotation
\newpage
\section{Mikhail Tal - Dieter Keller, Zurich 1959} 

\begin{multicols}{2}
    Here is one of Tal's most impressive games—so complicated that 
    Tal gave it in his book \emph{The Life and Games of Mikhail Tal} without
    any notes. I call this game an absolute TAL-Fahrt.
    \chessgameinfo{Zurich 1959}{Mikhail Tal}{Dieter Keller}{}{1959.05.27}{1-0}
    \newchessgame
    \mainline[level=1]{
        1.Nf3 Nf6 2.c4 e6 3.Nc3 d5 4.d4 c6 5.Bg5 dxc4 6.e4 b5}
    
    \chessboard 

    Tal could have also played \variation[invar]{7.e5 h6 8. Bh4 g5 9. Nxg5 hxg5 10. Bxg5 Nbd7},
    the ultra-sharp Botvinnik variation that required much more preparation.
        
    \mainline[level=1]{7.a4 Qb6 8.Bxf6 gxf6 9.Be2 a6 10.O-O Bb7}
    

    \chessboard
    
    \mainline[level=1]{11.d5 }
    
    \chessboard

    Black also has some other options. Reckoning that White has the initiative only
    because the Black king is in the center, he could also play \variation[invar]{11... Nbd7 
    \xskakcomment{ intending to move his king to the queenside. For example } 12. dxe6 fxe6 13. Nd4 c5  \xskakcomment{ the e6 pawn is now protected.}  14. Bh5+ Kd8 15. Nc2 b4}. Black has some advantage.
    
    \mainline[level=1]{11...cxd5 12.exd5}
    
    \chessboard

    \mainline[level=1]{12...b4}  
    
    Taking the central pawn is out of the question because it is too dangerous for the Black king.
    It is natural to play the text move. As mentioned before, \variation[invar]{12... Nbd7} covering the king would have been better.

    \chessboard

    Here White can choose between \variation[invar]{13. a5} and \variation[invar]{13. dxe6}.

    After \variation[invar]{13. dxe6 fxe6 14. a5 Qc6} White still has some advantage. By
    the way \variation[invar]{13. dxe6 bxc3 14. exf7 Kxf7 15. Bc4+} is too dangerous for Black.

    As a comparison, after \variation[invar]{13. a5}, \queen c6 is impossible. Therefore,
    a5 is the better move.
    
    \mainline[level=1]{13. a5}

    \chessboard

    We have here three candidate moves: \symqueen d8, \symqueen d6, \symqueen c7.

    \symqueen d8 is bad because \variation[invar]{14... Qd8 15. dxe6 Qxd1 16. exf7+ Kxf7 17. Bc4+ Kg7 18. Nxd1}
    White has more material and a lead in development, or \variation[invar]{14... Qd8 15. dxe6 fxe6 16. Nd4 bxc3 17. Bh5+ Ke7 18. Re1} White 
    has an unstoppable attack.

    \symqueen d6 is a good move. White cannot gain an advantage after taking the e6 pawn,
    because \symknight d4 doesn't win a tempo. \variation[invar]{14... Qd6 15. Ne4 Qf4! 16. dxe6! Qxe4 17. exf7+ Kxf7 18. Rc1 } White has an attack against the Black king.

    The text move is also good.
    
    \mainline[level=1]{13...Qc7 14.dxe6 }

    \chessboard

    Now taking the pawn is bad because \variation[invar]{14... fxe6 15. Nd4}.

    \mainline[level=1]{14... bxc3 15.Nd4 Rg8 16.Qa4+ }
    
    The last few moves are quite straightforward. Black declines \variation[invar]{16... Nc6 17. exf7+}.
    
    \mainline[level=1]{16...Kd8 17.g3}
    
    Don't forget to defend! 

    \chessboard

    We are now in the middlegame. In the next few moves, both players have some resources like playing cards.
    It is important to recognize these cards and decide when to play which one.

    White has:
    \begin{itemize}
        \item \symbishop xc4
        \item bxc3
        \item \symrook fd1 or \symrook ad1
        \item exf7
    \end{itemize}

    Black has:
    \begin{itemize}
        \item \symknight c6
        \item cxb2
        \item \symbishop c5
        \item \symbishop d5
        \item \symrook g5
    \end{itemize}

    White has compensation because Black's knight and rook are on the corner while White 
    has some attack against the Black king.

    We can eliminate some moves first: \index{Process of Elimination}
    \variation[invar]{17... Bc5 18. exf7 Qxf7 19. Rad1 Kc8 20. Bxc4 Bd5 21. Bxd5 Qxd5 22. Nb3} White has a big advantage.

    \variation[invar]{17... Nc6 18. exf7 Qxf7 19. Nxc6} White gains back the material and keeps the advantage. 

    \variation[invar]{17... cxb2 18. exf7 Qxf7 19. Rad1} White is crushing.

    This means only \symbishop d5 or \symrook g5 are playable.
    
    \mainline[level=1]{17...Bd5}
    
    \chessboard

    \mainline[level=1]{18.Rfd1?}

    \variation[invar]{18. bxc3 Bd6 19. Rfd1 fxe6 20. Bf3 Kc8 21. Nxe6 Bxe6 22. Bxa8}
    The material is roughly equal. White has a safer king and therefore some advantage.
    There is however no attack anymore.

    Black blocks the d-file. White must hurry to place his rook on the d-file.

    The question is which one. Noticing that
    Black can play cxb2 attacking the rook on a1, Tal should have eliminated \symrook fd1.
    Therefore, \symrook ad1 is the best move.

    \chessboard
    
    \mainline[level=1]{18...Kc8?}

    Now Black has some candidate moves: cxb2, \symking c8. The move cxb2 wins a tempo for 
    him and White must place his a1 rook on b1. On the other hand, once White plays bxc3, 
    his rook on a1 will be active. 
    
    Playing \symking c8 also allows White to play \variation[invar]{19. Qe8+ Kb7 20. bxc3} activating 
    the rook on a1 as we have discussed. By the way, \variation[invar]{19... Qd8 20. exf7} gives White an advantage.
    
    It is therefore better to play \variation[invar]{18... cxb2}. 

    \chessboard

    \mainline[level=1]{19.bxc3}
    
    Black gives White the opportunity to play \symqueen e8+. The opportunity disappears
    when Black plays \symbishop in the next move. So \variation[invar]{19. Qe8+ Kb7 20. bxc3} is better.
    
    \mainline[level=1]{19... Bc5!}
    
    The last card played!

    \chessboard
    
    \mainline[level=1]{20.e7} 
    
    White still has the \symbishop xc4 card. After \variation[invar]{
        20. Bxc4 Bxc4? 21. Qxc4 Bxd4 22. Rxd4 Qxc4 23. Rxc4+ Kd8 24. Rd1+ Ke7 25. exf7 Kxf7 26. Rc7+ Kg6 27. c4 }

    White has total domination. Black cannot bring up his knight and rook on the queenside.

    It is better for Black to play \variation[invar]{
        20... fxe6 21. Bxd5 exd5 22. c4 Bxd4 23. Rxd4 Nc6 24. Rxd5 Rd8 25. Rad1 Rb8 26. Rxd8+ Nxd8 27. Qc2 
    } The position is equal.

    \chessboard
    
    \mainline[level=1]{20... Nc6}
    
    For human players, it is easier to play \symqueen xe7 or \symbishop xe7, eliminating White's dangerous pawn. \symbishop xe7 is probably better 
    because it also eliminates White's \symbishop xc4 card.

    After \variation[invar]{20... Bxe7 21. Rab1? Nd7!} Black finally brings his knight and has some advantage.
    \variation[invar]{20... Bxe7 21. Nf5 Be6 22. Nd4 Bd5} leads to a draw after repetition.

    \chessboard

    \mainline[level=1]{21.Bg4+}
    
    Another beautiful line is:
    \variation[invar]{21. Nf5 Qe5 22. Bxc4 Bxf2+! 23. Kf1 Qxf5}

    \chessboard[setfen=r1k3r1/4Pp1p/p1n2p2/P2b1q2/Q1B5/2P3P1/5b1P/R2R1K2 w - - 0 24]
    
    Black is threatening mate. Now the show begins:
    \variation[invar]{24. Rxd5 Qh3+ 25. Kxf2 Qxh2+ 26. Kf1 Kc7 27. Rb1 Qh1+ 28. Kf2! }
    The material is equal and both kings are exposed. 

    \variation[invar]{24. e8=Q+!}

    \chessboard[setfen=r1k1Q1r1/5p1p/p1n2p2/P2b1q2/Q1B5/2P3P1/5b1P/R2R1K2 b - - 0 24]

    \variation[invar]{24... Rxe8 25. Rxd5 Qh3+ 26. Kxf2 Kc7 27. Kg1 } White has advantage because 
    his king is safer.

    \variation[invar]{24... Kc7!! 25. Qaxc6+ Bxc6 26. Qxf7+ Bd7 27. Qxd7+ Qxd7 28. Rxd7+ Kxd7 29. Bxg8 Rxg8 30. Kxf2 } Finally, a draw.

    Let's go back to the mainline.
    \mainline[level=1]{21...Kb7}
    
    \chessboard

    The position has changed completely. Both sides have played all the 
    cards, and therefore it is time for a new evaluation. Intuitively, Black 
    will soon activate all his pieces. It would be good for White if 
    he can find some immediate drawing chances.

    \mainline[level=1]{22.Nb5?}
    
    \variation[invar]{
        22. Rab1+ Ka7 23. Nxc6+ Bxc6 24. Qxc4 Rg5}  

    \chessboard[setfen=r7/k1q1Pp1p/p1b2p2/P1b3r1/2Q3B1/2P3P1/5P1P/1R1R2K1 w - - 1 25]
    
    Noticing that Black's rook can be trapped, White can play naturally. Black is forced to sacrifice.

    \variation[invar]{25. Kf1 Re5 26. f4 Bb5 27. Rxb5 axb5 28. Qxb5 Rxe7 
    29. Be2 Qc8 30. Rd6 Qh3+ 31. Ke1 Rxe2+ 32. Kxe2 } 

    \chessboard[setfen=r7/k4p1p/3R1p2/PQb5/5P2/2P3Pq/4K2P/8 b - - 0 32]

    The position is equal.

    Let's go back to the mainline.

    \chessboard

    \mainline[level=1]{22...Qe5!}
    
    Black can use \vocab{PoE}{process of elimination} to find the correct move.

    \variation[invar]{22... axb5 23. Qxb5+ Ka7 24. Qxc5+ Ka6 25. Rxd5 Rxg4 26. Qb5+ Ka7 27. Rd7 } White wins.

    The bishop on d5 must be protected. So \symqueen e5 is also a natural move.

    \chessboard

    \mainline[level=1]{23.Re1}
    
    \symrook xd5 doesn't work unfortunately because \variation[invar]{23. Rxd5 Qxd5 24. Rd1 Bxf2+ 25. Kf1 Qxb5 }. 
    \symrook ab1 is too slow because \variation[invar]{23. Rab1 Rxg4 24. Nd6+ Kc7 25. e8=N+ Rxe8 26. Nxe8+ Kc8 27. Rxd5 Qxd5 28. Nxf6 Bxf2+ 29. Kxf2 Qf5+ 30. Kg2 Qxf6 }. 

    Tal played the only practical move.

    \chessboard

    \mainline[level=1]{23...Be4?}

    Here again, using process of elimination can help to find the correct move.

    We know that \symrook ab1 is the resource for White. As we will see, White can 
    win material with this move. Therefore Black should reject this move. The only move possible 
    is \variation[invar]{23... Qg5}.

\mainline[level=1]{24.Rab1 Rxg4 25.Rxe4 Qxe4 26.Nd6+ Kc7 27.Nxe4 Rxe4 28.Qd1}

\chessboard

\mainline[level=1]{28...Re5??}

Keller was unable to withstand the tension. The e7 pawn has haunted him for a long time. 
Therefore he should have taken it. \variation[invar]{28... Bxe7} would have been better.

\chessboard

\mainline[level=1]{29.Rb7+! Kxb7 30.Qd7+ Kb8 31.e8=Q+ Rxe8 32.Qxe8+ Kb7 33.Qd7+ Kb8 34.Qxc6}

This is indeed a very complex game with lots of tactical elements. Finding the right 
move through calculation leads to fatigue and mistakes. As human players,
it is much easier to find the right move by recognizing the resources and using the process of elimination. 
\end{multicols}
\newpage
\section{Ding, Liren - Gukesh D, World Championship Match, Round 12}

\epigraph{I drank coffee. I changed my haircut a little}{\textit{Ding Liren}}

\keywords{Simple Play, Dynamic Play}

\begin{multicols}{2}
	\chessgameinfo{World Championship Match}{Ding, Liren}{Gukesh D}{12}{2025.02.03}{1-0}
It was the 12th game of a 14-game match, and Ding found himself trailing by one point after his recent loss. Playing with the white pieces, he knew he needed to make a strong move to regain momentum. In a strategic decision, he opted for a rather calm opening, setting the stage for the challenges that lay ahead. 

\newchessgame[
 id=main,
 storefen=example,
 event={Ding - Gukesh World Championship Match},
 white={Ding, Liren},
 black={Gukesh D},
 round={12}]
 \mainline[level=1]{1. c4 e6 2. g3 d5 3. Bg2 Nf6 4. Nf3 d4 5. O-O Nc6 6. e3 Be7 7. d3 dxe3 8. Bxe3 e5 9. Nc3 O-O 10. Re1 h6}

 \chessboard[
   	pgfstyle=straightmove,
	linewidth=0.05em,
 	markmove=d3-d4
 ]

 Ding Liren is known for his solid, strategic approach to chess. He excels in positional play and endgames, often outmaneuvering opponents with his patience and precision. Here the position is calm and it's all about piece maneuvering. 


He thought for a long time about his next move to provoke. 

 \mainline[level=1]{
11. a3 } 
In this position, White wants to push d3-d4. His knight on f3 is critical for the breakthrough and should not be exchanged.  \variation[invar]{11.h3} is a good alternative.

\mainline[level=1]{11...a5}

\chessboard[
   	pgfstyle=straightmove,
        linewidth=0.05em,
 	markmove=d3-d4,
 	markstyle=circle,
 	linewidth=0.05em,
 	markfields={b5},
 ]

It was understandable that Gukesh wanted to prevent b4 with this move. However, compared to the previous position,
he now has one serious weakness on b5 which White can easily exploit with his knight.

Black would love to move his knight on c6 to play c6, guarding the weakness on b5 and restricting the bishop on g2.
He cannot do so because the knight must guard e5. We can also understand Ding's next move, which keeps the 
pressure on e5 by preventing Black from playing \variation{11...Bg4}, exchanging the knight on f3 and releasing the pressure
on his e5 pawn.

\mainline[level=1]{12. h3} 

\resumechessgame
A direct \variation[invar]{12.d4 exd4 13. Nxd4 Nxd4 14. Bxd4 c6 15. c5} according to the engine. The position opens up and White has some advantage.

 \chessboard[
        setfen=r1bq1rk1/1p2bpp1/2p2n1p/p1P5/3B4/P1N3P1/1P3PBP/R2QR1K1 b - - 0 15
        ]

I don't know if the position favors White. After exchanging some pieces, Black should have more room to maneuver than 
several moves ago, where he had a cramped position. Chess law states it is better to avoid exchanging when one has a space advantage. This is exactly the case for White.

\chessboard[
   	pgfstyle=straightmove,
        linewidth=0.05em,
 	markmove=d3-d4,
 	markstyle=circle,
 	linewidth=0.05em,
 	markfields={b5},
 ]

It is instructive to observe the position here. White still has his trumps: d3-d4 and an outpost on b5, and Black can do nothing to parry. So there is no need for White to rush. With h3, White deprives Black of placing a bishop or knight on
g4. White slowly improves his position and Black has no counterattack. 

\resumechessgame[id=main]
\mainline[level=1]{12...Be6 13. Kh2} 

No need to rush! \vocab{prophylaxis}{Prophylaxis} against \symqueen d7 \index{Prophylaxis}.

\mainline{13...Rb8}

\chessboard[
   	pgfstyle=straightmove,
        linewidth=0.05em,
 	markmove={d3-d4,e5-e4},
 	markstyle=circle,
 	linewidth=0.05em,
 	markfields={b5},
 ]

Here White has more options. Again, \variation{14. d4 exd4 15. Nxd4 Nxd4 16. Bxd4 c6 17. c5} loses his advantage. White loses his trump b5 now because the c6 pawn guards it. 
\variation{14. Nb5 Nh7 15. Qd2 Ng5 16. Nxg5 hxg5 17. Rad1} 

\chessboard[
        setfen=1r1q1rk1/1pp1bpp1/2n1b3/pN2p1p1/2P5/P2PB1PP/1P1Q1PBK/3RR3 b - - 1 17,
	markstyle=circle,
 	linewidth=0.05em,
 	markfields={e4},
        ]
White keeps the advantage because Black now has a new weakness on e4.

Ding's next move is prophylaxis against any Black potential e4 thrust.
\mainline[level=1]{14. Qc2 Re8} 

\chessboard[
   	pgfstyle=straightmove,
        linewidth=0.05em,
 	markmove={d3-d4,e5-e4},
 	markstyle=circle,
 	linewidth=0.05em,
 	markfields={b5},
 ]

\mainline{ 15. Nb5}

\chessboard[
   	pgfstyle=straightmove,
        linewidth=0.05em,
 	markmove={d3-d4,e5-e4},
 ]

Finally the move! White achieves his goal on b5. Now White is preparing d4. 

\mainline{15... Bf5}

\chessboard[
   	pgfstyle=straightmove,
        linewidth=0.05em,
 	markmove={d3-d4,e5-e4},
 	markstyle=border,
 	linewidth=0.05em,
 	markfields={a1},
 ]

The rook on a1 is not so active. It can move to d1 to support the d3-d4 thrust. The next
move is logical. Even a World Chess Champion game can be so simple!

\mainline{16. Rad1 Nd7}

\chessboard[
   	pgfstyle=straightmove,
        linewidth=0.05em,
 	markmove={d3-d4,e7-g5},
 ]

White wants to play d4 and prevent Black from playing \variation[invar]{16...Bg5}. His next move is logical.

\mainline[level=1]{ 17. Qd2} 

\chessboard[
   	pgfstyle=straightmove,
        linewidth=0.05em,
 	markmove={d3-d4},
 ]

 Grandmaster Dorfman writes: ``If for one of the players the static balance is negative, he must without hesitation 
 employ dynamic means, and be ready to go in for extreme measures.''  \cite{Dorfman:2001}

 Here White has obviously a static advantage: he has a clear plan to improve his position while
 Black just has no clear plan. For this reason, Black must play dynamically!
\mainline{17... Bg6? } 

\variation[invar]{17... Nc5 18. d5 Nd3 19. d5 Nxe1 20. Qxe1 Nd4 21. Nfxd4 exd4 22. Nxd4 Bh7 23. Qxa5}

\chessboard[
   	setfen=1r1qr1k1/1pp2pp1/6bp/Q2P2b1/2PN4/P3B1PP/1P3PBK/3R4 w - - 1 24
 ]

White still has a static advantage, with the price of one exchange. Black has more
space to maneuver after exchanging some pieces. The position is much more playable
than in the actual game.

Go back to the actual game. White has prepared everything. It is now the time!

\chessboard[
   	pgfstyle=straightmove,
        linewidth=0.05em,
 	markmove={d3-d4},
 ]

\mainline[level=1]{18. d4 e4}

The only move. \variation[invar]{18...exd4 19. Bf4! Rc8 20. Nfxd4 Nxd4 21. Qxd4 Nc5 22. Nxc7!} White wins easily.

\chessboard[
 	markstyle=border,
 	linewidth=0.05em,
 	markfields={f3},
	pgfstyle=straightmove,
        linewidth=0.05em,
 	markmove={f3-g1, g1-e2, e2-f4},
 ]

\mainline[level=1]{ 19. Ng1 Nb6 20. Qc3! }

\chessboard[
	pgfstyle=straightmove,
        linewidth=0.05em,
 	markmove={d4-d5},
 ]

It is quite an instructive move. The queen deprives the knight on c6 of the a5 and e5 squares.
Meanwhile, d5 is coming.

\mainline{20...Bf6 21. Qc2 a4 22. Ne2 Bg5} 

\chessboard[
	pgfstyle=straightmove,
        linewidth=0.05em,
 	markmove={e2-f4},
 ]

\mainline{ 23. Nf4 Bxf4 24. Bxf4 Rc8 25. Qc3 Nb8}

\chessboard[
	pgfstyle=straightmove,
        linewidth=0.05em,
 	markmove={d4-d5},
 ]

\mainline{26. d5 Qd7 27. d6 c5 28. Nc7 Rf8 29. Bxe4 Nc6} 
(\variation[invar]{29...Bxe4 30. Rxe4 Nc6 31. Bxh6 gxh6 32. Rg4+ Kh7 33. Qg7#})

\mainline[level=1]{ 30. Bg2 Rcd8
31. Nd5 Nxd5 32. cxd5 Nb8 } 

\chessboard

Gukesh could have resigned here.

\mainline{33. Qxc5 Rc8 34. Qd4 Na6 35. Re7 Qb5
36. d7 Rc4 37. Qe3 Rc2 38. Bd6 f6} 

\chessboard

The live commentator said:``This is the only time Ding has calculated during the game!''

\mainline{39. Rxg7+} Black resigned.

 \chessboard

 Certainly this was not Gukesh's best game. Besides the game,
 it was more instructive to hear how Ding prepared for the game after the defeat the previous day:
 ``I drank coffee, I changed my haircut a little bit.''

\end{multicols}
 \subsection*{Lessons Learned}
 \begin{itemize}
    \item{Simplicity is often the best strategy. ``I drank coffee, I changed my haircut a little bit.''}
	\item{Play dynamically when one has a static disadvantage!}
\end{itemize}

\newpage
\section{Emanuel Lasker - Wilhelm Steinitz, World Championship 1894, Game 7}
\epigraph{The hardest game to win is a won game}{\textit{Emanuel Lasker}}

\keywords{Intermediate moves, Prophylaxis}

\begin{multicols}{2}
    \chessgameinfo{World Championship}{Lasker}{Steinitz}{7}{1894.08.26}{1-0}
\newchessgame[
    id=main,
    event={World Championship 1894},
    white={Lasker},
    black={Steinitz},
    round={7}
]

\mainline[level=1]{1. e4 e5 2. Nf3 Nc6 3. Bb5 d6 4. d4 Bd7 5. Nc3 Nge7 6. Be3 Ng6}

\begin{chessdiagram}
    \chessboard
\end{chessdiagram}

I usually don't comment on openings, especially from the old masters. 
Just observing the position we can conclude White is much better. Assuming White castles
on the queen side and Black castles on the king side. White is ready to launch with his h-pawn.
Black's knight on g6 is unfortunate. Not only does it waste a tempo (on f6 would be much better), 
but it is also a perfect target for White's h-pawn.

Black, on the other hand, needs much more time to prepare. His queen cannot be activated soon 
because his c-pawn is blocked by the knight on c6. If White plays \variation[invar]{7. d5}, closing the center, 
I can hardly see any counter play. Lasker has however a different plan.

\mainline[level=1]{7. Qd2 Be7 8. O-O-O a6 9. Be2 exd4 10. Nxd4 Nxd4 11. Qxd4 Bf6 12. Qd2 Bc6 13. Nd5 O-O }

\begin{chessdiagram}
\chessboard
\end{chessdiagram}

Optically, White has more more active pieces. \variation[invar]{14. h4} would be natural.
For example \variation[invar]{14. h4 Bxh4 15. g3 Bf6 16. f4 Re8 17. Bf3 Nf8 18. Qh2}

\begin{chessdiagram}
\chessboard[
    setfen=r2qrnk1/1pp2ppp/p1bp1b2/3N4/4PP2/4BBP1/PPP4Q/2KR3R b - - 4 18
]
\end{chessdiagram}

White has already doubled the heavy pieces on the h-file whilte Black has not launched the pawn!

We go back to the game.

\begin{chessdiagram}
    \chessboard
    \end{chessdiagram}

\mainline[level=1]{14. g4?}

The move loses a pawn without compensation.


\mainline{14... Re8 15. g5 Bxd5} 

\begin{chessdiagram}
\chessboard
\end{chessdiagram}

The candidate moves are exd5, gxf6, or \symqueen xd5.

After \variation[invar]{16. exd5 Rxe3 17. fxe3 Bxg5!}, Black solves all
his problems and the position is equal. However, such an exchange 
sacrifice becomes standard in the 20th century, decades after this game.
I am not sure if Steinitz could find this move.

After \variation[invar]{16. gxf6 Bxe4 17. Rhg1 Qxf6} Black is simply better while
White has no counter play.

Lasker chooses the third move.

\mainline[level=1]{ 16. Qxd5 Re5} 

\begin{chessdiagram}
\chessboard
\end{chessdiagram}

After \variation[invar]{17. Qxb7 Bxg5 18. Bxg5 Rxg5 19. Bc4 Rc5 20. Bd5 Rb8 21. Qxa6 Qf6},
Black has active pieces while White has one more pawn.

Most likely, Lasker is continuing the moves he has planned on his 14th move and
misses Steinitz' refutation.

\begin{chessdiagram}
\chessboard[
    setfen=1r4k1/2p2ppp/Q2p1qn1/2rB4/4P3/8/PPP2P1P/2KR3R w - - 1 22
]
\end{chessdiagram}

We go back to the current game. \mainline[level=1]{ 17. Qd2 Bxg5! 18. f4 Rxe4!}

\emph{
    This is the problem: doubling heavy pieces on the e-file
    allows Black to recover the piece
} (Neishtadt)

\mainline[level=1]{19. fxg5 Qe7}

% Dynamic play problem
\begin{chessdiagram}
\chessboard[
    setfen=r5k1/1pp1qppp/p2p2n1/6P1/4r3/4B3/PPPQB2P/2KR3R w - - 1 20
]
\end{chessdiagram}

Clearly Black has the static advantage: White is two pawns down and Black has more active heavy pieces on e-file.
Playing such a position needs nerve.

One possibility as recommended by Kasparov is take one pawn and then defend an inferior but defensible position.

\variation[invar]{20. Bf3 Rxe3 21. Bxb7 Rb8 22. Rhe1 Re5 23. Bxa6 Qxg5 24. Qxg5 Rxg5} 

\emph{Objectively, Kasparov is right. One pawn down gives greater saving chances than two pawns down.} (Dvoretsky)

Lasker chooses to continue the struggle:
\mainline[level=1]{20. Rdf1?! Rxe3 21. Bc4}

\begin{chessdiagram}
    \chessboard
    \end{chessdiagram}

The f7 pawn is under attack. \variation[invar]{21... Rf8} is an obvious
alternative. The knight can be activated by moving to e5 at some point.

\mainline[level=1]{21...Nh8} 

\emph{Typical Steinitz! The commentators admired this eccentric move, although it is apparently
not the strongest} (Kasparov)

I believe Lasker expects this move since it is typical Steinitz. 
 

\mainline[level=1]{ 22. h4 c6 23. g6}

\begin{chessdiagram}
\chessboard
\end{chessdiagram}

\mainline[level=1]{ 23... d5?} This move throws away all the advantage.

We can see from now on, the knight is stuck at the corner at the game. Black is therefore one piece down for
two pawns. Steinitz, however, still believes he has the advantage.

\variation[invar]{23... d5} is an automatic bad move. It closes the center,
it wins a tempo by chasing the White bishop. How can this be a wrong move,
it fails to see an intermediate move.

After \variation[invar]{23...hxg6 24. h5 gxh5 25. Rxh5 Re8 26. Rhh1} Black has a clear advantage because
he is up three pawns and White fails to start an attack on h-file. 

\begin{chessdiagram}
\chessboard
\end{chessdiagram}

\mainline[level=1]{24. gxh7+! Kxh7 25. Bd3+ Kg8 26. h5 Re8 27. h6 g6 28. h7+ Kg7}

\begin{chessdiagram}
\chessboard[
    setfen=4r2n/1p2qpkP/p1p3p1/3p4/8/3Br3/PPPQ4/2K2R1R w - - 1 29
]
\end{chessdiagram}

% TODO: add a prophylaxis problem
\mainline[level=1]{29. Kb1?! } \index{Prophylaxis!King Move} 

\emph{One of Lasker's characteristic ``changes of rhythm''. As long as his opponent
has not yet created any direct threats, White has a little time to make the useful
prophylaxy moves. In the ensuing complications, Black will no longer be able to exploit tactical resources involving
the enemy king's vulnerability. Such play requires both a healthy evaluating capacity andtremendous coolnesss.} (Dvoretsky)

This move has a good \vocab{prophylaxis}{prophylaxis}{} idea. However \variation[invar]{29. a3} first is more accurate \index{Prophylaxis}.

\begin{chessdiagram}
\chessboard[
    markstyle=border,
    linewidth=0.05em,
    markfields={h8},
]
\end{chessdiagram}

Black is playing practically one piece down, since his knight
is at the corner. Activating the knight urgent now.

\mainline[level=1]{29...Qe5} 
\variation[invar]{29... f6! 30. Qf2 \xskakcomment{ (\symqueen h2 is now impossible, due to back rank
weekness.)} Qe6 31. Ka2 Nf7} The knight is finally free again. We also see
why White's last move is an inaccuracy. If \variation[invar]{29. a3} has been played first,
White can play \variation[invar]{30. Qh2} after \variation[invar]{29... f6}.

This subtlety was also overlooked by Kasparov and Dvoretsky. 
White intends to play both \symking b1 and a3, but the order matters. 
Calculating the optimal sequence through brute force is practically impossible here. 
However, by comparing the consequences of each move, one can deduce that a3 must be played first, 
because \symking b1 creates a potential back-rank weakness that restricts White's options.

\mainline[level=1]{ 30. a3!}

Note that White's king castling is quite solid. The bishop on d3 protects the c2 pawn and e2 square, making Black's invasion on the second rank impossible. Black needs two tempi (c5 and then c4) to break this defense, which allows White to take his time to maneuver his queen.

\emph{In this game there is something of the `Tal' element: White's attack is rather abstract,
but it will not come to an end - all the time some threats arise!} (Kasparov)

\emph{Lasker's last two quiet moves were completely inexplicable to his comtemporaries: 
how can you play this way when two pawns are down?} (Kasparov)

\mainline{30...c5 31. Qf2 c4 32. Qh4 f6 33. Bf5}

\begin{chessdiagram}
    \chessboard[
    setfen=4r2n/1p4kP/p4pp1/3pqB2/2p4Q/P3r3/1PP5/1K3R1R b - - 1 33,
    markstyle=circle,
    linewidth=0.05em,
    markfields={g6},
    markstyle=border,
    linewidth=0.05em,
    markfields={h8},
    pgfstyle=straightmove,
    linewidth=0.05em,
    markmove={h1-g1},
]
\end{chessdiagram}


 The position is deeply complex and demands careful calculation.

\begin{itemize}
    \item{Where are the weaknesses?}

     The g6 square is the most vulnerable point in Black's camp.
    \item{Which is the worst-placed piece?}

     The h8-knight is completely sidelined, and there is no clear route to bring it back.
    \item{What is my opponent's idea?}

     White has nothing concrete yet, but \symrook hg1 followed by a sacrifice on g6 is in the air.
\end{itemize}

 White enjoys a static edge: his king is well protected and the knight on h8 remains sleeping.

 Black, in contrast, lacks counterplay. Occupying the second rank is either impossible or far too slow.

 With that in mind, Black must choose between several options:

\begin{itemize}
    \item {(Exchange piece) \symqueen g3}

    \variation[invar]{33... Qg3 34. Qh6+ Kf7 35. Ka2 \xskakcomment{ Prophylaxis! 35. \symrook hg1 is too early because 35... \symrook e1, exchanging all the rooks} Re1 \xskakcomment{Forced, to parry \symrook hg1} 36. Qd2 R1e5 37. Rh3 Qg5 38. Qxg5 fxg5 39. Bd7+ }
    
    \chessboard[setfen=4r2n/1p1B1k1P/p5p1/3pr1p1/2p5/P6R/KPP5/5R2 b - - 1 39]
    
    White has a clear advantage.
    \item {(Accept the challenge) gxf5}

    The threat is stronger than the execution. Now White has no more threat on g6.

    \variation[invar]{33... gxf5 34. Rhg1+ Kf7 35. Qh5+ Ke7 36. Rxf5 Qe6 37. Rxd5 Re1+ 38. Rd1 Rxg1 39. Rxg1 Kd8 }
    
    \chessboard[setfen=3kr2n/1p5P/p3qp2/7Q/2p5/P7/1PP5/1K4R1 w - - 1 40]
    
    The ending is equal.
    \item {(Create some counter play) c3}

    \variation[invar]{33... c3 34. Qh6+ Kf7 35. Rhg1 Rg3 36. Bd7 Re7 37. Rxg3 Qxg3 38. Rxf6+ Kxf6 39. Qf8+ Kg5 40. Qxe7+ Kh6 41. Qf8+ Kxh7 42. Qe7+ Kh6 43. Qf8+ Kh7 44. Qe7+ Kh6 45. Qf6 }
    
    \chessboard[setfen=7n/1p1B4/p4Qpk/3p4/8/P1p3q1/1PP5/1K6 b - - 7 45]
    
    The ending is equal.

    \variation[invar]{33... c3 34. b3 Kf7 35. Bd3 f5 \xskakcomment{ Intending \symrook xd3 then \symqueen e2} 36. Bxf5! gxf5 37. Qg5 Ke6 38. Rxf5 Qe4 39. Qf6+ Kd7 40. Rd1 Kc8 41. Rfxd5 Kb8 42. Rd7 Re1 43. Rxe1 Qxe1+ 44. Ka2 }

    \chessboard[setfen=1k2r2n/1p1R3P/p4Q2/8/8/PPp5/K1P5/4q3 b - - 1 44]

    The ending is equal.

     \item {(Bring the king to safety) \symking f7}

    Steinitz' choice. The middle game struggle continues.
\end{itemize}

 None of these lines is forced; over the board one must eliminate the wrong plan and select the move that best suits one’s style.

\mainline[level=1]{33...Kf7 34. Rhg1 gxf5 35. Qh5+ Ke7 36. Rg8}

\mainline[level=1]{36...Kd6}

\begin{chessdiagram}
\chessboard
\end{chessdiagram}
\emph{I believe it was exactly here that Steinitz made the decisive error:
unlike the other commentators, I fail to see where Black could havesaved himself after this.} (Dvoretsky)

This assessment is exaggerated; only after Black's next move does White obtain a tangible advantage.

\mainline{37. Rxf5}

% TODO: add an intermediate move problem
\begin{chessdiagram}
    \chessboard[
    setfen=4r1Rn/1p5P/p2k1p2/3pqR1Q/2p5/P3r3/1PP5/1K6 b - - 0 37
]
\end{chessdiagram}


\mainline[level=1]{37... Qe6?} 

The queen is under attack and it feels natural to move her immediately.
Yet both Kasparov and Dvoretsky overlooked
\variation[invar]{37... Re1+ 38. Ka2 Qe2 39. Rxd5+ Kc6 40. Rc5+ Kb6 41. Qxe2 R1xe2 42. Rc4}.
Once the queens come off, White's attack evaporates and the passed pawn is firmly blockaded.
The position should be equal. 

\mainline[level=1]{ 38. Rxe8 Qxe8 39. Rxf6+ Kc5 40. Qh6 Re7 41. Qh2 Qd7 42. Qg1+ d4 43. Qg5+ Qd5 44. Rf5 Qxf5 45. Qxf5+ Kd6 46. Qf6+ \xskakcomment{ 1-0}}

\end{multicols}

What an exhilarating fight this was. Lasker emerged two pawns down straight from the opening, yet he managed to keep complicating the position and piling up fresh questions for his opponent. For a long stretch the engine verdict is close to equal—though Steinitz was convinced he was the one pressing—because every time Black solved a problem, Lasker conjured another. Eventually, after neutralizing one threat after another, Steinitz finally slipped.

Game 7 marked the start of five consecutive losses to Lasker. This was an unprecedented humiliation for a man
who had been unbeaten in match play for over 25 years and had previously declared he would win without doubt.

Steinitz attributed his collapse to poor physical condition, particularly his disability which prevented him from walking and exercising properly, causing
``insomnia, rushing of blood to the head, and general depression.''

However, the true cause of Steinitz's defeat was not his physical condition, but rather that he was facing a form of chess he had never encountered before. Lasker was decades ahead of his time. He played this game in the style of Tal with an abstract attack, about 40 years before Tal was born. Even after 100 years, annotators failed to find Steinitz's last mistake even with the help of computers. Steinitz didn't need to be so harsh on himself---it was no wonder he failed to understand Lasker's revolutionary play and why he lost the game. 

\subsection*{Lessons Learned}

One must refrain from playing automatic moves and pay special attention to intermediate moves. In this game, Steinitz's \variation[invar]{23... d5} appeared natural---it closed the center and gained a tempo by attacking the bishop. However, it failed to account for the intermediate move \variation[invar]{24. gxh7+!}, which completely changed the evaluation of the position. Automatic moves often overlook tactical nuances that can turn a winning position into a losing one.

Before launching an attack, it is essential to play some quiet moves prophylactically, removing any potential counterattack in advance. Lasker's \variation[invar]{29. Kb1} and \variation[invar]{30. a3!} exemplify this principle. These seemingly passive moves eliminated tactical resources involving the enemy king's vulnerability, allowing White to proceed with his attack without fear of back-rank or back-rank-related tactics. Such prophylactic thinking requires both accurate evaluation and tremendous composure, especially when material is down.
\newpage
\section{Spassky - Polugaevsky, USSR Championship 1961}


\epigraph{Had White carried out his planned 34.Kf6!! Qxd4+ 35.Kf7! this would have been one of Spassky's best wins: such bold raids by the king under a hail of bullets can be counted on the fingers of one hand, and are part of the golden treasury of chess. \cite{kasparov:2004}}{\textit{Garry Kasparov}}

\begin{multicols}{2}
    \chessgameinfo{USSR Championship}{B.Spassky}{L.Polugaevsky}{10}{1961.01.26}{0-1}
    \newchessgame
    \mainline[level=1]{
    1. d4 Nf6 2. c4 e6 3. Nf3 b6 4. Nc3 Bb7 5. Bg5 Be7 6. e3 Ne4
7. Nxe4 Bxe4 8. Bf4 O-O 9. Bd3 Bb4+ 10. Kf1 Bxd3+ 11. Qxd3 Be7}

\begin{chessdiagram}
    \chessboard
\end{chessdiagram}

White has a lead in development, so playing for a kingside attack is logical.

\mainline[level=1]{12. h4 f5?!}

In defense, it is important not to create weaknesses for the attacker.
The text move closes the b1-h7 diagonal but makes a g4 thrust possible.

White has a simple plan: open the g-file and bring the queenside 
rook to g1 for the attack.

\mainline[level=1]{13. Ke2 d6 14. g4 Nd7 15. Rag1}

\begin{chessdiagram}
    \chessboard
\end{chessdiagram}

\mainline[level=1]{15... fxg4 16. Rxg4 Nf6
17. Rg5 Qd7}

\begin{chessdiagram}
    \chessboard
\end{chessdiagram}

White has two possible plans: double rooks on the g-file then play h5, or play h5 directly. 

If White doubles rooks on the g-file, Black can simply play ...\symrook f7. 
After h5 and ...h6, Black protects the g7 pawn and waits for hxg7. In that case, White's rook on g1 would be better placed on h1. 

Therefore White plays h5 directly.

\mainline[level=1]{18. h5 Ne8 19. Rg2 b5 }

\begin{chessdiagram}
    \chessboard
\end{chessdiagram}

White could have played h6 and then attacked the weak g6 pawn, gaining a material advantage:

\variation[invar]{
    20. h6 g6 21. Rxg6+! hxg6 22. h7+ Kh8 23. Be5+! 
    Ng7 \xskakcomment{ (20... dxe5 21. \symknight xe5 and then \symknight xg6)} 24. Qxg6 Bf6 25. Bxf6 Qf7 26. Bxg7+ Qxg7 27. Qxg7+ Kxg7 28. Ng5 
}

\begin{chessdiagram}
    \chessboard[setfen=r4r2/p1p3kP/3pp3/1p4N1/2PP4/4P3/PP2KP2/7R b - - 1 28]
\end{chessdiagram}

\mainline[level=1]{20. c5}

\begin{chessdiagram}
    \chessboard
\end{chessdiagram}

\mainline[level=1]{20... dxc5}

Black voluntarily weakens his e5 square. ...\symbishop f6 would have been better.

\mainline[level=1]{21. h6 Rf5}

The standard response g6 loses quickly.
\variation[invar]{21...g6 22. Rxg6+ hxg6 23. Qxg6+ Kh8 24. Ne5 }

\begin{chessdiagram}
    \chessboard
\end{chessdiagram}

\mainline[level=1]{22. Be5! c4 23. Qe4 }

\variation[invar]{23. Qxd5 exd5 24. hxg7} is more prosaic, exchanging pieces in a winning position. 

\mainline[level=1]{23...Qd5 24. Qg4 c3 25. b3 b4}

\begin{chessdiagram}
    \chessboard
\end{chessdiagram}

Black has created a protected passed pawn. If he can survive the attack, he will have an advantage.

White is playing for mate. 
Exchanging pieces with \variation[invar]{
    26. Bxg7 Qxf3+ 27. Qxf3 Rxf3 28. Kxf3
} would lead to a straightforward win. 

\mainline[level=1]{26. e4 Qb5+ 27. Ke3 Rf7 28. hxg7 Nf6}

``I had a lot of time. I thought: I'll have 7.5 out of 10, I'm the leader of the qualifying tournament. 
I was overjoyed when I played this forced variation'' (Boris Spassky).

\mainline[level=1]{29. Bxf6 Rxf6 30. Rxh7}

\begin{chessdiagram}
    \chessboard
\end{chessdiagram}

After Black takes the h7 rook, White will play \variation[invar]{
    31. Rh2 Kg8 32. Rh8+ Kf7 33. g8+
}. Black gives checks, hoping to save the game.

\mainline[level=1]{30... Rxf3+ 31. Kxf3 Qd3+ 32. Kf4 Bd6+ 33. Kg5 Kxh7 }

\begin{chessdiagram}
    \chessboard
\end{chessdiagram}

``Erratic thinking at moments of terrible tension, especially with the 
opponent's flag about to fall, occurs even with great players'' (Kasparov). 

``That's the position I wanted, and I saw that there's a very simple win: 
\variation[invar]{
    34. Kf6!! Qxd4 35. Kf7
} and, as Kazimirych would say, amen to the pies, and it would turn out very nicely, 
in Borisenko's words, but, alas, it didn't happen. After this game, I had a 
straight road to the Interzonal tournament, and then further... But something 
terrible happened.
I was haunted by that position for years. Now I'm not—such things happen once 
in a lifetime. But I wanted to do everything the best possible way, 
and I saw a one-move win. So I thought, if I have one move, 
why do I have to get the King to f6 and then f7? So, with 7.5/10 and 
leadership in mind, I played the next move'' (Boris Spassky).

\mainline[level=1]{34. Kh5} 

``And you know—I just forgot about the b5 square. Just forgot. 
Look, in chess, there are some situations when you lose a game, but still get 
something good out of it. You get useful experience, or you see that you 
shouldn't play so rashly and should calculate more precisely, or something'' (Boris Spassky).

\mainline[level=1]{ 34...Qb5+ 35. Kh4 Be7+
36. Kh3 Qg5 37. Qxg5 Bxg5 38. Rxg5 Rd8}

\begin{chessdiagram}
    \chessboard
\end{chessdiagram}

\mainline[level=1]{ 39. f4?}

\variation[invar]{
    39. Kg4 Kg8 40. Rc5 Rxd4 41. Rxc7 Rxe4+ 42. Kg5 Kh7 43. Rxa7 Re2 44. f3 Rg2+ 45. Kf4 Rxg7 46. Rxg7+ Kxg7 47. Ke3 
} would lead to a draw. Kasparov did not point this out in his commentary.

\begin{chessdiagram}
    \chessboard
\end{chessdiagram}

Black finds the right move by \vocab{prophylaxis}{Prophylaxis}{}. Black wins the endgame. \index{Prophylaxis}

\mainline[level=1]{39...Kg8! 40. Rc5 Rxd4
41. Rxc7 Rxe4 42. Kg4 e5 43. a3 Rxf4+ 44. Kg5 a5 45. axb4 axb4
46. Kg6 Rg4+ 47. Kf6 Kh7 48. g8=Q+ Kxg8 49. Kxe5 Rg1 50. Kf6
Rf1+ 51. Ke5 Rb1}
\end{multicols}

\subsection*{Lessons Learned}
It is far from the truth that Spassky was too fragile. He made 
one mistake when he could have mated his opponent. The endgame that followed 
was too difficult to handle. Even Kasparov did not find the 
right move decades later, analyzing with computers. 

Spassky's only mistake was playing too hastily when 
his opponent was in time trouble. By ignoring the clock and playing the right move,
he would have won the game with a brilliancy.

\section{Anatoly Karpov - Garry Kasparov, FIDE World Championship Match 1984/85, Round 3}
\epigraph{An impulsive, nervous reaction, the preceding game would appear to have put me in a not altogether correct
frame of mind, such that I could solve all my problems immediately with the help of a `sharp' pawn sacrifice}{\textit{Garry Kasparov}}

\keywords{Maroczy Bind, Hedgehog Position, Impulsive Reaction}
% My first annotation without books and computers. I chose this game
% because it is quite short. The strategic win from Karpov is also instructive according
% to chessgames.com

% I am happy with the result that I found out 12... Nc5 was bad. 
% I didnt figure out 16... d5 was explicitly because my analysis was not thourough enough and I 
% turned to books too early..

% The winning of Karpov was instructive. I should have taken more time.
\newchessgame[
 id=main,
 event={FIDE World Championship Match 1984/85},
 white={Karpov, Anatoly},
 black={Kasparov, Garry},
 round={3}]

\mainline{1.e4 c5 2.Nf3 e6 3.d4 cxd4 4.Nxd4 Nc6 5.Nb5 d6 6.c4 Nf6 7.N1c3
a6 8.Na3 Be7 9.Be2 O-O 10.O-O b6 } 

\chessboard[
  markstyle=circle,
  linewidth=0.05em,
  markfields={b6},
]

Black has weakness on b6. The next moves of White are logical, attacking 
the weakness and developing his pieces.


\mainline{11.Be3 Bb7} 
Black has a so-called hedgehog position. Typically Black wants to play 
b6-b5 and d6-d5 to free himself. With White's Maroczy bind (c4 and e4 pawns),
he can stop Blacking from playing d5. With White's next move, he stops both b5 and d5 break.

\chessboard[
    pgfstyle=straightmove,
    linewidth=0.05em,
    markmove={b6-b5, d6-d5},
    markstyle=circle,
    linewidth=0.05em,
    markfields={b6},
]

\mainline{12. Qb3}

\chessboard

b6 pawn is under attack. \variation[invar]{12... Nd7} is a natural move.
after \variation[invar]{13. Rad1 \xskakcomment{ Preventing Nc5 with \symbishop xc5. The bishop on
b7 is unprotected..} Qc7 \xskakcomment{intending \symknight c5}} Black is still in the game.

\mainline[level=1]{12... Na5?} 

Kasparov chose to force the matter. After this move White has 3 vs 1 
on the queenside and Black has a backward pawn on the d-file.

\chessboard

\variation[invar]{13. Bxb6 Nxb3 14. Bxd8 Rfxd8 15. axb3 Nxe4 16. Nxe4 Bxe4} is not good enough because
Black's bishop pair gives him enough chance to defend.

\mainline[level=1]{ 13.Qxb6
Nxe4 14.Nxe4 Bxe4 15.Qxd8 Bxd8} (\variation[invar]{15... Rxd8 16. Bb6} This is by the way another
drawback of the move \variation{12... Na5}) 


\mainline[level=1]{ 16.Rad1} 

Quite often the most natural move is the best.

\chessboard

% I wrote "This position must be estimated carefully before Black's 12th move."
% Actually Kasparov has expected only 16. Rfd1 in his preparation and was instantaneously nervous after
% seeing this move. If he could not foresee such a move, I shouldnt expect myself
% to foresee it on the board. 


\mainline{16... d5?}

\quote{An impulsive, nervous reaction, the
preceding game would appear to have put
me in a not altogether correct frame of
mind, such that I could solve all my
problems immediately with the help of a
`sharp' pawn sacrifice} \cite{Kasparov:2008}



\variation[invar]{16... Be7 17. Nb1 \xskakcomment{ (improving his worst piece) } Rac8} Black can still hold on.

\chessboard

\mainline{ 17.f3 Bf5 }

\chessboard

\mainline{18.cxd5
exd5 19.Rxd5 Be6 20.Rd6 Bxa2 21.Rxa6 Rb8 }

\chessboard

\mainline{22.Bc5! Re8 23.Bb5! }
Notice how Karpov used bishops to win tempi and protect
the b2 pawn from the b8 rook. Also note how the Black knight on a5 is doomed by the b5 bishop. 


\mainline{23... Re6
24.b4 Nb7 25.Bf2 Be7 26.Nc2 Bd5 27.Rd1 Bb3 28.Rd7! Rd8} (\variation[invar]{28... Bxc2 29. Rxe6}) \mainline[level=1]{29.Rxe6
Rxd7 30.Re1 Rc7 31.Bb6} 1-0

\subsection*{Lessons Learned}
\begin{itemize}
  \item{Clear your mind, no impulsive, nervous reaction!}
  \item{Evaluate carefully when forcing the exchanges}
  \item{One can make simple natural moves has the advantage}
\end{itemize}
\section{Boris Spassky - Tigran Vartanovich Petrosian, World Championship Match 1969 Game 5}


\newchessgame[
 id=main,
 event={Petrosian - Spassky World Championship Match},
 white={Boris Spassky},
 black={Tigran Petrosian},
 round={5}]

\mainline{
    1. c4 Nf6 2. Nc3 e6 3. Nf3 d5 4. d4 c5 5. cxd5 Nxd5 6. e4 Nxc3
    7. bxc3 cxd4 8. cxd4}
    
\chessboard [
    markstyle=border,
    linewidth=0.05em,
    markfields={c8},
]

I don't like Black's position. White already has a static advantage
because his pawn formation is better and Black has a passive
light square bishop.

On the other hand, White has some natural moves to develop:
\symbishop c4, O-O, \symrook ad1, \symrook fe1 etc.

It is however difficult to recommend any dynamic play for Black here.

\mainline{8... Bb4+ 9. Bd2 Bxd2+ 10. Qxd2 O-O 11. Bc4
    Nc6 12. O-O b6 13. Rad1 Bb7 14. Rfe1} 
    
\chessboard [
    pgfstyle=straightmove,
    linewidth=0.05em,
    markmove={d4-d5},
]

\mainline {14... Rc8 } 

White is planning the d5 thrust to create a central passed pawn.
Black could also tried
\variation[invar]{14... Na5 15. Bf1 Rc8 16. d5 Nc4 17. Bxc4 Rxc4}
    \variation{18. d6 Rxe4 19. Ne5 Rxe1 20. Rxe1 Bd5\xskakcomment{Black can defend the ending}}

% 2025. 11.5: I wrote the following during my own analysis. 
% the conclusion is premature because I miss error on Black's next move
% "Now White has a straight forward move, his light square bishop can
% exchange an important defender, Black's light square bishop 
% while the Black's knight can't help too much in the defense."

\chessboard [
	pgfstyle=straightmove,
    linewidth=0.05em,
 	markmove={d4-d5},
]


% 2025. 11.5: In my analysis, I saw the knight stuck at a5 for a long time.
% I failed to point out, that 16... Na5 is an error.
\mainline[level=1]{15. d5}

\chessboard
% 2025. 11.5: I missed the error completely. Alas, it is a critical position and I failed to 
% see it. 
Black should have played \varriation[invar]{15... Na5}
\variation{16. dxe6 Nxc4? 17. exf7+ Kh8 18. Qxd8 Rcxd8 19. Rxd8 Rxd8 20. e5} White wins.
\variation{16. dxe6 Qxd2 17. exf7+ Kh8 18. Nxd2 Nxc4 19. Nxc4 Rxc4 20. e5 Bc8 21. e6 Bxe6 22. Rxe6} equal.
Petrosian didn't like \variation{16. Bd3 exd5 17. e5! Nc4 18. Qf4} White has an attack. Indeed, Polugaevsky won a brilliant game
against Tal later. Black must find the only defense \symrook c6.

\mainline{15... exd5 16. Bxd5
    Na5 17. Qf4} 


\chessboard

Black has a desperate position. He cannot defend the d5 passed pawn.
White could also play \symqueen f5, \symknight g5 attack the h7 pawn.
    

\mainline{17... Qc7 18. Qf5 Bxd5 19. exd5 Qc2 20. Qf4 Qxa2 21. d6
    Rcd8 22. d7 Qc4 23. Qf5 h6 24. Rc1 Qa6 25. Rc7 b5}
   
\chessboard
    
\mainline{26. Nd4 Qb6
    27. Rc8 Nb7 28. Nc6 Nd6 29. Nxd8 Nxf5 30. Nc6 
}
1-0


\section{Magnus Carlsen - Ding Liren, Magnus Carlsen Chess Tour Finals}
% \keywords{London System, French Defence, Square Play, Diagonal Play}

\epigraph{ In difficult positions I make moves that do 
not lose by force.}{\textit{Anatoly Karpov}}

\keywords{London System, French Defence, Square Play, Diagonal Play}

\begin{multicols}{2}
\newchessgame[
 id=main,
 event={Magnus Carlsen Chess Tour Finals},
 white={Magnus Carlsen},
 black={Ding Liren},
 round={1}]

 \mainline[level=1]{
    1. d4 Nf6 2. Nf3 d5 3. Bf4 c5 4. e3 e6 5. c3 Bd6 6. Nbd2}
    
\chessboard[
    markstyle=border,
    linewidth=0.05em,
    markfields={c8},
]

A typical London System opening: with a hidden cunning idea. We 
reach the first critical moment of the game. White offers a pawn sacrifice. Black 
has more options to choose from. 

\begin{enumerate}
    \item{Decline the sacrifice}

    Black has a pawn stucture similar
    to French Defence. His light squared bishop is not active. He could
    choose to proceed with a normal French Defence setup:
    \variation[invar]{6... Bxf4 7. exf4 Qb6 8. Qc2 Qc7 9. g3 b6 10. Bg2 Bg7}

    \chessboard[
        setfen=rn2k2r/pbq2ppp/1p2pn2/2pp4/3P1P2/2P2NP1/PPQN1PBP/R3K2R w KQkq - 2 11
    ]
    \item{Accept the sacrifice, take on c3}

    \variation[invar]{6... cxd4 7. Bxd6 dxc3} White has compensation and can play \variation[invar]
    {8.Qa4+} or \variation[invar]{8.Ba3} as in the game.

    \chessboard[
        setfen=rnbqk2r/pp3ppp/3Bpn2/3p4/8/2p1PN2/PP1N1PPP/R2QKB1R w KQkq - 0 8
    ]

    \item{Accept the sacrifice, take on e3}

    As in the current game:    
\end{enumerate}
    
\mainline[level=1]{6... cxd4 7. Bxd6 dxe3 8. Ba3 exd2+ 9. Qxd2}

\chessboard[
    pgfstyle=straightmove,
    linewidth=0.05em,
    markmove={d2-g5},
    markstyle=border,
    linewidth=0.05em,
    markfields={c8},
    markstyle=circle,
    linewidth=0.05em,
    markfields={g7},
]


Here another critical moment.

Let us first list the static advantages for White:
\begin{itemize}
    \item{Bishop pair}
    \item{Better King safety}
    \item{Black has an inactive bishop on c8.}
    \item{Black has weakness on the dark squares.}
\end{itemize}
Dynamically White has lead in development for a sacrificed pawn.

Black has however a complete pawn structure. He can castle after some moves. So his
position is still defendable.

White intends to play \variation[invar]{10. Qg5} threatens the g7 pawn. One move is to play
\variation[invar]{10... Ne4 11. Qe3 Qb6 12. Nd4 f6}.
Black can move his king to f7:

\chessboard[
    setfen=r1b4r/pp3kpp/1qn1pp2/3p4/3Nn3/B1PBQ3/PP3PPP/R4RK1 w - - 4 14
]

In the game, Ding chose to play 9... \symknight c6. This move is per se not bad. 
He must however play accurately afterwards

\mainline[level=1]{9... Nc6 10. Qg5 Rg8 11. Bd3 h6
12. Qe3 Qb6 13. Qe2 Bd7 14. O-O O-O-O 15. b4} 

\chessboard[
    pgfstyle=straightmove,
    linewidth=0.05em,
    markmove={b4-b5},
    markstyle=circle,
    linewidth=0.05em,
    markfields={e5},
    markstyle=border,
    linewidth=0.05em,
    markfields={d7},
]

Another critical moment.
White threatens to play b5 to drive away the knight on c6 and then controls the e5 square with his knight:
Once White achieves his goal, Black has a desperate position. Therefore he must react now!

Understanding White's idea, Black can play \variation[invar]{15... e5 16. b5 e4 17. bxc6 Qxc6 18. Ne5 Qc7 
19. Nxf7 Bg4 20. Qe3 Qxf7 21. Bc2}. This move also activates the bishop on d7 as a bonus.

\chessboard[
    setfen=2kr2r1/pp3qp1/5n1p/3p4/4p1b1/B1P1Q3/P1B2PPP/R4RK1 b - - 1 21
]

Black equalizes.

In the game, Ding's move is a serious mistake. 
\mainline[level=1]{15... Kb8? 16. b5 \xskakcomment{ Of course!} Na5
17. Ne5 Be8 18. Bb4 Rc8 19. a4} 

\chessboard

Black is passive but has no weaknesses. He should have waited (e.g. \variation[invar]{19... Qc7}) and
let White to prove his advantage. Here it is important
to remember what Karpov said: in difficult positions I make moves that do 
not lose by force.

After the text move, White wins quickly.
\mainline[level=1]{19... Ne4 20. Bxe4 dxe4 21. Qxe4 f6
22. Qh7 Nb3 23. Qxg8 Nxa1 24. Qxe8 \xskakcomment{ 1-0}
 }

 \chessboard

\end{multicols}
 % 2025. 11.6: I am happy to find the critical positions on my own: 15... Kb8
 \subsection*{Lessons Learned}
 \begin{itemize}
    \item{Knowledge in other openings with similar pawn structure
 can help to find the right idea.}
    \item{Don't make moves that lose by force in difficult positions.}
 \end{itemize}
\section{Ding, Liren - Ian Nepomniachtchi 2023 World Championship Game 4}

\epigraph{
    Ian Nepomniachtchi is a great dynamic player. Such players often
    find it difficult to sit and defend passively. And it seems that
    this position requires exactly that.
}{Grandmaster David Navara}

\keywords{Pawn Sacrifice, Patience in Defense, Practical Play, Passed Pawns}

\begin{multicols}{2}
    \newchessgame[      
        id=main,
        event={FIDE World Championship 2023},
        white={Ding, Liren},
        black={Nepomniachtchi, Ian},
        round={4}]
    \mainline[level=1]{
        1. c4 Nf6 2. Nc3 e5 3. Nf3 Nc6 4. e3 Bb4 5. Qc2 Bxc3 6. bxc3 d6 7. e4 O-O 8. Be2 Nh5 9. d4}

    \chessboard

    There is a fierce fight in the center around d4. An alternative is \variation[invar]{9... Qf6 \xskakcomment{ a natural idea.}
    10. d5 Na5 11. g3 Bg4 } 

    \chessboard[
        setfen=r4rk1/ppp2ppp/3p1q2/n2Pp2n/2P1P1b1/2P2NP1/P1Q1BP1P/R1B1K2R w KQ - 1 12
    ]

    Black has a solid but passive position. It is probably not to Nepo's taste. 

    \mainline[level=1]{9...Nf4 10. Bxf4 exf4 11. O-O Qf6 12. Rfe1 Re8}
    
    \chessboard

    In the middlegame, it is often more about choosing moves and positions according to the 
    players' style than finding the objectively best move.

    White has a solid center and Black has a compact position. Continuing in this manner 
    would suit Ding better because Nepo, as a dynamic player, cannot sit and wait passively.

    \variation[invar]{13. c5 dxc5 14. e5 Qh6 15. Rad1 Bg4 16. Qb3} would be a good idea but not a good practical decision:

    \chessboard[
        setfen=r3r1k1/ppp2ppp/2n4q/2p1P3/3P1pb1/1QP2N2/P3BPPP/3RR1K1 b - - 4 16
    ]

    The position is sharp. White may have some advantage. However, the position 
    is open and Black has counterplay and open lines. What is the point of allowing
    a dynamic player tactical opportunities?

    Ding chooses a natural move, improving his position slowly and not allowing Nepo any counterplay.
    
    \mainline[level=1]{ 13. Bd3 Bg4 14. Nd2}
    
    \chessboard
    
    \mainline[level=1]{14...Na5?}
    
    A very strange move. The knight on a5 has no future. \variation[invar]{14...Rad8} would be a natural move.

    \chessboard
    
    \mainline[level=1]{ 15. c5! \xskakcomment{ With his sacrifice White activates his pieces.} dxc5 16. e5 Qh6 17. d5 Rad8 18. c4}
    
    \chessboard

    White has some advantage here. His center is strong and Black has 
    a passive knight on a5. 

    While the position may still be equal according to the computer, for human players White has a much easier position to play. He controls the center, while his opponent, as a dynamic player, can only sit and wait. At some point, his opponent would lose patience while defending and make mistakes, as happened in this game. 

    \mainline[level=1]{18... b6 19. h3 Bh5}

    \chessboard[
        pgfstyle=straightmove,
        linewidth=0.05em,
        markmove=f4-f3,
        markstyle=circle,
        linewidth=0.05em,
        markfields={e5},
    ]

    \begin{itemize}
        \item Where are the weaknesses?
    
        The e5 pawn is the pivot of the position and must be protected.
        \item Which is the worst-placed piece?
       
        The rooks must be activated.
        \item What is my opponent's idea?
    
        He wants to play \symknight f3, creating some counterplay.
    \end{itemize}
    
    By answering the questions above, White can find the next few moves:
    \begin{itemize}
        \item Move his bishop to e4 then f3 (if Black exchanges the bishop, White has a knight on f3, which further strengthens the e5 pawn).
        \item Move his queen to c3 to protect the e5 pawn.
        \item Double his rooks on the e-file to protect the e5 pawn.
    \end{itemize}

    Black, however, must play move by move.

    \mainline[level=1]{20. Be4 Re7 21. Qc3 Rde8 22. Bf3 Nb7 23. Re2} 
    
    As mentioned above, White has a clear plan and needs only to execute it, without thinking too much. 

    \chessboard

    Black makes a difficult decision here, allowing White to create a passed pawn but gaining 
    a good square on d6 for his knight.
    
    \mainline[level=1]{23... f6 24. e6 Nd6 25. Rae1} 
    
    \chessboard[
        setfen=4r1k1/p1p1r1pp/1p1nPp1q/2pP3b/2P2p2/2Q2B1P/P2NRPP1/4R1K1 b - - 2 25
    ]

    White has executed his plan and Black has defended well. How should Black defend next?

    I believe the question can be answered logically without calculating a lot.
    The main asset of White is the passed pawns, which have been blocked by the Black rooks.

    The rooks on the e-file are not doing much because there is no open file. The knights are 
    important pieces. The Black knight keeps an eye on e4 so that the rooks cannot attack the 
    f4 pawn. 
    
    At some point, White may play \symrook e4 to attack the f4 pawn and exchange the Black knight.

    Alternatively, White may play \symknight e4 to exchange the Black knight. In either case, e4 is 
    an important square and must be protected in advance. \variation[invar]{25... Bg6} is a good move.

    It is unclear how White can make progress here. In the actual game, Black chooses to defend
    actively and soon makes a severe mistake.
    
    \mainline[level=1]{25... Nf5?!} 
    
    \chessboard

    White's bishop cannot improve White's position, while Black's bishop can defend the important e4
    square. It is therefore logical to exchange the bishops and then occupy the e4 square with the rook.
    
    \mainline[level=1]{ 26. Bxh5 Qxh5 27. Re4 Qh6 }
    
    \chessboard
    
    \mainline{28. Qf3}

    Again, a very logical move, attacking the weak f4 pawn. Ding's moves 
    are natural, although not the perfect computer moves. By playing these moves,
    he sets problems for Nepo that cannot be solved using dynamic play---a 
    very practical choice!
    
    \chessboard

    After \variation[invar]{28... g5 29. g4 Nd6}, the position is still defensible for Black. 

    \mainline[level=1]{28...Nd4?? } 
    
    \chessboard

    I am not sure whether Nepo sees a trap that backfires because Ding has set a deeper one. After \variation[invar]{29. Qxf4 Qxf4 30. Rxf4 c6 31. Nf3 Nxf3 32. Rxf3 cxd5 33. cxd5 Rd8 34. Rd3 Rd6} 
    
    \chessboard[setfen=6k1/p3r1pp/1p1rPp2/2pP4/8/3R3P/P4PP1/4R1K1 w - - 3 35]

    White has no advantage.

    More plausibly, Nepo loses his patience in defense and wants to force a draw. 
    
    \chessboard

    \mainline[level=1]{29. Rxd4! } 
    
    Of course, the knight is much more valuable than the rook.

    \mainline[level=1]{29...cxd4 30. Nb3 g5 31. Nxd4 Qg6 32. g4 fxg3 33. fxg3 h5 34. Nf5 Rh7 35. Qe4 Kh8 36. e7 Qf7 37. d6 cxd6 38. Nxd6 Qg8 39. Nxe8 Qxe8 40. Qe6 Kg7 41. Rf1 Rh6 42. Rd1 f5 43. Qe5+ Kf7 44. Qxf5+ Rf6 45. Qh7+ Ke6 46. Qg7 Rg6 47. Qf8}

    \chessboard
\end{multicols}

\subsection*{Lessons Learned}

In the middlegame, it is often more important to choose positions that match your playing style than to find the objectively best move. Ding chose a solid, positional approach that suited him better than sharp tactical lines that would favor his dynamic opponent. He correctly avoided opening the position unnecessarily, which would have given Nepomniachtchi counterplay and tactical chances, even though it might have been objectively good. By choosing positions where he felt comfortable and his opponent would struggle, Ding created opportunities for mistakes.

A well-timed pawn sacrifice can activate your pieces and create a strong center. Sometimes material is less important than piece activity and positional control. When your pieces become active and you gain strategic advantages, a pawn sacrifice can be a powerful tool.

When defending a difficult position, patience is essential. Black's position was still defensible, but an impatient move led to immediate defeat. Dynamic players often struggle with passive defense, and maintaining patience can be the difference between holding the position and losing.

Ding's moves were natural and practical, even if not always the computer's top choice. By setting problems that his opponent couldn't solve with dynamic play, he achieved a practical advantage. Sometimes the best move is not the objectively strongest one, but the one that creates the most problems for your opponent in a practical game.

\section{Ding, Liren - Ian Nepomniachtchi 2023 World Championship Game 6}

\keywords{London System, Knight Maneuvering}
\begin{multicols}{2}
    \newchessgame[      
        id=main,
        event={FIDE World Championship 2023},
        white={Ding, Liren},
        black={Nepomniachtchi, Ian},
        round={6}]
    \mainline[level=1]{1. d4 Nf6 2. Nf3 d5 3. Bf4 c5 4. e3 Nc6 5. Nbd2 cxd4
    6. exd4 Bf5 7. c3 e6} 
    
    \chessboard[
    ]
    
    Modern chess is full of subtleties. The position is actually a reversed
    Queen's Gambit Declined Exchange Variation. Typically after 
    \variation[invar]{1. d4 d5 2. c4 e6 3. Nc3 Nf6 4. cxd5 exd5 5. Bg5 Be7 6. e3 c6}
    
    \chessboard[
        setfen=r1bqk2r/pp1nbppp/2p2n2/3p2B1/3P4/2NBP3/PP3PPP/R2QK1NR w KQkq - 2 8
    ]
    
    We have a so-called Karlsbad structure. White has the following typical ideas:

    \begin{itemize}
        \item{Minority Attack}

        White advances his queenside pawns with the main purpose of creating weaknesses in black’s structure.

        \chessboard[
            setfen=8/pp3ppp/2p5/3p4/3P4/4P3/PP3PPP/8 w KQkq - 2 8,
            pgfstyle=straightmove,
            linewidth=0.05em,
            markmove={b2-b4, b4-b5, b5-c6, b7-c6},
        ]
        
        \item{Playing for the e3-e4 push}

        This can be done with or without the support of the f pawn (by pushing f3) then e4.
    
        \chessboard[
            setfen=8/pp3ppp/2p5/3p4/3P4/4P3/PP3PPP/8 w KQkq - 2 8,
            pgfstyle=straightmove,
            linewidth=0.05em,
            markmove={e3-e4},
        ]   
    \end{itemize}
    
    Black must defend accordingly. His ideas are
    \begin{itemize}
        \item{Moving the knight from g8 to e4}
        \item{Moving the knight from b8 to c4 via d7 and b6.}
    \end{itemize}
    
    We will see in this game that Ding adopts both of Black's ideas from the Queen's Gambit Declined Exchange Variation.  
    Personally I am not a fan of Black's opening. If in the Queen's Gambit
    Declined Exchange Variation, Black can defend well. With one more
    tempo and colors reversed, White should have some advantage. How can Black still
    defend? 

    Let's go back to the current game.

    \chessboard[
        pgfstyle=straightmove,
        linewidth=0.05em,
        markmove={e6-e5},
    ]

    The position is static and slow, meaning both players must
    seek plans episode after episode to improve their positions.

    \begin{itemize}
        \item{Where are the weaknesses?}
    
        It is still too hard to tell.
        \item{Which is the worst-placed piece?}
       
        The bishop on f1 must be developed, then castling.
        \item{What is my opponent's idea?}
    
        He wants to play e5 to free himself.
    \end{itemize}

    Understanding these, White should develop his bishop and then castle.
    He should also pay attention to Black's e5 thrust. So \symrook e1 is natural.

    \mainline[level=1]{ 8. Bb5}

    \chessboard[
        pgfstyle=straightmove,
        linewidth=0.05em,
        markmove={e6-e5},
    ]

    \begin{itemize}
        \item{Where are the weaknesses?}
    
        It is still too hard to tell.
        \item{Which is the worst-placed piece?}
       
        The bishop on f8 must be developed, then castling.
        \item{What is my opponent's idea?}
    
        Not clear yet.
    \end{itemize}

    White has a strong bishop outside its pawn chain on f4. Exchanging it
    with \symbishop d6 and developing is logical. The next few moves are natural.

    \mainline[level=1]{ 8...Bd6 9. Bxd6 Qxd6 10. O-O O-O
    11. Re1 h6} 
    
    \chessboard [
        markstyle=circle,
        linewidth=0.05em,
        markfields={c5, e5},
        markstyle=border,
        linewidth=0.05em,
        markfields={a1, d2},
        pgfstyle=straightmove,
        markmove={e6-e5},
    ]
    
    \begin{itemize}
        \item{Where are the weaknesses?}
    
        c5 and e5 are weak. White may have knights to occupy these
        squares.
        \item{Which is the worst-placed piece?}
       
        Not clear, neither the rook on a1 nor the knight on d2.
        \item{What is my opponent's idea?}
    
        e5 thrust
    \end{itemize}
    
    \mainline{ 12. Ne5} 
    
    \begin{itemize}
        \item{Where are the weaknesses?}
    
        c5 and e5 are weak. White may have knights to occupy these
        squares.
        \item{Which is the worst-placed piece?}
       
        Not clear, either the rook on a1 or the knight on d2.
        \item{What is my opponent's idea?}
    
        e5 thrust
    \end{itemize}
    
    \chessboard [
        markstyle=circle,
        linewidth=0.05em,
        markfields={c5, e5},
    ]

    \begin{itemize}
        \item{Where are the weaknesses?}
    
        c5 and e5 are weak squares.
        \item{Which is the worst-placed piece?}
       
        Not clear.
        \item{What is my opponent's idea?}
    
        \symbishop xc6, then occupy c5 with his knight.
    \end{itemize}

    \mainline{12... Ne7 } 
    
    % TODO:
    % r4rk1/pp2npp1/3qpn1p/1B1pNb2/3P4/2P5/PP1N1PPP/R2QR1K1 w - - 2 13
    % play out!


    \mainline{13. a4 a6}
    This move is okay in itself. However, Black starts a wrong plan that
    causes his position to deteriorate soon.

    \mainline{ 14. Bf1}
    
    \chessboard [
        markstyle=circle,
        linewidth=0.05em,
        markfields={c5},
        markstyle=border,
        linewidth=0.05em,
        markfields={a1, d2},
        pgfstyle=straightmove,
        markmove={a4-a5},
    ]

    \begin{itemize}
        \item{Where are the weaknesses?}
    
        c5 is weak.
        \item{Which is the worst-placed piece?}
       
        Not clear.
        \item{What is my opponent's idea?}
    
        a5 fixing the pawn structure.
    \end{itemize}

    Focus on the position only, \variation[invar]{14... a5} is the right move.
    Ironically, the opponent's plan \variation[invar]{15. a5} exists
    only since Black's last move. It is also psychologically difficult to play
    \variation[invar]{14... a5}, admitting \variation[invar]{13... a6} was a wrong
    plan. 

    \mainline[level=1]{14...  Nd7?} 
    
    \mainline{ 15. Nxd7 Qxd7} Natural moves!
    \mainline{16. a5!} We already know White's plan.
    
    
    \mainline{16... Qc7} 
    
    \chessboard

    White intends to install his knight on c5. A direct \variation[invar]{17. Nb3}
    would be answered with \variation[invar]{17... Nc6}. Black can
    also assault with \variation[invar]{18... Bc2}. White cannot improve his position.

    The queen on c7 is a defender of White's idea. So he tries to exchange it.
    Ding does this skillfully.

    \mainline[level=1]{ 17. Qf3 Rfc8 18. Ra3!} 
    
    \mainline[level=1]{18... Bg6 19. Nb3 Nc6} \mainline[level=1]{ 20. Qg3} 
    
    \chessboard

    Here Black could have defended patiently by playing \variation[invar]{20... Qxg3 21. hxg3 Bf5 22. Nc5 Rc7}.
    His position would be quite solid.
    
    However, Nepo chooses an active plan by countering with e5 in the center. His pieces
    are too passive for such a dynamic plan. White soon gets the upper hand.
    \mainline[level=1]{20... Qe7
    21. h4!}  A typical pawn move to remove back rank weakness in a better position
    before launching an attack.
    
    
    \mainline{ 21... Re8 22. Nc5 e5?} 
    Result of a wrong plan.
    
    \mainline[level=1]{23. Rb3 Nxa5 24. Rxe5 Qf6 25. Ra3 Nc4
    26. Bxc4 dxc4 } 
    
    \chessboard

    Black should have foreseen and estimated this position before he started 
    his plan of countering in the center on his 20th move. His moves 
    are forced while White may still have some improvements.

    Who has a better position? Obviously White. He has full control of 
    the center. Had Nepo estimated this position correctly, he would have 
    chosen a different plan. 
    
    \chessboard
    \mainline{27. h5?  } 
    
    \variation[invar]{27. Nxb7 Rxe5 28. dxe5 Qb6 29. Nd6 Qxb2 30. Nxc4 } would be winning.
    Note White should keep his strong knight to attack the Black king.

    Missed an opportunity!
    \variation{27... Rxe5 28. dxe5 Qd8 \xskakcomment{ This move is hard to see.} 29. Qf3 Qd2 30. hxg6 Qe1+ 31. Kh2 Qxe5+ 32. Kg1 Qe1+ 33. Kh2 Qe5+ 34. g3 \xskakcomment{ Otherwise perpetual check} Qxc5 35. Qxf7+ Kh8 }
    would have saved the game.
    \mainline[level=1]{ 27... Bc2? 28. Nxb7 Qb6 29. Nd6 Rxe5 30. Qxe5 Qxb2}

    \chessboard

    In this position, White can use process of elimination to find the best move.

    \variation[invar]{31. Nxc4?! Qc1+ 32. Kh2 Bd3 33. Qe3 Qd1 34. Ne5 Qxh5+ 35. Qh3 Qxh3+ 36. gxh3 }
    Black has a good chance to defend. 

    \chessboard [
        markstyle=circle,
        linewidth=0.05em,
        markfields={f7},
        markstyle=border,
        linewidth=0.05em,
        markfields={a3}
    ]

    \begin{itemize}
        \item{Where are the weaknesses?}
    
        f7 pawn.
        \item{Which is the worst-placed piece?}
       
        The rook on a3.
        \item{What is my opponent's idea?}
    
        Not clear
    \end{itemize}

    Black has a weakness on f7. \symrook a5-c5-c7 is natural to attack it.
    By the way, White's rook is attacked. There is no other way to parry the threat.

    \mainline[level=1]{ 31. Ra5! Kh7 32. Rc5 Qc1+ 33. Kh2 f6 34. Qg3 a5 35. Nxc4 a4
    36. Ne3 Bb1 37. Rc7 Rg8 38. Nd5 Kh8 39. Ra7 a3 40. Ne7 Rf8
    41. d5 a2 42. Qc7 Kh7 43. Ng6 Rg8 44. Qf7 \xskakcomment{ 1-0}}
    \chessboard
    \end{multicols}


\subsection*{Lessons Learned}
A very deep game played by Ding. He gains an advantage in the middlegame and 
waits patiently until Nepo takes immature measures.

The London system can become a Queen's Gambit Exchange Variation reversed.
This suggests the question of how to handle a Queen's Gambit Exchange Variation.

One example of an answer is that both sides have chances, both sides have their 
own prospects for attack and it is the player who has the greater knowledge and 
skill (and so who can, for example, employ resources that are less obvious than 
the resources which the opponent can employ) who will get the better of it.

When defending, one must choose between passive and active defense.
One can only use active defense when one has practical chances to defend.
Otherwise, an active defense is just suicide.
\section{S.Karjakin - Magnus Carlsen 2013}

\keywords{Ruy Lopez, Prophylaxis, Process of Elimination, Playability}

\begin{multicols}{2}
\newchessgame

\mainline[level=1]{1. e4 e5 2. Nf3 Nc6 3. Bb5 a6 4. Ba4 Nf6 5. O-O Be7 6. Re1 b5 7. Bb3 d6 8. c3 O-O
9. h3 Nb8 10. d4 Nbd7 11. Nbd2 Bb7} 

\chessboard

A typical idea in Ruy Lopez. Black intends to play \symrook e8, \symbishop f8, exd4 and 
then take the e4 pawn. White must protect his e4 pawn sooner or later.

\mainline[level=1]{ 12. Bc2 Re8 13. a4 Bf8} 

\chessboard

White usually maneuvers his d2 knight to f5 via f1, e3 or g3. At the moment
this maneuver is not possible because Black can play exd4 then take the e4 pawn.

\mainline{ 14. Bd3 c6 15. Qc2 Rc8
16. axb5 axb5 17. b4 Qc7 18. Bb2 Ra8 19. Rad1 Bb6 20. c4 bxc4 21. Nxc4 Nxc4 22. Bxc4 h6
23. dxe5 dxe5 24. Bc3 Ba6 25. Bb3 c5 26. Qb2 c4 27. Ba4 Re6 28. Nxe5 Bb7}

\chessboard[
    markstyle=circle,
    linewidth=0.05em,
    markfields={c4, e4, e5},
]

A critical position! The position is full of tactics so that
it is difficult to estimate statically. At the moment both kings
are safe. Piece balance and pawn structure will change dramatically.

White's bishop is being attacked. Black has a weakness
on c4. A natural idea is \variation[invar]{29. Bb5 Ba6 30. Ra1 Bb7 31. Rxa8 Bxa8 
32. Bxc4 Rxe5 33. Bxf7+ Qxf7 34. Bxe5 Nxe4 }.

After a more or less forced line we have the following position. 

\chessboard[
    setfen=b4bk1/5qp1/7p/4B3/1P2n3/7P/1Q3PP1/4R1K1 w - - 0 35
]

White has one rook and two pawns for two pieces. His king is safe. 
The position should be playable for him.

Alternatively, White moves his bishop to c2. He is forced to weaken his 
king safety to protect his weak e4 pawn. 

\chessboard[
    setfen=r4bk1/1bq2pp1/4rn1p/4N3/BPp1P3/2B4P/1Q3PP1/3RR1K1 w - - 1 29
]

\mainline[level=1]{29. Bc2 Rae8 
30. f4 Bd6} 

\chessboard

It was quite easy for Black to find the previous moves. 
White has weaknesses on e5 and e4. Black only needed to double his 
rooks on the e-file and use his pieces to target the e5 square.

However, it is difficult for White to find the right plan. He has 
already weakened his king safety with his kingside pawn movements. He will
also have to play g3 at some point to further weaken his king. 
From this development, we can conclude that White's 29th move was wrong. \variation[invar]{29. Bb5}
would have been better.

Looking forward, White still must find his next move.

\chessboard[
    setfen=4r1k1/1bq2pp1/3brn1p/4N3/1Pp1PP2/2B4P/1QB3P1/3RR1K1 w - - 1 31,
    markstyle=circle,
    linewidth=0.05em,
    markfields={e4,e5},
    pgfstyle=straightmove,
    linewidth=0.05em,
    markmove={f6-h5},
]

\begin{itemize}
    \item{Where are the weaknesses?}

    e4 and e5.
    \item{Which is the worst-placed piece?}
   
    Not clear.
    \item{What is my opponent's idea?}

    He wants to play \symknight h5 to attack the f4 pawn.
\end{itemize}

When Black plays \symknight h5, White must play g3. As prophylaxis,
it is a good idea to protect the g3 pawn first. There are two options
\variation[invar]{31. Kh2} and \variation[invar]{31. Re3}.

Karjakin should have eliminated \variation[invar]{31. Kh2} first. 
His king is on the same diagonal as the Black queen and bishop. 
\variation[invar]{31. Re3} is the right move.
\variation[invar]{31. Re3 Nh5 32. g3 g5} 
(\variation{32...f6 33. Nxc4 Bxf4 34. gxf4 Nxf4 35. Bb3 \xskakcomment{ Black king
is now exposed, White has advantage.}}) \variation {33. Ba4 R8e7 34. Qe2} White is fine.

\mainline[level=1]{ 31. Kh2? Nh5 32. g3 f6 33. Ng6 Nxf4}

\chessboard

\variation[invar]{34. gxf4 Bxf4+ 35. Kh1 Rxe4 36. Bxe4 Rxe4 37. Kg1 Be3+ 38. Rxe3 Rxe3 39. Qh2 Qxh2+ 40. Kxh2 Rxc3 41. Qh3}
Black is winning. Karjakin finds the only defense.

\mainline[level=1]{34. Rxd6 Nxg6 35. Rxe6 Rxe6 36. Bd4 f5
37. e5} 

\chessboard

White is still paying debts for weakening his own king safety voluntarily.
Black can attack along the main diagonal. Actually he only needs to make
natural moves and again White must come up with good defense.

\mainline[level=1]{37... Nxe5! 38. Bxe5 Qc6}

\chessboard

Here again, White must find the right defense. 

\variation[invar]{39. Be4 fxe4 40. Re3} is his last chance. After the text move, he
has no more chances.

\mainline[level=1]{39. Rg1? Qd5 40. Bxf5 Rxd5 41. Bg4 h5 42. Bd1 c3 43. Qf2 Rf5 44. Qe3 Qf7 45. g4 Re5 46. Qd4 Qc7 \xskakcomment{ 0-1}}

\end{multicols}

\subsection*{Lessons Learned}
Practically a player should choose to play playable positions.
We see again and again during the game, White struggles
to find the right move while Black only needs to make
logical moves. Such play is of course tiring for White and 
at some point, he makes decisive mistakes and then loses the game.

King safety is always the most important thing. It is unwise 
to weaken king safety.

Prophylaxis and Process of Eliminations are important
tools to find the right move without calculating too much. 
At move 31, White finds the right idea with prophylaxis.
However, he fails to choose the right move with Process of 
Elimination.
\newpage
\section{Ding Liren - Levon Aronian, Alekhine Memorial 2013}

\begin{multicols}{2}
    \chessgameinfo{Alekhine Memorial}{Ding Liren}{Levon Aronian}{}{2013.09.18}{1-0}
    \newchessgame
\mainline[level=1]{1. d4 d5 2. c4 c6 3. Nf3 Nf6 4. Nc3 a6 5. e3 e6 6. c5 Nbd7 7. b4 b6 8. Bb2 a5 9. a3 Be7 10. Bd3 O-O 11. O-O Ba6 12. Ne1} 

\chessboard

\mainline{12...Bc4?}

Black has less space, therefore he should exchange the bishops. \variation[invar]{12...Bxd3} would have been better.
Aronian overestimates his passed pawn. The pawn has no future in the middle game because
it is blocked by the White pieces. On the other hand, he gives up the center.
White will soon have the initiative because he will occupy the center with e3-e4 thrust.

\mainline[level=1]{13. Bxc4 dxc4 14. Qe2 Rb8 15. Ra2 b5}

\chessboard[
    markstyle=circle,
    linewidth=0.05em,
    markfields={d6},
    pgfstyle=straightmove,
    markmove={e4-e5, e3-e4},
]

Black has created a protected passed pawn as he has planned. The price is however too high. 
White occupies the center and can maneuver freely while Black can only sit and wait in restricted space.

It is instructive to see how White exploits the weakness on d6:
\begin{enumerate}
    \item{Protect the d4 pawn}
    \item{Target d6 with bishop and knight}
    \item{Occupy d6 with the knight}
\end{enumerate}

It is also important to remember, don't rush!

\mainline[level=1]{16. e4 Rb7 17. Nc2 Nb8 18. Raa1 Qc8 19. Rad1 Rd8 20. Bc1 Na6 21. Bf4 Rbd7 22. h3 Ne8 23. Qe3 Bf6 24. e5! Be7 25. Ne4!
 Nac7 26. Nd6}
 
 \chessboard

 Look at the position. White dominates the center while Black is cramped in his own half.

\mainline[level=1]{26... Qa8 27. Qg3 Nd5 28. Ne3 Nc3 29. Rde1 Bxd6 30. exd6 Ne4 31. Qh4 Nd2}

\chessboard

\mainline[level=1]{32. Nd5! Nxf1 33. Nb6! Qa7 34. Rxf1 Nf6 35. Be5 Nd5 36. Nxd5 exd5}

% TODO: create a tactical problem
\chessboard[
    setfen=3r2k1/q2r1ppp/2pP4/ppPpB3/1PpP3Q/P6P/5PP1/5RK1 w - - 0 37
]

The final blow!
\mainline[level=1]{37. Bxg7 Kxg7 38. Qg5+ Kf8 39. Qf6 Kg8 40. Qg5+ Kf8 41. Qf6 Kg8 42. Re1 axb4 43. Re5 h6 44. Rh5 Qxa3 45. Qxh6 f6 46. Qxf6 \xskakcomment{ 1-0}
}

\end{multicols}

\subsection*{Lessons Learned}
It is crucial to carefully weigh the advantages and disadvantages of any strategic decision.
Giving up control of the center in exchange for a protected but immobile passed pawn is rarely a good trade.
The center provides flexibility and initiative, while a blocked passed pawn offers little practical value in the middlegame.

When liquidating an advantage while the opponent has no counterplay, thorough preparation is essential. 
One must protect all weaknesses and eliminate any potential counterplay before delivering the final blow. 
Rushing to convert an advantage can allow the opponent to generate unexpected complications.
\newpage
\section{Magnus Carlsen - Li Chao, Qatar Masters 2015}

\begin{multicols}{2}
\chessgameinfo{Qatar Masters}{Magnus Carlsen}{Li Chao}{}{2015.12.24}{1-0}
\newchessgame

\mainline[level=1]{1. d4 Nf6 2. c4 g6 3. f3 d5 4. cxd5 Nxd5 5. e4 Nb6 6. Nc3 Bg7 7. Be3 O-O 8. Qd2 Nc6 9. O-O-O f5 10. e5 Nb4 11. Nh3 Qe8 12. Kb1}

\chessboard

This is a typical position in the Grünfeld Defense. White has a pawn center, and Black's
active pieces allow for counterplay.

Here we see the famous quiet king move again. Carlsen moved his king to 
a safer square before starting his attack on the kingside. \variation[invar]{12. Nf4} would also be possible. 

\mainline[level=1]{12... a5}

The kings castle on opposite sides. Mate or being mated is the most probable
outcome. Therefore, tempi are precious. 
Black was trying to postpone \symbishop e6 as much as possible, since White
could win a tempo with \symknight f4 and then start the attack with h4.

White was trying to postpone \symknight f4 as much as possible, since Black 
could react with g5 and f4, halting White's attack.

Both sides were making useful moves until reaching a ``zugzwang'' position.

\mainline[level=1]{13. Be2 c6}

In the post-game interview, Magnus said that c6 was a good move,
but he was happy to see it because now the queen on e8 cannot 
really go to a4, and there will be no mate threat.

\mainline[level=1]{14. Rc1 Kh8 15. Ka1}

\chessboard

\mainline[level=1]{15... Be6}

Now Black played the unpleasant move!

Black could still play \variation[invar]{15... N6d5 16. Nxd5 Nxd5}. This 
position would have been much more solid than in the game.

\mainline[level=1]{16. Nf4 Qf7 17. h4}

\chessboard

What does White want? Obviously, he wants to play h5, opening the h-file. \variation[invar]{17... g5} would 
be refuted with \variation[invar]{18. Ng6}. As \vocab{prophylaxis}{prophylaxis}, Black should play \variation[invar]{17... Rfd8}
so that \variation[invar]{18. Ng6} is not possible. \index{Prophylaxis}

\mainline[level=1]{17... Bxa2}

The text move is too slow.

\mainline{18. h5}


\mainline[invar]{18...Kg8 }

Black must parry \symknight g6. However, it feels wrong. The king had moved from g8 to h8 and then back to g8 again.

\mainline[level=1]{19. hxg6 hxg6 20. g4 }


\chessboard

Here prophylactic thinking is helpful. What does White want? He wants to move his 
queen to the h-file, check on h7, and then take the pawn with his knight. Black
could drive the knight away with \variation[invar]{20... g5 21. Nh3 f4 22. Bf2 Bd5 23. Nxg5 Qg6} and
could still hold the position.

\mainline[level=1]{20...Bb3?}

Now we see how White makes natural moves to attack.

\mainline{21. Bd1! a4 22. Qh2 Rfd8 23. Qh7+ Kf8 }

\chessboard

\mainline[level=1]{24. d5! }

White needed to find the decisive blow. \variation{24. Nxg6 Ke8} goes nowhere. Noticing
that Black's queen lacks space, the text move blocks the support from Black's bishop, and the 
Black queen will be trapped.

\mainline[level=1] {24...Nc4 25. Nxg6+ Ke8 26. e6}

\chessboard

Apparently, the Black king is trapped now.

\mainline{26... a3!}

Exclamation mark for the entertainment value. 

\chessboard

Black is also threatening mate: \variation[invar]{27... axb2 28. Kb1 Ra1#}.

\mainline[level=1]{27. exf7+ Kd7 28. Ne5+ Bxe5 29. Qxf5+ Kc7 30. Qxe5+ Nxe5}

\chessboard 

Now the game is effectively over. White has a material advantage, and Black has no more mating threats.

\mainline[level=1]{31. Bxb3 axb2+ 32. Kb1 Nxc1 33. Rxc1 Kc8 34. dxc6 bxc6 35. f4} Black resigned.

\end{multicols}
\

\else
	% Lite build: skip ideas, games and annotations to speed up compilation.
	\newpage
\section{Catalan Opening: Setup Pawn Center}
If the circumstance allows, White can also setup a pawn center.

\begin{multicols}{2}
    \chessgameinfo{Amsterdam IBM Tournament}{B.Spassky}{D.Ciric}{}{1970}{1-0}

    \newchessgame
    \mainline[level=1]{
        1. d4 d5 2. c4 e6 3. Nf3 Nf6 4. g3 Be7 5. Bg2 O-O 6. O-O c6 7. b3 Nbd7 8. Bb2 b6 9. Nbd2 Bb7 10. Rc1 Rc8 11. e3}
    
        ``White is planning to place the queen on e2 and not on the usual c2 square, where
    it would be `exposed' to the c8 rook'' (Najdorf).

    \mainline[level=1]{11... c5 12. Qe2}
    
    \chessboard

    ``White is preparing \symrook fd1 and an eventual \symknight e5'' (Spassky).

    \mainline[level=1]{12... Rc7 13. cxd5}
    
    \chessboard
    
    Critical moment! The next move decides the character of the game.

    \variation[invar]{13... exd5 14. dxc5 bxc5 15. Ne5}

    \chessboard[
        setfen=3qr1k1/pbrnbppp/5n2/2ppN3/8/1P2P1P1/PB1NQPBP/2RR2K1 b - - 3 16
    ]

    Black has \vocab{hanging-pawns}{Hanging pawns}{two side-by-side pawns on adjacent files (here the c- and d-pawns) with no friendly pawns on neighboring files behind them to support them. They grant space and dynamic piece activity but can become long-term weaknesses if blockaded or forced to advance} on the c- and d-files. The position is playable for both sides.
    
    The text move allows White to setup a pawn center.
    \mainline[level=1]{13... Bxd5 14. e4 Bb7 15. e5 Nd5 16. Nc4}

    White has made some natural moves. 

    \mainline[level=1]{16...Qa8 }
    
    \chessboard 

    \mainline[level=1]{17. Nd6!}
    
    White sacrifices a pawn to activate his pieces (his pawn will fall in 
    the next a few moves). \index{Pawn Sacrifice!Piece Activity}

    \mainline[level=1]{17...Bxd6 18. exd6 Rc6 19. dxc5 bxc5 20. Ng5 Rxd6 21. Rfd1!}
    
    ``The strongest move, pinning the knight and bringing his last inactive piece into play'' (Bernard Cafferty).
    It is also good practical play, if one cannot find a decisive blow, just bring new 
    pieces into play.
    
    The double sacrifice on h7 and g7 is too hasty: \variation[invar]{21. Nxh7 Kxh7 22. Qh5+ Kg8 23. Bxg7 Kxg7 24. Qg5+ Kh8 25.Rc4
    Nf4! 26. Rxf4 Rd4!}

    \mainline[level=1]{21... Ra6}
    
    \chessboard

    \mainline[level=1]{22. Qe4}
    
    Amusingly, now the double sacrifice mentioned in the last move works because the Black rook has moved 
    to a6! Sometimes the opponent will help us. \index{Attack!Double Sacrifice}
    \variation[invar]{22. Nxh7 Kxh7 23. Qh5+ Kg8 24. Bxg7 Kxg7 25. Qg5+ Kh8 26.Rc4} White mate soon
    
    \mainline[level=1]{22...f5 23. Qc4 Qe8 24. Re1 Rxa2 25. Rxe6 Qa8 26. Bxd5 Bxd5 27. Qh4 h6}
    
    \chessboard

    \mainline[level=1]{28. Qxh6! Nf6 29. Rxf6}
    
    1-0. There is no defense after \variation[invar]{29... Rxf6 30. Qh7+ Kf8 31. Qh8+ Bg8 32. Bxf6}.
\end{multicols}
\fi

%----------------------------------------------------------------------------------------
%	BIBLIOGRAPHY
%----------------------------------------------------------------------------------------

\chapterimage{} % Chapter heading image
\chapterspaceabove{2.5cm} % Whitespace from the top of the page to the chapter title on chapter pages
\chapterspacebelow{2cm} % Amount of vertical whitespace from the top margin to the start of the text on chapter pages

%------------------------------------------------

\chapter*{Bibliography}
\markboth{\sffamily\normalsize\bfseries Bibliography}{\sffamily\normalsize\bfseries Bibliography} % Set the page headers to display a Bibliography chapter name
\addcontentsline{toc}{chapter}{\textcolor{ocre}{Bibliography}} % Add a Bibliography heading to the table of contents

%\section*{Articles}
%\addcontentsline{toc}{section}{Articles} % Add the Articles subheading to the table of contents

\printbibliography[heading=bibempty, type=article] % Output article bibliography entries

\section*{Books}
\addcontentsline{toc}{section}{Books} % Add the Books subheading to the table of contents

\printbibliography[heading=bibempty, type=book] % Output book bibliography entries

%----------------------------------------------------------------------------------------
%	INDEX
%----------------------------------------------------------------------------------------

% Workaround: index entries are off by 2 pages in the .idx file.
% Adjust the page number and link target used by \hyperpage (only used in the index).
\makeatletter
\renewcommand*{\hyperpage}[1]{%
  \begingroup
  \edef\Hy@page{\number\numexpr#1+1\relax}%
  \hyper@linkstart{link}{page.\Hy@page}%
  \Hy@page
  \hyper@linkend
  \endgroup
}
\makeatother

\cleardoublepage % Make sure the index starts on an odd (right side) page
\phantomsection
\printindex % Output the index

%----------------------------------------------------------------------------------------

\end{document}
