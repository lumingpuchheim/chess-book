\chapter*{Introduction}
\epigraph{
    ``Knowledge is power.''
    \begin{flushright}
        --- Petyr Baelish, \emph{Game of Thrones}
    \end{flushright}
}

One day I was studying how Ulf Andersson beat Ivanov in 2000 using 
the Catalan Opening. He steered the game into an endgame 
and then won it with a precise king maneuver. 

That same evening I came across the famous Rubinstein pawn 
endgame where he won with exactly the same king maneuver.
To my astonishment, in both games the defender had the same pawn structure and 
the attacker followed the same king invasion plan.

The Andersson game:
\chessboard[
    setfen=3r2k1/5p1p/p3pp2/1pb5/8/2N3P1/PP2PPKP/5R2 w - - 0 20
]

The Rubinstein game:
\chessboard[
    setfen=8/pp2kppp/4p3/8/1P6/P3PP2/5P1P/2K5 b - - 0 25
]

At that moment it was clear to me that Rubinstein's game must have 
inspired Andersson. He had prepared a Catalan Opening line 
specifically to reach a similar endgame and execute the same king maneuver.

This experience reminded me of the German sociologist Niklas Luhmann. 
He wrote his articles and books with the help of a vast Zettelkasten 
(slip-box) containing thousands of handwritten index cards, each 
recording a small idea and linked non‑linearly to related cards. 
This network of notes was central to his productivity and originality.

I have written this book using a small Zettelkasten of my own. 
You will see individual ideas I have collected, and you will also see 
how they are woven together in complete games. An opening idea to sacrifice 
a pawn may later reappear as an attacking idea in the middlegame or a winning 
plan in the endgame. I hope that, in the end, you will recognise these 
connections and win more games.

You may read the book in any order, since the ideas are largely self‑contained. 
In the section ``Index'' you will find an idea index, where you can look up 
all the ideas in the book. I hope you will find it useful.
\addcontentsline{toc}{chapter}{Introduction}
\section{Explanation of Symbols}

The chessboard with its coordinates:

\chessboard[
    setfen=rnbqkbnr/pppppppp/8/8/8/8/PPPPPPPP/RNBQKBNR w KQkq - 0 1,
    showmover=false,
]

\vspace{1em}

\begin{tabular}{@{}ll@{}}
    \symking & King \\
    \symqueen & Queen \\
    \symrook & Rook \\
    \symbishop & Bishop \\
    \symknight & Knight \\
    $\pm$ & White stands slightly better \\
    $\mp$ & Black stands slightly better \\
    $\pm$ & White stands better \\
    $\mp$ & Black stands better \\
    $+-$ & White has a decisive advantage \\
    $-+$ & Black has a decisive advantage \\
    $=$ & balanced position \\
    $!$ & good move \\
    $!!$ & excellent move \\
    $?$ & bad move \\
    $??$ & blunder \\
    $!?$ & interesting move \\
    $?!$ & dubious move \\
\end{tabular}


\section*{Visualization the Three Questions}
We use a simple graphical system to illustrate the three questions as used by Aagaard in his book ``Grandmaster Preparations Positional Play''. 

We will use circle to identify weakness. We will use square to identify the worst-placed pieces. We will use arrows to illustrate the opponent's ideas.


Let us look at an example 

\subsection*{Hikaru Nakamura - Vladimir Kramnik}

 \chessboard[
        setfen=3r2k1/p4pbp/b3p3/npqpP1pP/8/2P3P1/P3QPBN/R3R1K1 w - - 0 1,
 	markstyle=circle,
 	linewidth=0.05em,
 	markfields={g5, f6},
	markstyle=border,
	markfields={g2},
	pgfstyle=straightmove,
 	markmove=h7-h6
        ]

Black has weakness on g5 and f6. All his pieces are bad. White's worst piece is the bishop on g2.

Black intends to play h6 to protect his biggest weakness: g5 pawn. To prevent this move White can play h6 himself and then \symqueen h5 attack the weak pawn on g5.




\section*{Process of Elimination}
\section{Process of Elimination}
\vocab{PoE}{Process of Elimination}{Ruling out bad moves, not immediately finding the perfect one. Commonly used in calculation, defense, and endgames}
 is the method instead of searching for a brilliant move directly, 
you narrow the field\cite{Dvoretsky:2015}.

How it works in practice:

\begin{itemize}
\item List candidate moves that are natural or forcing.
\item Eliminate bad moves by concrete reasons.
\item Compare the remaining moves and choose the one that best fits the position.
\end{itemize}
\index{Process of Elimination} 
