\section{Resourcefulness in Chess}

Shackleton's story is not about victory. It is about what happens when everything goes wrong: the ship sinks, the plan fails, the future disappears. In those moments, resourcefulness becomes the difference between survival and surrender.

Chess offers the same test. Every player faces losing positions—material down, king exposed, pieces trapped, time running out. The natural response is resignation: accept defeat, stop fighting, give up. But resourcefulness means finding another way.

When material is lost, resourcefulness looks for activity. When the king is exposed, it finds counterplay. When pieces are trapped, it creates complications. When time is short, it simplifies to what matters most. The goal shifts from winning to surviving, from brilliance to persistence.

Shackleton did not cross Antarctica. He did not achieve his original dream. But he brought every man home. In chess, resourcefulness means the same: you may not win the game, but you can make your opponent work for it. You can create problems, find counterchances, turn a lost position into a difficult one, a difficult one into a draw, a draw into a win.

In section \ref{sec:lasker-steinitz-1894} we saw how Lasker prevented an immediate loss after he was down in material. Then he resourcefully created problems for Steinitz. The position was unclear and finally Steinitz lost a so-called ``winning game''.

We will present one more example.

\newpage
\section{Vladimir Kramnik - Viswanathan Anand, Belgrade Investbanka, 1997}

\begin{multicols}{2}
    \chessgameinfo{Belgrade Investbanka}{V.Kramnik}{V.Anand}{}{1997.11.12}{0-1}
    \newchessgame
    \mainline[level=1]{
    1. Nf3 Nf6 2. c4 e6 3. Nc3 d5 4. d4 c6 5. Bg5 h6 6. Bh4 dxc4 7. e4 g5 8. Bg3 b5 9. Be2 Bb7 10. e5 Nh5 11. a4 a6 }
    
    White finds a brilliant sacrifice. 

    \begin{chessdiagram}
        \chessboard
    \end{chessdiagram}

    \mainline[level=1]{12. Nxg5 Nxg3 13. Nxf7 Kxf7 14. fxg3}
    
    \begin{chessdiagram}
        \chessboard
    \end{chessdiagram}

    Black finds himself in a difficult position: he has a lag in development and 
    his king is exposed. What can he do now? Applauding the opponent's play 
    is at least too early. Regretting his last move doesn't help either. 

    Since his king is exposed, his next move is natural. Of course White will 
    have some initiative, but it is no good to feel overwhelmed now. Let's solve the problem step by step.
    
    \mainline[level=1]{14...Kg8 15. O-O Nd7 16. Bg4 Qe7 17. Ne4}
    
    Black's last two moves are natural, developing his knight with 15... \symknight d7 and 
    protecting his e6 pawn with 16... \symqueen e7, also the only move. 

    \begin{chessdiagram}
        \chessboard
    \end{chessdiagram}

    Certainly White intends to play 18. \symknight d6, forcing Black to play 18... \symrook b8.
    Black's pieces are passive. Anand decides to protect his f7 square in advance.
    \variation[invar]{17... c5 18. Nd6 Bd5 19. Bf3 Rh7} is a good alternative because 
    Black keeps his material advantage and neutralizes White's initiative, and it is unclear how White will continue in 
    the next few moves. Perhaps Anand rejects this move because he 
    doesn't see 19... \symrook h7. The text move is however also not bad. 
    It is up to White to show that he has compensation for the sacrifice before 
    Black completes his development. We will soon see that White makes a mistake.
    
    \mainline[level=1]{17...Rh7 18. Nd6 Rb8 }
    
    \begin{chessdiagram}
        \chessboard
    \end{chessdiagram}
    
    \mainline[level=1]{19. b4?}
    
    ``Vlad now sank into thought for a long time - I imagined that he was trying to decide which of several promising continuations to go for. As it turned out, he had spent a long time on many lines, and didn't find anything convincing. Then he saw a move which discourages 
    Black's 'only' resource - c6-c5 and decided to go for it.'' (Anand).

    It is instructive to see even grandmasters make such psychological mistakes, 
    thinking for a long time and then deciding on a move that has not been verified.
    That's definitely an error to avoid both on the chessboard and in life. 

    On the other hand, we see Anand rewards himself by defending calmly. He just 
    avoids worsening his position, and then a good surprise comes.
    
    \variation[invar]{
        19. axb5 cxb5 20. Nxb7 Rxb7 21. Rxa6 Rb6 22. Rxb6 Nxb6 23. Rf6
    } White is crushing.

    \begin{chessdiagram}
        \chessboard
    \end{chessdiagram}

    \mainline[level=1]{19...h5!}
    
    Black reminds White that he also has a king.
    
    \mainline[level=1]{20. Bh3}
    
    \variation[invar]{
        20. Bxh5 Qg5 21. Bf7+ Kh8 22. Nxb7 Qe3+ 23. Kh1 Qxg3 24. h3 Rxh3+ 25. gxh3 Qxh3+ 
    } Black gives perpetual check. White rejects it because he thinks his 
    position is better.
    
    \mainline[level=1]{20... Bh6 21. Kh1 Bg5 22. Qc2 Rg7 23. Qe2}
    
    \begin{chessdiagram}
        \chessboard
    \end{chessdiagram}

    Black has developed his pieces. Now he plans to move his bishop to a8, freeing 
    his rook on b8 from protecting the bishop. 
    
    \mainline[level=1]{23... Ba8 24. Qxh5 Rf8 25. Ne4 }
    
    \begin{chessdiagram}
        \chessboard
    \end{chessdiagram}
    
    \mainline[level=1]{25...c5! 26. Nxg5}
    
    \begin{chessdiagram}
        \chessboard
    \end{chessdiagram}

    \mainline[level=1]{26...Bd5!}
    
    ``Remarkable. Black simply ignores the fact that White has taken a piece and focuses on the real priority: 
    supporting the crucial e6-pawn.'' (Anand).

    \variation[invar]{
        26... Rxg5 27.Bxe6+ Kg7 28.Qh4 
    } is also playable. 
    
    \mainline[level=1]{27. Nf3?}

    Stockfish recommends
    \variation[invar]{
        27. Nxe6 Bxe6 28. bxc5 Bd5 29. c6 Rxf1+ 30. Rxf1 Nf8 31. e6 Qd6 32. axb5 axb5 33. c7 Rxc7 34. Rf7 Bxe6 35. Rf6 
    }; the game is equal.

    I doubt if a human player can see this. It is instructive to see 
    an endgame principle: the defender uses a bishop along one diagonal 
    to defend the passed pawns, and the attacker tries to 
    deflect it with 29. c6.

    \mainline[level=1]{27... cxb4 28. axb5 axb5}
    
    \begin{chessdiagram}
        \chessboard
    \end{chessdiagram}

    Now Black is simply better. His passed pawns on the queenside will decide 
    the game.
    
    \mainline[level=1]{29. Nh4 Qg5 30. Rxf8+ Kxf8 31. Qe8 Rf7 32. Nf3 Qg6 33. Qxb5 b3 34. Rf1 Qe6 35. Rg1 Qe3+ 36. Kh1 c3 37. Bxe6 Bxe6 38. d5 Rxf3 39. gxf3 Bh3 40. Qc4 Bxf1 41. Qg4+ Kh7 42. e6 Qh4+ 43. Kh1 Ng6}
    
\end{multicols}

\subsection*{Lessons Learned}

After Kramnik's brilliant sacrifice, Anand finds himself in a difficult position. His king is exposed, his development lags, and White has the initiative. The natural response might be panic or resignation. But Anand shows how to handle adversity: step by step, without weakening the position further.

First, he moves his king to safety. Then he develops his pieces, one by one. He doesn't try to solve everything at once. He doesn't make desperate moves that create new weaknesses. 

As he stabilizes, he begins to pose questions to White. With material down, how does White want to proceed? The burden shifts from defending to making the opponent prove their advantage. 

When Kramnik makes a mistake, Anand is ready. He doesn't just defend—he counterattacks. His sacrifice of the piece activates his entire army and unleashes his passed pawns. What began as a defensive struggle ends in victory.


