\chapter{Things Don't Change}

\section{What Doesn't Change: The Bezos Approach}
Jeff Bezos, the founder of Amazon, has always asked a fundamental question: what doesn't change? While others focus on trends, technologies, and shifting markets, Bezos built his strategy around constants. The answer he found was simple: customers always want lower prices, faster delivery, and more choices. These desires are timeless—they were true when Amazon started, and they remain true today.

This insight became the foundation of Amazon's entire business model. Instead of chasing the latest trend or trying to predict which technology would dominate, Bezos focused on serving these unchanging customer needs. Every decision, every investment, every innovation was measured against this question: does this help us deliver lower prices, faster service, or more selection? By building on what doesn't change, Amazon created a business that could adapt to any new technology or market shift, because the core value proposition remained constant.

The power of this approach is that it provides a stable foundation in a world of constant change. When you know what doesn't change, you can invest confidently in improving those areas, knowing your efforts will compound over time rather than becoming obsolete. You can ignore the noise of temporary trends and focus your energy on what truly matters.



\section{Endgame Theory: A Permanent Investment}
In chess, opening theories come and go. What was considered best practice a decade ago may be refuted today. New variations are discovered, engines find improvements, and the theoretical landscape shifts constantly. Following these changes is a waste of energy because the opening you study may be outdated by the time you need it, or the specific variation you memorize may never appear in your actual games. The opening is a moving target, and chasing it requires constant reinvestment of time and effort.

Endgame theory is different. Once you learn the fundamental principles of endgames—opposition, key squares, triangulation, the rule of the square, basic rook and pawn endings—this knowledge will probably never change. The endgame is built on mathematical and logical foundations that remain constant regardless of opening theory or engine improvements. A winning technique discovered a century ago is still winning today. A defensive method that saved a game in 1920 will still save a game in 2020.

This makes endgame study a uniquely stable investment. The time you spend learning endgame principles compounds over your entire chess career. Unlike opening preparation, which can become obsolete, endgame knowledge is permanent. You can study it thoroughly, master it deeply, and know that what you learn today will serve you for decades. This is why strong players consistently emphasize endgame study: it is the one area of chess where your investment in learning truly pays permanent dividends.



\section{Endgame Transitions}
Having knowledge about certain endgames will save us time and energy in calculation. Chess endgames change
from one to another. For example, a rook endgame may transpose into a queen and rook endgame/middlegame,
a pawn endgame, or a queen vs. rook endgame. It is therefore important not only to remember the tricks
in a certain endgame, but also to be aware of or proactively search for endgames that may arise from the
current position.

In chess, we are not only searching for the best move, but also for the best possible transposed position.
We have knowledge about certain endgames. This knowledge is useful when we can connect these
endgames to our current position. For example, pawn endgame positions in which two rook pawns are facing
each other, with one side having a distant passed pawn, are quite common. When we are defending a rook
endgame where the pawns are described as above, the question becomes: can we force a rook exchange, and
is the resulting pawn endgame a draw? If both answers are yes, we have a clear plan for how to defend.

The same principle applies to life. We face unknown situations constantly—new problems, unfamiliar challenges, unexpected circumstances. But we also have knowledge of known situations: patterns we've seen before, problems we've solved, principles we understand. The art lies in recognizing how to transpose the unknown situation into a known one.

When you encounter a difficult negotiation, you may not have faced this exact scenario before. But you know the principles of negotiation: understanding interests, finding common ground, creating value. The unknown situation becomes manageable when you transpose it into the framework of known principles.

When you face a career transition, the specific circumstances are new. But you understand the patterns of change: assessing your skills, identifying opportunities, building relationships. By transposing the unknown into the known, you create a path forward.

The skill is not in memorizing every possible situation, but in developing the ability to see how unfamiliar problems map onto familiar patterns. Just as a chess player learns to recognize when a complex position can be simplified into a known endgame, we learn to recognize when a complex life situation can be understood through principles we already know.