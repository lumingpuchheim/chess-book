One day I was studying how Ulf Andersson beat Ivanov in 2000 using 
the Catalan Opening. He steered the game into an endgame 
and then won it with a precise king maneuver. 

That same evening I came across the famous Rubinstein pawn 
endgame where he won with exactly the same king maneuver.
To my astonishment, in both games the defender had the same pawn structure and 
the attacker followed the same king invasion plan.

The Andersson game:
\begin{chessdiagram}
    \chessboard[
        setfen=3r2k1/5p1p/p3pp2/1pb5/8/2N3P1/PP2PPKP/5R2 w - - 0 20
    ]
\end{chessdiagram}

The Rubinstein game:
\begin{chessdiagram}
    \chessboard[
        setfen=8/pp2kppp/4p3/8/1P6/P3PP2/5P1P/2K5 b - - 0 25
    ]
\end{chessdiagram}

At that moment it was clear to me that Rubinstein's game must have 
inspired Andersson. He had prepared a Catalan Opening line 
specifically to reach a similar endgame and execute the same king maneuver.