\section{Example}
\index{Process of Elimination}
We will use \vocab{PoE}{Process of Elimination}{} to analyze the games
in this book. In this section, we just give a very simple example.

\begin{multicols}{2}
    \newchessgame[
        setfen=8/8/4K3/8/8/5p1k/5P1P/8 w - - 0 1
    ]

    \begin{chessdiagram}
        \chessboard
    \end{chessdiagram}

    White must approach to the f3 pawn, otherwise Black simply takes 
    h2 pawn and then f2 pawn. The candidate moves are: \symking d5, \symking e5, \symking f5.

    Note no matter which candidate move White plays, Black cannot 
    play \variation[invar]{1... Kxh2? 2. Ke4 Kg2 3. Ke3} Black must give up his 
    pawn because he is in zugzwang. Black's only move is ...\symking g2.

    After \variation[invar]{1... Kg2 2. h4 Kxf2 3. h5 Kg2 4. h6 f2 5. h7 f1=Q}

    If White's king is on f5, he will be checked and pawn will be lost since his king 
    is too far away.

    If White's king is on e5, his pawn will lost after \variation{6. h8=Q Qa1+}

    The only move left is \symking d5. 

    That's it. A very simple process.
\end{multicols}