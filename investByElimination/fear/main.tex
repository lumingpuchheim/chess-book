\chapter{Fear}

\section{Kodak: From Nostalgia to Bankruptcy}

For much of the twentieth century Kodak was inseparable from the idea of photography itself. Family albums around the world were filled with prints made from Kodak film. Holidays, birthdays, weddings and graduations were captured on yellow boxes of film and developed at local photo labs. The company’s slogan, ``Kodak moment'', became shorthand for a memory worth preserving. For many people, the warm colours and slight grain of a Kodak print are part of the visual texture of their childhood.

Inside the company, this success was built on a powerful business model. Kodak sold cameras, but the real profits came from film, chemicals and paper. Every roll of film meant multiple steps in a chain that the company controlled: buying the film, having it processed, ordering prints and enlargements. The more people photographed their lives, the more film they consumed. As the world grew richer and travel became easier, the number of ``Kodak moments'' multiplied.

It is tempting to imagine that digital photography was an external shock that came out of nowhere and destroyed this cosy world. But the uncomfortable truth is that Kodak itself helped create the technology that eventually undermined its own business. In the 1970s, a young engineer at Kodak, Steve Sasson, built one of the first digital cameras. The early prototypes were bulky, low resolution and experimental, but they contained the essential idea: an image could be captured electronically, stored and displayed without film.

When Sasson demonstrated the technology, the reaction inside Kodak was mixed. Technically, it was impressive. Commercially, it was seen as a distant possibility and, more importantly, as a threat. A future in which people could take and view thousands of pictures without ever buying film did not fit comfortably with a company whose profits depended on chemical-based photography. The digital camera was an invention that, if fully embraced, might eat the company’s own lunch.

Over the following decades, digital imaging improved step by step. Sensors became more sensitive, storage cheaper, displays sharper and software more capable. Competitors and new entrants began to release digital cameras and, later, camera phones. Each generation reduced the need for film a little more. Users discovered that they could see their pictures instantly, delete the bad ones and share the good ones without waiting for a lab. What started as a curiosity gradually reshaped everyday habits.

By the time digital photography reached the mainstream, Kodak was caught in a painful position. It had invested in digital technologies and owned valuable patents, but its culture, capital allocation and identity were still deeply tied to the old film-based model. The company tried to straddle both worlds: defending its traditional business while cautiously stepping into the new one. In the end, that hesitation proved costly. As film sales collapsed and digital competitors gained ground, Kodak filed for bankruptcy protection in 2012, leaving behind a powerful sense of nostalgia and a cautionary tale about innovation.

\section{How Fear Shapes Strategic Choices}

\vocab{Fear}{Fear}{a feeling of anxiety or concern about a future event or situation} \index{Fear} played a central role in Kodak’s slow response to the digital camera revolution. Management could see, at least in outline, what digital imaging might become. Their own engineers had demonstrated it. But fully committing to that future would have meant confronting a terrifying possibility: that the company’s most profitable business, the sale of film and processing, would shrink dramatically. Rather than build a new future that might threaten their present, it was psychologically safer to protect what already existed.

This is where fear and \vocab{wishful thinking}{Wishful thinking}{ beliefs based on what might be pleasing to imagine, rather than on evidence, rationality, or reality} \index{Wishful thinking} reinforce each other. It was comforting to believe that digital cameras would take a long time to become good enough or cheap enough for the mass market. It was comforting to imagine that professional photographers and serious amateurs would always prefer film, or that people would always want physical prints instead of digital images. Each of these beliefs delayed the moment when Kodak had to face the full implications of the technology it had helped invent.

In the language of the innovator’s dilemma, Kodak was trapped by its own success. The film business generated high margins and reliable cash flow, and the company’s processes were optimised to serve its most profitable customers. Digital photography, especially in its early stages, looked worse on every conventional metric: lower quality, uncertain demand, unfamiliar distribution channels and smaller margins. From the perspective of short-term financial performance, it made sense to starve the new business and feed the old one.

But innovation does not respect the comfort zone of incumbents. While Kodak hesitated, other companies were willing to embrace the new model fully, even if it cannibalised their existing products. As digital cameras improved and later merged with mobile phones, the basis of competition shifted. Users stopped buying film for every trip and started carrying a camera in their pocket at all times. The more people photographed digitally, the fewer reasons they had to buy anything from Kodak at all.

Fear of losing what they were—a dominant analogue film company—made it harder for Kodak’s leaders to become something new. By clinging to the familiar and assuming that change would be slow, they underestimated how quickly customer behaviour could shift once a better, more convenient option appeared. The lesson is not just about technology; it is about psychology. Companies can see the future in their laboratories and still fail to act on it, because acting would require them to dismantle the very business that made them successful.

\nocite{christensen:1997-innovators-dilemma}
