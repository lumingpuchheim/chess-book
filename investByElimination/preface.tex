Chess masters use \vocab{PoE}{Process of Elimination}{Ruling out bad moves, not immediately finding the perfect one. Commonly used in calculation, defense, and endgames} to find the best move.
How it works in practice:
\begin{itemize}
  \item List candidate moves that are natural or forcing.
  \item Eliminate bad moves by concrete reasons.
  \item Compare the remaining moves and choose the one that best fits the position.
\end{itemize}

\section*{The Pursuit of Unhappiness}
In chess, the \vocab{PoE}{Process of Elimination}{} helps players find good moves by systematically ruling out bad ones. Instead of searching for brilliance, they eliminate moves that lose material, weaken the position, or violate basic principles. What remains is a manageable set of serious candidates. The method works because avoiding disasters is often more reliable than finding perfection.

Paul Watzlawick, the Austrian-American psychologist and communication theorist, observed a similar pattern in human psychology. In his work on ``the pursuit of unhappiness,'' he identified how people systematically engage in behaviors that guarantee their own misery: dwelling on past mistakes, comparing themselves unfavorably to others, setting impossible standards, refusing to accept reality, or creating problems where none exist. These are not accidents; they are patterns people actively maintain.

Just as a chess player eliminates losing moves before analyzing subtle options, one can eliminate unhappiness-producing behaviors before seeking complex solutions. The connection is direct: if you stop doing the things that make you unhappy, happiness becomes the natural result. You do not need to pursue happiness directly; you simply need to stop pursuing unhappiness.

In chess, Process of Elimination works because there are far more bad moves than good ones. Similarly, in life, there are far more ways to create unhappiness than to create happiness. By systematically identifying and eliminating self-defeating patterns—complaining without action, holding grudges, seeking validation from others, or refusing to adapt—you narrow the field to behaviors that naturally lead to contentment.

Watzlawick's insight mirrors the chess principle: the path to improvement is not about adding brilliance, but about removing stupidity. A grandmaster does not win by finding the most brilliant move every turn; he wins by consistently avoiding blunders. Likewise, happiness does not come from achieving perfect circumstances, but from eliminating the mental habits that create suffering regardless of circumstances.

You can apply this in your own life. When you notice persistent unhappiness, start by listing the behaviors that actively maintain it: rumination, comparison, unrealistic expectations, or resistance to change. Eliminate those first. What remains may not be perfect, but it will be freer from self-imposed misery—much like a chess position becomes playable once the losing moves are removed. Over time, the habit of eliminating unhappiness-producing patterns creates space for contentment to emerge naturally, both in chess and in life.



\section*{Charlie Munger: Avoiding Stupidity}
In chess, strong players often use the \vocab{PoE}{Process of Elimination}{}: instead of trying to find the best move directly, they first rule out the clearly inferior ones. Moves that lose material for no reason, weaken the king, or violate basic principles are discarded. What remains is a much smaller set of candidate moves that deserve serious calculation. By systematically asking, “What can I safely throw away?”, the player frees mental energy to focus on the few moves that really matter.

Charlie Munger uses the same method in life and investing. Rather than starting with the question “What should I do?”, he often begins with “What must I avoid?”. He looks for obvious mistakes: businesses with dishonest management, fragile balance sheets, or economics he cannot understand. By eliminating these “bad moves” first, he narrows the field to a handful of sensible opportunities. This negative approach is not pessimism; it is disciplined filtering.

Just as a chess player eliminates blunders before analyzing subtle options, Munger eliminates stupidity before seeking brilliance. In both domains, the gains come not from genius, but from consistently avoiding disasters. A grandmaster may not always find the absolute best move, but he almost never plays a losing one. Likewise, Munger’s success comes less from predicting the future than from refusing to do obviously dumb things.

You can apply this mindset in your own decisions. When facing a difficult choice, start by listing everything that would clearly make the situation worse: lying, taking on unmanageable risk, ignoring important constraints, or acting in anger. Eliminate those options first. What remains will rarely be perfect, but it will be safer and more robust—much like choosing between several reasonable moves in a complex chess position. Over time, the habit of systematic elimination turns chaos into structure, both on the board and in life.




Both Watzlawick and Munger use the Process of Elimination to make better decisions.
One open question is, what is the criteria for the bad moves? 
Munger has listed some very practical criteria:  businesses
with dishonest management, fragile balance sheets, or economics he cannot understand.

\section*{Invest by Elimination}
Charlie Munger has analyzed the history of failures and developed his own list of criteria to avoid.
His list is certainly useful for him. I believe I need to develop my own list because 
I have access to the internet, which he did not use often, and I come from a different background than he did. 
In this book, I will list the criteria that help identify investments to avoid. 
Different companies exhibit different symptoms, and therefore the list will never be complete. 
As the Germans say, ``the way is the goal.'' So let's get started.


\vspace{2em}
\begin{flushright}
Ming Lu\\
\emph{Aegina, Greece}\\
\emph{January 10, 2026}
\end{flushright}
