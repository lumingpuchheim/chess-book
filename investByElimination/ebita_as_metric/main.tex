\section{EBITA as Metric}
Earnings before interest, taxes, and amortization (EBITA) is a measure of a company's profitability and value. As the name implies, EBITA adds interest, taxes, and amortization to the earnings equation. 

Buffett doesn't like EBITDA because it pretends real costs don't exist.

His core objection is that EBITDA ignores depreciation and amortization, which are not accounting tricks but the economic cost of wearing out assets. A business that needs factories, machines, trucks, or networks must reinvest cash just to stay competitive. EBITDA treats those reinvestments as optional, which inflates how “cheap” or “profitable” a business looks.

Second, EBITDA ignores capital intensity differences. Two firms can have the same EBITDA, but one may need massive ongoing capex while the other doesn’t. Buffett cares about owner earnings—cash that can actually be taken out of the business without hurting it—not a metric that hides future cash drains.

Finally, he associates EBITDA with promotional finance. It's often used in leveraged buyouts and weak businesses to justify high valuations and heavy debt. As Buffett once mocked it: “Does management think the tooth fairy pays for capital expenditures?”

In short: Buffett wants to know how much real cash a business can generate after staying alive. EBITDA doesn’t answer that.