\chapter{Rationalization}
\section{Decartes is wrong}
Descartes’ Error argues that reason does not operate independently from emotion. The traditional view—going back to Descartes—that good thinking requires suppressing emotion is wrong. Antonio Damasio shows that emotion is not the enemy of rationality; it is a precondition for it.

The core evidence comes from neurological patients with damage to brain areas involved in emotion (especially the ventromedial prefrontal cortex). These patients often retain normal IQ, logic, and language, yet they are incapable of making sound real-world decisions. They can analyze endlessly, but they cannot choose. Without emotional signals, every option appears equally plausible, so decision-making collapses.

Damasio introduces the somatic marker hypothesis. Past experiences create bodily signals—emotional markers—that are automatically triggered when we face similar situations. These markers act as a filter, quickly highlighting what is dangerous, promising, shameful, or rewarding. Reason then works on a narrowed set of options. When these markers are absent or impaired, reasoning becomes slow, ineffective, and detached from reality.

A crucial implication is that much of what we call ``reasoning'' is actually post-hoc justification. We feel an emotional pull or aversion first, and only afterward construct arguments that make the choice look rational. This explains why people can defend poor business or personal decisions with internally consistent logic, even when outcomes are clearly bad.

\nocite{damasio:1994-descartes-error}

\section{Emotions in Disguise: How Rationalization Works in Business}

In the previous chapters, we have seen how management, as people driven by emotions, caused harmful results. \vocab{Jealousy}{Jealousy}{a feeling of resentment or envy toward someone else's achievements, possessions, or advantages} \index{Jealousy} was rationalized by \vocab{wishful thinking}{Wishful thinking}{ beliefs based on what might be pleasing to imagine, rather than on evidence, rationality, or reality} \index{Wishful thinking}, which was then further rationalized and communicated as ``vision''. \vocab{Fear}{Fear}{a feeling of anxiety or concern about a future event or situation} \index{Fear} of losing what a company was led to defensive decisions that looked strategic but were really emotional reactions. Companies are run by people who have emotions, and these emotions are often hidden beneath layers of business language and strategic frameworks. As investors, we must be astute enough to identify and eliminate these companies from our consideration.

The best solution is to check if the arguments are sound. Often, abstract words like ``vision'', ``strategy'', ``experience'' and ``transformation'' are warning signals. These words are not inherently bad; good companies use them too. But there is a crucial difference: good companies replace these abstract terms with concrete content, making the words themselves become superfluous. When a company says it has a ``vision for the future'', you should be able to see specific products, markets, technologies or customer behaviors that make that vision tangible. When a company talks about its ``strategic direction'', you should be able to identify clear actions, investments and trade-offs that demonstrate what that strategy means in practice.

Mediocre companies use abstract words because they have nothing else, and they do not want to confess that they have not thought through their decisions. They rely on vague language to create the appearance of depth without actually having it. A company that says ``we are transforming our business through digital innovation'' but cannot explain which specific products are being built, which customer problems are being solved, or which revenue streams are being created is likely rationalizing uncertainty or failure rather than describing a real plan.

Even when arguments sound good, it is helpful to ask more questions: Why? What? How? Why is this the right time for this investment? What specific evidence supports this decision? How will this create value that did not exist before? If the answers remain abstract or circular, that is a red flag. If the answers become concrete and specific, that is a green light. The difference between rationalization and genuine reasoning is not in the sophistication of the language, but in whether the language connects to observable reality.

This is especially important when evaluating acquisitions, strategic pivots, or major investments. When Alibaba acquired Ele.me, the ``New Retail'' vision sounded impressive, but asking ``what specific synergies?'' and ``how will this overcome Meituan's network effects?'' would have revealed the emotional drivers beneath the strategic language. When Nokia described its risks in generic terms, asking ``what about the specific risk of failing to meet customer needs?'' would have exposed the gap between disclosed risks and real threats.

The challenge for investors is that rationalization can sound very convincing. People who are rationalizing their emotional decisions are not trying to deceive others; they are trying to convince themselves. That makes their arguments internally consistent and emotionally compelling. But consistency and emotional appeal are not the same as truth. The way to pierce through rationalization is to demand specificity, to ask for evidence, and to look for the concrete actions and outcomes that would make the abstract claims real. If those specifics are missing or weak, you are likely looking at emotion dressed up as strategy, and that is a company worth eliminating from your investment universe.