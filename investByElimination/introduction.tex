\chapter*{Introduction}
\addcontentsline{toc}{chapter}{Introduction}
\epigraph{Investing isn't about beating others at their game. It's about controlling yourself at your own game.}{Benjamin Graham}

\section*{Horse Race}

Even a fool knows how to pick a horse to bet on: those who run fast and those with less weight to carry.

Most horse race bets are based on pari-mutuel betting. All bettors put their money into a pool, the organizer takes a fee (typically 17\%), and the winner takes the rest.

To make money, it is not only about being right often enough, but also about being right when the odds are favorable. Occasionally, few people bet on a favorite horse, meaning the odds are favorable. Under such circumstances, one should bet heavily to win big.

The key insight: what other people are betting is not relevant. It is only about picking a good horse when the odds are favorable.

\section*{Investment}

Investment is remarkably similar to horse racing. We focus on picking a good company (a good horse) when the price (odds) is favorable. The crowd's enthusiasm or fear does not matter—only the quality of the company and the favorability of the price. 

\section*{Good Company}

A good company has an easy-to-understand business and generates good profits. Warren Buffett's investment philosophy centers on this principle: invest only in businesses you can understand. If you cannot explain how a company makes money in simple terms, you should not invest in it.

A good company has several characteristics. First, it operates a simple, understandable business model. The company's products or services are clear, its competitive advantages are obvious, and its path to profitability is straightforward. Complex businesses with opaque operations are more likely to hide problems or face unexpected challenges.

Second, a good company has good profit margins and strong returns on capital. It generates cash, not just revenue. A company that grows rapidly but burns cash is not a good company—it is a speculation. Buffett looks for businesses with pricing power, meaning they can raise prices without losing customers. This indicates a durable competitive advantage.

Third, a good company has a sustainable competitive advantage—what Buffett calls a ``moat.'' This could be a strong brand, network effects, economies of scale, or regulatory protection. The moat protects the company from competitors and allows it to maintain profitability over time.

Finally, a good company is run by honest, capable management. The managers should allocate capital wisely, treat shareholders fairly, and focus on long-term value creation rather than short-term metrics.

The key insight is that you do not need to understand every business. You only need to understand a few good businesses well enough to recognize when they are available at a good price. As Buffett says, ``It's far better to buy a wonderful company at a fair price than a fair company at a wonderful price.''

\section*{Good Price: Mr. Market}

Benjamin Graham introduced the concept of ``Mr. Market'' as a metaphor for the stock market. Imagine you own a small share of a business, and you have a partner named Mr. Market. Every day, Mr. Market offers to either buy your share or sell you more shares at a price he sets.

The catch is that Mr. Market is emotionally unstable. Some days he is euphoric and offers you very high prices, believing the business is worth far more than it actually is. Other days he is depressed and offers you very low prices, convinced the business is worthless. His mood swings wildly based on news, rumors, and his own psychological state—not the actual value of the business.

The key insight is that Mr. Market is there to serve you, not to guide you. You should not let his mood swings influence your judgment about the business's true value. When Mr. Market is euphoric and offers high prices, you can sell if you want. When he is depressed and offers low prices, you can buy if you want. But you should never let his emotional state dictate your decisions.

Most investors make the mistake of being influenced by Mr. Market. When prices are rising and everyone is optimistic, they feel compelled to buy, fearing they will miss out. When prices are falling and everyone is pessimistic, they feel compelled to sell, fearing further losses. This is backwards. You should want to buy when prices are low and sell when prices are high—exactly the opposite of what Mr. Market's mood suggests.

The discipline to ignore Mr. Market is what separates successful investors from the crowd. When everyone is selling and prices are depressed, that is often the best time to buy a good company at a good price. When everyone is buying and prices are inflated, that is often the best time to sell or avoid buying.

Mr. Market will always be there, offering you prices every day. You do not need to transact every day. You only need to transact when the price is favorable to you—when you can buy a good company at a price below its intrinsic value, or sell when the price exceeds its intrinsic value. The rest of the time, you can simply ignore Mr. Market's daily mood swings and focus on understanding the business itself.

\section*{Is Investment Simple?}

Yes, investment is simple in principle: you only need to pick a good company at a good price. But it is not easy in practice. The difficulty lies not in understanding the concept, but in executing it consistently.

Can you think independently and resist following the crowd? Can you avoid Deutsche Telekom in Germany during the 1990s bubble, when everyone was buying? Can you avoid dot-com companies during the internet bubble, when prices reached absurd levels? Can you avoid Enron in the early 2000s, when it was celebrated as an innovative energy company? Can you avoid Wirecard in the 2020s, when it was Germany's fintech darling?

These were all companies that looked attractive to the crowd at the time, but were terrible investments. The crowd's enthusiasm made them expensive, and their underlying businesses were either flawed or fraudulent. Avoiding these mistakes requires discipline, independent thinking, and the ability to say no when everyone else is saying yes.

This book is about how to avoid these mistakes—how to recognize the warning signs that separate good investments from bad ones, and how to resist the psychological pressure to follow the crowd.