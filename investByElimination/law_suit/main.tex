\chapter{Law Suit}

\section{Wirecard: The Rise and Fall}

Wirecard's story is one of the most dramatic corporate collapses in recent European history. Founded in 1999, the German payment processing company grew from a small startup to a DAX-listed technology giant valued at over EUR 24 billion at its peak. The company's narrative was compelling: it was building the infrastructure for the digital economy, processing payments for merchants around the world, expanding into new markets, and growing revenue at impressive rates.

But this chapter is not about Wirecard's accounting fraud or the audit failures that eventually exposed it. This chapter is about something else entirely: Wirecard's years of experience with lawsuits. For more than a decade before its collapse in 2020, Wirecard was engaged in an aggressive legal campaign against anyone who questioned its business model or financial reporting. The company sued journalists, short sellers, analysts, and critics, positioning itself as the victim of coordinated attacks rather than focusing on building a sustainable business.

\section{The Legal Offensive}

Wirecard's approach to criticism was systematic and aggressive. When journalists from the Financial Times began investigating the company's accounting practices and questioning its reported revenue from Asian markets, Wirecard did not respond with transparency or detailed explanations. Instead, it filed lawsuits. The company accused journalists of defamation, market manipulation, and spreading false information. It sought injunctions to prevent publication of critical articles and demanded retractions.

When short sellers and analysts raised questions about Wirecard's financial statements, the company's response was legal action. Wirecard sued hedge funds, research firms, and individual analysts who published skeptical reports. The company's legal strategy was clear: treat any criticism as an attack, any question as defamation, and any skepticism as market manipulation.

This legal offensive was not limited to financial critics. Wirecard also sued former employees who raised concerns, business partners who questioned transactions, and even regulators in some jurisdictions. The company built a reputation for being litigious, using the legal system as a weapon to silence critics and intimidate anyone who dared to question its operations.

\section{Playing the Victim}

What made Wirecard's legal strategy particularly revealing was how the company positioned itself. In press releases, investor communications, and public statements, Wirecard consistently portrayed itself as the victim of malicious attacks. The company claimed it was being targeted by short sellers who were spreading false information to profit from a declining stock price. It accused journalists of being part of a coordinated campaign to damage its reputation. It suggested that critics were working together to undermine a successful German technology company.

This victim narrative was powerful. It allowed Wirecard to dismiss legitimate questions as attacks. It enabled the company to rally support from politicians, investors, and the public who saw Wirecard as a German success story under siege. It created a narrative where anyone who questioned Wirecard was automatically suspect, and anyone who defended Wirecard was supporting a national champion.

But this focus on lawsuits and victimhood came at a cost. While Wirecard was spending millions on legal fees, filing lawsuits, and positioning itself as a victim, it was not focusing on what actually matters for a business: winning customers. A payment processing company's success depends on providing reliable, efficient, and trustworthy services to merchants. It depends on building relationships, solving customer problems, and creating value that keeps customers coming back. Wirecard's energy and resources were directed toward legal battles and public relations campaigns, not toward building a better product or serving customers more effectively.

\section{The Lesson: Focus on Customers, Not Lawsuits}

Wirecard's collapse in 2020, when it admitted that EUR 1.9 billion in cash probably did not exist, revealed the truth: the company had been fraudulent for years. But the warning signs were there long before, and they were visible in the company's approach to lawsuits.

A good company must focus on winning customers, not showing itself as a victim in lawsuits. When a company responds to criticism with legal threats rather than transparency, when it positions itself as a victim rather than addressing legitimate concerns, when it spends more energy on litigation than on improving its products and services, that is a red flag. Companies that are confident in their business model, their financial reporting, and their competitive position do not need to sue their critics. They can respond with facts, data, and results. They can let their performance speak for itself.

Wirecard's years of experience with lawsuits were not a sign of strength; they were a sign of weakness. The company used litigation as a shield to hide its problems, as a weapon to silence questions, and as a distraction from the fundamental issues with its business. When the truth finally emerged, all those lawsuits could not save the company. The legal victories were meaningless because the business itself was fraudulent.

For investors, the lesson is clear: be wary of companies that are quick to sue, that position themselves as victims, and that spend more time on legal battles than on building their business. A company that is focused on winning customers will respond to criticism with better products, clearer communication, and improved performance. A company that is focused on lawsuits is often trying to hide something. Wirecard's story shows that when a company's primary response to questions is legal action, those questions are usually worth asking—and the answers are usually worth avoiding.
