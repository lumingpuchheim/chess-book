\chapter{Company Without Expertise}

\section{Theranos: The Rise and Fall}

Theranos was founded in 2003 by Elizabeth Holmes, a 19-year-old Stanford dropout who claimed to have developed revolutionary blood-testing technology. The company's promise was audacious: it would revolutionize healthcare by enabling patients to run dozens of blood tests using just a few drops of blood from a finger prick, instead of the traditional method of drawing vials of blood from veins. The technology would be faster, cheaper, and less painful than existing methods, and it would make comprehensive health testing accessible to everyone.

For more than a decade, Theranos raised hundreds of millions of dollars from investors, reached a valuation of over USD 9 billion, and signed partnerships with major retailers like Walgreens and Safeway. Holmes became a media sensation, appearing on magazine covers, speaking at conferences, and being compared to Steve Jobs. The company's board included prominent names: former Secretaries of State Henry Kissinger and George Shultz, former Defense Secretary James Mattis, former Wells Fargo CEO Richard Kovacevich, and other distinguished figures from politics, business, and the military.

But in 2015, investigative reporting by the Wall Street Journal began to expose the truth: Theranos's technology did not work. The company had been using traditional blood-testing machines from other manufacturers and diluting samples to make them appear to work with smaller volumes. The revolutionary technology that Holmes had promised for more than a decade was a fraud. By 2018, Theranos had collapsed, Holmes and her business partner Sunny Balwani faced criminal charges, and investors had lost hundreds of millions of dollars.

\section{The Board Without Expertise}

What made Theranos's collapse particularly revealing was the composition of its board. The company's board of directors included former cabinet secretaries, military leaders, and business executives—impressive names that lent credibility and opened doors. But there was a critical problem: almost none of the board members had any expertise in biology, medicine, or laboratory science.

Henry Kissinger was a diplomat and political scientist. George Shultz was an economist and former Secretary of State. James Mattis was a Marine Corps general. Richard Kovacevich was a banker. These were accomplished people in their respective fields, but they had no background in the science that Theranos claimed to have revolutionized. They could not evaluate whether the company's technology was scientifically sound, whether its claims were feasible, or whether its research was valid. They could assess business strategy, market opportunities, and financial projections, but they could not assess the core product that the entire business was built on.

This lack of expertise had profound consequences. When Holmes presented her vision and claimed breakthrough technology, the board members had to take her word for it. They could not independently verify the science, question the technical feasibility, or challenge the engineering claims. They could evaluate the business opportunity—the market size, the potential partnerships, the revenue projections—but they could not evaluate whether the fundamental technology actually worked.

\section{The Inability to Decide the Right Direction}

A board without domain expertise cannot effectively guide a company in a technical field. When critical decisions need to be made about research priorities, product development, regulatory strategy, or technical partnerships, board members who lack expertise in the relevant field cannot provide meaningful oversight. They cannot distinguish between genuine progress and scientific theater, between real innovation and marketing claims, between feasible technology and impossible promises.

In Theranos's case, this meant that the board could not decide the right direction for the company. When questions arose about whether the technology was working, the board had to rely on management's assurances rather than independent technical evaluation. When decisions needed to be made about research investments, product development timelines, or regulatory submissions, the board could not assess whether the proposed direction was technically sound. When problems emerged—as they inevitably did—the board lacked the expertise to understand their severity or to guide the company toward solutions.

This created a dangerous dynamic. Holmes could present impressive-sounding technical claims, show demos that looked convincing, and cite scientific-sounding explanations, and the board had no way to verify whether any of it was real. The board's lack of expertise meant that it could not serve as an effective check on management's claims or decisions. It could approve strategies, authorize investments, and provide credibility, but it could not ensure that the company was heading in a technically sound direction.

\section{Why Such Companies Should Not Be Invested In}

The Theranos story illustrates a fundamental principle: companies operating in technical fields must have board members with relevant domain expertise. When a biotechnology company has no biologists on its board, when a pharmaceutical company has no medical experts, when a software company has no engineers, the board cannot effectively oversee the company's core business. It cannot evaluate the product, assess the technology, or guide strategic decisions in the company's area of expertise.

For investors, this is a critical red flag. A company without domain expertise on its board is flying blind in its most important decisions. The board may be able to evaluate financial performance, market opportunities, and business strategy, but it cannot evaluate whether the company's core product or service is sound, whether its technology is feasible, or whether its claims are valid. This creates a fundamental governance risk: the people responsible for overseeing the company cannot actually oversee what matters most.

In Theranos's case, this governance failure was catastrophic. The board's lack of biology and medical expertise meant that it could not detect the fraud until it was exposed by external journalists. By then, hundreds of millions of dollars had been invested, partnerships had been signed, and the company had reached a valuation of billions. The board's impressive names and political connections had helped the company raise money and sign deals, but they had not helped it build a real product or detect that the product did not exist.

The lesson for investors is clear: be wary of companies where the board lacks expertise in the company's core business. A board full of prominent names from other fields may look impressive, but it cannot effectively guide a company in a technical field. When evaluating a biotechnology company, look for board members with medical or scientific backgrounds. When evaluating a technology company, look for board members with engineering or technical expertise. When evaluating any company, ask whether the board has the knowledge to actually evaluate the company's core product or service.

Companies without domain expertise on their boards should not be invested in, because they cannot be effectively governed. The board cannot decide the right direction, cannot evaluate the core business, and cannot serve as an effective check on management. In Theranos's case, this governance failure allowed a fraud to continue for more than a decade, costing investors hundreds of millions of dollars. The impressive names on the board provided credibility, but they could not provide the expertise needed to detect that the entire business was built on a lie.
