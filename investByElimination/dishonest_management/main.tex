\chapter{Dishonest Management}

\section{Alibaba's Delivery War}
In early 2025, China’s on‑demand delivery market shifted from a two‑player fight into an all‑out instant‑commerce battle. Meituan had long led food delivery, commanding roughly 70 % of the market with hundreds of millions of daily orders. Alibaba’s Ele.me had been a distant second, and JD.com’s presence was small. But as consumers embraced “one‑hour” delivery for food and everyday goods, JD.com formally entered the fray in April with a dedicated local delivery push that rapidly gained traction in dozens of cities, pushing millions of daily orders and forcing the incumbents to respond aggressively. 

Alibaba, determined not to cede ground again, rebranded and tightly integrated its delivery business with Taobao’s broader commerce ecosystem. Ele.me was merged into a new Taobao Instant Commerce/Taobao Shangou offering that promised sub‑hour delivery for food, groceries, electronics and general retail, using Ele.me’s courier network to fulfill a wide range of orders placed directly inside Taobao. This aimed to turn delivery into a core driver of traffic and repeat engagement on its main marketplace. 

For much of 2025, Alibaba, Meituan and JD.com poured huge subsidies into food and instant deliveries — from free and near‑free orders to steep merchant incentives and massive rider bonuses — in an effort to win share from consumers and build loyalty. One weekend, Alibaba alone reported more than 80 million combined on‑demand orders in a single day, part of nationwide discount campaigns that overwhelmed merchants and upended local business models. 

The competition was fierce and expensive. Meituan, even though it remained the largest platform by volume, saw its profits shrink sharply — one report showed net profit down nearly 89 % in a quarter as costs for discounts and marketing climbed steeply. JD.com also continued to lose money even as its delivery volumes climbed. Alibaba’s overall earnings were hit as well, with profit declines reflecting its heavy investment in subsidies and capacity expansion. In total, analysts estimated that the three companies spent tens of billions of yuan on price cuts and promotions to compete. 

The escalation drew official attention. Chinese regulators repeatedly summoned the platforms and urged them to curb “disorderly competition”, warning that extreme subsidy wars were eroding the rights of merchants and riders and disrupting healthy market development. Regulators and the companies later pledged to pursue “rational competition” and limit excessive price‑based battles. 

By late 2025 the war had no clear winner but left deep marks: Meituan held its market leadership in food delivery, Alibaba grew its instant commerce footprint and daily order volumes, and JD.com carved out a visible third force. All three suffered margin compression and reported steeper losses or weaker earnings. Facing mounting financial pain and regulatory pressure, the platforms signaled a shift away from runaway subsidies toward more sustainable competition, even as they continued to invest in fast delivery and related services
\section{Alibaba's Quarterly Report}