\chapter{Dishonest Management}

\section{Alibaba's Delivery War}
In early 2025, China’s on‑demand delivery market shifted from a two‑player fight into an all‑out instant‑commerce battle. Meituan had long led food delivery, commanding roughly 70\%  of the market with hundreds of millions of daily orders. Alibaba’s Ele.me had been a distant second, and JD.com’s presence was small. But as consumers embraced “one‑hour” delivery for food and everyday goods, JD.com formally entered the fray in April with a dedicated local delivery push that rapidly gained traction in dozens of cities, pushing millions of daily orders and forcing the incumbents to respond aggressively. 

Alibaba, determined not to cede ground again, rebranded and tightly integrated its delivery business with Taobao’s broader commerce ecosystem. Ele.me was merged into a new Taobao Instant Commerce/Taobao Shangou offering that promised sub‑hour delivery for food, groceries, electronics and general retail, using Ele.me’s courier network to fulfill a wide range of orders placed directly inside Taobao. This aimed to turn delivery into a core driver of traffic and repeat engagement on its main marketplace. 

For much of 2025, Alibaba, Meituan and JD.com poured huge subsidies into food and instant deliveries — from free and near‑free orders to steep merchant incentives and massive rider bonuses — in an effort to win share from consumers and build loyalty. One weekend, Alibaba alone reported more than 80 million combined on‑demand orders in a single day, part of nationwide discount campaigns that overwhelmed merchants and upended local business models. 

The competition was fierce and expensive. Meituan, even though it remained the largest platform by volume, saw its profits shrink sharply — one report showed net profit down nearly 89\% in a quarter as costs for discounts and marketing climbed steeply. JD.com also continued to lose money even as its delivery volumes climbed. Alibaba’s overall earnings were hit as well, with profit declines reflecting its heavy investment in subsidies and capacity expansion. In total, analysts estimated that the three companies spent tens of billions of yuan on price cuts and promotions to compete. 

The escalation drew official attention. Chinese regulators repeatedly summoned the platforms and urged them to curb “disorderly competition”, warning that extreme subsidy wars were eroding the rights of merchants and riders and disrupting healthy market development. Regulators and the companies later pledged to pursue “rational competition” and limit excessive price‑based battles. 

By late 2025 the war had no clear winner but left deep marks: Meituan held its market leadership in food delivery, Alibaba grew its instant commerce footprint and daily order volumes, and JD.com carved out a visible third force. All three suffered margin compression and reported steeper losses or weaker earnings. Facing mounting financial pain and regulatory pressure, the platforms signaled a shift away from runaway subsidies toward more sustainable competition, even as they continued to invest in fast delivery and related services
\section{Alibaba's Quarterly Report}

In Alibaba's September 2025 quarter earnings call (see appendix \ref{appendix:alibaba_2025_q3_earnings_call}), the company presented a narrative focused entirely on growth metrics and market share gains, with no discussion of the underlying economics.

The CEO highlighted that ``Alibaba Group's performance maintained steady growth,'' with overall revenue increasing 15\% year-over-year (excluding RT-Mart and Intime). China e-commerce CMR increased 10\\% year-over-year, and Cloud Intelligence Group revenue increased 34\\% year-over-year. Alibaba Cloud achieved 34\\% revenue growth while external commercial revenue accelerated to 29\\%. AI-related product revenue achieved triple-digit year-over-year growth for the ninth consecutive quarter.

The report emphasized market share improvements across multiple segments: in the hybrid cloud market, Alibaba Cloud's growth exceeded 20\%, exceeding market average growth, with market share continuing to improve. Growth in the financial cloud market also led the market, with share continuing to improve. In China's AI cloud market, overall share maintained leadership, exceeding the sum of second through fourth place.

The Qwen App public beta version achieved over 10 million new downloads in just one week. The company emphasized its position as ``China's only company with leading models plus rich life and consumption scenarios,'' positioning itself as the future AI life entry point.

\section{The Numbers}
Some numbers in the last section are suspiciously round. ``over 10 million new downloads'', ``15\% year-over-year'', ``China e-commerce CMR (What is that?) increased 10\% year-over-year''.
I am at least not sure if they are exactly true. Real numbers can be like 13.7\%, but not 15\%.


\chapter{Resourcefulness}
\section{Endurance}

In 1914, Ernest Shackleton sailed south with a dream of crossing Antarctica. The ship was called Endurance, a name chosen for confidence, not prophecy.

Months later, the sea closed around them. Ice pressed in slowly, relentlessly, until the wooden hull began to scream. The men listened at night as the ship cracked and groaned, knowing there was nothing they could do. When Endurance finally broke apart and sank, it took with it the future they had imagined. All that remained was white, silence, and each other.

They lived on the ice, drifting without control. Days blended into weeks. Food shrank. The cold crept into bones and thoughts. Shackleton kept them moving, kept them busy, kept them believing the next day mattered. Birthdays were celebrated with crumbs. Hope was rationed carefully, never allowed to run out.

When the ice finally loosened, they climbed into three small lifeboats and pushed into a violent sea. Waves rose like walls. After days of exhaustion and terror, they reached Elephant Island—a place no ship ever visited. Safe, but trapped. Left behind.

Shackleton chose five men and a single boat. They vanished into the Southern Ocean, crossing more than a thousand kilometers of water known for swallowing ships whole. For sixteen days, they bailed water, fought freezing winds, and trusted a fragile craft not to fail. Every night could have been the last.

They reached South Georgia, but not salvation. Mountains stood between them and help. Shackleton and two others crossed the island on foot, through snow and ice no human had mapped, walking without rest because stopping meant freezing to death. After thirty-six hours, they stumbled into a whaling station. Bearded, broken, alive.

Rescue did not come quickly. Ice turned ships away again and again. Each failure meant more waiting, more uncertainty for the men left behind. Shackleton returned every time. He did not stop.

When he finally reached Elephant Island, months later, the men were still there. Watching. Waiting.

Not one had died.

The journey ended without triumph, without discovery, without glory. But every man went home. And long after the ice melted and the maps were forgotten, that remains.
\section{Resourcefulness in Chess}

Shackleton's story is not about victory. It is about what happens when everything goes wrong: the ship sinks, the plan fails, the future disappears. In those moments, resourcefulness becomes the difference between survival and surrender.

Chess offers the same test. Every player faces losing positions—material down, king exposed, pieces trapped, time running out. The natural response is resignation: accept defeat, stop fighting, give up. But resourcefulness means finding another way.

When material is lost, resourcefulness looks for activity. When the king is exposed, it finds counterplay. When pieces are trapped, it creates complications. When time is short, it simplifies to what matters most. The goal shifts from winning to surviving, from brilliance to persistence.

Shackleton did not cross Antarctica. He did not achieve his original dream. But he brought every man home. In chess, resourcefulness means the same: you may not win the game, but you can make your opponent work for it. You can create problems, find counterchances, turn a lost position into a difficult one, a difficult one into a draw, a draw into a win.

In section \ref{sec:lasker-steinitz-1894} we saw how Lasker prevented an immediate loss after he was down in material. Then he resourcefully created problems for Steinitz. The position was unclear and finally Steinitz lost a so-called ``winning game''.

We will present some more examples.

\begin{multicols}{2}
\chessgameinfo{USSR Championship}{B.Spassky}{L.Polugaevsky}{10}{1961.01.26}{0-1}

We have seen this game already from the perspective of Spassky. Now we will see it 
from the perspective of Polugaevsky, showing how to play to loss position.

\newchessgame[
    setfen=r3n1k1/p1pqb1pp/4p2P/1pp1Br2/3P4/3QPN2/PP2KPR1/7R b - - 2 22,
    moveid=22b
]

\begin{chessdiagram}
    \chessboard
\end{chessdiagram}


White has a deadly attack and more time on the clock. Resigning is out of the question.
If White plays the attack correctly, there is nothing to be done. 

Polugaevsky was in time scramble now and Spassky started to play quickly.
He could only hope Spassky made some mistake. But at least he must have some 
position to play with. What about create a passed pawn. If he survived the attack,
he would have advantage in the endgame. Of course it is a big "if", but 
at least there is something to hope for.

\mainline[level=1]{22...c4 23. Qe4}

\variation[invar]{23. Qxd5 exd5 24. hxg7} is more prosaic:
Exchanging pieces in a winning position. Admittedly, Black was also lucky because White was playing for mate.
If he chose to play prosaically, Black would have nothing to hope for.

\mainline[level=1]{23...Qd5 24. Qg4 c3 25. b3 b4}

\begin{chessdiagram}
    \chessboard
\end{chessdiagram}

Now Black has a protected passed pawn. White must prove that he 
can mate. 

\mainline[level=1]{26. e4 Qb5+ 27. Ke3 Rf7 28. hxg7 Nf6}

Black played the forced moves to avoid material loss.

\mainline[level=1]{29. Bxf6 Rxf6 30. Rxh7}

\begin{chessdiagram}
    \chessboard
\end{chessdiagram}

Black must reckon White intends to play 
\variation[invar]{
    31. Rh2 Kg8 32. Rh8+ Kf7 33. g8+
}.

Black can only give some checks now.

\mainline[level=1]{
    30... Rxf3+  31. Kxf3 Qd3+ 32. Kf4
}

\begin{chessdiagram}
    \chessboard
\end{chessdiagram}

\variation[invar]{
    32... Qd2+ 33. Ke5 Bd6+ 34. Kxe6 Re8+ 35. Kd7! Re7+ 36. Kd8!
} Black has no more checks. 

From the above variation, White has no chance to make a mistake.

In time scramble Polugaevsky again found the only move with 
some practical chance.

\mainline[level=1]{
    32... Bd6+ 33. Kg5 Kxh7
}

\begin{chessdiagram}
    \chessboard
\end{chessdiagram}

Now the decisive moment has arrived and Spassky made a mistake with \symking h5. 
He lost the endgame afterwards.

\end{multicols}



Throughout the earnings call, the focus remained on percentage increases, market share gains, and growth metrics. The company highlighted EBITA improvements across core segments, with Cloud Intelligence Group showing strong EBITA expansion driven by revenue growth. The narrative emphasized EBITA growth rates that outpaced revenue growth, suggesting improving operational efficiency and margin expansion.

The delivery war investments were framed as strategic market share gains rather than cost drivers. Instead, the narrative focused solely on the scoreboard: revenue growth, market share, EBITA improvements, and user acquisition numbers.

This pattern mirrors the Enron approach: emphasize impressive growth percentages and market leadership claims while avoiding discussion of whether that growth creates value for shareholders. When a company focuses exclusively on the scoreboard (revenue growth, market share) without addressing the playing field (profitability, cash generation, sustainable economics), investors should be skeptical.