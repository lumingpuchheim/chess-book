\chapter{My Experiences}

This story is according to my recollection. I have changed the names in the story.

I once worked for AlfaTec, a technology company producing RFID readers in Germany.

\section{The Takeover}

In 2017, Bergen Partners took over the company. Initially, they told the staff that they wanted to grow with the company and that nothing would change.

In those days, the company was doing well. The business was growing around 20\% every year. The staff were happy. At the end of 2018, they told the company to pay double bonuses for employees as a reward for their hard work.

An investment company like Bergen is only interested in the score. They required 20\% growth every year. The company had expanded significantly. I was very skeptical at that time. Growing by expansion is not sustainable. New staff must be trained. For old staff like me, why should I care about the growth when I only received my salary and bonus, which was only relevant to my years of service? On the other hand, I had purchased some Amazon stock. The price was also growing, perhaps no less than 20\%. What is the point of investing in a company, paying so much for new staff, when they could have sat on their hands by buying Amazon stock?

\section{The First Acquisition}

2019 was not so good, meaning not much growth. The company was still profitable while some other companies were fighting for survival. The CEO warned us that there might be no bonus at the end of the year.

If revenue was missing, then something had to be done. In those days, a German energy company wanted to buy products from AlfaTec via another company called Soprio. The company Soprio went bankrupt. In order to save the revenue, AlfaTec decided to buy Soprio.

As a value investor, I didn't like the deal at all. As the Poles say, ``not my circus, not my monkeys.'' Well, not exactly. I had to work with them for quite a few years. I was an engineer developing software in those days. The Soprio engineers—let me put it this way—there was a reason why the company went bankrupt. A hardware developer didn't know why their device drew more current when it was plugged in and asked me, a software developer, to investigate.

\section{New Management and Growth Mindset}

Then came 2020, the Covid-19 pandemic. Our old CEO stepped down because of health reasons. The staff had always liked him. The newcomer Jens was from Microsoft. The first thing he did was train us to have a ``growth mindset,'' because selling hardware wasn't enough for him. He also wanted to sell software. Somehow he got the idea of doing something called PACaaS (Physical Access Control as a Service). You see, the name is not pronounceable, and it never became a success. The project closed after he left the company.

``Growth mindset'' is a term used to describe the belief that one can improve one's abilities. It sounds great, like communism, and had never worked. There was only one training session by some external consulting company, and I turned off my camera because I didn't want to be recorded (No one could force me to do so. Data protection was very strict in Germany. I hated the law, but I liked to use it for my own benefit). The real reason was that it was a waste of time.

In 2020, the old head of development also stepped down because he couldn't take the job anymore. 
The newcomer was someone ``with experiences.'' I always had a hard time with him. 
There was one time Jens called all the company together so that the growth goal could be met. 


I also didn't like that Dirk had brought his family members into the company—his daughter and her boyfriend.
There was once a post on a recruiting platform criticizing the management, using him as an example.
HR had to respond by saying the hiring was based on qualifications, not connections.

I didn't like it, not because of moral reasons, but because why would you bring family members into a bad company? 
\section{More Failed Acquisitions}

Jens also bought another dying company, Tales in Bavaria, for their customers again. As Munger said, ``Only in fairy tales are emperors told they are naked.'' No one would say it was a bad deal. The Tales office was closed in 2023 and the staff dismissed.

He also bought a startup company in Berlin for PACaaS. The sole developer Fernando must have suffered a lot. I had my full sympathy for him. He had to work with Soprio engineers. I had pain because their code was a huge spaghetti mess, with copy and paste everywhere, and the developers didn't know what a function was. I told Dirk that I could not work like this, and he said I was too arrogant. Fernando was much more fluent in programming than I was. Good for him—he left the company two years after AlfaTec bought his company.

\section{The China Connection}

AlfaTec wanted to build a touch screen reader for a customer. The ex-manager of Soprio somehow managed to convince the company to use his connection in China. A one-man show who delivered different products to AlfaTec every time. I remember he was sitting in a Chinese hotel during the pandemic. Maybe the AlfaTec management was touched by this.

This Vincent guy was a musician, maybe a con man, but definitely not a good manager. There was always a struggle within AlfaTec development: one engineer, Anselm, worked mainly for him and his connection, and the rest worked separately. The reader had different problems: sometimes it triggered the alarm because its tamper switch was activated (Nothing serious, just like calling the police and telling them nothing happened), or the reading performance was bad, or sometimes the reader triggered a reading event even if there was no card.

The management wanted to sell the reader, but there was nothing to sell. Of course, Vincent was also stressed. Once in a meeting, he told the engineers that they were too bad and that he could manage by himself in China. Well, he finally managed this: at a trade fair, AlfaTec found out he was selling the same reader to the same customer, competing with his employer. The AlfaTec management was, of course, furious. The whole Soprio department was fired.

\section{Jens left the Company}
After several years, Bergen managed to sell the company to another investment company, SL Capital,
at a very high price: 10-fold the annual revenue. Assuming the company had a margin of 10\%, 
this meant the price was 100-fold the annual profit, while Amazon trades at only 30 times earnings.
Was the reader business still growing? 

The deal was supposed to close in 2024, but SL Capital couldn't or didn't want to complete the purchase for some reason. 
I speculate that interest rates at that time were too high, so SL Capital couldn't 
borrow the money. 

Jens was not happy. He said he could not sleep. He asked the staff to work harder to generate more revenue.

Cui bono? Who benefited? Dirk was not happy that I didn't pretend to be hardworking. 
I told him boldly that I had no skin in the game.

In the end, Jens left the company after breaking what had once been a healthy business.
A year later, Dirk stepped down as head of development due to health reasons.


\section{Summary}

Let me summarize: AlfaTec had bought two dying companies, Soprio and Tales, and closed them down. AlfaTec bought a startup for one developer, and he left the company two years later. None of the investments were successful. Respect!

By the way, Vincent is still working with Anselm and competing with AlfaTec with the same reader.

