\chapter{Cutting Throat Competition}

\nocite{cnbc_2025_pricewar}
\nocite{howtheybegan_meituan}
\nocite{deliverywar_2013}
\nocite{cnbc_2025_pricewar}

Chinese e-commerce markets are cut-throat. When a business opportunity appears promising, abundant capital floods in from multiple players, each willing to burn billions to capture market share. Companies enter markets, subsidize customers, and compete on price until losses become unsustainable. Then they declare a truce—only for the cycle to repeat when the next ``hot'' opportunity emerges.

This pattern reveals a fundamental investment principle: when everyone thinks a business is good, it cannot be a good investment. Abundant capital means intense competition, which destroys profitability. The companies that survive are often those with the deepest pockets, not the best business models. For investors, these markets become a race to the bottom where capital is destroyed rather than created.

%\epigraph{Games are won by players who focus on the playing field, not by those whose eyes are glued to the scoreboard}{Buffett}

\section{Community Group Buying War}
The basic setup is simple. People living in the same neighborhood are organized into a WeChat group. A local “group leader” posts daily offers for vegetables, fruit, meat, and other necessities. Residents place orders today, the platform ships the goods in bulk to the community the next day, and buyers pick them up from the leader's location. The promise was to use collective ordering to cut distribution costs and offer fresh food at very low prices.

It was widely believed to be a huge business because it seemed to combine several powerful forces. Groceries are a universal, high-frequency need, so the market size looked enormous. Delivering to one pickup point per community appeared much cheaper than delivering to each home. WeChat groups made user acquisition almost free, and many people assumed that once scale was reached, efficiency and data would eventually create healthy margins. This logic attracted all the major Chinese internet companies, who rushed in aggressively.

The problem was that early success was largely an illusion created by subsidies. Platforms burned enormous amounts of money to offer prices below cost. Users were not loyal to the model; they were loyal to cheap prices. When subsidies were reduced, demand dropped immediately. This exposed the underlying economics.

Fresh grocery retail has extremely thin margins by nature. Food spoils, demand fluctuates daily, and handling requires sorting, cold storage, and careful logistics. Even though deliveries are centralized, the system is still complex and costly. Managing thousands of small pickup points, dealing with returns, complaints, and quality issues turned out to be far more expensive than expected.

User behavior also worked against the model. Buyers compared prices constantly and switched platforms easily. Group leaders, who were supposed to be stable local anchors, often churned, negotiated aggressively, or moved to whichever platform paid higher commissions. At the same time, regulators stepped in to curb predatory pricing and subsidy wars, removing the main growth engine of the industry.

In the end, community group buying was re-evaluated. It was not an ``internet platform'' business with scalable margins, but a highly operational grocery and logistics business with brutal competition and very limited profit potential. It still exists, but only as a narrow, efficiency-driven operation for companies that can tolerate low margins and high execution pressure, not as the goldmine it was once imagined to be.

\section{The Delivery Wars: Alibaba vs. Meituan}

\subsection*{The 2014-2017 War: Focus on Market Share}

The first major delivery war began in 2014 when Meituan entered food delivery to compete with Ele.me, which was backed by Alibaba. From 2014 to 2017, both companies engaged in what became known as a ``pixel-level'' war across hundreds of cities in China.

The battle was fought entirely around market share metrics. Ground promotion teams from both sides engaged in physical altercations to secure exclusive merchant agreements. Subsidy wars reached extreme levels: if one side offered a 5 yuan discount, the other would offer 8 yuan. A 15 yuan lunch could cost users only 1 yuan after applying various coupons. Tens of billions of yuan were poured into this competition.

Both companies focused obsessively on the scoreboard—market share numbers, daily order volumes, number of cities covered, number of merchants signed. Meituan's ground promotion team reported daily KPIs: merchants signed, users acquired, orders processed. The entire war was measured in percentages and rankings, not profitability or sustainable business models.

The result: By 2018, Meituan had won the market share battle, but both companies had burned through massive amounts of capital in a largely unprofitable business. The victory was measured in market percentage points, not in building a sustainable, profitable operation. This war demonstrated what happens when companies focus on the scoreboard (market share) rather than the playing field (building a viable business).


\subsection*{The 2025 War: The Same Mistake Repeated}

The second delivery war began in February 2025 when JD.com entered the takeout dining market, challenging Meituan and Alibaba's Ele.me. In April, Meituan responded by launching a 24/7 ``flash shopping'' platform promising 30-minute deliveries. The competition escalated rapidly: JD.com announced 10 billion yuan in subsidies, Alibaba's Taobao Instant Commerce pledged 50 billion yuan over the next year, and Meituan offered coffee for as low as 2 yuan (30 cents).

Once again, the entire war was measured by market share metrics. Companies boasted about daily order volumes: Meituan reported a record 120 million orders in a single day, Alibaba claimed 200 million orders per day. JD.com focused on gaining market share, with analysts estimating it captured about 10\% of the instant delivery market with 20 million orders daily.

The result: All three companies burned through billions while focusing on the scoreboard. Meituan's shares fell 22\%, JD.com's fell 10\%. JD.com's push into food delivery may have generated a loss of more than 10 billion yuan in the second quarter alone. Meituan warned that increased competition would impact earnings. China's market regulator summoned all three companies in May, but the intervention had little effect—the price war continued.

This second war demonstrated the same fundamental problem: companies measuring success by market share percentages and daily order volumes rather than profitability or sustainable business models. The focus remained on the scoreboard, not the playing field.

\section{The Pattern}

A predictable cycle repeats itself across Chinese e-commerce markets. Someone identifies what appears to be a promising opportunity—a large market, high-frequency demand, or an innovative business model. The opportunity looks attractive, and word spreads quickly.

Then everyone rushes in. Fearing they will miss out on the next big thing, companies flood the market with abundant capital. No one wants to be left behind. The logic seems sound: if this is a good business, we must participate. If we don't, competitors will capture the market and we'll be excluded forever.

The market becomes cut-throat. With multiple well-funded players competing for the same customers, competition intensifies. Companies offer subsidies, discounts, and below-cost pricing to capture market share. They measure success by user growth, order volumes, and market percentage—metrics that can be bought with capital rather than earned through superior operations.

At the end, no one has real profit. The abundant capital that made the market attractive also makes it unprofitable. Companies burn through billions to gain market share, but the competition prevents anyone from achieving sustainable margins. The market becomes a race to the bottom where capital is destroyed, not created. The companies that ``win'' often do so by having the deepest pockets, not the best business model.

This pattern—opportunity discovery, capital flood, cut-throat competition, profit destruction—repeats because the fundamental mistake remains the same: when everyone thinks a business is good, it cannot be a good investment. The very factors that make a market attractive (large size, high frequency, obvious demand) also make it a target for abundant capital, which guarantees intense competition and destroyed profitability.

\section{War Narrative}
War narrative comes hand in hand with cutting throat competition.
Many Chinese managers don't just use the language of war as a metaphor — they think in it. Once business is framed as war, the most important question silently changes. It is no longer “is this a good business?” but “are we winning or losing?” And that shift is fatal.

Calling competition a war does three things. First, it moralizes endurance. Losses stop being a signal and become a badge of honor. Burning cash is reinterpreted as sacrifice. Anyone who questions the economics is accused of lacking resolve or vision. In war, retreat is shameful; in business, retreat can be wisdom. The metaphor makes that distinction disappear.
Second, the war framing removes agency. Wars are forced upon you. If you lose money, it is because the enemy attacked, not because the business model was flawed. Managers convince themselves they had no choice but to fight. This is comforting, because it turns bad decisions into acts of necessity. In reality, entering most of these “wars” was optional. The refusal to see that is not ignorance, but psychological self-protection.
Third, war thinking crowds out alternative strategies. In a real market, you can choose segments, design asymmetries, walk away, or wait. In a war narrative, there is only offense and defense. Every competitor must be confronted head-on. That is why you see companies copy each other into identical battlegrounds and then destroy margins together. The tragedy is not competition itself, but the lack of imagination caused by the metaphor.

There is also a deeper cultural layer. Modern Chinese business grew up alongside national narratives of survival, struggle, and catching up. Aggression, speed, and endurance are celebrated virtues. Questioning whether a fight should exist at all can sound unambitious or even disloyal to the organization. War language legitimizes extreme behavior and suppresses doubt.

The irony, is that these “wars” are often self-declared. No law forces a ride-hailing company to subsidize every ride. No natural monopoly demands nationwide price wars. These are strategic choices made under social pressure and reinforced by organizational incentives. Once made, they are narrated as fate.

The clearest businesses are run by people who resist this framing. They treat competition as selection, not war. If a market requires permanent subsidies, no pricing power, and constant vigilance against copycats, the correct response may be not to fight harder, but to exit earlier.

Seeing that you have a choice is the first strategic advantage. Many managers never realize that the war they are fighting is optional — and that is why they lose even when they “win.”