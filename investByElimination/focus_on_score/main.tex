\chapter{Focus on Score}
\nocite{ciw_alibaba_identity_crisis}
\epigraph{Games are won by players who focus on the playing field, not by those whose eyes are glued to the scoreboard}{Buffett}

\section{Alibaba's KPI Obsession}

In June 2025, a senior R\&D executive at Alibaba published a 10,000-character resignation essay on the company's internal network that exposed a fundamental shift in the company's culture. The memo revealed how Alibaba had transformed from a mission-driven company into one obsessed with metrics and KPIs.

``Alibaba used to be about ideals,'' the executive wrote. ``Now we talk about KPIs, stocks, mortgages. Customers are just data points to be harvested.''

This captures the essence of focusing on the scoreboard rather than the playing field. When a company treats customers as ``data points to be harvested'' rather than people to serve, it has lost sight of what creates real value. The scoreboard—daily active users, conversion rates, revenue per customer—becomes the goal, rather than building products and services that genuinely improve customers' lives.

\section{Failed Bets and Innovation Droughts}

The executive's analysis revealed a pattern of failed acquisitions over the past decade: Kuaishou competitor UC, music platform Xiami, video giant Youku, bike-sharing, and Southeast Asia's Lazada. ``With the exception of Amap (Gaode) and UC, most were failures,'' he noted.

The reason was clear: ``Obsessed with operational data, allergic to long-term product polishing.''

This is the scoreboard trap. Companies become so focused on hitting operational metrics—user growth, engagement numbers, market share—that they neglect the fundamentals of building great products. They optimize for the numbers that show up on dashboards rather than the customer experience that creates lasting value.

When you focus on the scoreboard, you make decisions based on what moves the metrics in the short term. You acquire companies to boost user numbers, you launch features to improve engagement scores, you optimize for conversion rates. But if you're not simultaneously building products that customers genuinely want and need, those metrics become meaningless. The scoreboard shows progress, but the playing field—the actual business—is deteriorating.

\section{From Mission to Metrics}

The executive traced Alibaba's decline to a loss of its founding mission. ``We've gone from changing the world to gaming the system,'' he wrote. The company that once aimed to ``make it easy to do business anywhere'' had become a machine for hitting quarterly targets.

This shift from mission-driven to metric-driven is fatal. When a company's north star becomes its KPI dashboard, employees optimize for the numbers rather than the mission. They learn to game the system—manipulating performance scores, inflating metrics, focusing on activities that look good in reports rather than activities that create real value.

The executive described how ``internal rivalries and performance-score manipulation have become rampant.'' When performance is directly tied to pay but assessments are hidden and managers wield wide discretion, the system becomes corrupt. ``Good performers burn out, bad ones survive,'' he wrote. The scoreboard becomes a game to be played rather than a measure of real achievement.


\section{The Scoreboard Trap}

Alibaba's story illustrates the scoreboard trap: when a company becomes obsessed with metrics, KPIs, and operational data, it loses sight of what actually matters. The numbers on the dashboard become the goal, and the business fundamentals—building great products, serving customers well, creating sustainable competitive advantages—become secondary.

This is why Buffett's distinction matters: games are won by players who focus on the playing field, not by those whose eyes are glued to the scoreboard. The scoreboard tells you what happened, but it doesn't tell you how to win. To win, you need to focus on the fundamentals—the playing field—and let the scoreboard take care of itself.

For investors, this means looking beyond the metrics companies report. Revenue growth, market share, user numbers—these are all scoreboard metrics. What matters is whether the company is building real value on the playing field: products customers love, sustainable competitive advantages, honest management, and a culture that values execution over reporting.

When you see a company that talks constantly about KPIs, operational data, and performance metrics, but struggles to execute, innovate, or retain talent, you're seeing a company focused on the scoreboard. And as Alibaba's story shows, that focus can lead to cultural decay, failed bets, and strategic malaise—even when the scoreboard numbers look impressive.