\chapter{Focus on Score}

\epigraph{Games are won by players who focus on the playing field, not by those whose eyes are glued to the scoreboard}{Buffett}

\section*{The Delivery Wars: Alibaba vs. Meituan}

The first major delivery war between Meituan and Alibaba in 2025 erupted as China’s key e-commerce platforms competed fiercely for faster and cheaper food and instant deliveries. Alibaba expanded its Taobao Instant Commerce and Ele.me services with a reported one-year subsidy programme (about ¥50 billion) designed to attract consumers and merchants, prompting Meituan to increase discounts (e.g., coffee for as low as 2 yuan) and defend its market share. Daily orders on these platforms hit record levels—Meituan reported over 120 million in a single day while Alibaba’s delivery business achieved around 80 million orders—driven by deep price cuts and promotional spending amid a broader instant delivery price war involving JD.com as well. The intense rivalry led to shrinking profits and warned losses for Meituan and its competitors, as defensive subsidies and rapid expansion strained earnings in a largely unsustainable battle for users and volume \cite{scmp2025_meituan_record,scmp2025_alibaba_subsidy,news_reuters_meituan_loss,news_bi_pricewar}.

second distinct phase of the delivery war emerged in mid- to late-2025 as Meituan, Alibaba and JD.com escalated subsidy competition into what analysts described as a “second season” of price warfare. Building on the early 2025 battles, this round involved extremely high subsidy spending across food, coffee and instant retail offerings, with daily orders pushed to unprecedented levels and platforms pouring money into incentives to capture share in China’s instant delivery market. Meituan responded to Alibaba’s and JD.com’s aggressive discounts by deepening its own promotions, while JD.com expanded its footprint, forcing all three players into increasingly steep discounting that magnified losses and eroded profitability through the third quarter of 2025 \cite{turn0search23,turn0search24}. By late 2025, the scale of competing offers and financial results showed the toll: Alibaba’s operating profits plunged year-on-year, Meituan recorded its largest quarterly loss in history, and all three companies signalled a shift away from raw subsidy competition toward cost control and efficiency as the “wildfire” of price discounts burned through over two thousand yuan in combined spending without sustainable gains in net profits \cite{turn0search24,turn0search19}.