\chapter{Unaware of Risks}

\section{The Story of Nokia and the iPhone}

For a long time Nokia felt like the natural choice when you wanted a mobile phone. In the early 2000s its devices were everywhere: in offices, in cafés, on school playgrounds. A Nokia phone was sturdy, the battery lasted for days, and people swapped stories about how their old handset had survived a fall down the stairs or a trip through the washing machine. The interface was not beautiful, but after a while you knew where everything was. You could call, send texts, set an alarm and maybe play a quick game of Snake while waiting for the bus.

When you bought a new Nokia, you expected it to work in more or less the same way as your previous one. The screen might be a bit bigger, the camera a little sharper, the casing slightly slimmer, but the logic of the menus and the feel of the keypad were familiar. You scrolled through lists, pressed physical buttons and navigated nested options. If you wanted something extra, like an email client or a simple game, you might download it from a website or get it pre-installed by your operator, but most people never bothered. The phone was mainly for calls and messages, with a few extras on the side.

Then the iPhone arrived. At first glance it did not look like a phone at all: just a piece of glass with a single button at the bottom. There was no keypad, no green and red call buttons, no familiar Nokia-style menu. Instead, you woke the device and were greeted by a grid of colourful icons. To open something, you tapped it with your finger. To move around a page, you slid your finger across the screen. To zoom, you pinched. It felt more like touching the content itself than operating a machine.

From a user’s point of view, the difference in daily life was striking. On a Nokia, writing a long text message meant pressing the same key several times to get the right letter, or later, using a small numeric keyboard with predictive text. On the iPhone, a software keyboard appeared on the screen, and although it took some getting used to, the large, clear letters made it easier to type quickly once you adapted. Browsing the web on a Nokia meant a cramped, simplified version of the internet. On the iPhone, web pages looked almost like they did on a computer, and you could move around them with your fingers.

Another contrast showed up when you wanted to add something new to your phone. With Nokia, installing an application often meant dealing with obscure files, compatibility warnings and confusing prompts. Many users simply never tried. With the iPhone, you opened the App Store, searched or browsed, tapped ``Install'' and watched a new icon appear on your home screen. Suddenly the phone was not just a fixed set of functions; it was something that could keep changing as new apps appeared. You could add a better calendar, a different email client, a game your friends were playing or a service you had just read about.

Even everyday details felt different. On a Nokia you might transfer music by connecting the phone to a computer with a cable and using manufacturer software, hoping that the files ended up in the right place. On the iPhone, you synced with iTunes and your music, photos and contacts appeared on the device. Later, with cloud services, backups and restores happened over the air. When you upgraded to a newer iPhone, much of your digital life came with you automatically. For a typical user, this made the device feel less fragile: losing or replacing the physical phone did not mean losing everything stored on it.

Visually and emotionally, the two brands occupied different spaces. A Nokia phone was practical, sometimes even a little plain. It was the device you trusted to survive a rough day. The iPhone, on the other hand, was something people showed to friends. They passed it around at dinners to demonstrate a new app or a photo, they talked about upcoming software updates, and they lined up outside stores when a new model was released. The phone started to feel less like a tool and more like a personal companion, an object that combined communication, entertainment and work in one place.

From the user’s side of the table, Nokia’s fall did not happen overnight. For a while many people kept their old devices and were sceptical of the all-screen design. But as more friends bought iPhones and began using them for maps, email, social media and games, the gap in experience became hard to ignore. A Nokia might still last longer on a single charge or take a decent photo, but the iPhone was where the new ideas appeared first. Over time, more and more users made the switch, not because they had read a technical comparison, but because the iPhone simply felt like the place where their digital lives were happening.

By the time Nokia started shipping touch-screen phones that resembled smartphones in shape, many users had already chosen their side. Their contacts, photos, apps and habits were now tied to the iPhone. Looking back from the perspective of someone who lived through that period as an ordinary phone user, the story of Nokia and the iPhone is less about specifications and more about how it felt to hold each device in your hand, how it fit into your day and where you sensed the future was heading.

\section{What Nokia Said Was Risky vs. What Was Really Risky}

If you read Nokia's 2008 annual report in appendix \ref{appendix:nokia_2008_risk_factors}, the ``Risk Factors'' section looks familiar to anyone who has studied large global companies. There are paragraphs about currency fluctuations, interest rate movements, changes in tax rules, macroeconomic slowdowns, political uncertainty and regulatory shifts. These are real risks, but they are also generic: almost every multinational with factories, sales offices and supply chains around the world faces them. You could almost replace the company name on the front page and the risk section would still read sensibly.

On paper, this list sounds thorough. It checks all the expected boxes and creates the impression that management has systematically mapped out the dangers ahead. Yet when you place those disclosed risks next to the story of the iPhone and the user experience, something important is missing. The real threat to Nokia was not a sudden swing in the euro or a small change in interest rates. It was the possibility that the company would fail to keep up with what customers actually wanted from a phone once it turned into a pocket computer.

For an ordinary user in 2008, the decisive questions were simple: Which phone makes it easiest to stay in touch with friends? Where do the best apps appear first? Which device feels like the future when you hold it? These questions almost never appear in a financial risk section, but they shape the survival of a consumer technology business far more than a one- or two-percentage-point move in exchange rates. By focusing on abstract financial and macroeconomic variables, Nokia’s official risk language said little about the very concrete danger of being left behind in the race to satisfy customers.

The irony is that the generic risks in the report were valid, but they were not distinctive. Currency and interest rate fluctuations can hurt any global company. They can be modelled, hedged and described with similar sentences year after year. The risk of not meeting customers’ evolving needs is different. It demands close observation of how people actually use your product, humility about past success and a willingness to question the assumptions behind your entire business model. That kind of risk is harder to quantify and harder to write about in a neat paragraph, so it is often softened or ignored.

Looking back, Nokia’s journey from market leader to marginal player shows what happens when a company’s formal risk disclosures and its real exposure drift too far apart. On the surface, the 2008 report presented a careful list of dangers that could affect earnings and cash flow. In reality, the company was already facing a much deeper threat: that users would gradually choose another device that served their needs better. Without a clear awareness of that risk, and without putting it at the centre of strategy, even a giant can go from top to flop in just a few years.

If you read those risk paragraphs again with the iPhone in mind, you can feel this gap. Nokia talks about global economic turmoil, emerging market instability, exchange rate swings, supplier failures and even the danger that ``our products, services and solutions include increasingly complex technologies.'' All of that may be true, but none of it forces you to ask the simple question an ordinary buyer would ask in a shop: \emph{Which phone actually serves my needs better today?} The risks are described from the balance sheet outward, not from the customer backward.

The deeper problem is not that the disclosed risks are wrong, but that they are generic and inward-looking. They could apply, with minor edits, to almost any large industrial or technology company. What is missing is an honest, specific recognition that the centre of gravity in mobile phones was moving from hardware specifications and operator relationships to software, ecosystems and everyday user experience. Failing to name that possibility explicitly, and to treat it as a central risk, is a sign that the company did not truly study how fragile its position was.

In that sense, the real risk was not just ``competition'' in the abstract, but failing to see and meet customers’ evolving needs when a new type of product appeared. The iPhone made it obvious, in the hands of millions of users, that a phone could be simpler, more powerful and more personal at the same time. Nokia’s own risk factors, focused on the usual macro and operational issues, did little to prepare investors or managers for that shift. The company was not only unlucky; it was unaware of the most important risk it faced, and that lack of awareness helped turn its story from top to flop.