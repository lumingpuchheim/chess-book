\chapter{Company Communication: Empty Content}

In current times, there are different sources of information about a company. The easiest 
and first step to estimate a company is to visit its homepage. 

Shareholder letter and quarterly reports are the other sources of information.

Bad companies have empty content. We will list some of them.

\section{Information Overload Without Clear Purpose}

A major European bank's homepage demonstrates a fundamental organizational problem: it presents information without establishing why any of it matters to a potential customer.

The homepage hero section features ``The Pinnacle of Green Construction''—an article about sustainable building techniques. This raises immediate questions: What does a bank have to do with green construction? Is this a service they offer? A research interest? A corporate responsibility initiative? The connection between the content and the bank's core business is unclear, leaving visitors confused about the company's actual value proposition.

As you scroll down, the homepage becomes a chaotic mix of unrelated topics presented with equal visual weight:
\begin{itemize}
    \item A Partnership with a Football Club
    \item Positive ESG Rating
    \item The Digital Euro
    \item What is next for Europe's economy?
    \item Stablecoin
    \item Sustainable Student Living
    \item Saving Energy with Networks
    \item Sustainable Building
\end{itemize}

There is no hierarchy, no prioritization, no indication of what matters most. Is the football partnership as important as their banking services? Should a potential customer care about stablecoin research as much as opening an account? The homepage treats everything as equally important, which means nothing stands out.

The navigation structure compounds the problem. Under ``What we do'' → ``Products and services'', you find:
\begin{itemize}
    \item Client logins, online banking
    \item Locations worldwide
    \item Mobile services
    \item Corporate Bank
    \item DWS
    \item Investment Bank
    \item Private Bank
    \item Reports
    \item Awards
\end{itemize}

This mixes infrastructure (client logins, locations) with business divisions (Corporate Bank, Investment Bank) with administrative items (Reports, Awards). There is no clear customer journey, no guidance on where to start, no indication of which services are for which type of customer.

After extensive scrolling, a potential customer still cannot answer basic questions: Why should I choose this bank? What problems do they solve for someone like me? What should I know about ``Sustainable Building'' as a potential banking customer? The homepage has become a content dump—everything the company wants to talk about, organized for internal convenience rather than customer understanding.

\section{Buzzwords instead of Meaning}

A major enterprise software company's homepage and product pages demonstrate how marketing language can become completely meaningless through the accumulation of industry buzzwords.

Their artificial intelligence product page describes solutions as ``Business AI'' that enable an ``intelligent enterprise'' through ``cloud-native, end-to-end solutions'' for ``next-generation businesses'' seeking ``digital transformation''. The homepage promises to help companies become ``future-ready'' with ``AI-powered insights'' and ``data-driven decision making''.

If you remove the buzzwords—``AI-powered'', ``cloud-native'', ``end-to-end'', ``intelligent'', ``next-generation'', ``digital transformation'', ``future-ready'', ``data-driven''—what remains? Very little. The language tells you nothing about what the software actually does, who it's for, or what problems it solves.

Consider this typical product description: ``ABC Business AI delivers intelligent, cloud-native, end-to-end solutions that empower next-generation enterprises to achieve digital transformation through AI-powered insights and data-driven decision making.''

After reading this, can you answer: What does the software do? What business problem does it solve? How is it different from competitors? The sentence is grammatically correct and uses all the right industry terms, but it communicates nothing concrete. It could describe any enterprise software company's AI product.

This is the hallmark of empty content: language that sounds impressive but conveys no actual information. The company is hiding behind industry jargon because they either don't know how to explain their value clearly, or they have nothing specific to say. Either way, it's a red flag for investors—if a company cannot clearly articulate what they do, how can they execute on it?

\section{Talking about themselves, not the Customer}

A security hardware company's brands page demonstrates the ultimate form of self-centered content: it presents information that only makes sense if you already know what the company does.

The page is titled ``Our brands'' and displays a full-screen grid of brand names: AD Systems, Austral Lock, AXA, Boss Door Controls, Bricard, Brio, Briton, CISA, Dexter, Dorcas, Falcon, FSH, Gainsborough, Glynn-Johnson, Interflex, Isonas, IVES, Krieger, Kryptonite, LCN, Legge, Locknetics, Normbau, plano, Republic, Schlage, Simons-Voss, SOSS Door Hardware, Stanley Access Technologies, Steelcraft, TGP, Trelock, Trimco, Unicel Architectural, Von Duprin, Yonomi, Zentra, Zero International.

That's it. Just a list of brand names covering the entire screen.

The page includes a brief introductory text: ``Our prominent security brands are sold across the globe. These brands offer a variety of products to keep our customers safe and secure throughout their daily lives and when security is needed the most.''

But this generic statement doesn't answer the fundamental questions: What do these brands actually do? What problems do they solve? Who are they for? What makes them different from each other or from competitors? Why should a potential customer care about any of these brands?

After viewing the page, you know the company owns many brands, but you have no idea what value the business provides. Are these brands for residential use? Commercial? Industrial? What types of security problems do they solve? What makes this company's approach better than alternatives?

The page assumes you already understand the security hardware industry and recognizes these brand names. It's organized for internal company purposes—showing off their brand portfolio—rather than helping potential customers understand what the business does or why they should care. This is empty content: information that exists but communicates nothing useful to someone trying to evaluate the company.

\section{Too many Products}

An identity and access management company's homepage demonstrates what happens when a company tries to be everything to everyone: the message becomes so diluted that it communicates nothing.

The homepage presents seven main solution categories: Access Control, Biometrics, Card Printing, Digital Security, Identity \& Access Management, RFID, and RTLS (Real-Time Location Systems). Under each category, there are multiple sub-features. For example, under Access Control alone, you find: Access Control Systems, Cloud-Based Physical Access Control, Contactless Payments and Ticketing, Seamless Mobile Access, Physical Identity \& Access Management, and Visitor Management.

But it doesn't stop there. The homepage also organizes everything by industry vertical: Banking \& Finance, Education, Enterprise, Government, Healthcare, Hospitality, Manufacturing \& Logistics, Oil \& Gas, Retail, Sports/Events, and Travel \& Transportation. Each industry gets its own description of how the solutions apply.

Then there's a separate ``Explore Our Products'' section listing: Biometrics, Cards and Credentials, Controllers, Door Readers \& Access Control Software, Events and Mass Transit Credentials \& Readers, Financial Instant Issuance, ID Badge and Card Printing, ID Document \& Ticket Readers, Identity \& Authentication, Real-Time Location Systems, RFID Tags/Readers/Modules, and Textile Management.

By the time you've scrolled through all of this, you've seen the same products and features organized in three different ways—by solution type, by industry, and by product category. The homepage becomes a comprehensive directory of everything the company does, but it tells you nothing about what they're best at, what their core competency is, or what makes them different from competitors.

This is the symptom of a company that kept adding products and messages without killing bad ones or establishing priorities. When everything is important, nothing is important. A potential customer cannot determine: What is this company's specialty? What should I focus on? Where do I start? The homepage has become a product catalog rather than a clear value proposition.

\section{Buzzwords Irrelevant to the Business}

An RFID reader manufacturer's homepage demonstrates how companies use trendy buzzwords that sound impressive but have nothing to do with their actual business.

The homepage prominently features interactive 3D models of ``Smart Office'', ``Smart Factory'', ``Smart Hospital'', and ``Smart Campus''. The hero section promises to ``unlock the Smart Office, the Smart Factory, the Smart Hospital and the Smart Campus of tomorrow'' through their ``innovative multi-technology readers''.

But what does the company actually do? They make RFID readers—hardware devices that read user IDs from cards or mobile devices. That's it. The core business is simple: a reader delivers a user ID to an access control system. The reader doesn't create a ``Smart Office'' or transform a factory into a ``Smart Factory''. Those are marketing fantasies that have nothing to do with the actual product.

The homepage describes their readers as enabling 
``unified access control solutions'' and promises to 
``define the future of access control''. 
But strip away the ``Smart'' prefixes and the grandiose language, 
and you're left with a company that makes RFID readers. 
The buzzwords—``Smart Office'', ``Smart Factory'', ``Smart Hospital'', ``Smart Campus''—sound impressive, but they describe outcomes that depend on entire systems, software platforms, and infrastructure decisions far beyond what a simple RFID reader provides.

This pattern continues in their content marketing. 
A blog post titled ``Migrating to Modern Access Credentials: 
A Getting Started Guide'' promises practical guidance. 
But the content immediately shifts to abstract industry trends: 
``Two key initiatives are leading the way''. 
The website itself is supposed to be a getting started guide, 
yet the content wastes the reader's time with vague industry commentary 
instead of actionable information.

The guide presents itself as a step-by-step process, 
but the steps reveal empty content. Step 3, ``Plan for the Future'', 
tells readers to ``adopt open standards and credential-agnostic reader hardware'' 
to ensure their system ``isn't tied to outdated or proprietary technology''— 
vague advice without concrete implementation steps. 
Step 4, ``Create a Migration Strategy'', instructs readers to 
``think about a step-by-step migration'' and ``define your security requirements up front'', 
but provides no actual framework or methodology for doing so. 
Step 6, ``Future-Proof Your Access Control'', essentially repeats Step 3's theme 
about adapting to future changes, recommending ``universal readers with remote update capabilities'' 
to enable ``new encryption methods or emerging standards''—the same abstract concept 
presented as a separate step. Most tellingly, the guide promises ``future-proof'' solutions 
throughout, but how can anyone guarantee something is ``future-proof'' 
when the future is inherently uncertain? 
This is marketing language masquerading as technical guidance— 
empty promises wrapped in the appearance of expertise.

This is empty content in its purest form: buzzwords that sound 
relevant but describe things the company doesn't actually do or control. The ``Smart'' prefix has become a meaningless marketing placeholder—you could attach it to anything (Smart Door, Smart Card Reader, Smart Badge) and it would sound equally impressive and equally meaningless. For investors, this is a red flag: if a company cannot accurately describe what they do without resorting to irrelevant buzzwords, they may not understand their own business model or competitive position.

\section{Overstressing Experiences}

When a football club adds Messi to the team as a player, nothing needs to be said. His achievements speak for themselves. When an unknown player is added, the best one can say is ``the player has experiences''. The word ``experiences'' hides the fact that there is nothing concrete to be said.

This pattern appears everywhere in corporate communication and resumes. Companies describe executives as having ``extensive experience in the industry''. Job candidates list ``years of experience'' without specifying what they actually accomplished. The word ``experiences'' becomes a placeholder for substance—a way to sound impressive without providing concrete information.

What are the experiences? If one can answer this question clearly—like ``he has created a company on his own'', ``she led a team that increased revenue by 50\%''—the word itself becomes superfluous. The concrete achievements are what matter, not the vague claim of having ``experiences''.

When you see a company homepage describing their team as having ``decades of combined experience'' or management bios emphasizing ``years of experience'' rather than specific accomplishments, be careful. The word ``experiences'' often signals that there are no specific achievements to highlight. A company with a track record of success will mention specific results. They don't need to hide behind the word ``experiences''.

For investors, this is a useful filter: if a company cannot point to concrete achievements and instead relies on vague claims about ``experiences'', they may not have much to show for their work. Real achievements don't need the word ``experiences'' to sound impressive.

\section{Superlatives}

Enron's 2000 Annual Report provides a masterclass in how companies use superlatives to create an illusion of success while hiding fundamental problems. The shareholder letter opens with: ``Enron's performance in 2000 was a success by any measure, as we continued to outdistance the competition and solidify our leadership in each of our major businesses.''

Notice the language: ``success by any measure'', ``outdistance the competition'', ``solidify our leadership''. These are absolute claims with no qualification. But what does ``success by any measure'' actually mean? Which measures? The letter doesn't specify, because the real measures—cash flow, sustainable profitability, honest accounting—would tell a different story.

The report can be found in appendix \ref{appendix:enron_2000_annual_report}. It is saturated with superlatives throughout:

\begin{itemize}
    \item ``record \$1.3 billion''—record net income, but how much was real versus accounting manipulation?
    \item ``unique and strong businesses''—unique how? Strong in what way?
    \item ``tremendous opportunities for growth''—tremendous is meaningless without specifics
    \item ``very large existing markets''—how large? What's the addressable market?
    \item ``enormous growth potential''—potential is not reality
    \item ``unparalleled liquidity''—unparalleled compared to whom?
    \item ``Astonishing Success of EnronOnline''—astonishing to whom? Based on what metrics?
    \item ``break-out year''—break-out in what sense? Revenue? Profit? Or just hype?
    \item ``compelling commercial model''—compelling to whom? Where's the proof?
    \item ``strong returns to shareholders''—strong compared to what? The market? Their cost of capital?
\end{itemize}

The pattern is clear: every claim uses the strongest possible language—``enormous'', ``record'', ``unparalleled'', ``astonishing'', ``tremendous''—but provides no context for comparison or verification. When a company describes everything as ``record-breaking'', ``unprecedented'', or ``industry-leading'' without concrete benchmarks, the superlatives become meaningless.

When you see a company using excessive superlatives—especially absolute claims like ``by any measure'', ``unparalleled'', or ``industry-leading'' without specific comparisons—be skeptical. Honest companies don't need to oversell their achievements. They let the numbers speak for themselves. Enron's use of superlatives was a red flag that the underlying business couldn't support the claims.