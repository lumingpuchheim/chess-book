\chapter{Jealousy}

\section{The Story of Alibaba, Meituan, and Ele.me}

Alibaba's involvement with Ele.me began in 2016, at a time when China’s food delivery market was becoming strategically important rather than just a side business. By then, food delivery was no longer only about meals; it was becoming the physical entrance to local services, payments, and last-mile logistics. Alibaba had already built dominance in e-commerce, but it was weak in instant, high-frequency offline consumption. In April 2016, Alibaba and Ant Financial invested about USD 1.25 billion into Ele.me, marking Alibaba’s clear decision to enter this battlefield.

This move was closely linked to Alibaba’s earlier relationship with Meituan. Before Meituan merged with Dianping in 2015, Alibaba had been one of Meituan’s shareholders. After the merger, Meituan increasingly aligned with Tencent’s ecosystem, integrating deeply with WeChat payments and traffic. Alibaba found itself in an awkward position: holding a minority stake in a platform that was becoming a strategic weapon for a rival ecosystem. Between 2016 and 2017, Alibaba exited its Meituan position. The sale was not about valuation; it was about control. Alibaba did not want to fight a critical local-services war as a passive shareholder. It wanted its own platform.

With that decision made, Ele.me became Alibaba’s chosen vehicle. Over the next two years, Ele.me was pulled tightly into Alibaba’s system. Alipay became the default payment method, merchant data was integrated, and Ele.me’s delivery network was treated as infrastructure rather than a standalone business. In April 2018, Alibaba completed the takeover, buying the remaining shares of Ele.me for roughly USD 9.5 billion. This turned Ele.me into a wholly owned subsidiary and a core part of Alibaba’s “New Retail” vision, where food, groceries, pharmacies, and daily goods could all be delivered within minutes.

The acquisition immediately escalated the delivery war. Meituan, backed by Tencent, already dominated food delivery and had stronger merchant density and user habits. Alibaba responded by heavily subsidizing Ele.me: free deliveries, aggressive discounts, and higher rider incentives. For a short period after 2018, Ele.me regained significant market share and even briefly claimed leadership in some cities. However, the underlying economics were brutal. Both sides burned cash, and scale advantages mattered more than capital alone. Meituan’s broader local-services ecosystem—hotels, travel, entertainment, and reviews—kept users inside its app more frequently, reinforcing its lead.

By the early 2020s, the result of the delivery war became clear. Meituan emerged as the long-term winner in food delivery, holding a dominant market share, while Ele.me settled into a solid but distant second place. Alibaba did not "lose" completely, but it failed to dislodge Meituan's leadership. Instead, Ele.me was gradually repositioned from a pure food-delivery challenger into part of Alibaba's instant retail and local-services infrastructure, eventually being absorbed more deeply into Taobao and Alibaba's broader commerce flow. The war reshaped the industry, entrenched two ecosystems, and confirmed that food delivery in China was less a profit center than a strategic gateway to consumer behavior.

\section{How Jealousy Drove Strategic Decisions}

When you look beneath the surface of Alibaba's acquisition of Ele.me, the decision reveals something uncomfortable: it was driven less by a clear-eyed assessment of business value and more by a reaction to what Tencent had achieved with Meituan. Alibaba bought Ele.me not primarily because it was a good business opportunity, but because Tencent had backed Meituan, and Alibaba could not bear to see its rival control such an important piece of the local-services market.

The psychology here is revealing. Alibaba had once been a shareholder in Meituan, but after Meituan aligned with Tencent's ecosystem, Alibaba found itself in the position of watching a company it had helped build become a strategic asset for a competitor. That shift triggered a defensive response. Rather than accept that Meituan had chosen a different partner, or that food delivery might not be the right battlefield for Alibaba to fight on, the company decided it needed its own platform, regardless of whether that platform could realistically compete.

This jealousy was then wrapped in strategic language. The "New Retail" vision became the official explanation: a grand plan to integrate online and offline commerce, to deliver everything within minutes, to create a seamless consumer experience. There is nothing wrong with that vision in principle, but in practice it served as a rationalization for an acquisition that was fundamentally about not letting Tencent win. The "New Retail" narrative made the Ele.me purchase sound like a forward-looking strategic move, when in reality it was a reactive, competitive response driven by fear of being left behind.

This is a form of \vocab{wishful thinking}{Wishful thinking}{ beliefs based on what might be pleasing to imagine, rather than on evidence, rationality, or reality} \index{Wishful thinking}. Alibaba wished that by acquiring Ele.me and spending billions on subsidies, it could replicate Meituan's success and prevent Tencent from gaining an advantage. It wished that the "New Retail" integration would create synergies powerful enough to overcome Meituan's head start and network effects. It wished that capital alone could win a war that was really about user habits, merchant relationships and ecosystem depth.

The outcome tells a different story. Despite massive investment, Ele.me remained a distant second to Meituan. The subsidies burned cash without creating sustainable competitive advantages. The "New Retail" integration, while not without value, did not transform Ele.me into a market leader. What Alibaba gained was not a victory over Meituan, but a costly defensive position that prevented Tencent from having the field entirely to itself.

The lesson here is about the danger of making strategic decisions based on what competitors have, rather than on what your own business genuinely needs. Jealousy can make a company chase markets and assets not because they fit its strengths, but because it cannot stand to see a rival control them. When that jealousy is dressed up as strategic vision, it becomes even more dangerous, because it allows management to convince themselves and investors that they are making rational choices when they are really making emotional ones. In Alibaba's case, the Ele.me acquisition was a multi-billion-dollar expression of competitive anxiety, rationalized as forward-looking strategy.

\section{The Irony of Selling What You Own}

There is a deeper irony in Alibaba's relationship with Meituan that reveals how jealousy can blind even experienced investors to their own best interests. Alibaba was not a latecomer to Meituan; it was one of the company's early investors, holding a stake that would later become extremely valuable. When Meituan merged with Dianping in 2015 and then aligned with Tencent's ecosystem, Alibaba found itself in a position that many investors would envy: it owned a piece of what was becoming the dominant food delivery platform in China.

But that ownership came with a psychological cost that Alibaba could not bear. Meituan's integration with WeChat payments and traffic meant that every transaction, every user interaction, every piece of data was flowing through Tencent's ecosystem. From Alibaba's perspective, this was not just a business partnership; it was a betrayal. A company that Alibaba had helped fund was now strengthening a competitor's position. Every time a user paid for a Meituan order through WeChat Pay, it reinforced Tencent's payment network at the expense of Alipay. Every time a user discovered Meituan through WeChat, it deepened Tencent's control over consumer behavior.

This jealousy was not about the financial value of the Meituan stake. It was about control and pride. Alibaba could not accept being a passive shareholder in a company that was actively helping Tencent build its ecosystem. The psychological discomfort of watching your investment become a strategic asset for your rival was too much to bear. So between 2016 and 2017, Alibaba sold its Meituan position. The decision was framed as strategic: Alibaba wanted its own platform, not a minority stake in someone else's. But beneath that rationalization lay a simple emotional truth: it could not stand to see Tencent benefit from something Alibaba had helped create.

The cost of this jealousy-driven decision was enormous. By selling its Meituan stake, Alibaba gave up ownership in what would become the clear winner of the food delivery war. Meituan's market value grew substantially in the following years, and Alibaba missed out on those gains. More importantly, by exiting Meituan, Alibaba forced itself into a position where it had to find an alternative platform to compete in food delivery. That alternative was Ele.me, which required a USD 9.5 billion acquisition plus billions more in subsidies, all to achieve a distant second-place position.

The irony is stark: Alibaba had already invested in the winning platform, but jealousy made it sell that position and then spend far more money trying to build or buy a competitor. If Alibaba had kept its Meituan stake and accepted that a successful investment could benefit multiple ecosystems, it would have retained financial exposure to the market leader while avoiding the massive costs of the Ele.me acquisition and subsidy war. Instead, jealousy turned a winning investment into a costly mistake.

This pattern reveals something important about how jealousy works in business decisions. It is not just about wanting what competitors have; it is also about not being able to tolerate seeing competitors benefit from what you have created or funded. Alibaba's jealousy of Tencent's relationship with Meituan was so strong that it preferred to sell a valuable stake and start from scratch rather than continue benefiting from an investment that also happened to help a rival. The financial logic was clear: keep the stake, enjoy the returns, and find other ways to compete. The emotional logic was different: if Tencent wins, we lose, even if we also win. That emotional logic cost Alibaba billions.