% Enron 2000 Annual Report - Appendix Content
% This file contains the content from Enron's 2000 Annual Report
% Source: https://enroncorp.com/corp/investors/annuals/2000/

\section*{Financial Highlights}

Financial highlights are presented in tabular format showing key financial metrics for 2000 compared to previous years. The highlights include revenue, net income, earnings per share, and other key financial indicators that demonstrate Enron's strong performance in 2000.

\section*{Letter to Shareholders}

Enron's performance in 2000 was a success by any measure, as we continued to outdistance the competition and solidify our leadership in each of our major businesses. In our largest business, wholesale services, we experienced an enormous increase of 59 percent in physical energy deliveries. Our retail energy business achieved its highest level ever of total contract value. Our newest business, broadband services, significantly accelerated transaction activity, and our oldest business, the interstate pipelines, registered increased earnings. The company's net income reached a record \$1.3 billion in 2000.

Enron has built unique and strong businesses that have tremendous opportunities for growth. These businesses—wholesale services, retail energy services, broadband services, and transportation services—can be significantly expanded within their very large existing markets and extended to new markets with enormous growth potential. At a minimum, we see our market opportunities company-wide tripling over the next five years.

Enron is laser-focused on earnings per share, and we expect to continue strong earnings performance. We will leverage our extensive business networks, market knowledge, and logistical expertise to produce high-value bundled products for an increasing number of global customers.

\subsection*{Competitive Advantages}

Our targeted markets are very large and are undergoing fundamental changes. Energy deregulation and liberalization continue, and customers are driving demand for reliable delivery of energy at predictable prices.

Enron's competitive advantages include:

\begin{itemize}
\item Robust network of strategic assets that we own or have contractual access to, which give us greater flexibility and speed to reliably deliver widespread logistical solutions.
\item Unparalleled liquidity and market-making abilities that result in price and service advantages.
\item Risk management skills that enable us to offer reliable prices as well as reliable delivery.
\item Innovative technologies such as EnronOnline to deliver products and services easily at the lowest possible cost.
\end{itemize}

These capabilities enable us to provide high-value products and services other wholesale service providers cannot. We can take the physical components and repackage them to suit the specific needs of our customers.

\subsection*{Astonishing Success of EnronOnline}

In late 1999 we extended our successful business model to a web-based system, EnronOnline. EnronOnline has broadened our market reach, accelerated our business activity and enabled us to scale our business.

With EnronOnline, we are reaching a greater number of customers more quickly and at a lower cost than ever. It's a great new business generator, attracting users who are drawn by the site's ease of use.

EnronOnline has enabled us to scale quickly, soundly and economically. Since its introduction, EnronOnline has expanded to include more than 1,200 of our products. It also has streamlined our back-office operations.

\subsection*{Enron Wholesale Services}

The wholesale services business delivered record physical volumes of 51.7 trillion British thermal unit equivalent per day (TBtue/d) in 2000, compared to 32.4 TBtue/d in 1999. As a result, wholesale services earnings increased significantly.

In North America, we deliver almost double the amount of natural gas and electricity than the second tier of competitors. Our network of 2,500 delivery points provides price advantages, flexibility and speed.

We are building a similar, large network in Europe. In 2000 we marketed 3.6 Bcf/d of natural gas and 53 million MWh in this market, a vast increase over 1999. As markets open, we tenaciously pursue the opportunities.

We are extending Enron's proven business approach to other markets, and integrating EnronOnline into all our businesses as an accelerator. Our growth rates are rising in areas such as metals, forest products and other commodities.

\subsection*{Enron Energy Services}

Our retail unit is a tremendous business that experienced a break-out year in 2000. We signed contracts with a total value of \$16.1 billion of customers' future energy expenditures, almost double the \$8.7 billion signed in 1999.

\subsection*{Enron Broadband Services}

We have created a new market for bandwidth intermediation with Enron Broadband Services. In 2000 we completed 321 transactions with 45 counterparties. We are expanding our broadband intermediation capabilities.

Part of the value we bring to the broadband field is network connectivity—providing the switches, the network intelligence and the intermediation skills to enable the efficient exchange of capacity.

Enron also has developed a compelling commercial model to deliver premium content-on-demand services via the Enron Intelligent Network. Content providers want to extend their established businesses and reach new customers.

\subsection*{Enron Transportation Services}

The new name for our gas pipeline group accurately reflects a cultural shift to add more innovative customer services to our efficient pipeline operations. To serve our customers more effectively, we are adding new services beyond traditional pipeline transportation.

\subsection*{Strong Returns}

Enron is increasing earnings per share and continuing our strong returns to shareholders. Recurring earnings per share have increased steadily since 1997 and were up 25 percent in 2000.

\section*{Enron Wholesale Services}

The wholesale services business delivered record physical volumes of 51.7 trillion British thermal unit equivalent per day (TBtue/d) in 2000, compared to 32.4 TBtue/d in 1999. As a result, wholesale services earnings increased significantly.

In North America, we deliver almost double the amount of natural gas and electricity than the second tier of competitors. Our network of 2,500 delivery points provides price advantages, flexibility and speed.

We are building a similar, large network in Europe. In 2000 we marketed 3.6 Bcf/d of natural gas and 53 million MWh in this market, a vast increase over 1999. As markets open, we tenaciously pursue the opportunities.

We are extending Enron's proven business approach to other markets, and integrating EnronOnline into all our businesses as an accelerator. Our growth rates are rising in areas such as metals, forest products and other commodities.

\section*{Enron Energy Services}

Our retail unit is a tremendous business that experienced a break-out year in 2000. We signed contracts with a total value of \$16.1 billion of customers' future energy expenditures, almost double the \$8.7 billion signed in 1999.

\section*{Enron Broadband Services}

We have created a new market for bandwidth intermediation with Enron Broadband Services. In 2000 we completed 321 transactions with 45 counterparties. We are expanding our broadband intermediation capabilities.

Part of the value we bring to the broadband field is network connectivity—providing the switches, the network intelligence and the intermediation skills to enable the efficient exchange of capacity.

Enron also has developed a compelling commercial model to deliver premium content-on-demand services via the Enron Intelligent Network. Content providers want to extend their established businesses and reach new customers.

\section*{Enron Transportation Services}

The new name for our gas pipeline group accurately reflects a cultural shift to add more innovative customer services to our efficient pipeline operations. To serve our customers more effectively, we are adding new services beyond traditional pipeline transportation.

\section*{Financial Review}

The Financial Review section* is available in PDF format. Please note that the PDF versions are quite large and will require some time to download. You will need the Adobe Acrobat 4.0 reader to view the downloaded file. The Financial Review contains detailed financial statements, management's discussion and analysis, and other financial disclosures required for the annual report.

\section*{Our Values}

\subsection*{Communication}

We have an obligation to communicate. Here, we take the time to talk with one another… and to listen. We believe that information is meant to move and that information moves people.

\section*{Board of Directors}

The Board of Directors includes:

\begin{itemize}
\item ROBERT A. BELFER (1, 3) - New York, New York - Chairman, Belco Oil \& Gas Corp.
\item NORMAN P. BLAKE, JR. (3, 4) - Colorado Springs, Colorado - Chairman, President and CEO, Comdisco, Inc., and former Chairman and CEO, USF\&G Corporation
\item RONNIE C. CHAN (2, 3) - Hong Kong - Chairman, Hang Lung Group Ltd.
\item JOHN H. DUNCAN (1, 4) - Houston, Texas - Former Chairman and CEO, Gulf States Utilities Company
\item WENDY L. GRAMM (2, 5) - Washington, D.C. - Senior Fellow, Mercatus Center, George Mason University, and former Chairman, U.S. Commodity Futures Trading Commission
\item ROBERT K. JAEDICKE (1*, 4) - Stanford, California - Dean Emeritus and Professor of Accounting, Graduate School of Business, Stanford University
\item CHARLES A. LEMAISTRE (2, 5) - Houston, Texas - President Emeritus, M.D. Anderson Cancer Center
\item JOHN MENDELSON (2, 4*) - New York, New York - Senior Managing Director, CIBC World Markets Corp.
\item JEROME J. MEYER (2, 3) - Tampa, Florida - Chairman, TECO Energy, Inc.
\item FRANK SAVAGE (2, 3) - New York, New York - Chairman, Alliance Capital Management International
\item JOHN WAKEHAM (1, 4) - London, England - Former Secretary of State for Energy, United Kingdom
\item HERBERT S. WINOKUR, JR. (2, 5*) - Greenwich, Connecticut - Chairman, Capricorn Holdings, Inc.
\item KENNETH L. LAY (1, 3*) - Houston, Texas - Chairman, Enron
\item JEFFREY K. SKILLING (1) - Houston, Texas - President and Chief Executive Officer, Enron
\item MARK A. FREVERT (3) - Houston, Texas - Vice Chairman, Enron
\end{itemize}

(1) Executive Committee (2) Audit Committee (3) Finance Committee (4) Compensation Committee (5) Nominating Committee * Denotes Chairman

\section*{Enron Corporate Policy Committee}

\begin{itemize}
\item KEN LAY - Chairman, Enron
\item JEFF SKILLING - President and Chief Executive Officer, Enron
\item CLIFF BAXTER - Vice Chairman \& Chief Strategic Officer, Enron
\item RICK CAUSEY - Executive Vice President \& Chief Risk Officer, Enron
\item RICK BUY - Chief Risk Officer, Enron Wholesale Services
\item STEVE KEAN - Executive Vice President, Government Affairs
\item MARK KOENIG - Executive Vice President, Investor Relations
\item MARK PALMER - Vice President, Corporate Communications
\item CINDY OLSON - Executive Vice President, Human Resources and Community Relations
\item KEN RICE - Chief Executive Officer, Enron Broadband Services
\item DAVID DELAINEY - Chief Executive Officer, Enron Energy Services
\item MARK FREVERT - Vice Chairman, Enron
\item JIM FALLON - President, Enron Wholesale Services
\item STEVE BERGER - Executive Vice President, Enron Transportation Services
\end{itemize}

\section*{Shareholder Information}

\subsection*{Transfer Agent, Registrar, Dividend Paying and Reinvestment Plan Agent (DirectService Program)}

EquiServe Trust Company, N.A.\\
P.O. Box 2500\\
Jersey City, NJ 07303-2500\\
(800) 519-3111\\
Website: http://www.equiserve.com

For additional shareholder information, visit http://www.enron.com

\vspace{0.5cm}

\textit{Source: Enron 2000 Annual Report, available at \url{https://enroncorp.com/corp/investors/annuals/2000/}}

\textit{Note: This document contains excerpts from Enron's 2000 Annual Report. Some section*s may require additional content extraction from the original PDF or web pages.}
